\chapter{Временная эволюция физических систем}

В этой главе не будет идти речи о выборе базиса. Это представление зависимости во времени, то есть разные способы введения времени в квантовой механике.

\textbf{Терминология:}
в квантовой механике принято говорить об уравнениях временной эволюции. Это более общее понятие, чем просто уравнение движения.\index{Уравнение временной эволюции}

Кроме слова <<представление>> также употребляют слово <<картина>>.\index{Представление! картина}

\section{Представление Шрёдингера}
\subsection{Оператор эволюции}
\Def Предположим, что имеется состояние, которое в начальный момент времени $t_0$ имеет вид $\qu{\psi(t_0)}$.

В результате эволюции состояние описывается волновым вектором $\qu{\psi(t)}$.
$$
    \qu{\psi(t)} = \hat U (t, t_0) \qu{\psi(t_0)}
$$
\Def Оператор $\Hat U$ называется \emph{оператором эволюции}.\index{Оператор! эволюции}

\textbf{Свойства:}
\begin{enumerate}
  \item Представим, что есть промежуточный момент времени, $t_0 < t' < t$.

  Тогда эволюцию можно представлять, как <<происходящую в два этапа>>:
  $$
    \Hat U(t, t') \Hat U(t', t_0) = \Hat U(t, t_0)
  $$
  При совпадающих моментах времени:
  $$
    \Hat U(t_0, t_0) = \hat 1
  $$
  \item Если внешние условия стационарны, то оператор эволюции не может зависеть индивидуально от одного момента времени и от другого момента времени. Он зависит от их разности, то есть определяется промежутком времени:
      $$
        \hat U(t, t_0) = \Hat U(t - t_0) \eqdef \Hat U(\tau)
      $$
      Преобразуем это равенство:
      $$
      \begin{cases}
        \hat U(\tau_1) \hat U(\tau_2) = \hat U(\tau_1 + \tau_2),\\
        \hat U(0) = \hat 1
      \end{cases}
      $$
  \item Если время обратимо:
  $$
    \begin{cases}
        \qu{\psi(t_0)} \to \qu{\psi(t)},\\
        \qu{\psi(t)} \to \qu{\psi(t_0)}
    \end{cases}
    \RA
    \Hat U^{-1} (t, t_0) = \hat U^{-1} (\tau) = \hat U(-\tau)
  $$
  $$
    \Hat U^{-1} (\tau) \Hat U(\tau) = \hat U(\tau) \Hat U^{-1} (\tau) = \hat 1
  $$
  Вывод: семейство операторов $\Hat U(\tau)$ образует однопараметрическую группу преобразований.
  \item (Чисто физическое требование). При этих преобразованиях не должно изменяться физическое содержание теории.

      Условие неизменности содержится в вероятности. Поставим условие неизменности скалярного произведения.
      \begin{eqnarray*}
        \qs{\phi(t)}{\psi(t)} &=& \qtri{\phi(t_0)}{\Hat U^+ (t-t_0) \Hat U(t-t_0)}{\psi(t_0)}\\
                              &=&\qs{\phi(t_0)}{\psi(t_0)} \Rightarrow {\Hat U}^+{\Hat U} = {\Hat 1}
      \end{eqnarray*}
\end{enumerate}%end of properties
Явный вид оператора $\Hat U(\tau)$ получается из уравнения Шрёдингера (которое постулируется квантовой механикой).
$$
    \left( i\hbar \ud{}{t} - \Hat H \right) \qu{\psi(t)} = 0
$$
$$
    \left( i \hbar \ud{}{t} - \hat H \right) \hat U (t, t_0) \qu{\psi(t_0)} = 0
$$
Если гамильтониан явно не зависит от времени, то
$$
    \hat U(t, t_0) = \exp \left( -\dfrac{i}{\hbar} \Hat H (t-t_0) \right)\index{Оператор! эволюции}
$$
Гамильтониан должен быть эрмитовым оператором, как главная наблюдаемая нашей системы:
$$
    \Hat H^+ = \hat H
$$
Значит, оператор $ \Hat U$ унитарный:
$$
    \Hat U^+ = \Hat U^{-1}
$$

\Def Описание эволюции системы, когда эволюция во времени заключается в том, что векторы состояний, или волновые функции зависят от времени, а операторы не зависят от времени, называется \emph{представлением Шрёдингера}.\index{Представление! Шредингера}

\subsection{Стационарные состояния}
Пусть задана система:
$$
    \ud{\hat H}{t} = 0
$$
Найдём решение уравнения Шрёдингера:
$$
    i\hbar \ud{}{t} \qu{\psi(t)} = \Hat H \qu{\psi(t)}
$$
методом разделения переменных.

Представим волновую функцию в виде произведения:
$$
    \qu{\psi(t)} = f(t)\cdot \qu{\psi},
$$
где $f(t)$ зависит от $t$, и $\qu{\psi}$ не зависит от времени.

Производим <<формальное деление>>:

$$
    \dfrac{i\hbar \ud{}{t} f(t)}{f(t)} = \dfrac{\Hat H \qu{\psi}}{\qu{\psi}} = \underbrace{E}_{\mathrm{const}}
$$
$$
    \begin{cases}
        \Hat H \qu{\psi} = E \qu{\psi},\\
        i\hbar \ud{}{t} f(t) = E f(t)
    \end{cases}
$$
Это уравнение на собственные векторы и значения гамильтониана, стационарное уравнение Шрёдингера\index{Уравнение Шредингера! стационарное}. Физически собственные значения гамильтониана~--- полная энергия системы.

Решение этого уравнения даёт:
$$
    f(t) = e^{-\frac{iEt}{\hbar}}, \quad f(0) = 1
$$
\textbf{Вывод:} в стационарном случае уравнение Шрёдингера имеет следующую систему частных решений:
$$
\begin{cases}
    \qu{\psi_E(t)} = \exp \left( -\dfrac{iEt}{\hbar} \right) \qu{\psi_E},\\
    \Hat H \qu{\psi_E} = E \qu{\psi_E}
\end{cases}
$$
Эти состояния называются \emph{стационарными}. Ещё раз подчеркнём, что \textbf{стационарные состояния~--- это те состояния, в которых энергия имеет вполне определённые значения, и более того, сохраняется во времени}.\index{Состояние! стационарное}

Общее решение записывается разложением по системе частных решений как по базису\footnote
{
  $\quad
    \ssum_{A} \eqdef \left\{
              \begin{array}{ll}
                \sum_n, & \hbox{дискретный спектр;} \\
                \int \cdots d A, & \hbox{непрерывный спектр.}
              \end{array}
            \right.
  $
}.
$$
    \qu{\psi(t)} = \ssum_E C_E \qu{\psi_E(t)} = \ssum_E C_E e^{- \frac{iEt}{\hbar}} \qu{\psi_E}
= \sum_n C_n e^{- \frac{iE_nt}{\hbar}} \qu{\psi_n} + \int dE \, C(E) e^{- \frac{iEt}{\hbar}} \qu{\psi_E}
$$
Общее решение раскладывается по полной системе собственных векторов гамильтониана единственным образом.
$$
    |C_E|^2 = W_{\qu{\psi}} (E) \such \qu{\psi(0)} = \qu{\psi_0} \to C_E
$$
Обычно, решая уравнение Шрёдингера, ограничиваются стационарным случаем. Общее решение ищут как линейную комбинацию частных решений.

\subsection{Квантовый аналог уравнения непрерывности}
\Rem Есть аналогия между квантовой механикой и классической теорией несжимаемой жидкости, которые, впрочем, не распространяются очень далеко.

\begin{tabular}{r|c|l}
  Уравнение Шрёдингера:  & $
    i \hbar \ud{}{t} \Psi (\vec r, t) = - \dfrac{\hbar^2}{2m} \vec \nabla^2 \Psi(\vec r, t) + U(\vec r) \Psi(\vec r, t)
$ & $\cdot \Psi^\ast$ \\[10pt]
  Координатное представление: &
  $
    -i \hbar \ud{}{t} \Psi^\ast (\vec r, t) = - \dfrac{\hbar^2}{2m} \vec \nabla^2 \Psi^\ast(\vec r, t) + U(\vec r) \Psi^\ast(\vec r, t)
$
   & $\cdot \Psi$ \\[10pt]
  Представление Шрёдингера: &
  $i\hbar \ud{}{t} (\Psi^\ast \Psi) = \dfrac{\hbar^2}{2m} \left(
        (\vec \nabla^2 \Psi^\ast)\Psi - \Psi^\ast(\vec \nabla^2 \Psi)
   \right)$
   &  \\
\end{tabular}\index{Уравнение Шредингера}\index{Представление! координатное}

Отсюда получается известный результат (уравнение непрерывности):
$$
        \underline{\ud{\rho}{t} + \div \vec j = 0},
$$
где
$$
    \rho = \Psi^\ast \Psi, \qquad \underline{\vec j = \dfrac{i \hbar}{2m} \left(
        \Psi(\vec \nabla \Psi^\ast) - \Psi^\ast(\vec \nabla \Psi)
    \right)}\index{Уравнение непрерывности}
$$
Если $\ud{\rho}{t} = 0$, то $\div \vec j = 0$

\section{Представление Гейзенберга}
Представление Гейзенберга~--- это другой способ введения времени в квантовой механике.\index{Представление! Гайзенберга}

\Rem Представления Шрёдингера и Гейзенберга образуют как бы <<крайние>> варианты, а вообще существует очень много <<промежуточных>> представлений.

В предыдущем параграфе мы пришли к заключению: при $\ud{\Hat H}{t} = 0$, имеем
$$
    \begin{cases}
    \qu{\psi(t)} = \hat U (t, t_0) \qu{\psi(t_0)},\\
    \Hat U(t, t_0)  = \exp \left( -\dfrac{i}{\hbar}\Hat H (t-t_0) \right)
    \end{cases}
$$
Положим $t_0 = 0$, и будем для описания эволюции использовать только начальный вектор состояния. Оказывается, что в этом случае возможно получить полное описание эволюции системы:
$$
    \qu{\psi(t_0)}=\qu{\psi(0)} = \qu{\psi_0}
$$
Совершим над векторами состояния и над операторами в представлении Шредингера унитарное преобразование при помощи оператора, обратного к оператору $\hat U$, чтобы перейти к векторам в момент времени $t_0$.

Пусть вектор в представлении Гейзенберга имеет вид $\qu{\psi_H}$, а в представлении Шрёдингера $\qu{\psi_S}$
$$
    \qu{\psi_H} = \qu{\psi_S(t_0)} = \qu{\psi_S(0)} = \hat U^{-1} (t, t_0) \qu{\psi_S (t)}
    = e^{i/\hbar \cdot \Hat H_s (t-t_0)} \qu{\psi_S(t)}
$$
$$
    \hat F_H = \hat U^{-1} (t, t_0) \hat F_S \hat U(t,t_0) = e^{i/\hbar \cdot \hat H_S (t-t_0)}
    \hat F_S e^{-i/\hbar \cdot \hat H_S (t-t_0)}
$$
$$
    \underline{\Hat H_H = \Hat H_S = \hat H}
$$
\subsection{Векторы состояний}
Оператор $e^{i/\hbar \cdot \Hat H_S (t-t_0)}$ как бы <<поворачивает назад>> вектор состояния в представлении Шрёдингера. При $t = t_0$ они совпадают.

Вектор $\qu{\phi_H}$ не зависит от $t$.

Вектор $\qu{\psi_S}$ зависит от $t$, зависимость определяется уравнением Шрёдингера.

\subsection{Операторы}
При $t = t_0$ операторы $\Hat F_H$, $\Hat F_S$ совпадают. В произвольный момент времени оператор будет зависеть от времени.

\Rem Окажется, что в нестационарном случае полученное ниже уравнение Гейзенберга тоже выполняется, но об этом будет сказано позже.

Положим $\ud{\hat H}{t} = 0$.
$$
    \Hat F_H = e^{i/\hbar \cdot \Hat H t} \, \Hat F_S \, e^{-i/\hbar \cdot \Hat H t}
$$
Каждый из трёх сомножителей может зависеть от времени, надо это учитывать при дифференцировании. Дифференцировать необходимо методом внимательного вглядывания в выражение для $\hat F_H$.
$$
    \dfrac{d \Hat F_H}{dt} = e^{\frac{i}{\hbar} \Hat H t} \ud{\Hat F_S}{t} e^{-\frac{i}{\hbar}\Hat H t}
    + \dfrac{i}{\hbar} \left(
        \comm{\Hat H}{\Hat F_H}
    \right)
$$
Уравнение Гейзенберга в краткой форме:
$$
    \underline{\dfrac{d\Hat F_H}{dt} = \left(\ud{\Hat F_s}{t}\right)_H + \dfrac{i}{\hbar} [\Hat H, \Hat F_H]}
$$\index{Уравнение движения Гайзенберга}
Индекс $H$ обозначает представления Гейзенберга.

\Def Представление, в котором эволюция во времени переносится на операторы, а векторы состояния от времени не зависят, называется \emph{представлением Гейзенберга}.\index{Представление! Гайзенберга}

\Rem
\begin{itemize}
  \item Напомним, что это же уравнение можно получить без предположения о стационарности системы
  \item Переход от представления Шрёдингера к представлению Гейзенберга, производится при помощи унитарного преобразования. Это значит, что эти представления \emph{унитарно эквивалентны}.
\end{itemize}

\section{Эволюция квантовомеханических средних}
Как нас учит принцип соответствия, средние значения должны вести себя так же, как их классические аналоги. Проверим это предположение.

Усредним уравнение Гейзенберга по состоянию в представлении Гейзенберга $\qu{\psi_H} = \qu{\psi_S(0)}$:

Рассмотрим
$$
    \qtri{\phi_H}{ \left( \dfrac{d \Hat F_H}{d t} = \big( \ud{\Hat F_S}{t} \big)_H + \dfrac{i}{\hbar} [\Hat H, \, \Hat F_H] \right)}{\phi_H}
$$

При этом
$$
\dfrac{d}{dt} \qfive{\psi_H}{\Hat U^+ \Hat U}{\Hat F_H}{\Hat U^+ \Hat U}{\psi_H}
= \qtri
    {\psi_H}
    {\Hat U^{-1} (t,t_0) \ud{\Hat F_S}{t} \Hat U(t, t_0)}
    {\psi_H}
    +
\dfrac{i}{\hbar}
\qfive{\psi_H}{\Hat U^+ \Hat U} { [\Hat H, \, \Hat F_H] } {\Hat U^+ \Hat U} {\phi_H}
$$
%\begin{eqnarray*}
%    \dfrac{d}{dt} \underbrace{\qtri{\psi_S(0)}{\Hat F_H}{\psi_S(0)}}_{\qtri{\psi_S(0)}{\Hat 1\Hat F_H\Hat 1}{\psi_S(0)}} &=&
%        \qtri{\psi_S(0)}{\Hat U^{-1} \ud{\hat F_s}{t} \hat U}{\psi_S(0)}
%         + \dfrac{i}{\hbar} \underbrace{\qtri{\psi_S(0)}{ [\Hat H, \Hat F_H] }{\psi_S(0)}}_
%         {\qtri{\psi_S(0)}{ (\Hat U^+ \Hat U) [\Hat H, \Hat F_H] (\Hat U^+ \Hat U) }{\psi_S(0)}}
%\end{eqnarray*}
Окончательный результат:
$$
\underline{    \dfrac{d}{dt} \qtri{\psi_S(t)}{\Hat F_S}{\psi_S(t)}
    = \qtri{\psi_S(t)}{ \ud{\Hat F_S}{t}}{\psi_S(t)}
    + \dfrac{i}{\hbar} \qtri{\psi_S(t)}{[\Hat H, \Hat F_S]}{\psi_S(t)}
}$$

\section{Сопоставление классической и квантовой механики. Теоремы Эренфеста}

\subsection{Сопоставление классических и квантовых уравнений}
В основе этого сопоставления лежит принцип соответствия. Хотим, чтобы уравнения квантовой механики имели вид классических.

Классика:
$$
    \dfrac{d F_{\text{classical}}}{dt} = \ud{F_{\text{classical}}}{t} + \Pois{F_{\text{classical}}}{H}
$$
Выражения для скобки Пуассона
$$
    \pois{u}{v} = \alpha [u, v] \to \alpha = -\dfrac{i}{\hbar}
$$

\subsection{Интегралы движения}
Под интегралами движения имеются в виду величины, сохраняющиеся во времени. Ввести их можно двумя способами:\index{Интегралы движения}
\begin{enumerate}
  \item
  $$
    \dfrac{d \Hat F_H}{dt} = 0
  $$
  Для того, чтобы это выполнялось, достаточно двух условий:
  \begin{equation}
    \begin{cases}
        [\Hat H, \Hat F] = 0,\\
        \ud{F}{t} = 0
    \end{cases}
    \label{eq::constant_motion_cond}  
  \end{equation}
  \item Вместо уравнения Гейзенберга можно положить в основу уравнение эволюции во времени квантово-механической системы.
      $$
        \dfrac{d \avr{\hat F}}{dt} = 0, \quad \forall \qu{\psi} \in \H
      $$
      Если среднее значение наблюдаемой, вычисленное в любом состоянии, не зависит от времени, то сама величина $ \avr{F}$ является интегралом движения. Для того, чтобы это выполнялось, достаточно (часто необходимо), чтобы выполнялись те же условия~\eqref{eq::constant_motion_cond}.
\end{enumerate}
