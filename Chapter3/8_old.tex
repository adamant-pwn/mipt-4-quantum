\subsection{Теоремы Эренфеста}
В представлении Гейзенберга (классика) выполнено:
\begin{eqnarray}
        \dfrac{d\vec r}{dt} &=& \dfrac{i}{\hbar} [\hat H, \hat{\vec r}]
        = \dfrac{i}{\hbar} \left[ \dfrac{{\Hat{\vec p}}^2}{2m} + U(\Hat{\vec r} ), \Hat {\vec r}  \right]
        = \dfrac{\Hat{\vec p}}{2m} \\
        \dfrac{d\vec p}{dt} &=& \dfrac{i}{\hbar} [\hat H, \hat{\vec r}]
        = \dfrac{i}{\hbar} \left[ U(\Hat{\vec r}), \Hat {\Vec p}  \right]
        = - \mathrm{grad}\, U(\Hat{\vec{r}})
\end{eqnarray}
Классическим уравнениям сопоставляются квантовые уравнения с помощью усреднения $\qtri{\psi_H}{\ldots}{\phi_H}$
получаем теоремы Эренфеста:
$$
    \begin{cases}
        \dfrac{d \avr{\Hat{\vec r}}}{dt} = \dfrac{\avr{\Hat{\vec p}}}{m}\\
        \dfrac{d \avr{\Hat{\vec p}}}{dt} =  - \mathrm{grad}\, U(\Hat{\vec r})
    \end{cases}
$$
\index{Теорема! Эренфеста}
\Rem
\begin{itemize}
  \item (Мнемоническое правило) Уравнения эволюции во времени средних значений координат и импульсов формально тождественны уравнениям Гамильтона классической механики, если все физические величины в классических уравнениях (в обеих частях) заменить на средние значения квантовых операторов.
  \item Полученные уравнения \emph{не являются классическими}. Это были бы классические уравнения, если бы можно было сделать замену
      $$
        \avr{ \ud{}{x} U(\Hat x) } = \ud{}{x} U( \avr{\Hat x} ),
      $$
      и аналогичные замены для других величин. Это выполнено, например, если $\Hat H$~--- многочлен 2 степени по $\Hat p$, $\Hat x$. Например, гамильтониан свободной частицы, или гармонического осциллятора.
\end{itemize}
