\subsection{Теоремы Эренфеста}

Представления Шредингера и Гайзенберга эквивалентны с точки зрения наблюдаемых следствий, потому что векторы и операторы в обоих представлениях связаны унитарным преобразованием. Однако работать в представлении Шредингера часто удобнее, поскольку решение уравнения Шредингера в координатном представлении сводится к решению дифференциального уравнения. Уравнения Гайзенберга решать сложнее -- они операторные. Тем не менее, есть "плюсы"{} и в представлении Гайзенберга. В представлении Гайзенберга наиболее ярко прослеживается связь между квантовой и классической механикой (принцип соответствия). Дело в том, что уравнения, являющиеся аналогами классических, записываются в этом представлении не для средних значений, а для самих операторов физических величин.

Действительно, в представлении Гейзенберга выполнено:
\begin{eqnarray}
        \dfrac{d\Hat{\vec r}}{dt} &=& \dfrac{i}{\hbar} [\hat H, \hat{\vec r}]
        = \dfrac{i}{\hbar} \left[ \dfrac{{\Hat{\vec p}}^2}{2m} + U(\Hat{\vec r} ), \Hat {\vec r}  \right]
        = \dfrac{\Hat{\vec p}}{m} \\
        \dfrac{d\Hat{\vec p}}{dt} &=& \dfrac{i}{\hbar} [\hat H, \hat{\vec p}]
        = \dfrac{i}{\hbar} \left[ U(\Hat{\vec r}), \Hat {\Vec r}  \right]
        = - \mathrm{grad}\, U(\Hat{\vec{r}})
\end{eqnarray}
\textbf{Теоремы Эренфеста} -- это обобщение уравнений Ньютона в квантовой теории. В квантовой теории классические уравнения обобщаются путем замены классических физических величин на средние значения операторов.

Действительно, усредним при помощи процедуры $\qtri{\psi_H}{\ldots}{\psi_H}$ данные уравнения по состоянию, взятому в представлении Гайзенберга, и получим теоремы Эренфеста:
$$
    \begin{cases}
        \dfrac{d \avr{\Hat{\vec r}}}{dt} = \dfrac{\avr{\Hat{\vec p}}}{m}\\
        \dfrac{d \avr{\Hat{\vec p}}}{dt} =  -\left\langle \mathrm{grad}\, U(\Hat{\vec r})\right\rangle
    \end{cases}
$$

\index{Теорема! Эренфеста}
\Th
Уравнения эволюции во времени средних значений координат и импульсов формально тождественны уравнениям Гамильтона классической механики, если все физические величины в классических уравнениях (в обеих частях) заменить на средние значения квантовых операторов.

\Rem
Полученные уравнения \emph{не являются классическими}. В этих уравнениях в правых частях стоят некоторые средние значения, для вычисления которых, вообще говоря, требуется знание волновой функции. Это были бы классические уравнения, если бы можно было сделать замену
      $$
        \avr{ \ud{}{x} U(\Hat x) } = \ud{}{x} U( \avr{\Hat x} ),
      $$
      и аналогичные замены для других величин. Это выполнено, например, если $\Hat H$~--- многочлен 2 степени по $\Hat p$, $\Hat x$. Например, гамильтониан свободной частицы, или гармонического осциллятора, или заряженной частицы в однородном магнитном (электрическом) поле.

