\chapter{Введение}
\Rem Экзамен будет проходить следующим образом: можно посмотреть те билеты, которые лежат на сайте, они не менялись в течение двух лет. Три из вопросов являются обязательными. (По билету из каждого семестра плюс задача).

Четвёртый вопрос~--- для получения дополнительных зачётных единиц (<<повышенный уровень>>). Эти темы будут помечены звёздочками.

\section{Предпосылки возникновения квантовой механики}
1900 г. -- первые открытия Макса Планка в квантовой механике\\
$$
a = \frac{\hbar^{2}}{me^{2}}=0,529\cdot10^{-8} \text {см -- характерный размер атома}
$$
\index{Характерный размер атома}
$
\lambda = \frac{\hbar}{mc} = 3,9\cdot10^{-11}$см~--- комптоновская длина волны (граница применимости одночастичной квантовой механики)

\index{Комптоновская длина волны}
Принципы классической физики:
\begin{itemize}
\item[a)] непрерывное изменение основных физических величин
\item[б)] причинность
\end{itemize}
Не поддающиеся объяснению с точки зрения классической физики эффекты:\\
\begin{enumerate}
  \item Излучение абсолютно черного тела
$$
\rho = \int\limits_{0}^{\infty}\rho_{\omega} d\omega \rightarrow \infty
$$
  \item Фотоэффект
  \item Линейчатость спектров
$$
\omega  = R\left(\frac{1}{n^{2}}-\frac{1}{m^{2}}\right)
$$
  \item Устойчивость атома
$$
w = \frac{2e^{2}}{3c^{2}}|\dot{\vec{v}}|^{2}
$$
\end{enumerate}

\section{Основные этапы построения квантовой механики}
\begin{itemize}
  \item Макс Планк, 1900 г. -- идея квантования
$$
E=n\varepsilon = n\hbar\omega
$$
$$
\hbar = 1,05\cdot10^{-27} \text{эрг$\cdot$с -- постоянная Планка (квант действия)}
$$
\index{Постоянная Планка}
$$
\hbar = 6,58\cdot10^{-22}\text{МэВ$\cdot$с}=1,05\cdot10^{-34}\text{Дж$\cdot$с}
$$
  \item Альберт Эйнштейн, 1905 г. -- дуализм света
$$
\varepsilon_{\text{ф}}=\hbar\omega, \qquad  \vec{p_{\text{ф}}}=\hbar\vec{k}
$$
Это математические выражения корпускулярно-волнового дуализма (слева~-- корпускулярные величины, справа~-- волновые)
\index{Корпускулярно-волновой дуализм}\\
  \item Нильс Бор, 1913 г. -- <<старая>> квантовая теория\\
Постулаты Бора:
\begin{itemize}
\item[1)] постулат стационарного состояния
\item[2)] постулат частот
\end{itemize}
$$
\omega = \frac{E_1-E_2}{\hbar}
$$ 
 \item Луи де Бройль, 1924 г. -- волновые свойства частиц\\
$$
E=\hbar\omega, \quad \vec{p}=\hbar\vec{k}, \quad \lambda_{\text{дБ}}=\frac{\hbar}{p}
$$
  \item Эрвин Шрёдингер, Вернер Гайзенберг, 1926 г. -- волновая, матричная механика
\end{itemize}

\section{Основные свойства, характерные для любого волнового движения}

\subsubsection{Волновая функция}
\index{Волновая функция}
Например, волновая функция монохроматической волны, распространяющейся в положительном направлении оси $X$ записывается так:
$$
\psi (x, t) = Ae^{-i(\omega t-kx)}
$$

\subsubsection{Волновой пакет}\index{Волновой пакет}
Волновую функцию волнового пакета можно записать в виде:
$$
\Psi (x, t ) = \int_{k_0-\Delta k}^{k_0+\Delta k} A(k)e^{-i(\omega (k)t-kx)}dk
$$
\begin{center}
\begin{tikzpicture}
\begin{axis}[
  ylabel={$\Psi(x)$},
  ylabel style={at=(current axis.right of origin),rotate=-90, anchor=north},
  ylabel shift = 5 cm,  
  samples=100,
  axis y line  = center,
  axis x line  = bottom,
  axis lines*=center,
  xtick={0}, 
  xticklabels={$0$},
  ytick=\empty,
  axis on top,
  xlabel={$x$},
  x label style={at={(axis description cs:1,0.1)}},
  xmax=8,
  xmin=-8,
  ymax = 0.22,
  ymin=0
  ]
  \addplot [draw=black, domain=-8:8] {gauss(0,2)} \closedcycle;
  \draw[->] (axis cs:0, 0.1) -- (axis cs:3, 0.1) node [above]
     {$\vec{v}_{\text{гр}}$};
  \draw[<->] (axis cs:-3.1, 0.06) -- (axis cs:3.1, 0.06) node [above,midway,text width=2cm] {$\Delta x$};
\end{axis}
\end{tikzpicture}
\end{center}

%\begin{figure}[h]
%\center{\includegraphics[width=0.6\textwidth]{pic/1/lecture.png}}
%\end{figure}

$\vec{v}_{\text{гр}}$ -- скорость центра масс волнового пакета\index{Групповая скорость}\\
$$
\Delta k\Delta x \ge 1
$$
$$
\Delta \omega \Delta t \ge 1
$$
Соотношения неопределенностей Гайзенберга:\index{Соотношение неопределенностей}
$$
\Delta p\Delta x \ge \hbar,\qquad  \Delta E \Delta t \ge \hbar
$$
Траекторий в квантовой механике нет в силу соотношения неопределенностей.\\
Макс Борн, 1926 г. -- статистическое (вероятностное) значение (интерпретация) волновой функции. \index{Волновая функция! статистическое значение}
$$
\Psi^{*} (\vec{r}, t)\Psi(\vec{r}, t)d^3x = w(t, \vec{r})d^3x
$$
$$
|\Psi (\vec{r}, t)|^2 = w(t, \vec{r})
$$
$$
\int \Psi^{*} (\vec{r}, t)\Psi (\vec{r}, t)d^3x = 1
$$

\textbf{Краткие итоги и систематизация результатов:}\\
\nopagebreak
\begin{tabular}{|c|c|}
\hline
Классическая теория & Квантовая теория\\
\hline
\multicolumn{2}{|c|}{Описание состояния}\\
\hline
$p_i(t), q_i(t)$ & $\Delta p\Delta x \ge \hbar, \Delta E\Delta t \ge \hbar$\\
$\dot{p_i} = -\frac{\partial{H}}{\partial{q_i}}$ & 1) ограниченность классических методов\\
$\dot{q_i} = \frac{\partial{H}}{\partial{p_i}}$ & 2) $\Psi (\vec{r}, t)$\\
\hline
\multicolumn{2}{|c|}{Причинность}\\
 \hline
$q_i(0), p_i(0)$ & $i\hbar \frac{\partial}{\partial{t}}\Psi (\vec{r}, t)=\left(-\frac{\hbar^2}{2m}\vec{\nabla}^2+V(\vec{r})\right)\Psi (\vec{r}, t)$\\
$$ &~--уравнение Шрединегера;\index{Уравнение Шредингера}\\
$$ & причинность носит не траекторный, а статистический характер\\
\hline
\multicolumn{2}{|c|}{Наблюдаемые величины}\\
\hline
$q(t), p(t)$ & постулат средних значений\index{Постулат средних значений}\\
$$ & $F \rightarrow \hat{F}$\\
$$ & $\avr{F}=\avr{\hat{F}}=\dfrac{\int\Psi^{*}\hat{F}\Psi d^3x}{\int\Psi^{*}\psi d^3x}$\\
\hline
\end{tabular}
