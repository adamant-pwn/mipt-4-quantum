\section{Линейный принцип суперпозиции состояний}
\subsection{Краткие замечания}
\index{Суперпозиция состояний}
Будем очень долго идти к тому, чтобы определить понятие <<состояния>>. Хотя бы потому, что состояние \emph{не обязательно} описывается волновой функцией. Пойдём другим путём.

При помощи координатной волновой функции \index{Волновая функция} можно описать движение. У квантовых объектов есть также внутренние свойства (поляризация фотона, спин электрона). Даже если квантовая частица имеет классический аналог, нам мешает описать её движение соотношение неопределённостей.

Очень важна вероятностная трактовка.

\subsection{Предварительное определение}
Дадим предварительное определение линейного принципа суперпозиции состояний на языке волновых функций.

\textbf{Формулировка}.
\begin{enumerate}
  \item Если квантовая система может находиться в состоянии \index{Состояние} $\psi_1$, $\psi_2$, то она может находиться и в состоянии
$$
    \Psi = c_1 \psi_1 + c_2 \psi_2, \, c_1, c_2 \in \C
$$
  \item Умножение на число $c \ne 0$ не изменяет состояние
$$
    \psi \to c\psi
$$
\end{enumerate}

\subsection{Два мысленных примера.}
\begin{itemize}
  \item \textbf{Опыт 1. Интерференция электронов, опыт Юнга.}

%  \Picw{pic/1/1.pdf}{0.4}
\begin{center}
\begin{tikzpicture}
\begin{axis}[
  axis lines=none,% 
%  domain=0.125:1000,
  xmin=-10, xmax=9,
  ymin=-12, ymax=12,
  samples=400
]
\pgftransformrotate{-90}
\addplot [draw=black, domain=-10:9] {10*gauss(1.5,1)} \closedcycle;
\addplot [draw=black, domain=-10:9] {10*gauss(-2.5,1)};
\addplot+[mark=none, black, domain=-7:6] {6.5 + cos(4.9 * (deg(x) - 1.5))};
\node[rotate=90] at (axis cs: 5,3) {$W_2(x)$};
\node[rotate=90] at (axis cs: -6,3) {$W_1(x)$};
\node[rotate=90] at (axis cs: 0,9.5) {$W(x)$};
\addplot[mark=none,black] coordinates {(1.5,-10) (1.5,4.5)};
\addplot[mark=none,black] coordinates {(-2.5,-10) (-2.5,4.5)};
\addplot[mark=none,black] coordinates {(-10,-9) (-3,-9)};
\addplot[mark=none,black] coordinates {(-2,-9) (1,-9)};
\addplot[mark=none,black] coordinates {(2,-9) (9,-9)};
\draw[latex-latex] (axis cs: -2.5,-10) -- (axis cs: 1.5,-10) node [left, midway, align=center,rotate=90] {$d$};
\draw[latex-latex] (axis cs: 7,-9) -- (axis cs: 7,0) node [above, midway, align=center,rotate=90] {$L$};
\end{axis}
\end{tikzpicture}
\end{center}

  Реализация \index{Опыт Юнга}
  \begin{enumerate}
    \item Открыта щель 1, закрыта щель 2. $W_1(x)$
    \item $W_2(x)$
    \item Обе открыты, $W(x) \neq W_1(x) + W_2(x)$ (!)
  \end{enumerate}
  Если длина волны Де-Бройля $\lambda_{\text{дБ}} \leqslant d < L$, то наблюдается весьма сложная интерференционная система минимумов, максимумов.

  Возможно выпускать частицы строго по одной. В результате получим то же самое: по прошествии некоторого времени получим интерференционную картину (которая собирается из точек или пятен).


  Значит, вероятностное поведение связано не с пучком электронов, а с каждый электроном в отдельности.

  ***

  Интерпретируем этот результат на языке волновых функций. Складываются не вероятности, а волновые функции:
  $$
    \Psi(\vec r, t) = \Big[
        \underbrace{\psi_1(\vec r)}_{hole \, 1} + \underbrace{\psi_2(\vec r)}_{hole \, 2}
    \Big] \exp{\left(-\dfrac{i E t}{\hbar}\right)}
  $$
  $$
    \big| \Psi(\vec r, t)\big|^2 = \Big| \psi_1(\vec r) + \psi_2 (\vec r) \Big|^2
  $$
  \Quest{С чем интерферирует электрон, если мы сопоставили волновую функцию каждому электрону в отдельности?}
  \Ans Ответ немного странный: он интерферирует сам с собой, и он как бы проходит через две щели сразу. С другой стороны, можно поставить эксперимент, с помощью которого можно узнать, через какую именно щель он пройдёт. Но тут уже вступает в силу концепция измерения: из состояния суперпозиции мы перевели электрон в состояние $\psi_1(\vec r)$. Измерение разрушает интерференционную картину и состояние суперпозиции.
  \item \textbf{Опыт 2. Поляризованные фотоны.}

  Рассмотрим классический пучок поляризованного света. Пусть поляризация линейная ($ \vec{\mathbf{E}} \perp \vec{\mathbf{H}}$,
  $\vec{\mathbf{E}}$ колеблется строго в определённой плоскости).

  На его пути стоит поляроид (поляризационный фильтр). Если направления поляризации совпадают, свет проходит, в противном случае~--- поглощается. Пусть свет поляризован под углом $\theta$ к оси поляроида.
  $$
    \vec E = \vec E_{\|} + \vec E_{\perp}
  $$
  \begin{itemize}
    \item До поляроида: $\vec E \exp(-i \omega t)$
    \item После поляроида: $\vec E_{\|} \exp(-i\omega t)$
    \item $\dfrac{I_{pass}}{I_{fall}} = \dfrac{|\vec E_{\|}|^2}{|\vec E|^2} = \cos^2 \theta$
  \end{itemize}
%  \Picw{pic/1/2.pdf}{0.3}
\begin{center}
\begin{tikzpicture}
\draw[-latex,thick] (0,0) -- (0, 5) node [above] {Ось поляроида};
\draw[-latex,thick] (0,0) -- (0, 2.5) node [left] {$\vec E_{\|}$};
\draw[-latex,thick] (0,0) -- (5, 0);
\draw[-latex,thick] (0,0) -- (3, 0) node [below] {$\vec E_{\perp}$};
\draw[-latex,thick] (0,0) -- (3, 2.5) node [above] {$\vec E$};
\draw[dashed] (0,2.5) -- (3,2.5);
\draw[dashed] (3,0) -- (3,2.5);
\end{tikzpicture}
\end{center}

  При этом интенсивность пропорциональна количеству падающих фотонов:
  $$
    I_{fall} \sim N_{fall}, \quad I_{pass} \sim N_{pass}.
  $$
  Отсюда
  $$
    \dfrac{I_{pass}}{I_{fall}} = \dfrac{N_{pass}}{N_{fall}} = \cos^2 \theta
  $$
  Будем считать, что фотон, поляризованный под углом $\theta$, находится в особом состоянии:
  $$
    \bra \text{Поляризованный под углом $\theta$}\ket = c_1 \bra \text{вдоль} \ket + c_2 \bra \text{перп.} \ket
  $$
  \Quest{Что такое $c_1, c_2$?}

  \Ans На квантовом языке это значит, что поляроид перевёл фотоны из состояния суперпозиции в состояние $c_1 \bra \text{вдоль} \ket$.

  При этом говорят, что $\dfrac{|c_1|^2}{|c_1|^2 + |c_2|^2}$~--- \emph{вероятность прохождения фотона}.

  По сути дела, поляроид можно трактовать как некий прибор, измеряющий состояние поляризации.
\end{itemize}

\subsection{Окончательная формулировка принципа}
\begin{enumerate}
  \item Если квантовая система может находиться в состояниях $\bra 1 \ket$, $\bra 2 \ket$, то она может находиться в состоянии $\bra 3 \ket = c_1 \bra 1 \ket + c_2 \bra 2 \ket$, при этом
      $
        \dfrac{|c_1|^2}{|c_1|^2 + |c_2|^2}
      $~--- вероятность найти систему в состоянии 1.
  \item Умножение на комплексное число, отличное от нуля, не меняет состояние.
  $$
    \bra 1 \ket \to c \bra 1 \ket ,\, (c \neq 0)
  $$
\end{enumerate}
Состояние обозначается при помощи угловых скобок $\bra \ldots \ket$, внутри которых ставятся числа, которые эти состояния характеризуют (обозначение Дирака).
\index{Скобки Дирака}