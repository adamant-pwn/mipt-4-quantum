\chapter{Симметрия в квантовой механике и законы сохранения}
\Rem Конкретно речь пойдёт о разработке симметрий, связанных с теорией групп и представлений.
Поэтому на лекции будут рассказываться части общей теории.
%заключительная лекция после трёх лекций 6 главы.
%семинары. на следующей неделе контрольная, сдача задания.
Важность симметрий для квантовой механики была установлена в работах классиков:
\begin{itemize}
  \item Вейль, 1928
  \item Вигнер, 1931
  \item Ван-дер-Варден, 1931
\end{itemize}
\section[Теория групп в квантовой механике]{Общее введение в теоретико-групповые методы в квантовой механике}
\subsection{Определение группы}
\Def \emph{Группой} $G$ называется множество элементов любой природы, конечное или бесконечное, такое что на этом множестве задана \emph{групповая операция} $\ast$, такая что
\begin{itemize}
  \item Если $a, b \in G$, то $a\ast b \in G$
  \item $a \ast (b \ast c) = (a \ast b) \ast c$, если   $a, b, c \in G$
  \item $\exists! \,  e \in G \such \, \any g \in G \quad e\ast g = g\ast e = g$
  \item $\any g \in G \,\, \exists g^{-1} \such \, g \ast g^{-1} = g^{-1} \ast g = e$
\end{itemize}
\subsection{Преобразование инвариантности в квантовой механике}
\subsubsection{Преобразование системы отсчёта}
%Для того, чтобы описать физическое явление, необходимо задать систему отсчёта.
Переходим из системы $S$ в систему $S'$.

Тогда в новой системе координат координаты преобразовываются следующим образом:
$$
    X = \{t, \vec r\} \to X ' = \Hat G X
$$
Оператор $G$ действует в пространстве координат, и принадлежит \emph{группе линейных преобразований координат}.

%Группы таких операторов для нас сейчас будут чрезвычайно важны.
\begin{itemize}
  \item \textbf{Пространственные трансляции}. Это преобразования вида
  $$
  \begin{cases}
    \vec r \to \vec {r'} = \vec r -  \vec a, &\\
    t \to t' = t &
  \end{cases}
  $$
  \item \textbf{Временн\underline{ы}е трансляции}:
  $$
    \begin{cases}
        t \to t' = t - \tau, & \\
        \vec r \to \vec {r'} = \vec r &
    \end{cases}
  $$
  \item \textbf{Пространственный поворот}
  $$
    \begin{cases}
        t \to t' = \Hat R \vec r, &\\
        t \to t' = t&
    \end{cases}  
  $$
  Матрица $\Hat R$ есть ортогональная матрица поворота
  \item Отражение координатной оси $\vec{r'} = -\vec r$
  \item Обращение времени $t' = -t$
\end{itemize}
\Rem Переход к новой системе отсчёта может изменить \emph{математическое описание физического явления}.
Другими словами, уравнения изменяются. Но при этом ни при каких обстоятельствах этот переход не должен изменять физического содержания теории. Под этим понимаются числа, которые можно из этой теории получить и проверить экспериментом. 
\pagebreak[3]
\Def \emph{Преобразования инвариантности}~--- это такие преобразования, при которых \emph{уравнения и граничные условия} остаются неизменными.

\subsubsection{Преобразования векторов состояний}
Если подействовать на пространство преобразованием $S \to S'$, то
$$
    \qu \psi \to \qu{\psi'} = \Hat U (S, S') \qu \psi
$$
Операторы преобразования векторов состояния действуют в пространстве векторов состояний.
\pagebreak[3]
\Rem
\begin{enumerate}
  \item Из-за равноправности систем отсчёта, оператор определяется не каждой из систем отсчёта в отдельности, а только элементом группы $G$: $GS = S'$:
      $$
        \Hat U(S, S') = \Hat U (\Hat G)
      $$
  \item Исходя из вероятностной интерпретации квантовой механики, основные величины, которые можно вычислить, и которые составляют физическое содержание теории, суть вероятности. Значит, они не должны изменяться при преобразованиях координат.
      
      Значит, не должны меняться модули скалярных произведений:
      $$
        \Big| \qs{\phi'}{\psi'} \Big| = \Big| \qs{\phi}{\psi} \Big|
      $$
\end{enumerate}
Когда мы говорили, что состояния описываются с помощью векторов состояний, это было не совсем правда. На самом деле, состояния описываются с помощью \emph{нормированных гильбертовых лучей}.

\Def Нормированным лучом называется множество
$$
      \qu\Psi = \big\{
          e^{i\alpha} \qu \psi, \quad \qu \psi \in \H, \quad \Im \alpha = 0, \quad \big\|\qu{\psi} \big\| = 1
      \big\}
$$
\Quest{Что практически означает, что $\qu \psi \in \qu \Psi$, $\qu \phi \in \qu \Phi$}

\Th (Вигнер).
Можно всегда выбрать систему базисных векторов, такую что скалярное произведение либо сохраняется, либо сопрягается:
\begin{itemize}
  \item Унитарные преобразования
  $$
      \qs{\psi'}{\phi'} = \qs{\psi}{\phi}
  $$
  \item Антиунитарные преобразования
  $$
      \qs{\psi'}{\phi'} = \qs{\psi}{\phi}^\ast
  $$
\end{itemize}
\textbf{В рамках нашего курса можно рассмотреть только унитарные преобразования, ибо антиунитарные преобразования затрагивают обращение времени.}

\Rem Оказывается, что во всех физически интересных случаях для операторов $\Hat U(G)$ выполняется
$$
    \Hat U(\Hat G_1) \Hat U(\Hat G_2) = \Hat U(\Hat G_1 \Hat G_2), \quad \Hat U(e) = \Hat 1
$$
Таким образом, для оператора $\Hat U$ групповая операция сохраняется. Говорят, что операторы $\Hat U$ осуществляют некоторое \emph{представление} группы преобразований координат. Полученное определение можно записать короче:

\Def Представление группы есть \emph{гомоморфизм} группы $G$ на группу линейных преобразований некоторого линейного пространства.

\subsection{Трансформационные свойства гамильтониана}
\begingroup
\textbf{Цель}: понять, при каких преобразованих Уравнение Шрёдингера сохраняет свой вид.
$$
    i \hbar \ud{}{t} \qu{\psi(t)} = \Hat H \qu{\psi(t)}
$$
\newcommand \shleft[1] {    i \hbar \ud{}{t} \qu{#1} }
\newcommand \shright[2] {     \Hat H \qu{#2} }

Применим к обеим частям уравнения оператор $\Hat U$:
\begin{eqnarray*}
    i \hbar \Hat U \ud{}{t} \qu{\psi(t)} &=& i \hbar \ud{}{t} \left( \Hat U \qu{\psi} \right)
    - i \hbar \left( \ud{\Hat U}{t} \right) \underset{\underset{\Hat U^{-1}\Hat U}{\uparrow}}{}\bra\psi(t)\ket\\
    &=& \Hat U \Hat H \Hat U^{-1} \underbrace{\Hat U \qu{\psi}}_{\qu{\psi'}}
\end{eqnarray*}

Уравнение должно сохранять свой вид:
$$
    i \hbar \ud{}{t} \qu{\psi'} = \Hat H \qu{\psi'}
$$
Таким образом:
$$
    i \hbar \ud{}{t} \qu{\psi'} = i \hbar
    \left(
        \ud{\Hat U}{t}
    \right)
    \Hat U^{-1} \qu{\psi'} + \Hat U \Hat H \Hat U^{-1} \qu{\psi'}_{}
$$
$$
    i \hbar \left(
        \ud{\Hat U}{t}    
    \right) \Hat U^{-1} + \Hat U \Hat H \Hat U^{-1} = \Hat H,
$$
откуда получается условие, которому должен удовлетворять оператор преобразования $\hat U$:
$$
    \ud{\Hat U}{t} + \dfrac{i}{\hbar} [\Hat H, \Hat U] = 0
$$

Если оператор $\Hat U$ не зависит от времени, то требование сводится к требованию коммутации с гамильтонианом:
$$
    \big[ \Hat U, \Hat H \big] = 0
$$
\endgroup

\textbf{Резюме:}
\begin{enumerate}
  \item Если \emph{унитарный оператор} коммутирует с гамильтонианом, то он описывает некоторое преобразование векторов состояний, которое оставляет систему неизменной.
  \item Если \emph{эрмитов} оператор коммутирует с гамильтонианом, то он описывает интеграл движения (физическую величину, которая остаётся неизменной).
\end{enumerate}

\section{Группы Ли}
%\Rem В этом параграфе мы будем избегать давать точных определений.
Главное свойство таких групп: <<непрерывность>>. Это группы, на множестве элементов которых можно ввести понятия <<близости>> и сходимости. Другими словами, на множестве элементов группы можно ввести \emph{топологию}, на основе которой можно ввести понятие сходимости.

Выше были определены \emph{топологические группы}. Они обладают, например, свойствами:
\begin{enumerate}
  \item $a \ast b$ непрерывно зависит от $a, b$
  \item $a^{-1}$ непрерывно зависит от $a$
\end{enumerate}

\subsubsection{Параметризация элементов группы Ли}
Будем предполагать, что любой элемент группы $\Hat G \in G$ может быть охарактеризован при помощи $n$ существенных вещественных параметров, то есть
$$
    \Hat G = G(a_1, \ldots, a_n)
$$
\emph{Существенность} означает то, что меньшим числом параметров охарактеризовать систему нельзя.

При этом групповая операция принимает вид:
$$
    G(a_1, \ldots, a_n) G(b_1, \ldots, b_n) = G(c_1, \ldots, c_n),
$$
где параметры $c_i$ есть функции
$$
    c_i = c_i(a_1, \ldots, a_n; b_1, \ldots, b_n)
$$
Выше были описаны \emph{параметрические группы}.

Чтобы превратить их в группы Ли, надо наложить некоторое требование на эти функции.

\Assume Функции $c_i(\cdots)$ должны быть \emph{бесконечно дифференцируемыми функциями своих аргументов}.

\Def Если группа удовлетворяет описанным выше свойствам, она называется \emph{группой Ли}.

%\textbf{Каноническая параметризация}
\subsection{Канонические параметризации}
\subsubsection{Однопараметрическая группа}
\begin{enumerate}
  \item $G(0, \ldots, 0) = e$
  \item $G(0, \ldots, \underset{\underset{k}{\uparrow}}{a_k}, \ldots, 0) = G(a_k)$
  \item $G(a_k)G(b_k) = G(a_k + b_k)$
\end{enumerate}
%Некоторым параметризациям (каноническим) придумали названия

\subsection{Инфинитезим\underline{а}льное преобразование}
\def \infzh {{инфинитезимальных}}
\def \infze {{инфинитезимальные }}
\def \infzj {{инфинитезимальный}}
Окрестность единичного элемента определяет всё. Исследовать её принято при помощи бесконечно малых (\infzh) преобразований.

\subsubsection{Генераторы группы Ли}
\Def \emph{Генераторы группы Ли} $f_k$~--- это такие операторы, что
\begin{eqnarray*}
    G(a_1, \ldots, a_n) &\backsimeq& G(0, \ldots, 0) + \left.\sum_{k=1}^{n} a_k \ud{G(a_1, \ldots, a_n)}{a_k}
    \right|_{a_1 = a_2 = \ldots = a_n = 0}\\
    &=& e + i \sum_{k=1}^{n} a_k f_k,
\end{eqnarray*}
где
$$
    f_k = \left.-i \ud{G(a_1, \ldots, a_n)}{a_k}\right|_{a_1 = \ldots = a_k = 0}
$$
Чтобы обосновать корректность данного определения, воспользуемся теоремой.

\Th (Адо)
Для любой группы Ли имеется изоморфная ей группа линейных преобразований некоторого линейного пространства.

\subsubsection{Представление группы Ли}
На представления накладывается дополнительное требование \emph{непрерывности}.

Выпишем \infze преобразование для элементов группы Ли:
$$
    \Hat U(a_1, \ldots, a_n) \backsimeq \Hat 1 + i \sum_{k=1}^{n} a_k \hat F_k,
$$
где
$$
    \Hat F_k = -i \ud{\Hat U(a_1, \ldots, a_n)}{a_k}
$$
$\Hat F_k$~--- генераторы представлений группы.

\subsubsection{Основные свойства генераторов}
\Quest{Почему у упомянутых генераторов самой группы и генераторов представлений должны быть одинаковые свойства?}
\begin{enumerate}
  \item Представление в целом гомоморфизм.
  \item Есть взаимно однозначное соответствие в некоторой окрестности (то есть имеется локальный изоморфизм окрестности единицы).
\end{enumerate}
Будем рассматривать \emph{только те} свойства, которые едины и для тех, и для других генераторов.

\textbf{Свойства.}
\begin{enumerate}
  \item Генераторы образуют $n$-мерное вещественное пространство с базисом
  $$
    \Hat F_1, \ldots, \Hat F_n
  $$
  \item
  $$
    [\Hat F_i, \Hat F_j] = C_{ijk} \Hat F_k
  $$
  Они называются \emph{структурные константы} группы. По этим числам можно определить структуру группы вблизи единичного элемента.
  \item Пространство, образованное генераторами группы, называется \emph{Алгеброй Ли}. Групповая операция определяется через коммутатор.    
\end{enumerate}
\subsection{Восстановление конечных элементов группы Ли по её генераторам}

\Th (Софус Ли, Норвегия, 1842-1899). Знания явного вида генераторов и алгебры Ли достаточно для однозначного восстановления конечного элемента группы. 

\Proof Пусть дана каноническая параметризация, причём $\hat U(a_1) \hat U(a_2) = \hat U(a_1 a_2)$

Проиллюстрируем утверждение для однопараметрической подгруппы, оставив без доказательства общий случай.

Схематичная последовательность действий:
$$
    \ud{}{a_1}, \quad a_1 \to 0, \quad a_2 \to a
$$
В результате применения получаем:
$$
    \left.\ud{\Hat U(a)}{a}\right|_{a_1 = 0}, \quad \Hat U(a) = \ud{\Hat U(a)}{a}
$$
$$
    -i \ud{\Hat U (a)}{a}
$$
Получили дифференциальное уравнение, которое решается с данным начальным условием
$$
    \begin{cases}
        -i \Hat F_k \Hat U(a) = \ud{\hat U}{a}\\
        \Hat U(0) = \Hat 1
    \end{cases}
$$
$$
    \Hat U(a) = e^{i a \hat F_a}, \quad \Hat U(a_k) = e^{i a_k \Hat F_k}
$$
Если в группе выделяются однопараметрические подгруппы, то так будет \emph{для любой однопараметрической подгруппы}.

Без доказательства:
$$
    \Hat U(a_1, \ldots, a_n) = e^{i \sum_{k=1}^{n} a_k \Hat F_k}
$$

\textbf{Вывод.}
\begin{enumerate}
  \item Мы хотели, чтобы преобразования векторов состояний происходили при помощи унитарных операторов. Для того чтобы $\Hat U(a_k)$ был унитарным, достаточно потребовать, чтобы $\hat F_k$ был эрмитовым, то есть
$$
    \Hat F_k^+ = \Hat F_k
$$
  \item При отсутствии явной зависимости от времени, из $[\hat U(a_k), \Hat H] = 0$ следует инвариантность. Для этого достаточно потребовать, чтобы гамильтониан коммутировал с генератором:
      $$
            [\hat H, \Hat F_k] = 0
      $$
      Это есть квантовый аналог теоремы Нётер. Инвариантности теории отвечает величина $\Hat F_k$, которая описывает физическую величину, которая описывает интеграл движения.
\end{enumerate}
