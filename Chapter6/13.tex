Итак, мы выяснили, что
$$
    \Hat{\vec L} = -i \hbar [\vec r \times \vec \nla]
$$
Чтобы найти всевозможные представления какой-то группы Ли (в нашем случае группы трёхмерных вращений), нужно знать коммутационные соотношения для компонент момента:
$$
    [\Hat J_i, \Hat J_j] = i \hbar e_{ijk} \Hat J_k
$$
Коммутационное соотношение, выписанное выше, совместимо и с целым, и с полуцелым квантованием.

\subsection{}
\begin{enumerate}
  \item $\vec r \to \vec r' = \Hat T(\vec \phi) \vec r$
  \item $\qu \psi \to \qu{\psi'} = \Hat U(\vec \phi) \qu{\psi}$
  \item $\Hat U(\dta \vec \phi) \approx \Hat 1 + \dfrac{i}{\hbar} \dta \phi (\vec n \Hat{\vec J})$
  \item В соответствии с теоремой Ли, оператор конечного поворота (генератор представления):
  $$
    \Hat U (\vec \phi) = \exp\left(\dfrac{i}{\hbar} \phi(\vec n \Hat {\vec J})\right)
  $$
\end{enumerate}
Коммутационные соотношения для $\Hat{\vec J}$ и для компонент углового момента одни и те же.


\textbf{Основные свойства генераторов, справедливые для любого представления}
\begin{enumerate}
  \item $[\Hat{J_i}, \Hat{J_j}] = i \hbar e_{ijk} \Hat J_k$
  \item $[\Hat{\vec J}^2, \Hat J_k] = 0$. Оператор квадрата углового момента коммутирует с любой компонентой углового момента.
      
      В теории групп Ли оператор квадрата углового момента коммутирует с любым оператором представления (потому что последний выражается через $J_{\bullet}$).

\Def \emph{Инвариантным оператором представления} называется оператор, коммутирующий со всеми операторами данного представления.

\def \Jsq {\Hat{\vec J}^2}
Но, с другой стороны, оператор $\Hat{ \vec{J}}^2$ тоже построен из операторов представления. Оператор $\Jsq$ называется \emph{оператором Казимира}.

    \item Как говорилось ранее, законы сохранения связаны с какими-то симметриями пространства и времени. 
    
    Если задано поле центрально-симметричного потенциала, то система должна быть инвариантна относительно поворота. Достаточно потребовать, чтобы
    $$
        [\Hat H, \Jsq] = [\Hat H, \Hat J_k] = 0
    $$
\end{enumerate}
\section{Неприводимые представления группы трёхмерных вращений}
\Def Если в пространстве представления группы есть нетривиальное инвариантное\wikiq{относительно чего?} подпространство (обычно на таком пространстве реализуется представление меньшей размерности), то это представление \emph{приводимое}. Если же такого подпространства нет, то оно является \emph{неприводимым}.

\textbf{Пример.} Для $SO(3)$ можно написать
$$
    \H = \bigoplus_{i=1}^{N} \H(i)
$$
Такое представление позволяет реализовать (блочно-)диагональный вид
$$
    \Hat U(\vec \phi) = \bigoplus_{i=1}^{N} \Hat U_i (\vec \phi)
$$
$$
    \Hat U(\vec \phi) = \begin{bmatrix}
                          \Hat U_1 (\vec \phi) & 0 & 0 \\
                          0 & \Hat U_2 (\vec \phi) & 0 \\
                          0 & 0 & \ddots \\
                        \end{bmatrix}
$$
\subsection{Лемма Шура, критерий неприводимости}
\Lem Если $\Hat F $ таков, что
$$
    [\Hat F, \Hat U (\Hat G)] = 0 \quad \forall G,
$$
то
$$
     \Hat F = f \Hat 1
$$
\Proof Пусть у оператора есть хотя бы один собственный вектор.
\begin{enumerate}
  \item 
  $$
    \Hat F \qu f = f \qu f
  $$
  Тогда натянем на этот собственный вектор пространство $H_f$:
  $$
    \H_f \subseteq \H
  $$
  Затем возьмём из этого подпространства вектор $\qu{\psi_f} \in \H_f$, действуем слева оператором $\Hat U$, $\Hat F$ (они коммутируют):
  $$
    \Hat U \Hat F \qu{\phi_f} = \Hat F \Hat U \qu{\psi_f} = f \Hat U \qu{ \psi_f}
  $$
  Итак, оператор $\Hat U$, действуя на операторы этого пространства, даёт векторы, принадлежащие этому же пространству.
  \item Пусть в пространстве представления нет инвариантых подпространств.
  $$
    \H_f = \H
  $$
  Отсюда следует, что
  $$
    \Hat F \qu{\psi} = f \qu{\psi},
  $$
  $$
    \Hat F = f \Hat 1
  $$
\end{enumerate}
Если представление является приводимым, и пространство этого представления разбито в прямую сумму, матрица имеет блочно-диагональный вид, то \textbf{всегда} можно найти оператор, коммутирующий со всеми операторами данного представления и не кратный единичной матрице.
$$
    \Hat F = \begin{bmatrix}
               f_1 \Hat 1 & 0 & 0 \\
               0 & f_2 \Hat 1 & 0 \\
               0  &  0 & \ddots \\
             \end{bmatrix}
$$ 
Если единственный такой оператор кратен единичному оператору, то представление неприводимо.

\textbf{Общая задача построения неприводимого представления группы SO(3)}

Задача сводится к построению неприводимого представления Алгебры Ли. Напомним, что эта задача уже решена. Нужно всего лишь согласовать терминологию. 

В квантовой механике решалась задача о нахождении общих собственных векторов из спектра
$$
\begin{cases}
    \Hat{\vec J}^2 \qu{j m} = \hbar^2 j (j+1) \qu{jm},\\
    \Hat{J_z} \qu{j m} = \hbar m \qu{jm}    
\end{cases}
$$
Отсюда получалась система векторов для $j = \mathrm{const}$, состоящая из $2j+1$ линейно-независимых векторов.

Операторы $\Hat J_k$ изображаются при помощи матриц $2 \times 2$.

Заключаем, что представление алгебры Ли группы  SO(3), реализованное в пространстве $E_{2j+1}$, является неприводимым представлением.

Если вспомнить решение задачи квантования углового момента, алгебра Ли группы SO(3) должна иметь неприводимое представление размерностей $(2j+1)$, где $j = 0, \dfrac12, 1, \ldots$

\Def \emph{Весом} неприводимого представления называется число $j$, при этом само неприводимое представление обозначается $D(j) = D^{(j)}$ 

\subsection{Простейшее неприводимое представление алгебры Ли группы SO(3)}
\begin{enumerate}
  \item Рассмотрим случай $j = 0$. Это называется \emph{скалярным представлением} (оно же тривиальное, оно же единичное).

Все генераторы нули. Есть единственный вектор $\qu{0,0}$. Единственный оператор единичный.
  \item \emph{Спинорное представление $D(\frac12)$}. $j = \frac12$, $m = \pm \frac12$
  
    Было введено обозначение
    $$
        \qu{\frac12, \pm \frac12} = \chi_{\pm \frac12}
    $$
    Матрицы Паули:
    $$
        \Hat F_k = \dfrac{1}{\hbar} \Hat S_k = \dfrac{1}{2} \Hat \sigma_k
    $$
    \Rem (Матрицы Паули и группа SU(2)) унитарные (унитарные) матрицы $2 \times 2$, $\det \Hat U = 1$.
    
    Матрицы Паули являются генераторами также и этой группы.
    
    \Quest{Каково соотношение между этой группой и SO(3)?}
    
    \Ans Имеется локальный изоморфизм.
    
    Рассмотрим глобальные соотношения между группами. Взаимно однозначного соответствия нет, имеется только гомоморфизм
    $$
        SU(2) \to SO(3)
    $$
    Каждому повороту отвечает \emph{ровно две матрицы} из SU(2).
    
    \Example Рассмотрим поворот на $\phi_3$ вокруг оси $Oz$
    $$
        e^{\frac i2 \phi_3 \Hat \sigma_z} = 
        \left.
        \begin{bmatrix}
          e^{\frac i2 \phi_3} & 0 \\
          0 & e^{-\frac i2 \phi_3} \\
        \end{bmatrix}
        \right|_{\phi_3 = 2 \pi} = -1
    $$
    Повороту на 0 отвечает 1, повороту на $2 \pi$ минус единица.
    
    Спин описывается именно таким представлением. Обратное соотношение является однозначным. Нужно превратить такое многозначное представление в однозначное, увеличив число элементов в исходной группе. Конструкция называется \emph{универсальная накрывающая группа} для группы трёхмерных вращений.
    
    Аналог: комплексная функция $\sqrt z$ является однозначной на многолистной Римановой поверхности.
    
    Когда говорят о спинорном представлении, имеют в виду универсальную накрывающую SU(2).

    \item $D(1)$. \emph{Векторное представление}. $j = 1$, $, = 0, \pm 1$.
    $$
        \Hat \jmath_x = \dfrac{1}{\sqrt 2} \begin{bmatrix}
                                             0 & 1 & 0 \\
                                             1 & 0 & 1 \\
                                             0 & 1 & 0 \\
                                           \end{bmatrix}, \quad
        \Hat \jmath_y = \dfrac{1}{\sqrt 2} \begin{bmatrix}
                                             0 & -i & 0 \\
                                             i & 0 & -i \\
                                             0 & i & 0 \\
                                           \end{bmatrix}, \quad
        \Hat \jmath_z = \dfrac{1}{\sqrt 2} \begin{bmatrix}
                                             1 & 0 & 0 \\
                                             0 & 0 & 0 \\
                                             0 & 0 & -1 \\
                                           \end{bmatrix}, \quad
    $$
    Матрицы подобны, существует $\Hat L$, такое, что
    $$
        \Hat f_k = \Hat L \hat \jmath_k \Hat L^{-1}
    $$
\end{enumerate}
\section{Спин и полный момент}
Будем считать, что у частицы есть внутренняя степень свободы~--- \emph{спин}.
$$
    \psi(\vec r) \in L_2 (\R^3)
$$
Если функция является однокомпонентной, то на выходе получается только орбитальный момент с целочисленным квантованием. Для полуцелого квантования этого недостаточно.

Рассмотрим многокомпонентную функцию, принадлежащую пространству $L_2$.
$$
    \psi_\mu (\vec r) \in L_2 (\R^3) \otimes \C^{(2s+1)},
$$
число $s$ описывает квадрат спина, $\mu = 2s + 1$.

При поворотах эта функция преобразуется \emph{двояким образом}:
$$
    \phi_{\mu}' (\vec r) = \Hat U(\vec \phi) \psi_\mu (\vec r) = \Hat Q_{\mu \nu} (\vec \phi) \psi_\nu (\Hat T^{-1} (\vec \phi) \vec r),
$$
где $\Hat Q_{\mu \nu} (\vec \phi)$~--- матрица неприводимого представления $SO(3)$
\subsection{Инфинитезимальное преобразование}
Формула Эйлера:
$$
    \Hat T^{-1} (\dta \vec \phi) \approx \vec r + \dta \phi[\vec n \times \vec r]
$$
$$
    \Hat Q_{\mu \nu} (\dta \vec \phi) \approx \dta_{\mu \nu} + i \dta \phi(\vec n \Hat{\vec F})_{\mu \nu}
$$
\begin{eqnarray*}
    \psi'_{\mu} (\vec r)& \approx& \left[\hat 1 + \dta \phi (\vec n [\vec r \times \vec \nla])\right] \cdot
    \left[ \dta_{\mu\nu} + i \dta \phi (\vec n \Hat{\vec F})_{\mu \nu} \right] \psi_{\nu} (\vec r)\\
    &\approx& \left(\hat 1 + i \dta \phi (\vec n 
    \left\{
        -i [\vec r \times \vec \nla] \dta _{\mu \nu} + \Hat 1 (\Hat{\vec F})_{\mu \nu}
    \right\})    
    \right) \psi_{\nu } (\vec r)
\end{eqnarray*}
$$
\underline{    \hbar \Hat F_k = \Hat L_k \dta_{\mu \nu} + 1 \cdot \hat S_k}
$$
$$
    \begin{cases}
        \Hat L_k = -i \hbar[\vec r \times \vec \nla],\\
        \Hat S_k = \hbar \Hat F_k
    \end{cases}
$$
Записываем то же самое в операторном виде:
$$
    \hbar \Hat{\vec{F_k}} = \Hat{\vec J} = \Hat{\vec L} \otimes \Hat 1 + \Hat 1 \otimes \Hat{\vec S}
$$
Итак, компоненты орбитального спина аддитивны.

Это представление называется \emph{прямым произведением представлений}, более подробно пойдёт речь об этом в следующем семестре. Закон сохранения приобретает вид закона сохранения полного момента $\Hat J_k$, $\Hat{\vec J}^2$

\section{Группа пространственной инверсии. Чётность.}
\subsection{Определение группы}
Группа состоит из двух элементов: инверсии и тождественного элемента.
$$
    \vec r \to \vec {r'} = - \vec r, \quad t' = t
$$
\subsection{Вектора состояния}
$$
    \Hat U (\Hat G) = \Hat I, \quad \qu \psi \to \qu{\psi'} = \Hat I\qu \psi
$$
\subsection{Унитарность}
$$
    \Hat I^+ \Hat I = \Hat I \Hat I^+ = \Hat 1, \quad \Hat I^2 = \Hat 1
$$
Отсюда следует, что оператор эрмитов. Таким образом, оператору сопоставляется некая физическая величина, называемая \emph{чётность}. 
\subsection{Собственные значения}
$$
     \Hat I^2 = \Hat 1,
$$
$$
    \Hat I \qu \psi = \lam \qu \psi, \quad \lam^2 = 1
$$
$$
    \left[
        \begin{array}{c}
          \lam = +1 \\
          \lam = -1
        \end{array}
    \right.
$$
Значения $\lam$ называется \emph{чётностью}.
\subsection{Средние значения}
Среднее значение должно перейти так
$$
    \qtri{\psi}{\Hat x_k}{\psi} \to \qtri{\psi}{x'_k}{\psi}, \quad \Hat x_k \to \Hat x_k' = \Hat I^{-1} \hat x_k \Hat I
$$
Должен меняться знак:
$$
    \qtri{\psi}{x_k'}{\psi} = - \qtri{\psi}{\hat x_k}{\psi}
$$
Значит, имеет место антикоммутация
$$
    \{ \Hat x_k, \Hat I \} = 0, \quad \{ \Hat p_k, \Hat I \} = 0
$$
Аналогично можно показать, что для момента коммутация есть:
$$
    [\Hat J_k, \Hat I] = 0
$$
Этим состояниям уже можно приписать определённую чётность, в отличие от предыдущего случая
\subsection{Теорема Нётер}
Оператор инверсии одновременно является унитарным и эрмитовым. Система инвариантна относительно инверсии. Получается некоторый интеграл движения.
$$
    [\Hat H, \Hat I] = 0
$$
Таким образом, чётность является физической величиной.