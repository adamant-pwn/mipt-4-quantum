\section{Преобразования инвариантности в координатном представлении}
Пусть имеется переход из новой системы координат в старую\wikiq{?} при помощи преобразования $\Hat G$:
$$
    S \to S'
$$

Это вызывает преобразование векторов состояния:
$$
    \qu \phi \to \qu{\psi'} = \hat U (\Hat G) \qu \psi
$$

В качестве модели выбираем простую систему: \emph{частица без внутренних степеней свободы}.

Эта система описывается при помощи координатной волновой функции из $L_2$:
$$
    \psi(\vec r) = \qs{\vec r}{\psi} \in L_2 (\R^3)
$$
\Quest{Что будет происходить с описанной волновой функцией при преобразовании координат?}

\Ans Несколько раньше было сказано, что все операторы в системе без внутренних степеней свободы, есть функции от $\Hat p, \Hat x$. В этом случае в координатном представлении все матричные операторы либо дельта-функции, либо их производные (которые тоже выражаются через дельта-функции\footnote{$\dta'(x) = -\dfrac{1}{x}\dta(x)$}).

Пусть $\hat u$~--- какой-то дифференциальный оператор,
$$
    \qtri{\vec r}{\Hat U (\Hat G)}{\vec r'} = \hat u (\Hat G) \dta^{(3)} (\vec r - \vec r')
$$
\begin{eqnarray*}
    \qs{\vec r}{\psi'} &=& \psi'(\vec r) = \qtri{\vec r}{\hat u(\Hat G)}{\psi}\\
    &=& \int\limits_{-\infty}^{+\infty} \qtri{\vec r}{\hat u (\Hat G)}{\vec r'}\qs{\vec r'}{\psi} d^3 x'\\
    &=& \int\limits_{-\infty}^{+\infty} \hat u(\Hat G) \dta^{(3)} (\vec r - \vec r') \psi'(\vec r') d^3 x'\\
    &=&\hat u(\Hat G) \psi(\vec r)
\end{eqnarray*}

При этом не указан характер действия этих операторов. Заполним этот пробел.

Зафиксируем точку пространства $\vec r$. При изменении системы координат координаты $\vec r$ стали $\vec r'$, точка осталась неподвижна: $\qu \psi \to \qu{\psi'}$.

При этом волновая функция преобразовывается по правилу:
$$
    \qs{\vec r}{\psi} \to \qs{\vec r'}{\psi'}
$$
Так как \emph{состояние} не изменилось (это следствие общего квантово-механического принципа об инвариантности физических законов в разных системах координат), то
$$
    \psi(\vec r) = \psi' (\vec r')
$$
Заметим, что
$$
    \psi'(\vec r) = \psi' \big(\Hat G(\hat G^{-1} \vec r)\big) = \psi'\big( (\Hat G^{-1} \vec r)' \big) = \psi(\Hat G^{-1}\vec r)
$$
\Rem Приведенные рассуждения верны для \emph{однокомпонентных систем}.

\textbf{Утверждение.} Совокупность операторов $\Hat u(\Hat G)$ реализует некоторое унитарное представление группы преобразований систем координат. 
\begin{enumerate}
  \item Линейность очевидна.
  \item Выполнение группового закона $\Hat U(\hat G)$ проверяется простым упражнением.
  \item Унитарность (сохранение скалярного произведения). При линейных заменах модуль якобиана под интегралом остаётся неизменным.
  \item Преобразование инвариантно, поэтому сохраняет вид уравнения Шрёдингера.
  
  В координатном представлении оно имеет вид
  $$
    i \hbar \ud{}{t} \psi(\vec r, t) = \hat H \psi(\vec r, t)
  $$
  Действуя слева оператором $\hat u$, не зависящим от времени, получаем:
  $$
    \hat H(\vec r, \vec \nla) \to \Hat H' (\vec r, \vec \nla) = \hat u(\hat G) \Hat H \hat u^{-1} (\hat G) = 
    {\Hat H}({\Hat G}^{-1} \vec{r}, {\Hat G}^{-1} \vec{\nabla})
  $$
  Отсюда получается дополнительное условие на гамильтониан:
  $$
    \Hat H(\vec r, \vec \nla) = \hat H(\vec r', \vec \nla')
  $$
  То есть функциональная зависимость в старых переменных и в новых переменных должна быть одинаковой.
\end{enumerate}

\section{Группа пространственных трансляций}
Будем обращать внимание на связь между свойствами:
\begin{itemize}
  \item симметрии пространства-времени
  \item наличия интегралов движения
\end{itemize}
В классической механике эта связь была установлена теоремой Нётер.

%\Th Инвариантности гамильтониана относительно преобразований какой-нибудь группы отвечает интеграл движения.
\subsection{Преобразование координат}

Пусть $O' = O + \vec a$. Тогда
$$
    \vec r' = \vec r - \vec a = \Hat T_{\vec a} \vec r
$$
Данная группа является \emph{трёхпараметрической группой Ли}.

\subsection{Представление группы в пространстве векторов состояний}
Рассмотрим инфинитезимальное преобразование, отвечающее переносу на бесконечно малый вектор $\dta \vec a$.
$$
    \Hat U(\dta \vec a) \approx \hat 1 + i \sum_{k=1}^{3} \dta a_k \Hat F_k = \Hat 1 + i (\dta \vec a \Hat {\vec F})
    \overset{\mathclap{\substack{\text{просто переобозначение}\\ \downarrow \\ \\}}}{=} 
    \hat 1 + \dfrac{i}{\hbar} (\dta \vec a \hat{\vec p})
$$
(здесь $\hbar$ это не постоянная Планка, а $\vec p$~--- не импульс.)

Уточним действие этого оператора в пространстве волновых функций. Функции преобразуются вот так:
$$
    \psi'(\vec r) = \hat u (\dta \vec a) \psi(\vec r) = \psi(\hat T_{\dta \vec a}^{-1} \vec r) = \psi(\vec r + \dta \vec a)
$$
Раскладывая по Тейлору:
$$
    \psi'(\vec r) \approx \big(\Hat 1 + (\dta \vec a \vec \nla) \big) \psi(\vec r)
    = \big(\hat 1 + \dfrac{i}{\hbar} (\dta \vec a \hat{\vec p}) \big) \psi(\vec r)
$$

Но на самом деле с точностью до константы $\hbar$,
$$
    \hat{\vec p} = \hbar \hat{\vec F} = -i\hbar \vec \nla
$$
Это равенство верно для всех представлений группы.

Итак, $\hat{\vec p}$~--- генераторы бесконечно малых трансляций.

Воспользуемся теоремой Ли, и получим унитарный оператор, дающий конечные трансляции:
$$
    \hat u(\vec a) = e^{\frac{i}{\hbar} \vec a \hat{\vec p}}
$$
\Lyrdig \textbf{средние значения.}
Как говорилось ранее, с физической точки зрения, систему характеризуют вероятности и её средние значения (которые вычисляются в текущем состоянии, в том состоянии, в котором сейчас находится система).

У нас есть преобразование координат $\vec r \to \vec r' = \Hat G \vec r$, и семейство операторов $\hat u(\hat G)$.

При помощи этих операторов можно осуществить преобразования \emph{трёх разных типов}. 
\begin{enumerate}
  \item $\qu \psi \to \qu{\psi'} = \hat U(\hat G) \qu \psi$
  $$
    \uq{\psi} \to \uq{\psi'} = \uq \psi \Hat U^{-1}
  $$
  $$
    \Hat F \to \Hat F
  $$
  По типу 1 преобразуются только векторы состояний, но не операторы!
  $$
    \qtri{\psi}{\Hat F}{\psi} \to \qtri{\psi'}{\Hat F}{\psi'} = \qtri{\psi}{\hat U^{-1}\Hat F\Hat U}{\psi} 
  $$
  При этом среднее значение оператора $\Hat F$, вообще говоря, меняется. Изменился физический способ описания системы, произошёл сдвиг среднего значения (например, координата поменялась). В некоторых обстоятельтвах сдвига может не быть (трансляционные инварианты). Вероятности (скалярные произведения) не изменяются.
  \item Преобразовываются операторы, но не векторы состояний!
  $$
    \qu \psi \to \qu \psi, \qquad \uq \psi \to \uq \psi
  $$
  $$
    \hat F \to \Hat F' = \Hat U^{-1} \Hat F \Hat U
  $$
  Действительно,
  $$
    \qtri{\psi}{\hat F}{\psi} \to \qtri{\psi}{\hat F'}{\psi} = \qtri{\psi}{\Hat U^{-1}\hat F\hat U}{\psi}
  $$
  При преобразованиях типа 1 и 2 одинаковым образом преобразуются средние значения.
  
  \Rem Если $[\Hat F, \Hat U] = 0$, то средние значения не меняются.
  \item Преобразуются и вектора состояний и операторы.
  $$
    \begin{cases}
        \qu \psi \to \qu{\psi'} = \Hat U \qu \psi\\
        \uq \psi \to \uq{\psi'} = \qu \psi \Hat U^{-1}\\        
        \Hat F \to \hat F' = \Hat U \Hat F \Hat U^{-1}
    \end{cases}
  $$
  При этом
  $$
    \qtri{\psi}{\Hat F}{\psi} = \qtri{\psi'}{\Hat F'}{\psi'}
  $$
  Это преобразование устроено совсем по-другому, чем первые два типа. При этом ни одно из наблюдений не меняется. Это аналог канонических преобразований в классической механике. 
\end{enumerate}
Вернёмся к группе трёхмерных трансляций.

\subsection{Остальные выводы}
Если преобразовывать операторы (то есть тип 2), то посмотрим на оператор координаты и импульса:
$$
    \Hat{\vec p} \to \hat{\vec p'} = e^{-\frac{i}{\hbar} \vec a \hat{\vec p}} \hat{\vec p} e^{\frac{i}{\hbar} \vec a \hat{\vec p}} = \hat {\vec p}
$$
Это пример \emph{трансляционно-инвариантной} величины.
$$
    \Hat{\vec r} \to \Hat{\vec r'} = e^{-\frac{i}{\hbar} \vec a \hat{\vec p}} \hat{\vec r} e^{\frac{i}{\hbar} \vec a \hat{\vec p}} = \Hat{\vec r} + e^{-\frac{i}{\hbar} \vec a \hat{\vec p}} [\Hat{\vec r}, e^{\frac{i}{\hbar} \vec a \hat{\vec p}}]
    = \Hat{\vec r} - \vec a \cdot \Hat 1
$$
\subsubsection{Однородность пространства}
Это физическое свойство пространства, такое, что свободные частицы не должны зависеть от выбора начала системы координат. 

Из теоремы Нётер следовало бы, что тут должна быть какая-нибудь симметрия.

Под $a_k$ подразумевается $x, y, z$:
$$
    [\Hat H, \Hat U(a_k)] = 0, \quad \to \quad [\Hat H, \Hat p_k] = 0
$$
\textbf{Вывод.} Импульс свободной частицы сохраняется и измерим вместе с энергией. Это квантовомеханический аналог закона инерции Ньютона.

\subsubsection{}
Из жёстких требований на Гамильтониан:
$$
    \Hat H(\Hat{\vec r}, \Hat{\vec p}) = \Hat H(\vec r', \vec p') = \Hat H(\Hat{\vec r} - \vec a, \Hat{\vec p})
$$
Гамильтониан \emph{не должен} зависеть от координат.

\section{Группа временных трансляций}
Свойства физической системы не должны зависеть от положения начала отсчёта времени.
\subsection{Преобразования времени}
$$
    t \to t' = t - \tau
$$
\subsection{Преобразование векторов состояний}
$$
    \qu{\psi'(t)} = \Hat U(\tau) \qu{\psi(t)}
$$
По сути семейство этих операторов есть семейство операторов \emph{эволюции}.
\subsection{Преобразование волновой функции}
Из физических соображений $\psi'(t') = \psi(t)$, откуда
$$
    \psi'(t) = \hat u (\tau) \psi(t) = \psi(t + \tau)
$$
Рассмотрим оператор {\infzh} сдвигов во времени.
\begin{eqnarray*}
    \psi'(t) = \hat u(\dta \tau) \psi(t) = \psi(t + \dta \tau)\\
    &\approx& (1 + \dta \tau \ud{}{t} ) \psi(t)\\
    &\approx& (1 + i \dta \tau \Hat F_t) \psi(t)
\end{eqnarray*}
Поэтому
$$
    \Hat F_t = -i \ud{}{t}, \qquad \Hat F_t = -\dfrac{1}{\hbar} \hat H
$$
Получили условие стационарности:
$$
    [\Hat F_t, \Hat H] = 0, \qquad \ud{\Hat H}{t} = 0
$$
Однородности времени отвечает закон сохранения энергии.

\section{Группа трёхмерных вращений}
Эта группа обычно называется $O(3)$, а если рассматривается группа без отражений, то $SO(3)$ (special orthogonal 3-parametric).

\Quest{Как задать поворот в 3-мерном пространстве?} 

\Ans Обозначим общий элемент группы $\vec \phi$, который сонаправлен с вектором $\vec n$, задающим ось вращения (поворот по часовой стрелке). Величина вектора равна по определению величине угла поворота, $\|\vec \phi\| < \pi$. Если нужен поворот на больший угол, меняем направление вектора.

$$
    \Hat R_{\vec \phi} = \Hat R_{\phi \vec n}
$$ 
Однопараметричность группы:
$$
    \vec \phi = \{\phi_1, \phi_2, \phi_3\}
$$
Обратим внимание на то, что пространство параметров топологически устроено довольно сложно.

Это многообразие называется $\R P^3$. Двумерный аналог это проективная плоскость.

\subsection{Повороты вокруг оси $Ox$}
$$
    \begin{cases}
        x' &= x\\
        y' &= y \cos \phi_1 + z \sin \phi_1\\
        z' &= z \cos \phi_1 - y \sin \phi_1
    \end{cases}
$$
В матричном виде:
$$
    \begin{bmatrix}
      x' \\
      y' \\
      z' \\
    \end{bmatrix}
    =
    \begin{bmatrix}
      1 & 0 & 0 \\
      0 & \cos \phi_1 & \sin \phi_1 \\
      0 & -\sin \phi_1 & \cos \phi_1 \\
    \end{bmatrix}
    \begin{bmatrix}
      x \\
      y \\
      z \\
    \end{bmatrix}
    = 
    \Hat R_{(\phi_1)}
        \begin{bmatrix}
      x \\
      y \\
      z \\
    \end{bmatrix}
$$
Ищем инфинитезимальный оператор:
$$
    \Hat f_1 =\left. -i \ud{}{\phi_1} \Hat R (\phi_1) \right|_{\phi_1 = 0}
    =
    \begin{bmatrix}
      0 & 0 & 0 \\
      0 & 0 & -i \\
      0 & i & 0 \\
    \end{bmatrix}
$$
$$
    \Hat f_2 =
    \begin{bmatrix}
      0 & 0 & i \\
      0 & 0 & 0 \\
      -i & 0 & 0 \\
    \end{bmatrix}, \quad 
    \Hat f_3 =
    \begin{bmatrix}
      0 & -i & 0 \\
      i & 0 & 0 \\
      0 & 0 & 0 \\
    \end{bmatrix}    
$$
Перемножением находим важные соотношения:
$$
    [\hat f_1, \hat f_2] = i \hat f_3, \qquad [\Hat f_i, \Hat f_j] = i e_{ijk} \Hat f_k
$$
Эта алгебра Ли изоморфна алгебре матриц Паули.
\subsection{}
Ищем \infze операторы для бесконечно малых поворотов.

Формула Эйлера:
$$
    \vec r' \approx \vec r - \dta \phi [\vec n \times \vec r]
$$
Введём обозначение $\hat L$, такое, что
$$
    \hat u(\dta \vec \phi) \approx \hat 1 + i \dta \phi(\vec n \Hat{\vec F}) =
    \hat 1 + \dfrac{i}{\hbar} \dta \phi(\vec n \Hat{\vec L})
$$
Передвигая члены смешанного произведения, получим:
\begin{eqnarray*}
    \phi' (\vec r) &\approx& \hat u(\dta \vec \phi) \psi(\vec r)\\
    &=& \psi(\vec r + \dta \phi[\vec n \times \vec r])\\
    &\approx& (\hat 1 + \dta \phi[\vec n \times \vec r] \vec \nla)\\
    &=& (\Hat 1 + \dta \phi (\vec n[\vec r \times \vec \nla])) \psi(\vec r)
\end{eqnarray*}
Представление группы трёхмерных вращений реализовано с точностью до множителя постоянной Планка с компонентами орбитального момента. Обратим внимания на то, что у системы не было никаких своих степеней свободы.