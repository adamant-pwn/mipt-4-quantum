%\chapter{Магнитное поле}
\setcounter{chapter}{7}
\chapter{Движение в центрально-симметричном поле}
\section{Уровни энергии}
Запишем снова уравнение Шрёдингера
$$
    i \hbar \ud{}{t} \Psi(\vec r, t) = \Hat H \Psi (\vec r, t)
$$
Напомним, что если $\ud{\Hat H}{t} = 0$, то уравнение решается методом разделения переменных,
$$
    \left\{
        \begin{array}{c}
            \Psi(\vec r, t) = \exp (- \frac{i E t}{\hbar}) \psi(\vec r),\\
            \Hat H \psi(\vec r) = E \psi(\vec r)
        \end{array}
    \right.
$$
Свободному движению соответствует плоская волна. 
В декартовых координатах вместо $\Hat H = \dfrac{\Hat{\vec p}^2}{2m}$ пишем $-\dfrac{\hbar^2}{2m} \Dta$

В импульсном представлении:
$$
    \psi_{\vec p} (\vec r) = \dfrac{1}{(2 \pi \hbar)^{3/2}}
$$
Соответствующий уровень энергии вырожден.
$$
    E_{p} = \dfrac{\vec p^2}{2m}
$$

\section{Общий подход к задаче о движении в центрально-симметричном поле.}
Для центрально-симметричного поля:
$$
    U(\vec r) = U(|\vec r|) = U(r), \quad r = \sqrt{x^2 + y^2 + z^2}
$$
\subsection{Гамильтониан}
$$
    H_{\mathrm{classical}} = \dfrac{\vec p^2}{2m} + U(r) = \dfrac{p_{r}^2}{2m} + \dfrac{\vec L^2}{2mr^2} + U(r)
$$
Переход к квантовым величинам происходит с помощью расстановки <<крышечек>>.
$$
    \Hat H = \dfrac{\Hat{\vec p}^2}{2m} + U(r) = \dfrac{\Hat{p_{r}}^2}{2m} + \dfrac{\Hat{\vec L}^2}{2mr^2} + \Hat U(r)
$$
Выбираем сферическую систему координат:
$$
    \begin{cases}
        x = r \cos \phi \sin \theta\\
        y = r \sin \phi \sin \theta\\
        z = r \cos \theta
    \end{cases},
    \quad 
    \begin{array}{c}
        0 \leqslant \theta \leqslant \pi\\
        0 \leqslant \phi \leqslant 2 \pi
    \end{array}
$$
При этом
$$
\begin{cases}
    \Hat p_{r}^2 \to - \hbar^2 \Dta_r = - \hbar^2 \left( 
        \dfrac{\partial^2}{\partial r^2} + \dfrac{2}{r} \ud{}{r}
    \right)\\
    \Hat{\vec L^2} \to -\hbar^2 \Dta_{\theta, \phi}
    =
    -\hbar^2 \left(
        \dfrac{1}{\sin \theta} \ud{}{\theta} \left(
            \sin\theta \ud{}{\theta}
        \right)
        + \dfrac{1}{\sin^2 \theta} \left(\ud{}{\phi}\right)^2
    \right)
\end{cases}
$$
\subsection{Полный набор взаимно коммутирующих наблюдаемых}
Из написанного выше можно сделать вывод, что
$$
    [\Hat H, \Hat{\vec L}^2] = [\Hat H, \Hat L_z] = 0
$$
Значит, операторы имеют полную общую систему собственных векторов, можно составить систему
$$
    \begin{cases}
        \Hat H \psi(\vec r) = E \psi(\vec r)\\
        \Hat{\vec L}^2 \psi(\vec r) = \hbar^2 \ell (\ell + 1) \psi(\vec r)\\
        \Hat L_z \psi(\vec r) = \hbar m \psi(\vec r)
    \end{cases}
$$
Общее решение должно иметь вид
$$
    \psi(\vec r) = \underbrace{R(r)}_\text{радиальная часть} \underbrace{Y(r)}_\text{угловая часть}
$$
$$
    Y(\theta, \phi) = Y_{\ell}^{(m)} (\theta, \phi)
$$
С учётом того, что $Y$~--- сферическая функция, получаем \emph{радиальное уравнение Шрёдингера}:
$$
\left\{
    \dfrac{\Hat p_r^2}{2m} + \dfrac{\hbar^2 \ell (\ell+1)}{2mr^2} + U(r)
\right\}
R(r) = E R(r)
$$
Надо принять к сведению, что если записать функцию в таком виде, то каждая из этих частей (угловая и радиальная) нормируется раздельно на единицу. Нормировочный интеграл записывается в виде произведения нормировочных интегралов. 
$$
    \underbrace{\int_{0}^\infty \underbrace{r^2 dr}_{\text{якобиан}} R^\ast R}_{=1} \oint \underbrace{d\Omega Y^\ast Y}_{=1} = 1
$$
Итак, получено уравнение
$$
    \begin{cases}
        R(r) = \dfrac{\chi(r)}{r}\\
        \chi(0) = 0
    \end{cases}
$$
Выписанные решения очень похожи на решения одномерного уравнения Шрёдингера.

$$
    \left\{ \dfrac{d^2}{dr^2} +\dfrac{2m}{\hbar^2} (E - U^{eff})
    \right\}
    \chi(r) = 0
$$
Укороченное уравнение Шрёдингера:
$$
    U^{\mathrm{eff}} = U(r) + \underbrace{\dfrac{\hbar^2 \ell(\ell+1)}{2mr^2}}_{\text{потенциал отталкивания}}
$$
\section{Атом водорода}
Эта задача является точно решаемой задачей.

Кулоновский потенциал притяжения:
$$
    U(r) = - \dfrac{Z e^2}{r}
$$
$Z$ --- заряд ядра. Если $Z > 1$, то атом \emph{ионизованный}.

Уравнение Шрёдингера принимает вид
$$
    \left\{
        \dfrac{d}{dr^2} + \dfrac{2}{r} \dfrac{d}{dr}
    \right\} R(r) + 
    \left\{
        -\dfrac{\ell(\ell+1)}{r^2} + \dfrac{2m}{{\hbar}^2} \dfrac{Z e^2}{r} + \dfrac{2mE}{\hbar^2}
    \right\} R(r) = 0
$$
Мы увидим, что эффективный потенциал имеет вид:
%рисунок
\subsection{Атомная система единиц}
\begin{enumerate}
    \item Характерная единица длины: размер атома.
$$
    a = \dfrac{\hbar^2}{me^2} \approx 0.529 \cdot 10^{-8} \text{ см}
$$
На основании этой длины строятся все остальные величины.
  \item Из соотношения неопределённостей получаем \emph{атомную единицу импульса}
$$
    p_{a} = \dfrac{\hbar}{a} = \dfrac{me^2}{\hbar}
$$
  \item $E_a = \dfrac{e^2}{a} = \dfrac{pa^2}{m} = 27.21$ Эв, атомная единица энергии (Хартри)
  \item $V_a = \dfrac{p_a}{m} = \dfrac{e^2}{\hbar}$
  
  Отсюда
  $$
    \dfrac{v_a}{c} = \dfrac{e^2}{\hbar c} = \alpha \approx \dfrac{1}{137}
  $$
  $\alpha$~--- \emph{постоянная тонкой структуры}.
\end{enumerate}
Переход к атомной систем единиц заключается в следующем.
$$
    \begin{cases}
        r \to p = \dfrac{r}{a};\\
        E \to \eps = \dfrac{E}{E_a}
    \end{cases}
$$
$$
    \hbar = m = e = 1
$$
При такой замене (обезразмеривании) уравнение Шрёдингера в атомных единицах принимает вид
$$
    \left\{
        \dfrac{d}{d\rho^2} + \dfrac{2}{\rho} \dfrac{d}{d\rho}
    \right\} R(\rho) +
    \left\{
        -\dfrac{\ell(\ell+1)}{\rho^2} + \dfrac{2Z}{\rho}  -\kappa^2
    \right\} R(\rho) = 0, -\kappa^2=\frac{2E}{E_a}
$$
%3.3
\subsection{Ход решения уравнения}
Полностью проводить выкладки не будем, это вроде как очевидно, а хотелось бы посмотреть на то, что менее очевидно.

Посмотрим на асимптотическое поведение уравнения при $\rho \to \infty$.
$$
    \left(
        \dfrac{d^2}{d \rho^2} - \kappa^2
    \right) R_{\infty} (\rho) = 0, \quad R_{\infty} (\rho) = C_1 e^{-\kappa \rho} + C_2 e^{\kappa \rho}, \,\, C_2 = 0
$$
При $\rho \to 0$:
$$
    \left(
        \dfrac{d^2}{d\rho^2} + \dfrac{2}{\rho} \dfrac{d}{d \rho} - \dfrac{\ell(\ell+1)}{\rho^2} 
    \right)R_0 (\rho) = 0
$$
$$
    R_0(\rho) = \rho^q, \qquad q(q-1) + 2q - \ell(\ell+1) = 0
$$
Получаем систему на коэффициенты:
$$
    \begin{cases}
        q_1 = \ell\\
        q_2 = - (\ell+1)
    \end{cases},
    \to \quad R_0(\rho) = C_1 \rho^\ell + C_2 \rho^{-(\ell+1)}, \quad C_2 = 0
$$
Рассмотрим ещё один случай, подставляя функцию следующего вида, где $u$~--- произвольная функция
$$
    R(\rho) = e^{-\kappa \rho} \rho^{\ell} u(\rho)
$$
$$
    \rho u'' + u' \big(
        2(\ell+1) - 2\kappa \rho + u(2z - 2 \kappa(\ell+1))
    \big) = 0
$$
Ищем решение в виде бесконечных рядов.
\begin{eqnarray}
    u(\rho) &=& \sum_{k=0}^{\infty} a_k p^k,\\
    u'(\rho) &=& \sum_{k=0}^{\infty} a_{k+1} (k+1) \rho^k,    \\
    u''(\rho) &=& \sum_{k=0}^{\infty} a_{k+1} (k+1) k \rho^{k-1}
\end{eqnarray}
Все эти функции нужно подставить в это уравнение.
$$
    \sum_{k=0}^\infty \rho^k
    \left[
        a_{k+1} \big( 
                        k(k+1)
                        + 2 (\ell + 1) (k+1)
                \big)
        + a_k \big(
                    2 Z - 2 \kappa (\ell+1) - 2\kappa k
              \big)
    \right] = 0
$$
Чтобы эта сумма была равна нулю, необходимо и достаточно, чтобы все коэффициенты были равны нулю.

Получается рекуррентное соотношение
$$
\underline{    a_{k+1} = a_k \dfrac
                    {2(\kappa (\ell + k + 1) - Z)}
                    {(k+1)( k+ 2 (\ell+1))}}
$$
\textbf{Анализ.} 

Надо потребовать ограниченность функции $U(\rho)$ при $\rho \to \infty$.

В соответствии с рекуррентными соотношениями при достаточно больших $k$ эти слагаемые одно знака. Основной вклад в эту сумму дают слагаемые с большим значением $k$.
$$
    \left|
    \dfrac{a_{k+1}}{a_k}
    \right| \approx \dfrac{2 \kappa}{k}
$$ 
Сравним это выражение с функцией $e^{2 \kappa \rho}$. Её разложение имеет вид
$$
    e^{\kappa \rho} = 1  + \dfrac{2 \kappa \rho}{1!} + \ldots + \dfrac{(2 \kappa \rho)^k}{k!} + \ldots
$$
При больших значениях $\rho$ функция ведёт себя как $e^{2 \kappa \rho}$.

$$
    R(\rho) \sim e^{\kappa \rho}, \quad \rho \to \infty
$$

\textbf{Обрыв ряда.}

Если при каком-то $k$ множитель при $a_k$ будет равен нулю, то бесконечный ряд превратится в многочлен.

Пусть $k = n_r$ (\emph{радиальное квантовое число}).
$$
    a_{n_r} \neq 0, \quad \forall k > n_r \,\, a_k = 0
$$
Условие обрыва ряда:
$$
    \dfrac{z}{\kappa} - \ell - 1 = n_r
$$
Удобно ввести обозначение.

\begin{itemize}
  \item Главное квантовое число. $n = n_r + \ell + 1 = 1, \, 2, \,  3, \ldots$
\end{itemize}

Вспоминая, как выражается энергия через квантовое число, получаем энергетический спектр водорода.
$$
    \mathcal{E}_n = - \dfrac{Z^2}{2n^2} =\dfrac{E_n}{E_a}
$$
$$
    - \kappa^2 = \dfrac{2 E}{E_a} = 2 \mathcal{E} < 0
$$
\textbf{Уравнение на собственные функции.}

$$
    \psi_{n\ell m} (\vec r) = R_{n, \ell}  (\rho) Y_{\ell} ^{(m)} (\theta, \phi)
$$
$$
    R_{n, \ell} (\rho) = C_{n, \ell} e^{-\kappa \rho} \rho^{\ell} U_{n, \ell}(\rho)
$$
\Def \emph{Присоединённый полином Лаггера}
$$
    U_{n, \ell} (\rho) = L_{n_r}^{(2 \ell + 1)} (2 \kappa \rho)
$$
$$
    L_k^s (x) = e^x x^{-s} \dfrac{d^k}{dx^k} (e^{-x} x^{s+k})
$$
\Quest{Какой физический смысл радиального квантового числа?}

\Ans Это просто степень полинома. Кроме того, это число узлов в радиальной волновой функции.

Это частный случай так называемой \emph{осцилляционной теоремы}.

\Th (Одномерный случай). Число нулей волновой функции совпадает с номером энергии связанного состояния, если их нумеровать начиная с нулевого.

\textbf{Нормировка.} 
$$
    1 = \int\limits_0^\infty \rho^2 d \rho R_{n, \ell} (\rho) R_{n, \ell} (\rho)
$$
$$
    C_{n_r \ell} = \kappa ^{3/2} \sqrt{\dfrac{4}{n(n-\ell-1)! (n+2)!}}
$$
Решение:
{\small
$$
    \phi_{n \ell m} (\vec r) = \left( \underbrace{\dfrac{z}{na}}_{\kappa} \right)^{3/2}
    \sqrt{\dfrac{4}{n(n-\ell-1)! (n+\ell)!}}e^{-(rz/na)}
    \left( \dfrac{2Z}{na} r\right)^\ell L_{n-\ell-1}^{(2\ell+1)}
    \left(
        \dfrac{2Z}{na} r
    \right)
    \sqrt{\dfrac{2\ell+1}{4 \pi} \dfrac{(\ell-m)!}{(\ell+m)!}} e^{im\phi} P_{\ell}^{(m)} (\cos \theta)
$$
}
\subsection{Вырождение}
Квантовые числа пробегают следующие значения.
\begin{itemize}
\item Главное квантовое число $n = 1,2,3, \ldots$
\item Орбитальное квантовое число $\ell = 0, 1, 2, \ldots, n-1$
\item Магнитное квантовое число $m = 0, \pm 1, \pm 2, \ldots, \pm l$
\end{itemize}

Кратность вырождения уровней энергии равна
$$
    \sum_{\ell=0}^{n-1} (2\ell + 1) = n^2
$$
\Th Предположим что наблюдаемые $\Hat A, \Hat B$ коммутируют с $\Hat F$, но не коммутируют между собой. Тогда спектр оператора $\Hat F$ вырожден.

\Proof Допустим, что у оператора $\Hat F$ есть хотя бы один собственный вектор.
$$
    \Hat F \qu{F_i} = F_i \qu{F_i}
$$
Действуем оператором $\Hat A$ слева, 
$$
    \Hat A \Hat F \qu{F_i} = \hat F \hat A \qu{F_i} = F_i \Hat A\qu{F_i}
$$
В силу невырожденности, каждому значению отвечает только один собственный вектор, поэтому
$$
    \Hat A \qu{F_i} = A_i \qu{F_i}
$$
Аналогично, для оператора $\Hat B$:
$$
    \Hat B \qu{F_i} = B_i \qu{F_i}
$$
Строя коммутатор, получаем
$$
    (\Hat A \Hat B - \Hat B \Hat A) \qu{F_i} = (A_i B_i - B_i A_i) \qu{F_i} = 0
$$
Противоречие.

\textbf{Вырождение по магнитному числу.}

Это происходит с любой центрально-симметричной системой, в силу того, что
$$
    [\Hat H, L_z] = 0, \quad [\Hat H, \Hat L_\pm] = 0, \quad [\Hat L_z, \Hat L_{\pm}] \neq 0
$$
Действуем оператором $\Hat L_{\pm}$ на собственные векторы гамильтониана,
$$
    \Hat H \psi_{n \ell m \pm 1} (\vec r) = E \psi_{n \ell m\pm1} (\vec r)
$$
Кратность вырождения $2 \ell +1$.

Гамильтониан должен коммутировать с оператором конечных поворотов:
$$
    [\Hat H, \Hat U(\phi)] = 0
$$
Для этого необходимо и достаточно предположить (см. выше) коммутируемость с $\Hat{\vec L}^2$, $\Hat L_i$.

В пространстве $E$ размерности $2 \ell + 1$ реализовано представление
$$
    E_{(2\ell+1)} \to P(\ell) \to E_n
$$
Кратность вырождения гамильтониана равна размерности пространства неприводимого представления.

Физическая причина явления: \emph{изотропия пространства}.

\textbf{Вырождение по орбитальному квантовому числу.}

Это вырождение очень нетипично (см. также изотропный одномерный осциллятор). Также называется <<\emph{случайное}>>, или \emph{кулоновское} вырождение. 

Для кулоновского поля есть инвариант, который называется вектором \emph{Лапласа-Рунге-Ленца} (эксцентриситет).

В классике:
$$
    \vec \eps = \dfrac{1}{z} [\vec L \times \vec p] + \dfrac{\vec r}{r}
$$
Пояснение: если умножить на $\vec r$ скалярно, получим
$$
    \vec{\eps} \vec{r} = - \dfrac{\vec L^2}{z} + r
$$
$$
    r = \dfrac{\vec L^2 / z}{1 - \eps \cos \phi}
$$
Получили уравнение эллипса.

\textbf{Квантовый случай}
\emph{Оператор эксцентриситета выглядит следующим образом:}
$$
    \Hat{\vec \eps^2} = \dfrac{1}{2z} \left(
        [\Hat{\vec L} \times \Hat{\vec p}]
        -
        [\Hat{\vec p} \times \Hat{\vec L} ]
    \right) + \dfrac{\vec r}{r}
$$
Уровни энергии вырождаются по $\ell$:
$$
    [\Hat H, \Hat \eps_i] = 0, \quad [\Hat H, \Hat{\vec L}^2] = 0,
$$
$$
    [\Hat{\vec L}^2, \Hat \eps_i] \neq 0
$$
Получается так, что одному и тому же уровню энергии отвечают различные значения $\ell$. Преобразовываться надо по неприводимым представлениям.  Истинная группа симметрии атома водорода: SO(4)~--- четырёхмерные вращения обычного Евклидова пространства. Алгебра Ли состоит из шести генераторов. 