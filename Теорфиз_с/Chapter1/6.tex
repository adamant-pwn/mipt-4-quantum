\setcounter{section}{5}
\section{Задача Кронига-Пенни}
%\marginpar{Метод Крамерса}
\Task{6}
$$
    U(x) = \dfrac{\hbar^2}{m}\kappa^2 \sum_{n=-\infty}^{n=+\infty} \dta(x - na)
$$
\Picw{./pic/6/Potential-actual.PNG}{0.3}
\Quest{Какую систему может описывать этот потенциал?}

\Ans Это может являться кристаллом (так как имеет периодическую структуру).

\Rem Задача из задания с помощью переобозначений сводится к этой.

\Quest{Сколько связанных состояний?}

\Ans Так как пиков бесконечно много, то и связанных состояний тоже бесконечно много.

\Quest{Как устроены уровни энергии в кристалле?}

\Ans В кристалле есть \emph{разрешённые} и \emph{запрещённые зоны}.

Запишем уравнение Шрёдингера для этого потенциала:
$$
    \psi'' (x) + \dfrac{2m}{\hbar^2} (E - V(x)) \psi(x) = 0
$$
Периодичность потенциала:
$$
    U(x+a) = U(x)
$$
Так как наше уравнение второго порядка то для каждогшо уровня энергии есть два линейно независимых решения. Из без ограничения общности можно выбрать вещественными.

Пусть $\psi_1(x), \psi_2(x)$~--- линейно независимые решения.

Тогда $\psi_1(x+a)$, $\psi_2 (x+a)$~--- тоже решения. Таким образом, эти решения можно записать в виде линейной комбинации:
$$
    \begin{cases}
        \psi_1(x+a) = a_{11} \psi_1(x) + a_{21} \psi_2(x),\\
        \psi_2(x+a) = a_{12} \psi_1(x) + a_{22} \psi_2(x)
    \end{cases}
$$
В матричном виде:
$$
    \Big( \psi_1(x+a), \psi_2(x+a) \Big) = \big(\psi_1(x), \psi_2(x) \big)
    \underbrace{\begin{bmatrix}
      a_{11} & a_{12} \\
      a_{21} & a_{22} \\
    \end{bmatrix}}_{A}
        \eqno (\ast)
$$
Матрица $A$ называется \emph{матрицей трансляции.}

\textbf{Утверждение 1:}
$$
    \det A = 1
$$
\Lem Определитель Вронского любых двух решений $\psi_1(x), \psi_2(x)$, где
$$
    \psi''(x) - q(x) \psi(x) = 0, \qquad q(x) = \dfrac{2m}{\hbar^2} (E - U(x)),
$$
равен константе.

\Proof
$$
    W(\psi_1(x), \psi_2(x)) =
    \begin{vmatrix}
      \psi_1(x) & \psi_2(x) \\
      \psi_1'(x) & \psi_2'(x) \\
    \end{vmatrix}
    = \psi_1(x) \psi_2'(x) - \psi_2(x) \psi_1'(x)
$$
Подставляя в исходное уравнение:
$$
    \begin{cases}
        \psi_1''(x) + q(x) \psi_1(x) = 0,\\
        \psi_2''(x) + q(x) \psi_2(x) = 0,\\
    \end{cases}
$$
Домножая первое на $\psi_2(x)$, второе на $\psi_1(x)$, вычитая, получаем:
$$
    \psi_1(x) \psi_2''(x) - \psi_2(x) \psi_1''(x) = \dfrac{d}{dx} \big(
        \psi_1(x) \psi_2'(x) - \psi_2(x) \psi_1'(x)
    \big) = 0
$$
Лемма доказана.

***

Теперь, используя лемму, докажем, что детерминант матрицы трансляций равен единице.

Дифференцируя $(\ast)$, получаем:
$$
    \Big( \psi_1'(x+a), \psi_2'(x+a) \Big) - \big(\psi_1'(x), \psi_2'(x) \big)
    \underbrace{\begin{bmatrix}
      a_{11} & a_{12} \\
      a_{21} & a_{22} \\
    \end{bmatrix}}_{A}
$$
Соединяя это со $(\ast)$, получаем:
$$
    W(\psi_1(x+a), \psi_2(x+a)) = W(\psi_1(x), \psi_2(x)) \cdot \det A
$$
Так как множитель равен константе, то он равен 1.

\textbf{Утверждение 2 (теорема Флоке):} для периодического потенциала всегда можно выбрать такие линейные комбинации (не обязательно вещественные) $\phi_1$, $\phi_2$, такие, что
$$
    \begin{cases}
        \phi_1(x+a) = \lam_1 \phi_1(x),\\
        \phi_2(x+a) = \lam_2 \phi_2(x)
    \end{cases}
    \eqno (\ast \ast)
$$
Речь идёт о том, что $\lam_1$, $\lam_2$~--- собственные значения матрицы трансляции, мы хотим её диагонализовать.

\Rem Диагонализация этой матрицы возможна, так как кратных корней нет.

$$
    \begin{vmatrix}
      a_{11} - \lam & a_{12} \\
      a_{21} & a_{22} - \lam \\
    \end{vmatrix}
    = (a_{11} - \lam)(a_{22} - \lam) - a_{12} a_{12} = 0
$$
$$
    \lam^2 - (\underbrace{a_{11} + a_{22}}_{\tr A = 2 \rho}) \lam + 1 = 0
$$
Инварианты матрицы: определитель и след.
$$
    \det A =
    \begin{vmatrix}
      \lam_1 &  \\
       & \lam_2 \\
    \end{vmatrix} = 1
$$
Вспомним, какие физические граничные условия накладываются на волновые функции. Покажем, что $\lam$ лежат на единичной окружности.
\begin{enumerate}
  \item
    $$
        |\lam_1| > 1, \quad \phi_1(x + na) = \lam_1^n \phi_1(x),
    $$
    $\psi_1(x)$ неограниченно растёт при $x \to \infty$.
  \item
    $$
        |\lam_1| < 1,
    $$
    $\phi_1(x)$ неограниченно растёт при $x \to -\infty$
    Эти функции не являются нормируемыми даже в смысле плоских волн.
  \item
    $$
        |\lam_1| = 1
    $$
    В этом случае есть нормальная физическая трактовка. Собственные числа являются комплексно сопряжёнными.
\end{enumerate}
\textbf{Утверждение:}
$$
    \lam_{1,2} = e^{\pm i K a}, \quad \textrm{Im} \, K = 0
$$
Рассмотрим уравнение на собственные функции и собственные значения оператора трансляции:
$$
    \psi(x+a) = \Hat T_a \phi(x) = \lam \psi(x)
$$
\begin{enumerate}
  \item $|\lam| = 1$
  \item $\lam(a) \cdot \lam(b) = \lam(a+b)$
\end{enumerate}
Из этих двух соображений следует, что
$$
    \lam_{1, 2} = e^{\pm i K a}
$$
***

\textbf{Резюме:}

Для периодического потенциала было выяснено, что:
$$
    \psi(x + na) = e^{i K a \cdot n} \phi(x)
$$
\textbf{Теорема Бл\underline{о}ха:}
общее решение имеет вид
$$
    \phi_K(x) = e^{i Kx} u_K(x),
$$
где $u(k)$~--- \emph{любая} периодическая функция, с периодом $a$.

$\phi_K(x)$ называются \emph{функциями Бл\underline{о}ха}.

\Rem Константа $K \in \R$ отчасти играет роль импульса в твёрдом теле, и называется \emph{квазиимпульс}.
Вообще говоря, это вектор (в нашем одномерном рассуждении это не проявилось), и определяет поведение функции при трансляциях.

Надо сказать, что это не обычный импульс. Он неоднозначно определён:
\begin{enumerate}
  \item $e^{2\pi i n} = 1$, $n = 0, \pm 1, \ldots$

  Функции $K$, $K' = K + \dfrac{2\pi n}{a}$ являются квазиимпульсами.
  \item $[\Hat H, \Hat p] = 0$
  \item $K$ не является собственным значением оператора импульса $-i \hbar \ud{}{x}$
\end{enumerate}
Этот квазиимпульс связан через специальное уравнение (дисперсионное отношение) с энергией $E = \dfrac{\hbar^2 k^2}{2m}$.

В кристалле разрешённые зоны энергии: $K \in \R$.

***

Перед тем, как проводит расчёты, напомним, что была введена величина $\rho = \dfrac12 (a_{11} + a_{22})$.
Поэтому
$$
    \rho = \dfrac12 (\lam_1 + \lam_2) = \Cos  Ka
$$
Вернёмся к конкретному потенциалу, данному в условии.

\textbf{План:}
\begin{itemize}
  \item Рассмотрим ячейку 1: $0 < x < a$, и соседнюю ячейку 2: $a < x < 2a$.
  \item Напишем 2 линейно независимых решения для ячейки 1
  \item При помощи матрицы сдвигов перенесём решение в ячейку 2
  \item Сошьём решения при $x = a$.
  \item Из условий сшивки получаем два соотношения, для двух функций. Из четырёх соотношений получаем матрицу $A$.
\end{itemize}
Выберем решения (пусть нас не волнует, что они не нормируемые).

Ячейка 1:
$$
    \begin{cases}
        \psi_1(x) = \Cos kx,\\
        \psi_2(x) = \Sin kx
    \end{cases}
$$
Ячейка 2:
$$
    \begin{cases}
        \psi_1(x) = a_{11} \Cos k(x-a) + a_{21} \Sin k(x-a),\\
        \psi_2(x) = a_{12} \Cos k(x-a) + a_{22} \Sin k(x-a)
    \end{cases}
$$
Условия сшивки: конечный разрыв производной и непрерывность:
$$
     \begin{cases}
        \psi(a+0) = \psi(a-0),\\
        \psi'(a+0) - \psi'(a-0) = 2\kappa_0 \psi(a)
     \end{cases}
$$
Внимательный наблюдатель может заметить, что во втором соотношении не такой знак, как в предыдущих задачах на конечный разрыв. В предыдущих задачах мы рассматривали ямы, а здесь~--- пики.
$$
    \begin{cases}
        a_{11} = \Cos ka,\\
        a_{12} = \Sin ka
    \end{cases}
$$
Производим дифференцирование для второго соотношения:

Ячейка 1:
$$
    \begin{cases}
        \psi_1'(x) = -k \Sin kx,\\
        \psi_2'(x) = k \Cos kx,\\
    \end{cases}
$$
Ячейка 2:
$$
    \begin{cases}
        \psi_1'(x) = -k a_{11} \Sin k(x-a) + k a_{21} \Cos k(x-a),\\
        \psi_2'(x) = -k a_{12} \Sin k(x-a) + k a_{22} \Cos k(x-a)
    \end{cases}
$$
Сшивая условия, получаем:
$$
    \begin{cases}
    k a_{21} + k \Sin ka = 2 \kappa_0 \Cos ka,\\
    k a_{22} - k \Cos ka = 2 \kappa_0 \Sin ka,\\
    \end{cases}
$$
откуда
$$
    \begin{cases}
    a_{21} = -\Sin ka + \dfrac{2 \kappa_0}{k} \Cos ka,\\
    a_{22} = \Cos ka + \dfrac{2 \kappa_0}{k} \Sin ka,
    \end{cases}
$$
Находим след:
$$
    \underline{\rho = \dfrac12 (a_{11} + a_{22} ) = \Cos ka + \dfrac{\kappa_0}{k} \Sin ka}
$$
Кроме того, получаем дисперсионное соотношение:
$$
    \underline{\Cos Ka = \Cos ka + \dfrac{\kappa_0}{k} \Sin ka}, \quad k = \dfrac{1}{\hbar} \sqrt{2mE}
$$
\Picw{./pic/6/kronig_p.jpg}{0.3}
Если $K \in \R$, то соответствующие зоны энергии разрешённые. В противном случае частица в кристалле распространяться не может.

На следующем рисунке изображена \emph{дисперсионная кривая} $E(K)$:
\Picw{./pic/6/var_mit.png}{0.4}

1 зона Бриллюэна: $-\pi < ka < \pi$.

2 зона Бриллюэна: $-2\pi < ka < -\pi$ (левая), $\pi < ka < 2\pi$ (правая).

Заштрихованы соответственно 1 запрещённая зона, 2 запрещённая зона, и так далее.

\begin{enumerate}
  \item $E$ такова, что $\rho < 1$,
  $$
    |\Cos ka| < 1 \, \to K \in \R, \quad |\lam| = 1
  $$
  \item $E$ такова, что $\rho > 1$,
  $$
    |\Cos ka| > 1 \, \to K \in \C.
  $$
  Поскольку кристалл \emph{безграничный}, то такие состояния существовать не могут
  \item Граница запрещённой зоны: $|\rho| = 1$.
  $$
    |\Cos Ka| = 1, \quad Ka = \pi m
  $$
  Последнее называется условием полного внутреннего отражения (условие Брегга-Вульфа)
\end{enumerate}

\Rem Иногда рисуют \emph{приведённую дисперсионную кривую.}
\Picw{./pic/6/var_mit2.png}{0.5}

\textbf{Источники вдохновения:}
\begin{tabular}{l}
  \texttt{http://web.mit.edu/course/6/6.732/www/new\_part1.pdf} \\
  \texttt{http://sibsauktf.ru/courses/foet/Foet\_2.htm} \\
  \texttt{http://dic.academic.ru/dic.nsf/enc\_physics/3649/КРОНИГА}  \\
\end{tabular}


