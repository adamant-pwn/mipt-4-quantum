\chapter[Математический аппарат квантовой механики]{Математический аппарат квантовой механики и его физическая интерпретация}
\section{Пространство состояний}

Физическое состояние системы описывается в квантовой механике при помощи векторов состояний\index{Состояние}, то есть физическому состоянию системы ставится в соответствие некоторый объект линейного пространства. Состояния обозначаются угловыми скобками\index{Скобки Дирака} $\bra \ldots \ket$, внутри которых ставятся числа, которые называют \emph{квантовые числа}.

\emph{Примеры:}
\begin{itemize}
  \item $\bra \vec p \ket$~--- импульс
  \item $\bra \vec r \ket$~--- координата
  \item $\bra n \ell m \ket$~--- квантовые числа
  \item $\bra \phi \ket$
  \item $\bra \psi \ket$
\end{itemize}
Аксиомы линейного пространства:
\begin{enumerate}
%  \item $\bra \chi \ket = c_1 \bra \psi \ket  + c_2 \bra \phi \ket  $
  \item $\qu{\psi} + \qu{\phi} = \qu{\phi} + \qu{\psi}$
  \item $\big(\qu{\psi} + \qu{\phi}\big) + \qu{\chi} = \qu{\psi} + \big(\qu{\phi} + \qu{\chi}\big)$
  \item $c \big(\qu{\psi} + \qu{\phi}\big) = c \qu{\psi} + c \qu{\phi}$
  \item $(c_1 + c_2) \qu{\psi} = c_1 \qu{\psi} + c_2 \qu{\psi}$

  ***
  \item $\bra \psi \ket \to c \bra \psi \ket ,\, (c \neq 0)$
  \item $0 \cdot \qu{\psi} = \qu{0} = 0$. Считается, что нулевой вектор не описывает никакого состояния (в противном случае, по предыдущему свойству, он бы описывал любое состояние). С другой стороны, если вероятность найти объект равна нулю, то считается, что его нет.
  \item Пространство наделено \emph{скалярным произведением}:
  $$
    \qs{\phi}{\psi} \such
  $$
  \begin{enumerate}
    \item $\qs{\psi}{\phi}^\ast = \qs{\phi}{\psi}$
    \item $
        \left.
        \begin{array}{c}
          \qu{\tilde \phi} = \lam_1 \qu{\phi} \\
          \qu{\tilde \psi} = \lam_2 \qu{\psi}
        \end{array}
        \right\} \to
        \qs{\tilde \phi}{\tilde \psi} = \lam_1^\ast \lam_2 \qs{\phi}{\psi}
    $
  \end{enumerate}
  \item Пространство наделено нормой:
  $$
    \| \qu{\psi} \| = \sqrt{\qs{\psi}{\psi}}
  $$
  При этом $\| \qu{\psi} \| ^2 \geqslant 0$ и $\| \qu{\psi} \| ^2 = 0$, если $\qu{\psi} = 0$.
  \item В этом пространстве определён предельный переход по норме: любая фундаментальная последовательность сходится к элементу этого пространства.
\end{enumerate}
Пространство, которое характеризуется этими свойствами, будем называть \emph{гильбертовым}. \index{Пространство! гильбертово}

\textbf{Постулат:} в пространстве существует хотя бы один счётный базис (пространство сепарабельно).

***

\textbf{Пример.} Рассмотрим конечномерное пространство $\H$:
$$
    \qu{\psi} \to \vec \psi = \stolb{\psi_1}{\psi_2}
$$
Обозначим пространство линейных отображений (функционалов) на $\H$ через $\H^\ast$. Данное пространство называется сопряженным (дуальным). \index{Пространство! сопряженное} Эти пространства изоморфны, и изоморфизм будем обозначать <<плюсиком>>:
$$
    \uq{\psi} = \qu{\psi}^{+}
$$
$$
    \text{Если} \, \qu{\psi} \in \H, \, \text{тогда} \, \uq{\psi} \in \H^\ast
$$
При этом
$
    \forall \qu{\psi} \in \H \such \uq{\phi}
$~--- линейный функционал, такой, что на аргументе $\qu{\psi}$ он принимает значение $\qs{\phi}{\psi}$

***

Область определения функционала $L_2(\R^3)$.
$$
    \qs{\phi}{\psi} \eqdef \int \phi^\ast (\vec r) \psi(\vec r) \,d^3 x
$$

\section{Динамические переменные}
Каждой динамической переменной, относящейся к данной динамической системе, ставится в соответствие \emph{линейный оператор.}

\textbf{Символика:} Динамической переменной $A$ соответствует оператор $\Hat A$.\index{Оператор}

Линейный оператор определяет в пространстве $\H$ линейное отображение области
$$
    D_{\Hat A} \set \H \to R_{\Hat A} \set \H,
$$
то есть областью определения оператора может являться не всё пространство.

\subsection{Свойства операторов}
\begin{enumerate}
  \item Линейность.
  \item Умножение операторов: $\Hat A_1 \Hat A_2 \qu \psi \neq \Hat A_2 \Hat A_1 \qu \psi$.

  \Def \emph{Коммутатором} $\Hat A_1, \Hat A_2$ называется величина $[\Hat A_1, \Hat A_2]$\index{Коммутатор}
  $$
    [\Hat A_1, \Hat A_2] \eqdef \Hat A_1 \Hat A_2 - \Hat A_2 \Hat A_1.
  $$
  Если операторы коммутируют, то $[\Hat A_1, \Hat A_2] = 0$.
  \item Оператор, эрмитово сопряжённый данному.
\index{Оператор! эрмитово сопряженный}
  Пусть $\Hat A \qu \psi = \qu \chi$. Тогда
  $$
  \begin{array}{l}
    \qu \chi ^+ = \big(\Hat A \qu \psi\big)^+, \\
    \uq \chi \eqdef \uq \psi \Hat A^+
  \end{array}
  $$
  (Обратите внимание, что оператор $\Hat A^+$ действует на $\uq \psi$ справа налево. До этого такое непривычное обозначение ни в одном предмете не встречалось. Существование такого оператора $\Hat A^+$ в произвольном гильбертовом пространстве было строго доказано на последней лекции по функциональному анализу. \Alex)
  \item $\qs{\chi}{\phi} = \uq \psi \Hat A^+ \qu \phi$
  \item $\qs{\chi}{\phi} = {\qs{\phi}{\chi}}^\ast = \uq \phi \Hat A {\qu \psi}^\ast$, откуда
  $$
    \uq \psi \Hat A^+ \qu \phi = \uq \phi \Hat A \qu \psi ^\ast
  $$
  \item \Def Эрмитов (симметричный) оператор\index{Оператор! эрмитов (симметричный)}~--- это такой оператор $\Hat A$, что $\Hat A^+ = \Hat A$, то есть
  $$
    \uq \psi \Hat A \qu \phi = \uq \phi \Hat A \qu \psi ^\ast \quad \forall \qu \phi, \qu \psi \in D_{\Hat A}
  $$
  \Def Оператор является \emph{самосопряжённым}\index{Оператор! самосопряженный}, если области определения, к тому же, совпадают.
\end{enumerate}
