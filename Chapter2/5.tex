%point 2
\subsection{Основные формулы и операции в $\alpha$-представлении}
\begin{enumerate}
  \item Вектор в $\alpha$-представлении: $\C_i = \qs{\alpha_i}{\psi}$ или $\qs{x}{\psi} = \psi(x)$
  \item Оператор: $\Hat A \qu \psi = \qu \phi$
  $$
    \qtri {\alpha_i}{\Hat A}{\psi} = \qs{\alpha_i}{\phi}
  $$
  Вставляя единичный оператор $\Hat 1 = \sum_j \qu{\alpha_j}\uq{\alpha_j}$, получаем:
  $$
        \qtri {\alpha_i}{\Hat A}{\psi} = \sum_j \underbrace{\qtri{\alpha_i}{\Hat A}{\alpha_j}}_{A_{ij}} \qs{\alpha_j}{\psi}
  $$
  $A_{ij}$~--- матричные элементы оператора $\Hat A$. Также называется $\Hat A$ в $\alpha$-представлении.\index{Оператор! в $\alpha$-представлении}
  В непрерывном случае это ядро интегрального оператора\index{Оператор! интегральный! ядро}.
  \item $\Hat \alpha \qu{\alpha_i} = \alpha_i \qu{\alpha_i}$
  $$
    \qtri{\alpha_i}{\Hat \alpha}{\alpha_j} = \alpha_i \dta_{ij}
  $$
  \item Все действия с операторами переходят в действия над матрицами.
\end{enumerate}
%point 3
\subsection{Смена представления. Переход от одного представления к другому}
\begin{itemize}
\item
$$
    \big\{\qu{\alpha_n}\big\} \to \big\{\qu{\beta_n}\big\}
$$

Пусть вектор в $\beta$-представлении имеет вид:
$$
    \qs{\beta_m}{\psi}
$$
Тогда, вставляя единичный оператор, получаем:
$$
     = \sum_n \underbrace{\qs{\beta_m}{\alpha_n}}_{\text{<<матрица>> перехода}} \qs{\alpha_n}{\psi}
     = \sum_n {\qs{\alpha_n}{\beta_m}}^\ast \qs{\alpha_n}{\psi}
     = \sum_n \psi_{\beta_m}^\ast(\alpha_n) \psi(\alpha_n)
$$
Рассмотрим оператор $F$ в $\beta$-представлении:
\item
$$
    F_{\beta \beta'} = \uq{\beta} \Hat 1 \Hat F \Hat 1 \qu{\beta'} =
$$
$$
    = \sum_{\alpha \alpha'} \qs{\beta}{\alpha} \qtri{\alpha}{\Hat F}{\alpha'} \qs{\alpha'}{\beta'}
$$
$$
    = \sum_{\alpha \alpha'} {\qs{\alpha}{\beta}}^\ast \qtri{\alpha}{\Hat F}{\alpha'}\qs{\alpha'}{\beta'} =
$$
$$
    = \sum_{\alpha \alpha'}\psi_\beta^\ast (\alpha) F_{\alpha \alpha'} \psi_{\beta'} (\alpha')
$$
Таким образом,
$$
    \psi_\beta(\alpha) = \qs{\alpha}{\beta}
$$
\Example \textbf{Переход от координатного представления к импульсному представлению}
Пусть вектор $\psi$ имеет $p$-представление,\index{Представление! импульсное}
$$
    \qs p \psi = \phi(p)
$$
Вставляем в скалярное произведение единичный оператор в виде
$$
     \Hat 1 = \int\limits_{-\infty}^{+\infty}\qu{x} \uq{x} dx
$$
Получаем:
$$
    \phi(p) = \int\limits_{-\infty}^{+\infty}dx \qs px \qs x \psi =
    \int\limits_{-\infty}^{+\infty}dx \underbrace{{\qs xp}^\ast}_{\psi_p^\ast (x)} \qs x \psi
    =
$$
$$
    = \dfrac{1}{\sqrt{2 \pi \hbar}} \int\limits_{-\infty}^{+\infty} e^{\frac{-ipx}{\hbar}} \psi(x) dx
$$\index{Представление! координатное}
Резюме: для того, чтобы осуществить переход, необходимо знать матрицу перехода $\qs{\beta}{\alpha}$. Такая матрица может быть и квадратной (только в том случае, если есть взаимно однозначное соответствие между элементами базиса, если они связаны линейным оператором). Такого может  и не быть, например, если осуществляется переход от дискретного базиса к непрерывному.

\item \textbf{Общие свойства матрицы перехода}
\begin{enumerate}
  \item

$$
    \qs{\beta_m}{\psi} = \sum_n \underbrace{\qs{\beta_m}{\alpha_n}}_{S_{mn} \leftarrow S} \qs{\alpha_n}{\psi}
$$
$$
    \qs{\alpha_n}{\psi} = \sum_m \underbrace{\qs{\alpha_n}{\beta_m}}_{T_{nm}} \qs{\beta_m}{\psi}
$$

\textbf{Утверждение 1:} $T = S^{+}$
  \item
$$
    \qs{\alpha_n}{\alpha_k} = \sum_m \qs{\alpha_n}{\beta_m} \qs{\beta_m}{\alpha_k} = \dta_{nk}
$$
$$
    S^+ S = 1
$$
Аналогично, $T^+ T = 1$

Объединяя эти утверждения, получаем \textbf{утверждение 2:}
$$
    S^+S = 1; \qquad SS^+ = 1
$$
\Rem Эти <<единички>> могут иметь разную размерность. Если же между элементами базиса имеется взаимно однозначное соответствие, то
$$
    \qu{\beta_m} = \Hat U^+ \qu{\alpha_m}
$$
$$
    \qs{\beta_m}{\alpha_n} = \qtri{\alpha_m}{\Hat U}{\alpha_n}
$$
Условия, которые накладываются на $\Hat U^+$, имеют вид:
$$
    \Hat U^+ \Hat U = \Hat U \hat U^+ = \Hat 1, \qquad \Hat U^+ = \Hat U^{-1}
$$
\Def Такие операторы называются \emph{унитарными}.\index{Оператор! унитарный}
\end{enumerate}
\end{itemize}

\subsection{Свойства унитарного оператора}
\begin{enumerate}
  \item \textbf{Уравнения на собственные значения и собственные векторы}
  $$\left\{
  \begin{array}{l}
    \Hat U \qu f = \lam \qu f \\
    \uq f \Hat U^+ = \lam^\ast \uq f
  \end{array} \right.
    \to
  \uq f \Hat U^+ \Hat U \qu f = \lam^\ast \lam \qs ff
  $$
  $$
    |\lam|^2 = 1 \quad \to \quad \lam = e^{\pm i \alpha}, \quad \mathrm{Im} \, \alpha = 0
  $$
  \item Если $\Hat U_1$, $\Hat U_2$~--- унитарные операторы, то оператор
  $$
    \Hat U_3 = \Hat U_1 \Hat U_2
  $$
  тоже является унитарным оператором
  \item \textbf{Унитарные преобразования не изменяют собственных значений наблюдаемых.}

  Пусть есть уравнение
  $$
    \Hat A \qu A = A \qu A
  $$
  Записываем выражение
  $$
    \Hat U \Hat A \Hat U^+ \Hat U \qu A = A \Hat U \qu A
  $$
  Обозначим преобразованный вектор $\Hat U \qu A = \qu{A'}$.
  Исходя из $\Hat U \Hat A \Hat U^+ = \Hat A'$, можно написать:
  $$
    \Hat A' \qu{A'} = A \qu{A'}
  $$
  \item \textbf{Унитарные преобразования не меняют эрмитовости}
  $$
    \Hat A^+ = \Hat A \quad \to \Hat {A'}^+ = \Hat A'
  $$
  \item \textbf{Унитарные преобразования не изменяют коммутационные соотношения}

  Если
  $$
    \Hat A, \Hat B \quad \to \quad [\Hat A, \Hat B] = i \Hat C,
  $$
  то
  $$
    [\Hat A', \Hat B'] = i \Hat C'
  $$
  \item Значения матричных элементов и средние значения наблюдаемых не изменяются при унитарных преобразованиях.
      $$
        \qtri{A}{\underbrace{\Hat F}_{\Hat 1 \Hat F \Hat 1}}{B} =
        \underbrace{\uq{A} \Hat U^+} \underbrace{\Hat U \Hat F \Hat U^+} \underbrace{\Hat U \qu{B}}
      $$
\end{enumerate}
\textbf{Вывод:} ни одно из наблюдаемых квантовой механикой чисел не меняется при унитарных преобразованиях. Поэтому с точки зрения наблюдаемых на эксперименте величин, физическое содержание теории остаётся неизменным.

Кроме того, окажется, что это всё связано со скобками Пуассона. В классической механике унитарные преобразования принято называть \emph{каноническими}.

\section{Квантовые скобки Пуассона}

Как известно, одна из особенностей квантовой механики заключается в том, что физическим величинам сопоставляются операторы. В общем случае они могут не коммутировать друг с другом:
$$
\left\{
\begin{array}{c}
  A \to \Hat A \\
  B \to \Hat B
\end{array}
\to [\Hat A, \Hat B] \neq 0
\right.
$$
\Quest{\underline{Как} физическим величинам сопоставить операторы?}\index{Коммутатор}

Пусть классические величины $q_i, p_i$~--- обобщённые координаты и импульс, $q_i = \{x, y, z\}$.

План:
\begin{enumerate}
  \item Разработать закон, по которому $q_i$ сопоставляется оператор $\Hat q_i$, $p_i \to \Hat p_i$, и чтобы между ними воспроизводились правильные коммутационные соотношения
  \item Физические величины не исчерпываются координатами и импульсами, можно строить сложные функции от этих величин: $f(\Hat p)$, $f(\Hat q)$.
\end{enumerate}
Начнём со второго вопроса.
\subsection{Функции наблюдаемых}
\index{Функция от оператора}
$$
    \Hat \alpha \to f(\Hat \alpha) = ?
$$
\Def Пусть имеется некоторая числовая функция $f(\xi)$, что на всей вещественной прямой справедливо разложение в ряд Тейлора:
$$
    f(\xi) = \sum_{n=0}^\infty \dfrac{f^{(n)} (0)}{n!} \xi^n
$$
На основании этого, сопоставим функции от оператора $\alpha$ степенной ряд с теми же коэффициентами:
$$
    f(\Hat \alpha) = \sum_{n=0}^{\infty} \dfrac{f^{(n)} (0)}{n!} {\Hat \alpha}^n
$$
\Rem У этого определения есть недостаток: не все функции являются аналитическими. В связи с этим дадим более общее определение. Как известно, оператор $\alpha$ определяет базис состояний:
$$
    \Hat \alpha \qu{\alpha_i} = \alpha_i \qu{\alpha_i}
$$
Достоверно известно, что значение динамической переменной $\alpha$ равно $\alpha_i$ с вероятностью $1$.
Для того, чтобы определить действие оператора $f(\alpha)$ в полном пространстве состояний, достаточно определить его действие на векторы базиса.

\Def  Пусть оператор $f(\Hat \alpha)$ действует на элементы базиса следующим образом:
$$
    f(\Hat \alpha) \qu{\alpha_i} = f(\alpha_i) \qu{\alpha_i}
$$
Тогда действие на произвольный вектор $\qu \psi \in \H$ определяется следующим образом:
$$
    f(\Hat \alpha) \underbrace{\qu \psi}_{\Hat 1 \cdot \qu \psi} =
    \sum_i f(\Hat \alpha) \qu{\alpha_i} \qs{\alpha_i}{\psi}
    = \sum_i f(\alpha_i) \qu{\alpha_i}\qs{\alpha_i}{\psi}
$$
$$
    f(\Hat \alpha) \eqdef \sum_i f(\alpha_i) \qu{\alpha_i} \uq{\alpha_i}
$$
Это ещё называется спектральное разложение оператора $f(\Hat \alpha)$.\index{Спектральное разложение оператора} Два определения эквивалентны, если функция разложима в ряд Тейлора.
\subsection{Классические скобки Пуассона}

Надо ввести коммутационные соотношения. Для этой цели мы будем использовать принцип соответствия, и будем строить наше квантово-механическое описание так, чтобы оно примыкало к классическому настолько тесно, насколько это возможно.
$$
    \pois uv \eqdef \sum_{i=1}^\ell \Big( \ud {u} {q_i} \ud{v}{p_i} - \ud {u} {p_i} \ud{v}{q_i} \Big)
$$\index{Скобки Пуассона! классические}
\textbf{Свойства:}
\begin{enumerate}
  \item $\pois {u_1 + u_2}{v} = \pois{u_1}{v} + \pois{u_2}{v}$
  \item $\pois uv = - \pois vu$
  \item $\pois uc = 0, \quad C = \mathrm{const}$
  \item \textbf{Дифференцирование:}
  $$
    \pois{u}{v_1 v_2} = \pois{u}{v_1}v_2 + \pois{u}{v_2}v_1
  $$
  \item \textbf{Тождество Якоби:}
  $$
    \Pois{u}{\pois vw} + \Pois {w}{\pois uv} + \Pois{v}{\pois wu} = 0
  $$\index{Тождество Якоби}
\end{enumerate}
\subsection{Квантовые скобки Пуассона}
Произведём сопоставление:
$$
    \pois uv \to {\pois uv}_{qu}
$$\index{Скобки Пуассона! квантовые}
Пусть квантовые скобки Пуассона удовлетворяют пяти свойствам, плюс возможная некоммутативность операторов $u, v, \pois uv$.\footnote{Имеется в виду, скажем, что $\pois{u}{v_1} v_2 \neq v_2\pois{u}{v_1}$, и в свойстве 4 надо сохранять порядок следования сомножителей:}
$$
    \qpois{u}{v_1v_2} = v_1 \qpois{u}{v_2} + \qpois{u}{v_1} v_2
$$
$$
    \qpois{u_1 u_2}{v} = u_1 \qpois{u_2}{v} + \qpois{u_1}{v} u_2
$$
$$
    \qpois{u_1 u_2}{v_1 v_2} = v_1 \qpois{u_1 u_2}{v_1} + \qpois{u_1 u_2}{v_1} v_2=
$$
$$
    v_1 u_1 \qpois{u_2}{v_2} + \underline{v_1 \qpois{u_1}{v_2} u_2} + \underline{u_1 \qpois{u_2}{v_1} v_2} + \qpois{u_1}{v_1} u_2 v_2
$$
Раскроем скобки другим способом:
$$
    \qpois{u_1 u_2}{v_1 v_2} =
    u_1 v_1 \qpois{u_2}{v_2} + \underline{u_1 \qpois{u_2}{v_1} v_2} + \underline{v_1 \qpois{u_1}{v_2} u_2} + \qpois{u_1}{v_1} v_2 u_2
$$
После приведения подобных, получаем условие, которое нужно наложить на скобку, чтобы операторы совпадали:
$$
    (u_1 v_1 - v_1 u_1) \qpois{u_2}{v_2} = \qpois{u_1}{v_1} (u_2 v_2 - v_2 u_2)
$$
Производя <<формальное деление>>:
$$
    \dfrac{\pois{u_1}{v_1}}{(\comm{u_1}{v_1})} = \dfrac{\pois{u_2}{v_2}}{(\comm{u_2}{v_2})} = \mathrm{const}.
$$
Получаем:
$$
    \qpois uv = \alpha (\comm uv) = \alpha [u, v]
$$\index{Скобки Пуассона! квантовые}
Это означает, что $\alpha$~--- это какая-то фундаментальная константа. Ответ на её значение даёт эксперимент. В последующих лекция будет показано, что $\alpha = -\dfrac{i}{\hbar}$.

\Def
$$
    \qpois uv \eqdef -\dfrac i \hbar [u, v]
$$
\subsection{Фундаментальная скобка Пуассона}
Речь идёт о координатах и импульсах.

В классической механике
$$
    [p_i, q_j] = [p_i, p_j] = [q_i, q_j] = 0
$$
Но классические скобки Пуассона отличны от нуля:
$$
  \left\{
    \begin{array}{l}
      \pois{q_i}{p_j} = \dta_{ij} \\
      \pois{q_i}{q_j} = \pois{p_i}{p_j} = 0
    \end{array}
  \right.
  \to
$$
\textbf{Предположение:}
$$
  \left\{
    \begin{array}{l}
      \qpois{q_i}{p_j} = \dta_{ij} \\
      \qpois{q_i}{q_j} = \qpois{p_i}{p_j} = 0
    \end{array}
  \right.
$$
Фундаментальные коммутационные соотношения (условия квантования):
\index{Скобки Пуассона! фундаментальные}\index{Фундаментальные коммутационные соотношения}
$$
\left\{
\begin{array}{l}
     [q_i, p_j] = i \hbar \dta_{ij} \\[0pt]
     [p_i, p_j] = [q_i, q_j] = 0
\end{array}
\right.          
$$
