%\Reminder

% \S 9 квантовая скобка Пуассона
В классической механике принято называть $q_i, p_i$ \emph{дополнительными}\index{Величины! дополнительные} в смысле соотношения неопределённостей.
Дополнительные не коммутируют, из них нельзя составить полный набор.

\section{Координатное представление в квантовой механике}
Координатное представление ещё иногда называют $X$-представлением.\index{Представление! координатное}
\subsection{Базис}
Базис в пространстве $\H$~--- это обобщённые собственные векторы оператора координат:
$$
    \Hat x \qu{x'} = x' \qu{x'}
$$
\Rem Декартова система имеет преобладающее значение, хотя иногда встречается и сферическая система координат и другие.

Когда речь идёт об абстрактном гильбертовом пространстве, то элементы $p_i, q_j$ являются элементами каких-то абстрактных алгебр с операцией $[p_i, q_j]$, $[p_i, p_j]$, $[q_i, q_j]$. Представление канонических коммутационных соотношений: переход к эрмитовым операторам.\index{Оператор! эрмитов (симметричный)}

Надо предположить, что у оператора $\Hat x$ есть хотя бы один собственный вектор, отвечающий вещественному собственному значению.

\Th Оператор $\Hat x$ сам по себе образует полный набор\index{Полный набор}\footnote{Полный набор взаимно коммутирующих наблюдаемых. Это означает, что не требуется вводить дополнительных коммутирующих операторов.}.

\Rem Если
$$
    \exists \Hat F \such [\Hat F, \hat x] = 0,
$$
то
$$
    \Hat F = F(\Hat x)
$$
\subsection{Спектр оператора $\Hat x$}
Дальнейшие утверждения будут доказаны на семинарах.

Спектр оператора $\Hat x$ непрерывен и заполняет всю вещественную ось.

\Proof Введём оператор $\Hat T_a \eqdef \exp\left(\frac{i}{\hbar} a \Hat p\right),$ $\mathrm{Im} \, a = 0$

Можно заметить, что
$$
    \Hat T_a^+ = \Hat T_a^{-1} = \Hat T_{-a} \text{ -- оператор трансляции на $a$}
$$\index{Оператор! трансляции}
Для дальнейших рассуждений понадобится коммутатор
$$
    [\Hat T_a, \Hat x] = a \Hat T_a \LRA \Hat x \Hat T_a = \Hat T_a (\Hat x - a\cdot \Hat 1)
$$
Подействуем на $\qu{x'}$ слева двумя операторами:
$$
    \Hat x\Hat T_a \qu{x'} = \Hat T_a (\Hat x - a) \qu{x'} = (x' - a) \Hat T_a \qu{x'}
$$
Получили новый собственный вектор (ненулевой в силу существования обратного оператора), соответствующий собственному значению $(x'-a)$.

Норма вектора:
$$
    \Hat T_a^+ \cdot \Hat T_a =  \Hat 1
$$
Таким образом, $\big\| \Hat T_a \qu{x'} \big\| = \big\| \qu{x'} \big\|$

Итак, можно получить любой его собственный вектор, соответствующий любому вещественному собственному значению.
Отсюда следует непрерывность спектра вещественных собственных значений.

\Rem Из сказанного выше следует неограниченность оператора $\Hat x$

\subsection{Эрмитовость оператора $\Hat x$}
\begin{eqnarray*}
    \Hat x \qu{x'} &=& x' \qu{x'}\\
    \uq{x'} \Hat x &=& \uq{x'} x'\\
    \Hat x^+       &=& \Hat x
\end{eqnarray*}
Полное доказательство ждёт в задании.

\subsection{Ортонормированность и полнота}
$$
    \qs{x}{x'} = \dta{(x-x')}
$$
$$
    \int_{-\infty}^{+\infty} \qu{x} \uq{x} dx = \Hat 1
$$
\subsection{Волновая функция}
$$
    \qu{\psi} = \int_{-\infty}^{+\infty} \qu{x} \underline{\qs x\psi} dx
$$
Координатами этого вектора является совокупность чисел $\qs{x}{\psi}$, и континуум компонент вектора $\psi$ в базисе собственных векторов оператора $\Hat x$ образует функцию
$$
    \psi(x) = \qs{x}{\psi}
$$
Иногда она называется \emph{матричной реализацией}
\subsection{Скалярное произведение}
Перейдём к координатному представлению скалярного произведения $\qs \psi \phi$.
$$
    \qs \psi \phi = \qtri{\psi}{\Hat 1}{\phi}
$$
\begin{eqnarray*}
    \qs \psi \phi &=& \int_{-\infty}^{+\infty} \qs{\psi}{x} \qs{x}{\phi} dx\\
    &=& \int_{-\infty}^{+\infty} {\qs{x}{\psi}}^\ast \qs x \phi dx\\
    &=& \int_{-\infty}^{+\infty} \psi^\ast (x) \phi(x) dx
\end{eqnarray*}
Аналогично,
\begin{eqnarray*}
    & \big\| \qu \psi \big\|^2\\
    &= \qs{\psi}{\psi}\\
    &= \int_{-\infty}^{+\infty}  \big|\psi(x)\big|^2 dx
\end{eqnarray*}

Таким образом, пространство $\H$ реализовано в виде пространства $L_2 (\R)$
\subsection{Собственные функции оператора $\Hat x$ в координатном представлении}
Собственные вектор $\qu{x'}$, по определению, имеет вид
$$
    \Hat x \qu{x'} = x' \qu{x'}
$$
$$
    \mybox{\psi_{x'} (x) = \qs{x}{x'} = \dta(x - x')}
$$
\subsection{Оператор координаты в $x$-представлении}
\index{Оператор! координаты}
Для произвольного вектора $\qu{\psi}$
$$
    \Hat x \qu \psi = \qu \phi
$$
Проектируем это равенство на $x$-базис, то есть скалярно умножаем на $\uq{x}$:
$$
    \qtri{x}{\Hat x \Hat 1}{\psi} = \qs{x}{\phi} = \phi(x)
$$
\begin{eqnarray*}
    \phi(x)&=&\int_{-\infty}^{+\infty} \qtri{x}{\Hat x}{x'}\qs{x'}{\psi} dx'\\
    &=&\int_{-\infty}^{+\infty} \underbrace{x' \dta(x - x')}_{\text{ядро интегрального оператора}} \psi(x') dx'\\
    &=&x\psi(x)
\end{eqnarray*}
При переходе к координатному базису действие сводится к умножению на вещественное число $x$, то есть
$$
    \phi(x) = x \psi(x)
$$
\Rem Запись вида $\Hat x \psi(x) = x \psi(x)$ математически некорректна, так как объекты слева и справа от равенства записаны в разных пространствах.
\subsection{$U(\Hat x)$ в $x$-представлении}
\index{Оператор! $U (\hat{x})$}
$$
    U(\Hat x) = \int\limits_{-\infty}^{+\infty} U(x') \qu{x'} \uq{x'} dx'
$$
$$
    U(\Hat x) \qu{\psi} = \qu{\phi}
$$
\Excse
$$
    U(x) \psi(x) = \phi(x)
$$
\textbf{Правила работы с $\dta$-функциями.}
\begin{eqnarray}
    \int\limits_{-\infty}^{+\infty} f(x) \dta(x - x_0) dx &=& f(x_0)\\
    \dfrac{1}{2\pi} \int\limits_{-\infty}^{+\infty} e^{ikx} dk &=& \dta(x) \quad \text{(фурье)}\\
    x \dta (x) = 0, \quad x \dta'(x) &=& -\dta(x)
\end{eqnarray}
\subsection{Оператор $\Hat p_x$ в $x$-представлении}
\index{Оператор! импульса}
$$
    \int\limits_{-\infty}^{+\infty} \qtri{x}{\Hat p_x}{x'} \qs{x'}{\psi} dx' = \qs{x} \phi
$$
При этом
$$
    \comm{\Hat x}{\Hat p_x} = i \hbar \Hat 1
$$
$$
    \qtri{x}{(\comm{\Hat x}{\Hat p_x})}{x'} = i\hbar \qs{x}{x'}
$$
Подействовав первым оператором $\Hat x$ налево, вторым оператором $\Hat x$ направо, получаем:
$$
    (x - x') \qtri{x}{\Hat p_x}{x'} = i\hbar \dta(x - x')
$$
Получили функциональное уравнение на функцию $f(x, x') = \qtri{x}{\Hat p_x}{x'}$ (для класса обобщённых функций):
$$
    \underline{(x - x') \qtri{x}{\Hat p_x}{x'} = i\hbar \dta(x - x')}
$$
Пользуясь правилом (3), получаем:
$$
    \underline{\qtri{x}{\Hat p_x}{x'} = i\hbar {\dta}'(x' - x)}
$$
Теперь можно получить \emph{матричную реализацию} этого оператора:
\begin{eqnarray*}
    \phi(x) &=& \int\limits_{-\infty}^{+\infty} i \hbar \dta'(x' - x) \psi(x') dx'\\
    &=& -i\hbar \int\limits_{-\infty}^{+\infty} \dta(x' - x) \ud{}{x'} \psi(x') dx'\\
    &=& -i \hbar \ud{}{x} \psi(x)
\end{eqnarray*}
\Rem Функция $\phi(x)$ задана неоднозначно:
$$
    \qtri{x}{\Hat p_x}{x'} = i \hbar \dta'(x' - x)
$$
$$
    \phi(x) = -i \hbar \ud{}{x} \psi(x) + f(x) \psi(x)
$$
\textbf{Объяснение:}

Этот вид решений даёт унитарно эквивалентное представление коммутационных соотношений. Как известно, унитарные преобразования не изменяют наблюдаемых величин, физическое содержание теории остаётся неизменным. Поэтому среди всех таких представлений можно выбрать наиболее простое ($f(x) = 0$)

\textbf{Резюме:}

\begin{enumerate}
  \item От произвольного гильбертова пространства $\H$ мы перешли к конкретному пространству $L_2 (\R)$. Тогда операторам соответствуют их конкретные реализации:
      \begin{itemize}
        \item $\Hat x \to x$
        \item $\Hat p_x \to -i \hbar \ud{}{x}$
        \item $\Hat {\vec p} \to -i\hbar \vec \nabla$
        \item $\Hat U(x) \to U(x)$
      \end{itemize}
  \item Основные динамические переменные~$p, x$. Остальные величины (в классике) выражаются через них.
  
  \Quest{Есть ли другие такие операторы в квантовой механике?}
  
  \Ans Операторы $\Hat x$, $\Hat p$ образуют неприводимый набор (в $\H$ нет нетривиального замкнутого инвариантного по отношению к этому набору подпространства).\wikiq{и что?}
  
  Пример: дифференциальный оператор:
  $$
    \qtri{x}{\Hat F}{x'} = \Hat \F \dta(x - x')
  $$\index{Оператор! дифференциальный}
  
 \begin{eqnarray*}
    \qtri{\psi}{\Hat F}{\phi} &=& \int\limits_{-\infty}^{+\infty} dx\, dx' \, \qs{\psi}{x}
    \qtri{x}{\Hat F}{x'} \qs{x'}{\phi} dx'\\
    &=& \int\limits_{-\infty}^{+\infty} dx dx' \psi^\ast(x) \Hat \F \dta (x - x') \phi(x') \\
    &=& \int\limits_{-\infty}^{+\infty} dx \psi^\ast (x) \Hat \F \phi(x)
 \end{eqnarray*}
 \item Соотношение неопределённостей.\index{Соотношение неопределенностей}
 $$
    \left[ -i\hbar \ud{}{x}, \, x \right] = \ldots = -i\hbar \Hat 1
 $$
 Можно поступить наоброт: ввести операторы, и из их конкретного вида получить коммутационные соотношения.
 \item Уравнение Шрёдингера в координатном представлении:\index{Уравнение Шредингера! в координатном представлении}
 $$
    i\hbar \ud{}{t} \qu{\psi(t)} = \Hat H \qu{\psi(t)}, \quad \Hat H = \dfrac{\Hat{p}_x^2}{2m} + \Hat U(x)
 $$
 $$
    \underline{
        i\hbar \ud{}{t} \Psi(x, t) = \left( - \dfrac{\hbar^2}{2m} \left(\ud{}{x}\right)^2 + U(x) \right) \Psi(x, t)
    }
 $$
\end{enumerate}
\subsection{Собственные функции оператора $\Hat{p}_x$ в $x$-представлении.}
%
$$
    \Hat {p}_x \qu{p'}  = p' \qu{p'} \quad \to \quad \qs{x}{p} = \psi_p (x)
$$
\Rem индекс $p$ означает, что это собственная фукнкция оператора импульса

Из условия нормировки:
$$
    -i\hbar \ud{}{x} \psi_p (x) = p \psi_p(x),
$$
$$
    \underline{\psi_p(x) = C \exp\left( \dfrac{ipx}{\hbar}\right)}, \quad |\psi_p ( \pm \infty)|< \infty
$$
\Proof
$$
    \underbrace{\qs{p}{p'}}_{\qtri{p}{\Hat 1}{p'}} = \dta(p - p') 
$$
\begin{eqnarray*}
    &\int\limits_{-\infty}^{+\infty} dx \qs px \qs x {p'}\\
    &= \int\limits_{-\infty}^{+\infty} \psi^\ast_p(x) \psi_{p'} (x) dx\\
    &= C^2 \int\limits_{-\infty}^{+\infty} \exp\left( \dfrac{-i \hbar (p-p')}{\hbar} \right) dx\\
    &= C^2 2\pi \hbar \dta(p-p')
    &= \dta(p-p')
\end{eqnarray*}
\section{Импульсное представление в квантовой механике}
По-другому импульсное представление иногда называется $p$-представлением.\index{Представление! импульсное}

Будут описаны только основные моменты, потому что идея похожа на $x$-представление.

\subsection{Базис}
$$
    \Hat p \qu{p'} = p' \qu{p'}
$$
\subsection{Спектр}
Спектр является непрерывным и заполняет всю вещественную ось,
$$
    -\infty < p < +\infty
$$
\subsection{Ортонормированность и полнота системы собственных векторов}
$$
    \qs p{p'} = \dta(p - p')
$$
$$
    \int\limits_{-\infty}^{+\infty} \qu p \uq p dp = \Hat 1
$$
\subsection{Волновая функция}
$$
    \qu \psi = \int\limits_{-\infty}^{+\infty} \qu p \qs p \psi  dp \to
$$
$$
    \Phi(p) = \qs p \psi, \qquad |\Phi(p)|^2 dp = W(p) dp
$$
\subsection{Конкретные реализации}
\begin{eqnarray*}
    \Hat p \to p,\\
    \Hat x \to i\hbar \ud{}{p}
\end{eqnarray*}
Чтобы проследить перемену знака, надо подставить $\Hat p$, $\Hat x$ в их коммутатор.

$$
    \Hat H = \frac{{\Hat p}^2}{2m} + U({\Hat {\vec{r}}}) \rightarrow 
        \frac{p^2}{2m} + U(i{\hbar}\frac{\partial}{\partial p})
$$
$$
    i{\hbar}\frac{\partial \qu \psi}{\partial t} = {\Hat H} \qu \psi
$$
$$
    \int ih \frac{\partial}{\partial t} {\qu p} \underbrace{\qs p \psi}_{\Phi(p)} dp =
        \int {\Hat H}{\qu p} \underbrace{\qs p \psi}_{\Phi(p)} dp
$$
$$
    {\Hat r} {\qu \psi} = {\qu \phi}
$$
$$
    {\qs p \phi} = \int \underbrace{{\uq p} \vec{r} {\qu p'}}_{i{\hbar}{\delta}'(p-p')\Phi(p)} 
        {\qs p' \psi} dp'
$$