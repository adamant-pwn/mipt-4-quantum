%Really 2.3 (?)
\subsection{Проблема собственных векторов и значений оператора}

\begin{itemize}
  \item \Th Если оператор эрмитов\index{Оператор! эрмитов (симметричный)}, то собственные значения\index{Собственное значение} соответствующей ему динамической переменной вещественны.

\Proof
$\uq{\psi} \Hat A \qu{\psi} = \uq{\psi} \Hat A \qu{\psi}^\ast = \lam^\ast \qs{\psi}{\psi}^\ast = \lam^\ast \qs{\psi}{\psi}$

Таким образом, $\lam = \lam^\ast$.

  \item Проблема собственных значений линейного эрмитова оператора.

  Будем считать, что спектр оператора $\Hat A$ дискретный\index{Спектр! дискретный}, то есть
  $$
    \Hat A \qu{\psi_n} = \lam_n \qu{\psi_n}.
  $$
  Может оказаться так, что каждому собственному значению $\lam_k$ отвечает ровно один собственный вектор. В этом случае $\lam_k$ называется \emph{простым.}\index{Собственное значение! простое}

  В случае, если собственному значению отвечает сразу несколько собственных векторов, например
  $$
    \lam_n \to \{ \qu{\psi_n^{(1)}}, \ldots, \qu{\psi_n^{(\ell)}} \}
  $$
  Тогда собственное значение называется вырожденным, число $\ell$ называется \emph{кратностью вырождения}.\index{Собственное значение! вырожденное}\index{Кратность вырождения}

  \Def Максимальное количество линейно независимых собственных векторов, отвечающих данному собственному значению, называется \emph{кратностью вырождения}.  Она может изменяться от 2 до бесконечности (включительно).

    \item
  \Reminder Собственные векторы эрмитова оператора, отвечающие различным собственным числам, являются ортогональными.

  \Proof
    Пусть
    $$
    \left\{
        \begin{array}{l}
                  \Hat A \qu{\psi_n} = \lam_n \qu{\psi_n}, \\
                  \Hat A \qu{\psi_m} = \lam_m \qu{\psi_m}
        \end{array}
    \right.
    $$
    Из первого уравнения:
    $$
        \uq{\psi_n} \Hat A^+ = \lam_n \uq{\psi_n}
    $$
    <<Умножим справа>> на $\qu{\psi_m}$:
    $$
        \uq{\psi_n} \Hat A^+ \qu{\psi_m} = \lam_n \qs{\psi_n}{\psi_m}
    $$
    Таким образом,
    $$
        \uq{\psi_n} \Hat A^+ \qu{\psi_m} = \uq{\psi_m} \Hat A \qu{\psi_n}^\ast
         \underbrace{=}_{\text{т.к. $\Hat A = \Hat A^+$}} \uq{\psi_n}\underbrace{\Hat A \qu{\psi_m}} = \lam_m \qs{\psi_n}{\psi_m}
    $$
    Получили:
    $$
        (\lam_n - \lam_m) \qs{\psi_n}{\psi_m} = 0
    $$
    Если $\lam_n \neq \lam_m$, то $\qs{\psi_m}{\psi_m} = 0$, собственные векторы эрмитова оператора ортогональны.

    Договоримся о следующем обозначении:
    $$
        \Hat A \qu{A_n} = A_n \qu{A_n}, \qquad \qs{A_n}{A_m} = \dta_{nm}
    $$
Второе выражение -- условие нормировки.
\index{Условие нормировки}
    \item Пусть есть вырождение:
  $$
    \lam_n \to \{ \qu{\psi_n^{(1)}}, \ldots, \qu{\psi_n^{(\ell)}} \}
  $$
  При этом довольно очевидно, что если взять их линейную комбинацию
  $$
    \qu{\psi_n^{(s)}} = \sum_{i=1}^\ell C_i^{(s)} \qu{\psi_n^{(i)}},
  $$
  то она тоже будет являться собственным вектором. Говорят, что собственному значению отвечает собственное пространство размерности $\ell$.

  Коэффициенты $C_i^{(s)}$ удобно выбирать так, чтобы векторы были ортонормированными.

  \item \textbf{Ядерная спектральная теорема}\footnote{Для простых конечномерных систем это утверждение мы доказывали во втором семестре линейной алгебры. Для бесконечномерных систем, да ещё и с неограниченными операторами всё гораздо сложнее. Поэтому в курсе теорема приводится без доказательства.\Lector}: Система собственных векторов линейного эрмитова оператора, а точнее говоря самосопряжённого, является \emph{полной}.\index{Теорема! ядерная спектральная}

  \Def Система называется \emph{полной}, если она образует базис\index{Полная система}\footnote{На лекции разгорелись холивары, на тему того, что можно называть базисом, а что просто полной системой. Написанное определение действительно задаёт базис, но полная система в функциональном анализе определяется по-другому. В квантовой механике оно без надобности. \Sergey} в пространстве состояний, то есть
    $$
\forall \qu{\psi} \in \H \quad \exists ! \, C_i \such \qu{\psi} = \sum_n C_n \qu{A_n}
    $$
    Чтобы определить коэффициенты разложения, скалярно домножим равенство на любой из базисных векторов:
    $$
        \qs{A_m}{\psi} = \sum_n C_n \qs{A_m}{A_n} = \sum_n C_n \dta_{mn} = C_m
    $$
    $$
        C_n = \qs{A_n}{\psi}
    $$
    \item   \textbf{Условие полноты системы собственных векторов оператора $\Hat A$}

    \Reminder В двумерной модели скалярное произведение моделировалось следующим образом:
    $$
        \qs{\phi}{\psi} \to (\phi_1^\ast \, \phi_2^\ast) \pair{\psi_1}{\psi_2}
    $$
    Давайте определим матрицу
    $$
        \pair{\psi_1}{\psi_2} (\phi_1^\ast \, \phi_2^\ast) =
        \xymatrix
        {\psi_1 \phi_1^\ast}{\psi_1 \phi_2^\ast}
        {\psi_2 \phi_1^\ast}{\psi_2 \phi_2^\ast}
    $$
    В более короткой форме:
    $$
        \qu{\psi}\uq{\phi}
    $$
    $$
        \qu{\psi} = \sum_n C_n \qu{A_n} = \sum_n \qu{A_n} \uq{A_n} \psi \ket
        =
        \underbrace{\Big(
            \sum_n \qu{A_n} \uq{A_n}
        \Big)}_{\text{оператор}}
        \qu{\psi}
    $$
    Выходит, в больших скобках выписан тождественный, или единичный оператор:\index{Оператор! тождественный (единичный)}
    $$
    %\mbox{
        \sum_n \qu{A_n} \uq{A_n} = \Hat 1
    %}
    $$\index{Условие полноты}
    Рассмотрим одно из слагаемых, подействуем им на произвольный вектор $\psi \in \H $:
    $$
        \qu{A_n}\underbrace{\qs{A_n}{\psi}}_{C_n} \eqdef \Hat \P_n \qu{\psi} = C_n \qu{A_n}
    $$
    \Def $\Hat \P_n = \qu{A_n}\uq{A_n} $~--- \emph{проектор} на одномерное подпространство, натянутое на собственный вектор, отвечающий собственному значению $A_n$.\index{Оператор! проектор}

    Сумма всех этих проекторов является (в силу полноты) проектором на всё $\H$:
    $$
        \sum_n \qu{A_n} \uq{A_n} = \Hat 1
    $$
    \Rem \begin{itemize}
           \item Во многих книгах для сокращения такие операторы называют \emph{наблюдаемыми}. Под этим подразумевается то, что результаты измерений соответствующей физической величины$^{\text{[какой?]}}$ являются наблюдаемыми.\index{Оператор! наблюдаемый}
           \item Мы предполагали, что спектр является дискретным и невырожденным. Можно обойтись без этого предположения. Для начала поправим невырожденность: умножение, которое описывается выше, должно идти не по собственным векторам, а по собственным подпространствам. Непрерывный случай будет рассмотрен позже.
         \end{itemize}
\end{itemize}

\section{Описание временн\underline{о}й эволюции системы}

В квантовой механике постулируется, что квантовый оператор зависит от времени как от параметра. \textbf{Времени не сопоставляется никакой оператор, и оно не является динамической переменной.} Уравнение, описывающее динамику системы (независимый постулат квантовой механики)~--- уравнение Шрёдингера.\index{Уравнение Шредингера} В этом уравнении $\Hat H$~--- оператор Гамильтона (гамильтониан) \index{Оператор! гамильтониан}
$$
    i \hbar \ud{}{t} \qu{\psi(t)} = \Hat H \qu{\psi(t)}
$$
\Quest{Откуда брать гамильтониан?}

\Ans Пусть мы рассматриваем квантовую систему, которая имеет классический аналог, тогда есть принцип соответствия, откуда $H(q, p) \to H(\Hat q, \Hat p)$. Если классического аналога нет, то тут всё не так просто\footnote{Мы положили в основу принцип квантовой суперпозиции, и линейное уравнение Шрёдингера.
Возможен и нелинейный вариант (это лежит за пределами нашего курса). Нелинейное уравнение Шрёдингера могло иметь следующий вид:\index{Уравнение Шредингера! нелинейное}
$$
    i \hbar \ud{}{t} \Psi(t, x) = \Big\{
        -\dfrac{\hbar^2}{2m} \dfrac{\partial^2}{\partial x^2} - \underbrace{G}_{\mathrm{const}} |\psi|^2
    \Big\} \Psi
$$}.

\section{Физическая интерпретация математического аппарата}
\begin{itemize}
  \item
  \Quest{Какие значения может принимать динамическая переменная A?}

  \Ans Только те значения, которые принадлежат множеству собственных значений, которые сопоставляются оператору $\Hat A$.

  \item \Quest{Что можно сказать о состоянии, в котором достоверно (с вероятностью 1) известно, что динамическая переменная $A$ имеет значение $A_n$?}\index{Состояние}

  \Ans Это состояние описывается собственным вектором $\qu{A_n}$ оператора $\Hat A$, отвечающим собственному значению $A_n$: 
  $$
    \Hat A \qu{A_n} = A_n \qu{A_n} \to \qu{A_n}
  $$
  \item \Quest{Система находится в состоянии $\psi$, которое не является собственным вектором оператора $\Hat A$. Что можно сказать о значении динамической переменной $A$ в этом состоянии?}

  \Ans Разложим состояние по собственным векторам:
  $$
    \qu{\psi} = \sum_n \qu{A_n} \qs{A_n}{\psi}
  $$
  Вероятность того, что при измерении состояния $\qu{\psi}$ мы обнаружим динамическое состояние переменной $A$, равное $A_n$, равно квадрату модуля разложения вектора $\qu{\psi}$ по системе собственных векторов оператора $\Hat A$.
  $$
    W_{\qu{\psi}}\big(A = A_n\big) = \Big| \qs{A_n}{\psi} \Big|^2
  $$
  Иными словами, частота выпадения такого состояния равна $\big| \qs{A_n}{\psi} \big|^2$.

  \Rem Состояние $\qu{\psi}$, $c \qu{\psi}$ одно и то же, поэтому нормировка не обязательна, а формулу можно обобщить и на этот случай:
  $$
    W_{\qu{\psi}} (A = A_n) = \dfrac{ \big| \qs{A_n}{\psi} \big|^2 }{\qs{\psi}{\psi}}
  $$
\end{itemize}
\textbf{Пример:} вычисление квантово-механического среднего.

\Quest{Чему равно среднее значение динамической переменной $A$ в состоянии $\qu{\psi}$?}
Обозначим его $ \langle A \rangle_{\qu{\psi}}$

$$
   \langle A \rangle_{\qu{\psi}} \overset{\text{Формула Байеса}}{=}
\sum_i A_i W_{\qu{\psi}} (A = A_i) \,\, = \,\, \sum_i A_i \big| \qs{A_i}{\psi} \big|^2 =
$$
$$
    = \sum_i A_i \qs{A_i}{\psi}^\ast \qs{A_i}{\psi} = \sum_i \underbrace{A_i}_{\in \mathbb{C}} \qs{\psi}{A_i}
    \qs{A_i}{\psi} = \sum_i \uq{\psi} A_i \qu{A_i} \qs{A_i}{\psi} =
$$
Воспользуемся тождеством $\Hat A \qu{A_i} = A_i \qu{A_i}$
$$
    = \sum_i \uq{\psi} \Hat A \qu{A_i} \qs{A_i}{\psi} =
$$
В силу линейности дираковских обозначений:
$$
    = \uq{\psi} \Hat A \underbrace{\sum_i \qu{A_i} \uq{A_i}}_{ = \Hat 1} \qu{\psi}
    = \uq{\psi} \Hat A \qu{\psi}
$$
Таким образом,
$$
    \langle A \rangle_{\qu{\psi}} = \uq{\psi} \Hat A \qu{\psi}
$$
или, без условия нормировки:
$$
    \langle A \rangle_{\qu{\psi}} = \dfrac{\uq{\psi} \Hat A \qu{\psi}} {\qs{\psi}{\psi}}
$$

\textbf{Упражнение:}
$$
    \sum_i \big| \qs{A_i}{\psi}\big|^2 = 1
$$
\Rem $W_{\qu{\psi}} (A = A_i) = |\qs{A_i}{\psi}|^2 = \langle \psi \underbrace{| A_i \rangle \langle A_i |}_{\Hat \P_i} \psi \rangle = \langle \Hat \P_i \rangle_{\qu{\psi}}$

\section{\small Обобщение результатов на непрерывный спектр \advanced}
\index{Спектр! непрерывный}
Можно показать что, пространство $\H$ может быть разбито на прямую сумму подпространств
$$
    \H = \H(1) \oplus \H(2),
$$
при этом в $\H(1)$ спектр дискретный, а в $\H(2)$ нет ни одного собственного вектора.

\textbf{Пример:}
действуем на функцию Шрёдингера операторами координаты и импульса:\index{Оператор! координаты}\index{Оператор! импульса}
$$
\left\{
\begin{array}{l}
      -i \hbar \ud{}{x} \psi_p (x) = p \psi_p (x) \\
        x \psi_{x_0} (x) = x_0 \psi(x_0)
\end{array}
\right.
\to
\left\{
\begin{array}{l}
  \psi_p (x)= A \exp\left(\dfrac{ipx}{\hbar}\right) \\
  \psi_{x_0} (x) = B \dta(x - x_0)
\end{array}
\right.
$$
Они не принадлежат пространству $\H(1)$.

\Def \emph{Резольвента} оператора $\Hat A$:\index{Резольвента оператора}
$$
    \Hat R = (\Hat A - \lam \Hat 1)^{-1}
$$
\Def
\begin{itemize}
  \item Те значения $\lam$, при которых резольвента существует и является ограниченным оператором, называются \textbf{регулярными точками} этого \emph{оператора}.\index{Точки! регулярные}
  \item Те значения $\lam$, при которых резольвента не существует, или является неограниченным оператором называются \emph{точками спектра}. При этом возможны два случая:\index{Точки! спектра}
      \begin{enumerate}
        \item $\Hat A \qu{A_i} = \lam \qu{A_i} \to $ решение в $\H$ существует $\to$ $\lam = A_i$
        \item $\Hat A \qu{A_i} = \lam \qu{A_i} \to $ не имеет решения в $\H$. Тогда резольвента существует и является неограниченным оператором. Это есть точки непрерывности.\index{Точки! непрерывности}
  \item Важно понимать, что два написанных выше свойства являются \emph{формальными определениями}, а не свойствами, которые нужно доказывать.
      \end{enumerate}
\end{itemize}
