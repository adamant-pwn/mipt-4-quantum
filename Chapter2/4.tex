%\S 5 Обобщение результатов на непрерывный спектр.
\subsection{Формальный рецепт}
Рецепт состоит из нескольких шагов:
\begin{enumerate}
  \item Обобщённое условие полноты.\index{Условие полноты! обобщенное}
  $$
    \ssum_{A} \qu A \uq A = \Hat 1,
  $$
  где
  $$
    \ssum_{A} = \left\{
              \begin{array}{ll}
                \sum\limits_n, & \hbox{дискретный спектр;} \\\index{Спектр! дискретный}
                \int \cdots d A, & \hbox{непрерывный спектр.}\index{Спектр! непрерывный}
              \end{array}
            \right.
  $$

  \item Обобщённое условие нормировки.\index{Условие нормировки! обобщенное}
  $$
    \qu{\psi} = \int \qu{A''} \uq{A''} \psi \ket d A''
  $$
  <<Спроектируем>> на $A'$:
  $$
    \qs{A'}{\psi} = \int \underbrace{\qs{A'}{A''}}_{\dta(A' - A'')} \qs{A''}{\psi} d A''
  $$
  При этом
  $$
    \qs{A'}{A''} = \Dta_{A'A''} =
    \left\{
      \begin{array}{ll}
        \dta_{A'A''}, & \hbox{дискретный спектр;} \\
        \dta(A' - A''), & \hbox{непрерывный спектр.}
      \end{array}
    \right.
  $$
  \item Вероятность обнаружить динамическую переменную в состоянии $A = A_i$:
  $$
    W_{\qu{\psi}} (A = A_i) \to \qquad
    W(a) d a \such a \in [a, a + d a]
  $$
  Главная сложность: наличие состояния с бесконечной нормой. Вести конкретные вычисления неудобно. Для этого был придуман ряд способов, чтобы этот момент обойти (например нормировка на элемент периодичности). В этом случае спектр назывался \emph{квазидискретным},\index{Спектр! квазидискретный} и т.д. В целом: эти состояния не являются нормируемыми в обычном смысле, выпадают из теории, потому что исходному пространству они не принадлежат.

  Физически нереализуемые состояния\index{Состояние! физически нереализуемое}. Такие состояния нельзя получить в реальном эксперименте (например, идеальную плоскую монохроматическую волну).
\end{enumerate}

\subsection{Краткие основы математического метода}

\comment{Этот метод придумал Гельфанд и Костюченко в 1955 году.}

Метод состоит в том, чтобы несколько пересмотреть основы и сформулировать понятие \emph{оснащённого гильбертова пространства}. \index{Пространство! оснащенное гильбертово}В нём, наряду с обычными векторами, присутствуют также ненормируемые векторы.

Пусть $\H$~--- гильбертово пространство (<<обычное>> пространство, к которому мы привыкли). Для определённости берём $L_2(\R^3)$. В этом пространстве выделяется $\Omega \set \H$~--- подпространство <<хороших>> векторов\footnote{Аналогия: перед тем, как вводить обобщённые функции, нам необходимо было ввести пространство функций с ограниченным носителем. При этом пространство линейных операторов на этом более узком множестве оказалось более широким множеством.}~--- пространство Шварца $\mathbb{S}$ бесконечно дифференцируемых функций, убывающих на бесконечности.

\Reminder Есть взаимно однозначное соответствие между линейными функционалами $\H^\ast$ и элементами $\H$:
$$
    \H \cong \H^\ast
$$
Введём множество $\Omega^\ast$~--- сопряжённое к $\Omega$. Пространств функционалов уже может содержать <<всякие пакости>>. Это множество называют \emph{пространство распределений} Шварца\index{Пространство! распределений Шварца}. В этом пространстве присутствуют обобщённый функции умеренного роста (например, дельта-функции и плоские волны).

\Def Тройка пространств $\big(\Omega, \H, \Omega^\ast\big)$ называется \emph{триплетом Гельфанда},\index{Триплет Гельфанда} или \emph{оснащённым гильбертовым пространством}. При этом
$$
    \Omega \set \H = \H^\ast \set \Omega^\ast
$$

\textbf{Идея:} искать собственные векторы линейного самосопряжённого оператора не в пространстве $\H$, а в пространстве $\Omega^\ast$.

\Th Самосопряжённый оператор в оснащённом гильбертовом пространстве обладает полной системой обобщённых собственных векторов, отвечающих вещественным собственным значениям.\index{Оператор! самосопряженный}

\section{Одновременная (совместная) измеримость наблюдаемых}
\index{Совместная измеримость}
\comment{Прохожий, посторонний.}
\Rem Ранее мы занимались изучением статистических результатов измерения одной физической величины. В жизни мы не ограничиваемся одной физической величиной. Для начала предположим, что имеется две физические величины $A$ и $B$.
$$
    \begin{array}{c}
      A \to \Hat A \to A_i \qquad (i = 1, 2, 3, \ldots) \\
      B \to \Hat B \to B_j \qquad (j = 1, 2, 3, \ldots)
    \end{array}
$$
Пусть существует вектор $\qu \psi \in \H$, такой, что
\begin{equation}
    \begin{cases}
      \Hat A \qu \psi = A_i \qu \psi \\
      \Hat B \qu \psi = B_j \qu \psi
    \end{cases}\tag{*}
\end{equation}
Это означает, что $\qu \psi$~--- это состояние, в котором достоверно известно, что физическая величина $A$ имеет значение $A_i$, а величина $B$ имеет значение $B_j$.

Введём обозначение:
$$
    \qu \psi = \qu{A_i B_j}
$$
Затем предположим, что у системы уравнений (*) имеется такое множество решений, что $\qu \psi$ образуют полную систему\index{Полная система} векторов в пространстве $\H$. Тогда любой вектор $\qu \psi$ может быть разложен по этой системе:
$$
    \qu \phi = \underbrace{\sum_{ij} \qu{A_i B_j} \uq{A_i B_j}}_{ = \Hat 1} \phi \ket.
$$
Это даёт выражение для совместной вероятности:
$$
    W_{\qu \phi} (A = A_i, \, B = B_j) = \Big| \qs{A_i B_j}{\phi} \Big| ^2.
$$
\Rem В этом случае оказывается, что две физические величины являются совместно (независимо) измеримыми. Ещё говорят, что в этом случае у экспериментатора имеется возможность построить универсальный прибор, который измерит обе физические величины, причём измерение одной физической величины никак не повлияет на измерение другой.

Другими словами, если мы \emph{сначала} измерим $A$, то состояние $\qu{A_i B_j}$ может измениться, и состояние $B$ измерить уже не получится.

\Th Для того, чтобы наблюдаемые $A, B$ были совместно измеримы, необходимо и достаточно, чтобы операторы $\Hat A, \Hat B$ коммутировали, то есть
$$
 \text{коммутатор}   [\Hat A, \Hat B] \eqdef \Hat A \Hat B - \Hat B \Hat A = 0
$$\index{Коммутатор}
\Proof
\begin{itemize}
  \item $\RA$

  Возьмём произвольный вектор $\qu \psi \in \H$. Сформируем для него выражение
  $$
    (\Hat A \Hat B - \Hat B \Hat A) \qu \psi =
  $$
  разложим его по полной системе собственных векторов:
  $$
    \sum_{ij}(\Hat A \Hat B - \Hat B \Hat A) \qu {A_i B_j} \qs{A_i B_j}{\psi} =
  $$
  $$
    = \sum_{ij}\underbrace{(A_i B_j - B_j A_i)}_{ = 0} { \qu {A_i B_j} \qs{A_i B_j}{\psi} = 0}
  $$
  \item $\LA$

  \Rem При определённых оговорках на их область определения операторы будут самосопряжёнными.

  \begin{enumerate}
    \item   Предположим, что $[\Hat A, \Hat B] = 0$, причём спектр оператора $\Hat A$~--- простой.\index{Спектр! простой} Это означает, что уравнение
        $$
            \Hat A \qu {A_i} = A_i \qu{A_i}
        $$
        имеет одномерное собственное подпространство из векторов $\qu{A_i}$.
        $$
            \Hat B \Hat A \qu{A_i} = \Hat A \underbrace{\Hat B \qu{A_i}} = A_i \underbrace{\Hat B \qu{A_i}}
        $$
        Обратим внимание, что оба вектора $\Hat B \qu{A_i}$, и $\qu{A_i}$ являются \emph{собственными векторами} оператора $\Hat A$, причём, как было упомянуто выше, это собственное подпространство является одномерным.

        Это означает, что векторы $\Hat B {\qu{A_i}}$, $\qu{A_i}$ коллинеарны, то есть
        $$
            \Hat B \qu{A_i} = B_j \qu{A_i}
        $$

        Поэтому система собственных векторов оператора $\Hat A$ (к тому же, полная), оказывается системой собственных векторов для оператора $\Hat B$, то есть они имеют общую полную систему векторов.
    \item Будем считать, что оператор $\Hat A$ имеет \emph{вырожденный} спектр,\index{Спектр! вырожденный} то есть собственному значению $A_i$ отвечает набор
        $$
            A_i \to \set{\qu{A_i^{(1)}}, \ldots, \qu{A_i^{(\ell)}} }
        $$
        Аналогично, $\Hat B \qu{A_i^{(k)}}$ является линейной комбинацией собственных векторов $\Hat A$: $\qu{A_i^{(\ell)}}$:
        $$
            \Hat B \qu{ A_i^{(k)} } = \sum_{m=1}^\ell C_m^{(k)} \qu{ A_i^{(m)} }
        $$
        Так как $\Hat B$~--- наблюдаемый, то он обладает полной системой собственных векторов в пространстве $\H$. Этим свойством он обладает в любом подпространстве, в частности, и в собственном подпространстве оператора $\Hat A$, которое мы обозначим $\H_\ell$.

        Обозначим:
        $$
            \qu{\widetilde{A_i}} = \sum_{m=1}^\ell C_m \qu{A_i^{(m)}}
        $$
        В силу того, что у $\Hat B$ найдётся полная система собственных векторов,
        $$
            \Hat B \qu{\widetilde{A_i}} = B_j \qu{\widetilde{A_i}}
        $$
        Таким образом,
        $$
            \qu{\wt{A_i}} = \qu{A_i B_j}
        $$
    \item Если пространство имеет бесконечную размерность $\ell = \infty$, то утверждение тоже верно, но доказательство выходит за рамки курса лекций.
  \end{enumerate}%=>
\end{itemize}%proof finish

\Rem (интерпретация теоремы).
\begin{enumerate}
  \item Предположим, что есть наблюдаемая физическая величина, и соответствующий ей оператор $\Hat A$ с вырожденным спектром. Тогда нужно определить совместную с ней наблюдаемую $\Hat B$, имеющую различные собственные значения в этих состояниях.

      Может оказаться так, что хотя бы два состояния имеют одинаковый набор собственных значений в этих состояниях. Тогда нужно ввести третью наблюдаемую величину.

      Так делается до тех пор, пока каждый общий собственный вектор не будет иметь индивидуальный набор собственных значений

      \Quest{Почему так может получиться?}

      \Ans Это является физическим предположением.

      \Def Система совместных наблюдаемых называется \emph{полной},\index{Полная система! совместных наблюдаемых} если никакие два общих собственных вектора не характеризуются одним и тем же набором собственных значений для всех наблюдаемых.

  \item
        Таким образом, мы приблизились к новому пониманию понятия \emph{состояния}.

        \Def \emph{Состояние системы}~--- это совокупность собственных значений какой-либо полной системы совместных наблюдаемых.\index{Состояние! системы}

        \Quest{А как же принцип суперпозиции?}

        \Ans Его нужно держать в голове, суперпозиции таких состояний тоже должны являться состояниями.
  \item
        \Rem Условие теоремы не запрещает двум некоммутирующим операторам иметь общий собственный вектор. Главное, чтобы их не было <<много>>~--- не может быть полной системы таких векторов.
\end{enumerate}

\section{Некоммутирующие наблюдаемые. Соотношение неопределённостей}
Пусть имеется две физические величины, которым сопоставлены операторы $\Hat A, \Hat B$, причём $[\Hat A, \Hat B] = i \Hat C$.

\Rem $\Hat A$, $\Hat B$ должны быть самосопряжёнными, потому что они наблюдаемые.

$$
\begin{array}{c}
  \avr {\Hat A} = \uq \psi \Hat A \qu \psi \\
  \avr {\Hat B} = \uq \psi \Hat B \qu \psi
\end{array}
$$
\begin{itemize}
  \item Обозначим
  $$
\begin{array}{c}
  \Dta \Hat A = \Hat A - \avr{\Hat A}\cdot \Hat 1 \\
  \Dta \Hat B = \Hat B - \avr{\Hat B}\cdot \Hat 1
\end{array}
$$
    \Exercise
    $$
        [\Dta \Hat A, \Dta \Hat B] = i \Hat C
    $$
  \item Пусть $\gamma = \gamma^\ast$
  $$
  \begin{array}{c}
        \qu{\phi} = (\Dta \Hat A - i \gamma \Dta \Hat B)\qu \psi \\
        \uq{\phi} = \uq \psi (\Dta \Hat A + i \gamma \Dta \Hat B)
  \end{array}
  $$
  Учитываем: $\Hat A^+ = \Hat A, \Hat B^+ = \Hat B$.

  $\avr{\Hat A}^\ast = \avr{A}$, $\avr{\Hat B}^\ast = \avr{B}$

  $$
    \| \qu{\phi} \|^2 = \qs{\phi}{\phi} =
  $$
  $$
    = \uq \psi \left(
        \Dta \Hat A + i \gamma \Dta \Hat B
    \right)\left(
        \Dta \Hat A - i \gamma \Dta \Hat B
    \right) \qu \psi =
  $$
  $$
    = \avr{\Dta \Hat A^2} + \gamma^2 \avr{\Dta \Hat B^2} + \gamma \avr{\Hat C} \geqslant 0
  $$
  Это выражение является квадратным трёхчленом относительно переменной $\gamma$. Дискримант должен быть неположительным:
  $$
    \avr{\Hat C^2} - 4 \avr{\Dta \Hat A^2} \avr{\Dta \Hat B^2} \leqslant 0
  $$
  Отсюда получается \emph{общее соотношение неопределённостей}:\index{Соотношение неопределенностей! общее}
  $$
    \avr{\Dta \Hat A^2} \avr{\Dta \Hat B^2} \geqslant \dfrac{1}{4} \avr{\Hat C^2}
  $$
\end{itemize}
\Rem
\begin{itemize}
  \item Если $\Hat A = \Hat x$ и $\Hat B = \Hat p$, тогда 
  $$
    [\Hat x, \Hat p] = i \hbar \cdot \Hat 1 \to \quad \avr{\Dta \Hat p^2} \avr{\Dta \Hat x^2} \geqslant \dfrac{\hbar^2}{4}
  $$
  \item <<Анонс>>:
  $$
    \avr{\Dta \Hat A^2} = \avr{ ( \Hat A - \avr{\Hat A} )^2 } = \avr{\Hat A^2} - \avr{\Hat A}^2
  $$
  Например, для собственного вектора дисперсия равна нулю. Действительно, в этом состоянии достоверно известно, чему равно значение действия оператора.
\end{itemize}

\section{Элементы дираковской теории представлений}
\subsection{$\alpha$--представление}
\Rem Идея следует из конечномерной теории: любой вектор из трёхмерного пространства задаётся своими координатами.

Выберем базис как множество собственных векторов оператора $\Hat \alpha$.
$$
    \Hat \alpha \qu{\alpha_i} = \alpha_i \qu{\alpha_i}
$$
\begin{enumerate}
  \item $\qs{\alpha_i}{\alpha_j} = \dta_{ij}$
  \item $\sum_{i} \qu{\alpha_i} \uq{\alpha_i} = \hat 1$
\end{enumerate}
При этом
$$
    \begin{array}{l}
      \forall \qu \psi \in \H\\
      \qu \psi = \sum_i \qu{\alpha_i} \qs{\alpha_i}{\psi}, \qquad C_i  = \qs{\alpha_i}{\psi}
    \end{array}
$$
\Def Набор чисел $C_i$ называется \underline{\emph{вектор $\psi$ в $\alpha$-представлении}}.\index{Представление! $\alpha$-представление}

Базис может оказаться и непрерывным. В этом случае разложение имеет вид:
$$
   \qu \psi = \int \qu{\alpha} \underbrace{\qs{\alpha}{\psi}}_{ = \psi(\alpha)}  d \alpha
$$
Тогда $\psi(\alpha)$ называется \emph{волновой функцией в $\psi$-представлении.}
$$
    \Hat{\qu{x}} = x \qu{x} \to \qs{x}{\psi} = \psi(x)
$$
%to be continued

