\subsection{Расчёт в сильном поле. Эффект Пашена-Бака}
Как говорилось ранее, под сильным полем подразумевается такое поле, что
$$
    \left(
        \dfrac{e}{a^2}
    \right)\alpha \ll \mathcal H \ll \left(\dfrac{e}{a^2}\right) \alpha^{-1}
$$
где $\alpha = \dfrac{e^2}{\hbar c}$~--- постоянная тонкой структуры.

Применим теорию возмущений, а затем наложим квазирелятивистские поправки.
$$
    \hat H^{(0)} = \hat H^{\text{без поля}} - \hat V^{\text{кв.рел.}} = \hat H^{S}
$$
$\hat H^S$~--- Шрёдингеровский гамильтониан.
$$
    \hat V = \Omega (\hat L_z + 2 \hat S_z), \quad \Omega = \dfrac{e \mathcal H}{2mc}
$$
Собственные векторы этой системы имеют вид
$$
    \qu{n \l s m_\l m_s} = \underbrace{\qu{n \l m \l}}_{\psi_{n\l m}(\bf r)} \otimes \underbrace{\qu{s m_s}}_{\chi}
$$
В базисе из этих векторов диагонален не только гамильтониан, но и оператор возмущения, потому что они являются собственными векторами для набора
$$
    \hat{\bf L}^2, \hat{\bf S}^2, \hat L_z, \hat S_z
$$ 
Таким образом, поправки для энергии в магнитном поле равны
$$
    E_{\H}^{(1)} = \hbar \Omega (m_\l + 2 m_s)
$$
$$
    \omega_{21} = \dfrac{E_2^{(0)} - E_1^{(0)}}{\hbar} + \Omega (\Delta m_\l + 2 \Delta m_s)
$$
Повторно применяя теорию возмущений, можно получить и другие поправки.

\subsection{Наблюдаемые частоты. Правила отбора}
Правила отбора связывают квантовые числа в начальном и конечном состоянии.
$$
    \qtri{f}{\hat{\bf d}}{i} \neq 0 \RA \qtri{f}{\hat{\bf d}}{i} \neq 0
$$
Как и в предыдущем пункте, рассмотрим случаи слабого и сильного поля.

\subsubsection{Слабое поле.}

Состояние электрона в атоме водорода описывается функцией, которая является общим собственным вектором операторов
$$
    \hat{\bf L}^2, \hat{\bf S}^2, \hat L_z, \hat S_z \, \to \, \qu{n \l s j m_j}
$$
Известно значение коммутатора
$$
    [\hat J_i, \hat x_j] = i \hbar e_{ijk} \hat x_k
$$
Это говорит о том, что оператор $\hat x$ является векторным оператором относительно поворотов в пространстве, компоненты которого являются компоненты углового момента. $\H^{orb} \otimes \H^{spin}$.

Для $m = 0, \pm 1$ оператор $\hat X_m^1$ является неприводимым тензорным оператором ранга 1.

Как следует из теоремы Вигнера-Эккарта, он имеет матричный вид
$$
    \qtri{\gamma_2 j_2 m_2}{\hat X_m^1}{\gamma_1 j_1 m_1} = C_{jm j_1 m_1}^{j_2 m_2} \langle \gamma_2 j_2 \| \hat X^1 \| \gamma_1 j_1 \rangle
$$
Здесь $\langle \gamma_2 j_2 \| \hat X^1 \| \gamma_1 j_1 \rangle$~--- это приведённый матричный коэффициент.

Для того, чтобы коэффициенты Клебша-Гордана были отличны от нуля, надо, чтобы выполнялись соотношения
$$
    \left\{
      \begin{array}{lcl}
        \bf j_2 & = & \bf j + \bf j_1 ,\\
        m_2 & = & m + m_1.
      \end{array}
    \right.
$$
Другими словами, из векторов $\{\bf j_2, \bf j, \bf j_1\}$ надо уметь строить треугольник, с учётом квантования.

Суммарный момент $j_2$ изменяется в пределах $j_1+1, j_1, j_1 - 1$.

Кроме того, $m_2 = m_1 - 1, m_1, m_1 + 1$

Правила отбора принимают вид
$$
    \left\{
      \begin{array}{lcl}
        \Delta m & = & m_2 - m_1 = 0, \pm 1 ,\\
        \Delta j & = & j_2 - j_1 = 0, \pm 1.
      \end{array}
    \right.
$$
На самом деле, при значении $\Delta j = 0$ возникает неточность, надо исключить случай $j_1 = j_2 = 0$.

\subsubsection{Сильное поле}

Состояние электрона в атоме описывается функцией
$$
    \qu{n \l m_\l} \otimes \qu{s m_s},
$$
которая является общей собственной функцией операторов
$$
    \hat{\bf L}^2, \hat{\bf S}^2, \hat L_z, \hat S_z
$$
Кроме коммутатора $[\hat J_i, \hat x_j]$, в прошлом семестре был получен коммутатор
$$
    [\hat L_i, \hat x_j] = i \hbar e_{ijk} \hat x_k
$$
Оператор $\hat x$ является вектором не только для полного пространства состояний, а также относительно поворотов в орбитальном пространстве.

$$
    \qtri{\gamma_2 \l_2 m_2}{\hat X_m^1}{\gamma_1 \l_1 m_1} = C_{j_1 m \l_1 m_1}^{\l_2 m_2} \langle \gamma_2 j_2 \| \hat X^1 \| \gamma_1 j_1 \rangle
$$ 
Отсюда возникают правила отбора для слабого поля
$$
    \left\{
      \begin{array}{lcl}
        \Delta m_\l & = & m_{\l_2} - m_{\l_1} = 0, \pm 1 ,\\
        \Delta \l & = & \l_2 - \l_1 = 0, \pm 1.
      \end{array}
    \right.
$$

\subsubsection{Правила отбора по чётности}
Как известно, чётность~--- это собственное значение оператора инверсии.
$$
    \hat I\qu{\gamma \l} = (-1)^\l \qu{\gamma \l} = \prod(\l) \qu{\gamma \l}
$$
\begin{eqnarray*}
    \hat I^{-1} \hat{\bf r} \hat I &=& - \hat{\bf r}_j - \qtri{\gamma_2 \l_2}{\hat{\bf r}}{\gamma_1 \l_1}\\
&=& \qtri{\gamma_2 \l_2}{\hat I^{-1}\hat{\bf r} \hat I}{\gamma_1 \l_1}\\
&=& \prod (\l_1) \prod(\l_2) \qtri{\gamma_2 \l_2}{\hat{\bf r}}{\gamma_1 \l_1}
\end{eqnarray*}
Правило отбора:
$$
    \boxed{
    \prod(\l_1) \prod(\l_2) = -1
}, \qquad \text{то есть } \, \l_1 + \l_2 = 2n+1
$$
Надо отметить, что такая жёсткая связь есть только у атома водорода.

\subsubsection{Правило отбора по спину}
Так как $\hat{\bf J} = \hat{\bf L} + \hat{\bf S}$.

Отсюда
$$
    [\hat S_i, \hat x_j] = 0
$$
Таким образом, в спиновом пространстве $\mathcal S$ оператор $\bf x$ является скалярным оператором.

Снова можем применить теорему Вигнера-Эккарта. Из неё следует, что
$$
    \Delta S = 0, \quad \delta m_s = 0
$$
Вспомним, что для эффекта Зеемана наблюдаемые линии имели вид
$$
    \omega_{12} = \omega_0 + \Omega (\Delta m_\l + 2 \Delta m_s),
$$
где $\Delta m_s = 0, \Delta m_\l = 0, \pm 1$. Нормальный эффект Зеемана переходит в аномальный.

В этом и состоит эффект.

\textbf{Примечание.} Разные случаи рассматривались, чтобы выделить группы квантовых чисел. Эти правила годятся для любых случаев, где эти числа встречаются. Правила носят <<запретительный>> характер. Таким образом, излучение возможно лишь тогда, когда выполнены все правила отбора.

\chapter{Тождественные частицы в квантовой механике}
\def \H {\mathcal H}
\section{Принцип тождественности (неразличимости) частиц}
\subsection{Пространство состояний $N$ частиц}
Пока частицы не обязательно тождественные (общий случай).

Для того, чтобы построить состояния $N$ частиц, надо стартовать с пространства состояний для одной частицы.

Рассмотрим самый простой случай бесспиновой частицы. Её пространство $\H^{(orb)}$ можно представить в виде пространства $L_2(\mathbb R^3)$ интегрируемых функций.

Спиновые частицы (спин $\frac12$): $\H^{(orb)} \otimes \H^{(spin)}$, которое представимо в виде $L(\R^3) \otimes \mathbb C^{2}$

Это пространство мы, в зависимости от контекста, называем $\H$.

Пусть имеется $N$ частиц. Тогда пространство состояний описывается как
$$
    \H(1) \otimes \H(2) \otimes \ldots \otimes \H(N) = \H
$$
Эксперимент говорит о том, что это годится лишь для случая, когда среди частиц нет одинаковых. В случае одинаковых частиц уточнения к теории даёт принцип тождественности. 

Всю теорию будем строить в координатном представлении. Элементом пространства будет $N$-частичная волновая функция
$$
    \Psi(\xi_1, \ldots, \xi_N),
$$
где в случае бесспиновой частицы, $\xi_i = \bf r_i$, а в случае спиновой $\xi_i = (\bf r_i, s_i)$, если спин $\frac12$. 

Как выглядит скалярное произведение в этом пространстве?
$$
    \qs{\Psi}{\Phi} = \sum_{S_1, \ldots, S_N} \int d^3 x_1 \ldots d^3 x_N \Psi^\ast(\xi_1, \ldots, \xi_N) \Phi(\xi_1, \ldots, \xi_N)
$$
\subsection{Группа перестановок из $N$ элементов}
Напомним необходимые свойства этой группы. Каждый её элемент~--- это перестановка, как подсказывает нам её название.

Элементы этой группы будут также операторами в исходном пространстве. Например,
$$
    \hat 1 = \begin{pmatrix}
               1 & \cdots & N \\
               1 & \cdots & N \\
             \end{pmatrix}
$$
Особую роль будут играть \emph{транспозиции} $\pi_{ij}$, которые переставляют местами два элемента.

Любую конечную перестановку можно представить в виде композиции транспозиций. Количество транспозиций в разложении может быть разным, но чётность этого числа предопределена. Каждой перестановке можно сопоставить её чётность.
$$
    (-1)^{[\pi]}
$$ 
Формально, оператор опеределяется как
$$
    \hat{\mathcal P}_{\pi} \Psi(\xi_1, \ldots, \xi_N) = \Psi(\xi_{i_1}, \ldots, \xi_{i_N})
$$
Эти операторы реализуют унитарное представление.

В этом пространстве можно выделить два пространства, инвариантных относительно группы перестановок (пространство симметричных и антисимметричных функций).
\def \mP{\hat{\mathcal P}}
$$
    \mP_{\pi} \Psi_S (\xi_1, \ldots, \xi_N) = \Psi_S (\xi_1, \ldots, \xi_N)
$$
$$
    \mP_{\pi} \Psi_A (\xi_1, \ldots, \xi_N) = (-1)^{[\pi]} \Psi_A (\xi_1, \ldots, \xi_N)
$$
Если $N = 2$, то других инвариантных подпространств нет. Любую функцию от двух аргументов можно единственным образом представить в виде суммы чётной и нечётной части.

Иначе говоря,
$$
    \H = \H_A \oplus \H_S
$$ 
Если $N > 2$, то появляются и другие, более сложные инвариантные подпространства.

\subsection{Тождественность частиц в квантовой механике}
В классике частицы тоже иногда называют одинаковыми. На самом деле, в классике они строго индивидуальны, за каждой частицей можно проследить, потому что у них есть \emph{траектории}.  Зная начальные координаты и скорости, решая задачу Коши, находим уравнения движения.

В квантовой механике, даже если в начальный момент времени создать локализованные пакеты, то они <<расплываются>>. Начинают проявляться волновые свойства частиц, например, интерференция.

Имеет смысл характеризовать $N$-частичную систему $N$-частичной функцией, симметрия которой следует из симметрии гамильтониана. Который пока не выписан.

\textbf{Гамильтониан $N$ тождественных частиц}
$$
    \hat H = \sum_{i=1}^{N} \left\{
        \dfrac{\hat{\bf p_i}^2}{2m} + U(\bf r_i)
\right\} + \sum_{i<j} \Phi(|\bf r_i - \bf r_j|)
$$
Так как гамильтониан симметричен, то для любого оператора $\mP_{\pi}$
$$
    [\hat H, \mP_\pi] = 0
$$
Должны быть другие $N$-частичные физические наблюдаемые. Чтобы частицы были тождественные, эти операторы тоже должны коммутировать со всеми операторами перестановки частиц.
$$
    [\hat F, \mP_\pi] = 0
$$
\textbf{Принцип.} $N$ частиц тождественные, если гамильтониан и все физические наблюдаемые не меняются при любой перестановке частиц. 

\textbf{Операторы, описывающие наблюдаемые величины}

Для того, чтобы описать состояние частицы, нужна полная система взаимно коммутирующих наблюдаемых.
Для одной частицы $\hat a_i = \{\hat{\bf p_i}, m_i\}$.

Для $N$ частиц $\hat A = \{\hat a_1, \ldots, \hat a_N\}$. Необходимо использовать $N$-частичные операторы. Среди них нужно рассматривать только симметричные функции.

Мы сильно сузили возможные операторы. Если на пространство $\H$ не наложить дополнительных условий, то потеряется свойство полноты. Постулат тождественности должен помочь наложить эти ограничения, чтобы восстановить свойства полноты.

\textbf{Постулат тождественности.}\footnote{В нерелятивистской квантовой механике нужны всего два факта со стороны. Первый факт~--- это спиновые операторы, второй факт~--- это постулат тождественности. Первое мы получили из уравнения Дирака, второе просто примем как данность. Он не выводится из других положений квантовой механики, но следует из более общей теории.} Истинным пространством состояний является пространство $\H_S$ или $\H_A$. Притом выбор одного из этих пространств связан только с природой рассматриваемых частиц. Конкретнее~--- со спином.

\begin{itemize}
  \item Если спин целый, то рассматриваются бозоны, для которых выполняется статистика Бозе-Эйнштейна, рассматриваются симметрические волновые функции.
  \item Если спин полуцелый, то выполняется статистика Ферми-Дирака, рассматриваются антисимметрические волновые функции.
\end{itemize}
\subsection{Принцип Паули} Рассмотрим систему из $N$ тождественных фермионов, которые между собой не взаимодействуют.

Тогда гамильтониан имеет аддитивную структуру,
$$
    \hat H = \sum \hat H_i
$$ 
Если для каждого из них есть полная система собственных функций
$$
    \hat H_i \, : \, \{\psi_{n_i} (\xi_i)\},
$$
то для гамильтониана $\hat H$ имеем
$$
    \tilde\Psi(\xi_1, \ldots, \xi_N) = \psi_{n_1} (\xi_1) \ldots \psi_{n_N} (\xi_N) 
$$
Этого недостаточно, надо антисимметризовать, для этого используется опеределитель Слеттера
$$
    \Psi_A (\xi_1, \ldots, \xi_N) = \dfrac{1}{\sqrt{N!}}
\begin{vmatrix}
  \psi_{n_1} (\xi_1) & \ldots & \psi_{n_1} (\xi_N) \\
  \vdots & \vdots & \vdots \\
  \psi_{n_N}(\xi_1) & \ldots & \psi_{n_N} (\xi_N) \\
\end{vmatrix}
$$
Не для любого набора функций можно построить волновую функцию в виде детерминанта. Если среди волновых функций есть хотя бы две одинаковых, то детерминант занулится.

\textbf{Принцип Паули.} В каждом однофермионном состоянии может находиться не более одной частицы. 