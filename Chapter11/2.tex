%. 
\subsection{Правило квантования Бора-Зоммерфильда}

Предположим, что имеется потенциальная яма, удовлетворяющая всем условиям квазиклассического приближения.
%
% \               / : E
%  \             /
%   \ ~~~~~~~~  /  ~ :
%  **\*********/****** U
%     \_______/
%  I     II    III
%     a     b
Если строить решение ВКБ, удовлетворяющее всем условиям, то представляет интерес решение, которое убывает на $\pm \infty$. 

При переходе через точки $a$, $b$ нужно использовать правило номер 1. (см. выше).

$$
    \psi_{x < b} (x) \approx \dfrac{a_1}{p} \sin \left(
        \dfrac{1}{\hbar} \int\limits_{x}^{b} p(x') dx' + \pi/4
    \right) = \dfrac{a_1}{\sqrt{p}} \sin \left(
        z_1 + \pi/4
    \right)
$$
$$
    \psi_{x > a} (x) \approx \dfrac{a_1}{p} \sin \left(
        \dfrac{1}{\hbar} \int\limits_{a}^{x} p(x') dx' + \pi/4
    \right) = \dfrac{a_2}{\sqrt{p}} \sin \left(
        z_2 + \pi/4
    \right)
$$
Теперь необходимо поставить достаточно жёсткое условие. Нужно потребовать, чтобы в области $a < x < b$ обе эти функции совпадали во всех точках. 
 
Вспомогательная лемма:
$$
    z_2 = \dfrac{1}{\hbar}\int_{a}^{b} p(x') dx' - z_1
$$

Условие совпадения функций:
$$
    \dfrac{a_1}{\sqrt p} \sin \left(
        z_1 + \dfrac{\pi}{4}
    \right) = \dfrac{a_2}{\sqrt p} \sin \left(
        z_2 + \dfrac{\pi}{4}
    \right)
$$
Подставляем в правое равенство $z_2$. Получаем
$$
    \dfrac{a_1}{\sqrt p} \sin \left(
        z_1 + \dfrac{\pi}{4}
    \right) = \dfrac{-a_2}{\sqrt p} \sin \left(
        z_1 - \dfrac{1}{\hbar} \int\limits_{a}^{b} p(x') dx' - \dfrac{\pi}{4}
    \right)
$$
Нужно потребовать совпадения фаз и амплитуд:
$$
    z_1 + \dfrac{\pi}{4} = z_1 - \dfrac{1}{\hbar} \int\limits_{a}^{b} p(x') dx' - \dfrac{\pi}{4} + \pi(n+1),
$$
$$
    a_2 = (-1)^n a_1
$$
$$
    \boxed{
        \int\limits_a^b p(x)dx = \pi \hbar \left(n + \dfrac{1}{2}\right)
    }
$$
Можно записать правило в виде взятия интеграла по полному периоду частицы, от $a$ до $b$, а затем от $b$ до $a$
$$
    \oint p(x) dx = 2 \pi \hbar \left(
        n + \dfrac{1}{2}
    \right),
$$
где $p(x) = \sqrt{2m (E - U(x))}$

\textbf{Замечания.}
\begin{itemize}
  \item Слева стоит адиабатический инвариант. То, что мы рассмотрели, есть не что иное как квантование адиабатических инвариантов. Классическая теорема о том, что адиабатический инвариант плавно изменяется при малом изменении параметра. С точки зрения квантовой механики, система остаётся в том же квантовом состоянии, $n = \mathrm{const}$.
  \item Смысл величины $n$. Через него определяются $E_n$, уровни энергии, которые появляются при квантовании по Бору-Зоммерфельду. С другой стороны, это ещё и число узлов волновой функции. Если стартовать от точки $x = a$, то фаза синуса равна $\pi / 4$. При $x = b$ мы получаем $\pi (n + 1/ 2) = \pi n + \dfrac{3 \pi}{4}$. Другими словами, функция $n$ раз пройдёт через 0.
      
      Это частный случай \emph{осцилляционной теоремы} о том, что номер уровня энергии совпадает с числом узлов.
      
  \item Квазиклассический метод (ВКБ) на самом деле плохо работает при маленьких значениях классического $p$, и не работает в классических точках поворота. Нужно иметь возможность отойти от этих точек на большое расстояние по сравнению с длиной волны. Нужно, чтобы между точками поворота умещалось большое число длин волн. Напоминаем, что расстояние между узлами как раз имеет порядок длины волны.
      
      $$
        \lam (x) = \dfrac{\hbar}{p(x)}
      $$
      
      Такое возможно при $n \gg 1$. Квазиклассический случай --- это по сути случай больших квантовых чисел.
      
  \item Можно ли при $n \gg 1$ выкинуть слагаемое $\dfrac{1}{2}$? Что мы вообще отбрасываем?
  
  На самом деле отбрасываются слагаемые порядка $\hbar^2$, поэтому можно показать, что слагаемое $\dfrac{1}{2}$ отбросить нельзя.
  
  \item  Можно взглянуть на интеграл $\oint (\cdots)$ с точки зрения фазовой плоскости. У нас имеется две координаты $(x, p)$. С точки зрения фазовой плоскости это площадь, которая ограничивает состояние с энергией $E < E_n$. Отсюда следует, что осуществляя шаг $E_n \to E_{n+1}$, площадь увеличивается на $2 \pi \hbar$.
      
      На одно квантовомеханическое состояние в фазовой плоскости приходится клетка площадью $2 \pi \hbar$.
      
      Отсюда возникает известная формула статистической физики:
      
      Число квантовых состояний на единицу фазового объёма равно $\Dta N$
      $$
        \Dta N = \dfrac{\Dta p \Dta x}{2 \pi \hbar}
      $$
\end{itemize}

\subsection{Туннельный эффект при прохождении частиц через потенциальный барьер}
\begin{verbatim}
% U(x)
%
% I         II              III
%       _____________
%      /             \
%     |               |
%****/**************************** E
%   /                  \
%  |                    \
% /                      \
% ----------------------------------> x
%    a                 b
\end{verbatim}
Из-за плавающего изменения функциии барьер должен быть очень широким.

На $+\infty$ будет прошедшая волна, а до барьера~--- падающая и отражённая.

Выберем функцию, которая будет описывать состояние частицы в области \texttt{III}.
$$
    \psi_{x > b} \approx \dfrac{A}{\sqrt p}  \exp \left(
        \dfrac{i}{\hbar} \int\limits_{b}^{x} p(x') dx' + \dfrac{i\pi}{4}
    \right)
$$
Это волна, которая распространяется слева направо.

Найдём плотность тока вероятности
$$
    j_x = \dfrac{i \hbar}{2 m} (\psi {\psi'}^\ast - \psi^\ast \psi')
$$
При дифференцировании нужно дифференцировать только показатель экспоненты (подумайте, почему)\footnote{Возможно, это как-то связано с порядком нашего приближения}.

$$
    j_x = \dfrac{i \hbar}{2 m} \dfrac{|A|^2}{p(x)} \left[ e^{i(z + \pi/4)} \left(
        -\dfrac{i p(x)}{\hbar}
    \right) e^{-i (z + \pi /4)} - k_1 c_1 \right] = \dfrac{|A|^2}{m} > 0
$$
Действительно получили распространяющуюся волну.

Подходим к точке поворота $x = b$. Нужно применить правило соответствия номер 3.

Согласно этому правилу, в области $x < b$
$$
    \psi_{x < b} \approx \dfrac{A}{\sqrt p} e^{1 / \hbar \int\limits_{x}^{b} |p(x')| dx'}
$$
Продолжая эту запись, получаем
$$
    \psi_{x < b} = \dfrac{A}{\sqrt p} \exp \left(
        \underbrace{\dfrac{1}{\hbar} \int\limits_{a}^{b} |p(x') dx'}_{\gamma} - \dfrac{1}{\hbar} \int\limits_{a}^{x} |p(x') dx'
    \right)
$$
Хотим составить квазиклассическое решение в области $\mathrm{I}$. Для перехода из области $\mathrm{III}$ в область $\mathrm{I}$ используем правило согласования 1.
$$
    \psi_{x < a} \approx \dfrac{2A e^{\gamma}}{\sqrt p} \sin \left(
        \dfrac{1}{\hbar} \int\limits_x^a p(x') dx' + \dfrac{\pi}{4}
    \right) = \dfrac{2 A e^{\gamma}}{2i \sqrt p}  \exp \left(
        \dfrac{i}{\hbar} \int\limits_x^a p(x') dx' + \dfrac{i \pi}{4}
    \right) - \exp \left(
        -\dfrac{i}{\hbar} \int\limits_x^2 p(x') dx' - \dfrac{i \pi}{4}
    \right)
$$ 
Первое слагаемое соответствует падающей волне, второе~--- отражённой.

Чтобы найти коэффициент прохождения по общему правилу, запишем формулу Гамова (1928)
$$
\boxed{
    D = \dfrac{|j_x^{pass}|}{j_x^{fall}} = e^{-2 \gamma} = e^{-\frac{2}{\hbar} \int\limits_a^b |p(x)| dx} 
}
$$
\textbf{Замечания.}
\begin{itemize}
  \item При таком подходе коэффициент отражения равен 1. Экспоненциально малые поправки в нашем методе отбрасываются.
  \item Отсюда вытекает парадокс: барьер выступает в роли источника частиц. В рамках ВКБ все рассуждения проведены строго.
  \item Что происходит, если мы сталкиваемся с нарушением условий применимости метода ВКБ? Тогда нужно выписывать точное решение.
\end{itemize}
