\section{Атом гелия}
\def \bf {\boldsymbol}
Из <<сложных атомов>> это самый простой.

У гелия есть ядро, заряд 2, два электрона, с координатами $\bf r_1, \bf r_2$, относительная координата
$$
    \bf r_{12} = \bf r_1 - \bf r_2
$$
\centerline{\includegraphics[width=0.4\textwidth]{pic/11/1.pdf}}
% Рисунок атома гелия
Гамильтониан атома гелия
$$
    \hat H = \dfrac{\bf p_1^2}{2m} + \dfrac{\bf p_2^2}{2m} - 2\dfrac{e^2}{r_1} - 2\dfrac{e^2}{r_2} + \dfrac{e^2}{r_{12}}
$$
Поскольку релятивистские эффекты проявляются только в членах порядка $(v/c)^2$ (спинорбитальное взаимодействие), в этом гамильтониане никаких следов спина нет. 

\subsection{Постановка задачи}
Так как взаимодействие от спина не зависит, в волновой функции координатные функции отделяются от спиновых:
$$
    \Phi(\bf r_1, \bf r_2, s_1, s_2) = \Phi(\bf r_1, \bf r_2) \chi(s_1, s_2)
$$
$$
    \hat H \Psi(\bf r_1, \bf r_2) = E \Psi(\bf r_1, \bf r_2)
$$
Сложность задачи состоит в том, что задача не имеет точного аналитического решения (как и любая уважающая себя задача трёх тел). Следовательно, нужно искать приближённое решение.

\subsection{Теория возмущения}
Очень многое зависит от того, как мы разделим гамильтониан на основную часть и возмущение.

Естественно разделить гамильтониан на кулоновское взаимодействие и всё остальное:
$$
    \hat V = \dfrac{e^2}{r_{12}}
$$
Очень трудно оценить малый параметр. На самом деле, он окажется порядка $\dfrac{1}{3}$, приближение очень неточное. Существуют более точные оценки, но для начала мы пойдём самым простым, <<лобовым>> путём, и посмотрим, какие следствия отсюда можно получить.

Разбивая оставшуюся часть гамильтониана на две части, получаем уравнение на собственные функции в виде
$$
    \big(\hat H_1^{(0)} + \hat H_2^{(0)} - E^{(0)}
    \big) \Psi^{(0)} (\bf r_1, \bf r_2) = 0
$$
$$
\boxed{
    E^{(0)} = E_{nk}^{(0)} = -  \left(
        \dfrac{1}{n^2} + \dfrac{1}{k^2}
    \right) \dfrac{4e^2}{2a}
}
$$
$$
    \Psi_{nk}^{(0)} = \left\{
                        \begin{array}{lcl}
                          \psi_n(\bf r_1) \psi_k(\bf r_2) & , & \text{первый элемент в $n$, второй в $k$} ,\\
                          \psi_k(\bf r_1) \psi_n(\bf r_2) & , & \text{наоборот}.
                        \end{array}
                      \right.
$$
Имеет место \emph{обменное вырождение} состояний. 
$$
    \Psi_{\pm}^{(0)} = \dfrac{1}{\sqrt 2} \big(
    \psi_n(\bf r_1) \psi_k (\bf r_2) \pm \psi_k (\bf r_1) \psi_n(\bf r_2)
\big)
$$
$$
    \qtri{\Psi_{\pm}^{(0)}}{\dfrac{e^2}{r_{12}}}{\Psi_{\pm}^{(0)}} =
\int d^3 \bf r_1 d^3 \bf r_2 \dfrac{e^2}{|\bf r_1 - \bf r_2|} \Psi_{\pm}^{\ast(0)} \Psi_{\pm}^{(0)} = 0
$$
Подынтегральное выражение поменяет знак, если поменять местами $\bf r_1, \bf r_2$. Построенные функции являются правильными волновыми функциями нулевого приближения. 
$$
    \boxed{
    E_{\pm}^{(0)} = \dfrac{e^2}{2} \int d^3 \bf r_1 d^3 \bf r_2 \dfrac{
    | \psi_n(\bf r_1) \psi_k(\bf r_2) \pm \psi_k(\bf r_1) \psi_n(\bf r_2) |^2
}{r_{12}}
}
$$
Вводим следующие обозначения
$$
    \rho_{nn} = \rho_{nn} (\bf r) = e \psi_n^\ast(\bf r) \psi_n(\bf r)
$$
Это \textbf{плотность заряда}, которая создаётся в точке с координатой $\bf r$.

\textbf{Обменная плотность заряда:}
$$
    \rho_{nk} = \rho_{nk}(\bf r) = e \psi_n^{\ast} (\bf r) \psi_{k} (\bf r)
$$
Это плотность заряда, которая создаётся электроном, \emph{как бы находящимся} одновременно в состояниях $n, k$.

\textbf{Кулон:\footnote{Coulomb (фр.)}}
$$
\boxed{
    C = \int d^3 \bf r_1 d^3 \bf r_2 \dfrac{\rho_{nn}(\bf r_1) \rho_{kk} (\bf r_2)}{r_{12}}
}
$$
Это есть электростатическая энергия взаимодействий электронов, один из которых находится в состоянии $n$, а другой~--- в состоянии $k$. 

\textbf{Обменный интеграл:\footnote{Austausch (нем.)}}
$$
\boxed{
    A = \int d^3 \bf r_1 d^3 \bf r_2 \dfrac{\rho_{nk}(\bf r_1) \rho_{kn}(\bf r_2)}{r_{12}}
}
$$
Обменная энергия взаимодействия электрона, который \emph{как бы находится} одновременно в состояниях $n, k$.

Оказывается, что в введённых нами обозначениях
$$
\boxed{
    E_{\pm}^{(1)} C \pm A
}
$$
\subsection{Учёт спина}
Напомним, что в исходном гамильтониане нет никакого спина. В точном решении отделяются координатные и спиновые части.

В соответствии с постулатом тождественности (два электрона это два фермиона), волновая функция должна быть антисимметризована относительно обмена частиц ($\bf r_1 \leftrightarrows \bf r_2$). 

Если вся функция должна быть антисимметричная, то возможны два случая:
\begin{itemize}
  \item $\Phi_A(\cdot) = \Psi_S(\cdot) \chi_A$
  \item $\Phi_A(\cdot) = \Psi_A(\cdot) \chi_S$
\end{itemize}
Буквы $A, S$ означают соответственно антисимметричность или симметричность.

\def \H {\mathcal H}
$$
    \H^{(spin)} = \H^{(spin)}_1 \otimes \H^{(spin)}_2
$$
Волновая спиновая функция должна быть общей собственной функцией операторов
$$
    \hat {\bf S}^2, \hat S_z, \qquad \hat{\bf S} = \hat{\bf S} (1) + \hat {\bf S} (2)
$$
Вспоминаем, какие собственные числа могут быть у операторов $\hat{\bf S}$, $\hat S_z$.
\begin{itemize}
  \item $\chi(1,1) = \alpha (1) \alpha (2)$
  \item $\chi(1, 0) = \dfrac{1}{\sqrt 2} \big(
    \alpha(1) \beta(2) + \beta(1) \alpha(2)
\big)$
  \item $\chi(1,1) = \beta(1) \beta(2)$
    \item $\chi(0,0) = \dfrac{1}{\sqrt 2} \big(
    \alpha(1) \beta(2) - \beta(1) \alpha(2)
\big)$
\end{itemize}
Здесь $\alpha = (1 \, 0)^\top = \chi_{1/2}$, $\beta = (0 \, 1)^{\top} = \chi_{-1/2}$

Если суммарный спин равен нулю, то спин антисимметричный.

\textbf{Анализ.} Существенную роль играет тот факт, что $A > 0$.  В нашем курсе это не доказывается, предлагается обдумать в качестве не очень сложного упражнения.

Пример. Из правила Хунда: суммарный спин имеет минимальное значение: $S = 1$. Итак, через антисимметрию волновой функции мы получили условия на спин (при этом в гамильтониане он не участвовал!).

\emph{Обменное вырождение есть явление сугубо квантовое, не имеет классического аналога. Оно связано с принципом тождественности частиц, выражающимся в антисимметрии волновой функции, и приводящим к особым корреляциям в движении электронов, в соответствии со статистикой Ферми-Дирака.}

Взаимодействие проявляется в том, что уровни энергии \textbf{зависят от суммарного спина}.

\textbf{Парагелий, ортогелий.} Уровни энергии атома гелия можно классифицировать по значению суммарного спина. $S = 1$~--- парагелий, $S = 0$~--- ортогелий.

С точки зрения излучения они себя ведут как два независимых вещества. Квантовые переходы возможны только если $\Delta S = 0$.

\emph{В основном состоянии ($n = 1, k = 1$) может находиться только парагелий.}

Этому основному состоянию отвечает следующая электронная конфигурация : $1s1s = 1s^2$.

Ортогелий: $1s2s$. Это состояние метастабильное, переход к основному состоянию занимает порядка года.

\textbf{Терм.} Это уровни энергии, соответствующие $S, L, J$.
$$
    {}^{2S + 1} {L}_{J} \qquad \to \qquad \boxed{
    {}^{1} {S}_{0}
}
$$

\section{Квантование свободного электромагнитного поля}
\subsection{Классическое поперечное электромагнитное поле}
Такое поле описывается векторным потенциалом $\bf A(\bf r, t)$, который удовлетворяет уравнению Даламбера:
$$
    \left(
    \Delta - \dfrac{1}{c^2} \dfrac{\pd^2}{\pd t^2}
\right) \bf A (\bf r, t) = 0
$$
Выберем некоторую калибровку потенциала.
$$
    \mathrm{div} \, \bf A = 0
$$
В этой калибровке
\def \rot {\mathrm{rot}\,}
\def \div {\mathrm{div}\,}
\def \H {\mathcal H}
\def \E {\mathcal E}
$$
    \left\{
      \begin{array}{lcl}
        \bf {\mathcal H} & = & \rot A ,\\
        \bf {\mathcal E} & = & -\dfrac{1}{c} \ud{\bf A}{t}.
      \end{array}
    \right.
$$
$$
    \bf A (\bf r, t) = \dfrac{1}{L^{3/2}} \sum_{\bf k} \bf A(\bf k, t) e^{i \bf k \bf r}
$$
Рассматриваются граничные условия типа коробки,
$$
    e^{i k_i (x_i + L)} = e^{i k_i x_i}, \qquad k_i = \dfrac{2 \pi}{L} n_i, \quad n_i = 0, \pm 1, \pm 2, \ldots
$$
Подставляя полученную функцию в уравнение Даламбера, получаем уравнение
$$
    \ddot{\bf A} (\bf k, t) + \omega^2 \bf A(\bf k, t) = 0, \qquad \omega = k \cdot c
$$
Каждая амплитуда Фурье по отдельности удовлетворяет уравнению гармонического осциллятора.

Запишем общее решение:
$$
     \bf A (\bf k, t) = \bf A(\bf k) e^{-i \omega t} + B(\bf k) e^{i \omega t}
$$
Когда мы подставляем решение в исходное уравнение, величины $\bf \E, \bf \H$ вещественные, но про потенциал никто ничего не обещает. Для удобства хотим, чтобы $\bf A$ тоже был вещественным. Для этого достаточно, чтобы
$$
    B (\bf k) = A^{\ast} (- \bf k)
$$ 
В итоге получим после подстановки в решение:
\begin{equation}
    \bf A(\bf r, t) = \dfrac{1}{L^{3/2}} \sum_{\bf k}
\left(
    \bf A(\bf k) e^{-i(\omega t - \bf k \bf r)} + \bf A^{\ast} (\bf k) e^{i (\omega t - \bf k \bf r)}
\right) 
\label{eq::solution}
\end{equation}

\textbf{Полная энергия системы.}

Обозначим полную энергию системы 
$$
    H = \dfrac{1}{8 \pi} \int d^3 \bf r (\bf \H^2 + \bf \E^2)
$$

Зная потенциал, можно получить полную энергию системы. Из условия ортогональности экспонент можно получить упрощение для слагаемых $\bf \H^2, \bf \E^2$:
$$
    \int d^3 x e^{i (\bf k - \bf k') \bf r} = \dta_{\bf k \bf k'}
$$
Если применить условие $\div \bf A$ = 0 к выражению~\eqref{eq::solution}, то получим \emph{условие поперечности}:
$$
    \big(\bf k \bf A(\bf k)\big) =     \big(\bf k \bf A^{\ast}(\bf k) \big) = 0
$$

Проделывая всю эту процедуру, получаем ответ\footnote{Для удобства, которое мы поймём позже, не приводим подобные слагаемые}:
$$
\boxed{
    H = \dfrac{1}{4 \pi} \sum_{\bf k} k^2 \big(
    \bf A^{\ast} (\bf k) \bf A(\bf k) + 
\bf A (\bf k) \bf A^{\ast}(\bf k)
\big)
}
$$
Переходим к нормированным амплитудам.
$$
    \bf A(\bf k) = \sqrt{\dfrac{2 \pi c^2 \hbar}{\omega}}
$$
$$
    \bf A(\bf r, t) = \dfrac{1}{L^{3/2}} \sum_{\bf k}\sqrt{\dfrac{2 \pi c^2 \hbar}{\omega}}
\left(
    \bf{a}_{\bf k} e^{-i(\omega t - \bf k \bf r)} + \bf{a}_{\bf k}^{\ast} (\bf k) e^{i (\omega t - \bf k \bf r)}
\right)
$$
В таком виде:
$$
    H = \sum_{\bf k} \dfrac{\hbar \omega}{2} 
(
    \bf a_{\bf k}^{\ast} \bf a_{\bf k} + \bf a_{\bf k} \bf a_{\bf k}^{\ast}
)
$$
\textbf{Поляризация электромагнитных волн.}

Когда мы записываем амплитуду $\bf a = \{a_1, a_2, a_3\}$, нужно учесть, что между ними есть связь:
$$
    (\bf k \bf a_{\bf k}) = (\bf k \bf a_{\bf k}^{\ast}) = 0
$$
Амплитуды Фурье перпендикулярны волновому вектору. Две независимые компоненты отвечают поляризациям.

Проиллюстрируем процесс выбора этих компонент. Пусть
$$
    \bf a = C_1 \bf e_1 + C_2 \bf e_2,
$$ 
где $\bf e_1, \bf e_2$ выбираются из условий
$$
    (\bf k \bf e_1) = (\bf k \bf e_2) = (\bf e_1 \bf e_2) = 0
$$
$$
    \bf e_2 = \dfrac{[\bf k \times \bf e_1]}{|\bf k|}
$$
\begin{itemize}
  \item Если $\bf e_1, \bf e_2$ вещественные, то поляризация линейная.
  \item Если $\bf e_{\pm} = \dfrac{1}{\sqrt 2} (\bf e_1 \pm i \bf e_2)$, то это циркулярная поляризация.
\end{itemize}

Новый вид полной энергии поля.
$$
    H = \sum_{\bf k} \sum_{\lambda = 1, 2} (C_{k \lambda}^{\ast} C_{k \lambda} + C_{k \lambda} C_{k \lambda}^{\ast}) \dfrac{\hbar \omega}{2}
$$
\subsection{Вторичное квантование электромагнитного поля}
\textbf{Что такое первичное квантование?} Результатом первичного квантования было уравнение Шрёдингера, которое сопоставило частице волну. Введение квантовых операторов, которые подчинаются уравнению эволюции в форме Гайзенберга
$$
    \dfrac{d \hat F}{d t} = \dfrac{i}{\hbar} [\hat H, \hat F]
$$ 
Вторичное квантование~--- это получение описания поля в терминах частиц, то есть сопоставление волновому процессу частиц. В данном случае это будут фотоны.

У электромагнитного поля <<первичного квантования>> нет, поэтому термин не очень удачный.

В результате мы надеемся получить описание поля при помощи векторов состояний, подчиняющихся принципу суперпозиции. При этом векторы $\bf A, \bf \H, \bf \E$ станут операторами.

Нужно наложить соответствующие коммутационные соотношения (пока держим это в голове).

Заменим числа $C$ на операторы:
$$
\left\{
  \begin{array}{lcl}
    C_{\bf k \lambda} & \to & \hat C_{k \lambda} = \hat C_{\alpha} ,\\
    C^\ast_{\bf k \lambda} & \to & \hat C_{k \lambda}^{+} = \hat C_{\alpha} .
  \end{array}
\right.
$$ 
Эти операторы должны удовлетворять уравнению Гайзенберга.
$$
    \hat C_{\alpha} (t) = \hat C_{\alpha} e^{- i \omega t}, \quad
    \hat C_{\alpha}^{+} (t) = \hat C_{\alpha}^+ e^{+ i \omega t}
$$
Дифференцируя, получаем
\begin{eqnarray*}
    - \omega C_{\alpha} &=& \dfrac{1}{\hbar} (\hat H \hat C_{\alpha} - \hat C_{\alpha} \hat H)\\
&=& \dfrac{1}{\hbar} \sum_{\alpha'} \dfrac{\hbar \omega'}{2}\Big(
    (\hat C_{\alpha'}^{+} \hat C_{\alpha'} + \hat C_{\alpha'} \hat C_{\alpha'}^{+}) \hat C_{\alpha} - 
\hat C_{\alpha}(\hat C_{\alpha'}^{+} \hat C_{\alpha'} + \hat C_{\alpha'} \hat C_{\alpha'}^{+}) \hat C_{\alpha}
\Big)\\
&=& \sum_{\alpha'} \dfrac{\omega'}{2} \Big(
\begin{array}{c}
      \hat C_{\alpha'}^{+} (\hat C_{\alpha'} \hat C_{\alpha} - \hat C_{\alpha} \hat C_{\alpha'}) +
\hat C_{\alpha'} (\hat C_{\alpha'}^{+} \hat C_{\alpha} - \hat C_{\alpha}\hat C_{\alpha'}^{+})
 \\
  - (\hat C_{\alpha} \hat C_{\alpha'}^{+} - \hat C_{\alpha'}^{+} \hat C_{\alpha}) \hat C_{\alpha'}
- (\hat C_{\alpha} \hat C_{\alpha'} - \hat C_{\alpha' \hat C_{\alpha}}) \hat C_{\alpha'}^{+}
\end{array}
\Big)
\end{eqnarray*}
Коммутационные соотношения
$$
\boxed{
    \left\{
      \begin{array}{lcl}
        [\hat C_{\bf k \lambda}, \hat C_{\bf k' \lambda'}] & = & [\hat C_{\bf k \lambda}^+, \hat C_{\bf k' \lambda'}^+] ,\\{}
        [\hat C_{\bf k \lambda}, \hat C_{\bf k' \lambda'}^+] & = & \dta_{\bf k \bf k'} \dta_{\lambda \lambda'}.
      \end{array}
    \right.
}
$$
$$
    \boxed{
    \hat H = \sum_{\bf k} \sum_{\lambda} \hbar \omega \left(
    \hat C_{\bf k \lambda}^+ \hat C_{\bf k \lambda} + \dfrac{1}{2}
\right)
}
$$
Эти коммутационные соотношения и дают нам вторичное квантование.    