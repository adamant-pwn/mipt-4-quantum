\subsection{Пространство состояний}
Изучим структуру оператора Гамильтона.
$$
    \Hat H = \sum_{\bf k} \sum_{\lambda} \hbar \omega (\hat C_{\bf k \lambda}^{+} \hat C_{\bf k \lambda} + \frac12)
$$
Он раскладывается в сумму гамильтонианов гармонических осцилляторов, коих бесконечно много
$$
    \hat H = \sum_{\bf k \lambda} \hat H_{\bf k \lambda}
$$
Имеется полная система собственных векторов у каждого из $\hat H_{\bf k \lambda}$
$$
    \qu{N_{\bf k \lambda}}
$$
Их тензорное произведение даёт собственный вектор для исходного гамильтониана
$$
    \qu{\ldots N_{\bf k \lambda} \ldots} \to \{N_{\bf k \lambda}\}
$$
Каждый собственный вектор характеризуется набором чисел $N_{\bf k \lambda}$, где
$$
    \hat N_{\bf k \lambda} \eqdef \hat C_{\bf  \lambda}^+ \hat C_{\bf k \lambda}
$$
\def \Ckl {\hat C_{\bf k \lambda}}
$$
    \Hat H \qu{\ldots N_{\bf k \lambda} \ldots} = E_{\{N_{\bf k \lambda}\}} \qu{\ldots N_{\bf k \lambda} \ldots}
$$
Аналогичные формулы были описаны в начале курса для гармонического осциллятора. Каждое следующее состояния можно получить из нулевого состояния по формуле
$$
    \prod_{\bf k \lambda} \dfrac{(\Ckl^+)^{N_{\bf k \lambda}}}{\sqrt{N_{\bf k \lambda}!}} \qu{0}
$$
\textbf{Корпускулярная интерпретация.}

\begin{itemize}
  \item $\{ N_{\bf k \lambda} \}$ --- число заполнения --- это число фотонов с определённым значением волнового вектора $\bf k$ и поляризации $\lambda$.
  \item $\Ckl, Ckl^+$~--- операторы рождения и уничтожения фотонов состояния с волновым вектором $\bf k$, поляризацией $\lambda$.
      \def \Nkl{N_{\bf k \lambda}}
      $$
        \Ckl^+ \qu{\ldots N_{\bf k \lambda} \ldots} = \sqrt{N_{\bf k \lambda} + 1} \qu{\ldots \Nkl \ldots}
      $$
      $$
        \Ckl \qu{\ldots N_{\bf k \lambda} \ldots} = \sqrt{N_{\bf k \lambda}} \qu{\ldots \Nkl \ldots}
      $$
  \item $\qu{0}$~--- состояния без реальных фотонов. Среднее число фотонов в таком состоянии равно нулю, но тем не менее возможны виртуальные процессы типа аннигиляции. Это есть особое состояние электромагнитного поля с наименьшей энергией, и фотоны могут рождаться из этого вакуума.
  \item Все рассматриваемые кванты электромагнитного поля являются неразличимыми, про каждый индивидуальный фотон ничего сказать невозможно. В каждом состоянии число этих фотонов может изменяться от $0$ до $\infty$, то есть рассматриваются \emph{бозоны}.
\end{itemize}

\section{Взаимодействие квантовой системы с электромагнитным излучением}
% Квантовая теория излучения. Под квантовой системой будем подразумевать систему типа атома. В частности, часто можно рассматривать атом водорода.
\def \Ckl {\hat C_{\bf k \lambda}}
Как квантовая, так и классическая теория хочет определить две вещи.
\begin{itemize}
  \item \emph{Частота излучения.} В квантовой теории $\omega_{nn'} = \dfrac{E_n - E_{n'}}{\hbar}$.
  \item \emph{Мощность излучения.} Это есть энергия, которая излучается в единицу времени. Мощность отличается от интенсивности излучения. Это отличие имеет значение для релятивистских частиц. Об этом подробнее можно прочитать в задачнике.
\end{itemize}

\subsection{Основные уравнения и приближения}
В основе теории лежит временное уравнение Шрёдингера.
\def \bf{\boldsymbol}
$$
    \left\{ i \hbar \ud{}{t} - \dfrac{1}{2m} \big(
        \hat {\bf p} - \dfrac{e}{c} \bf A
    \big)^2 - U(r) \right\} \Psi(\bf r, t) = 0
$$
$$
    \bf A(\bf r, t) = \bf A_{\text{внеш}} + \bf A_{\text{излучения}}
$$
При этом
$$
    \hat{\bf A}(\bf r, t) = \dfrac{1}{L^{3/2}} \sum_{\bf k}\sum_{\lambda} \sqrt{\dfrac{2 \pi c^2 \hbar}{\omega}}
    \{
        \bf e_{\bf k \lambda} \Ckl e^{-i(\omega t - \bf k \bf r)} + (\bf e_{\bf k \lambda} \Ckl e^{-i(\omega t - \bf k \bf r)} )^{\ast}
    \}
$$
здесь $\bf e_{\bf k \lambda}$~--- вектор поляризации.

Кулоновская калибровка.
$$
    \mathrm{div}\, \bf A = 0
$$
Возводя в квадрат, будем считать, что квадрат потенциала пренебрежимо мал. В кулоновской калибровке операторы $\hat{\bf p}, \hat{\bf A}$ коммутируют.  Таким образом, задачу можно сформулировать следующим образом:
$$
    \left\{
        i \hbar \ud{}{t} - \hat H^{(0)} - \hat V^{(0)}
    \right\} \Psi(\bf r, t) = 0
$$
$$
    \hat H^{(0)} = \dfrac{\hat{\bf p}^2}{2m} + U(\bf r), \quad \hat V^{(1)} = -\dfrac{e}{mc} \bf A \hat{\bf p}
$$
Оператор возмущения записывается в виде суммы двух слагаемых
$$
    \hat V^{(1)} = \hat V + \hat V^+
$$
$$
    \hat V^{+} = -\dfrac{e}{mc L^{3/2}} \sum_{\bf k} \sum_{\lambda} \sqrt{\dfrac{2 \pi c^2 \hbar}{\omega}}
    \left(
        e^{i (\omega t - \bf k \bf r)} \Ckl^+ (\bf e_{\bf k \lambda}^\ast \hat {\bf p})
    \right)
$$
\subsection{Временная теория возмущений. Золотое правило Ферми}
Рассмотрим одно слагаемое и выражения для $\hat V^+$,
$$
    \hat V_{\bf k} = \hat V_{\bf k}^{0+} e^{i\omega t}
$$
При помощи временной теории возмущений найдём вероятность переходов в единицу времени.
$$
    W_{nn'} = \dfrac{2 \pi}{\hbar} |V_{\bf k nn'}^{0+}|^2 \dta(E_{n'} - E_n + \hbar \omega)
$$
При фиксации состояний излучение будет идти только на частоте $\omega \to \omega_{nn'}$.

\def \Nkl {N_{\bf k \lambda}}

В невозмущённой задаче
$$
    \qu{E_n} \otimes \qu{\ldots \Nkl \ldots} = \qu{E_n; \, \ldots N_{\bf k \lambda} \ldots}
$$
Рассмотрим однофотонный процесс, для него выполняется условие
$$
    \omega = c k
$$
Хотим, чтобы одновременно излучался фотон, имеющий ровно такую связь.

Что означает, что должен излучаться только один фотон?

Общая структура матричного элемента должна иметь вид
$$
    V_{nn'}^{0+} = \qtri
    {E_n; \, \ldots \Nkl' \ldots}
    {\hat V^{0+}}
    {E_n; \, \ldots \Nkl \ldots}
$$
Число фотонов должно увеличиться на единицу:
$$
    \Nkl' = \Nkl +1
$$
$$
    \boxed{
        \qtri
        {\ldots N_{\bf k \lambda} + 1 \ldots}
        {\Ckl^+}
        {\ldots \Nkl \ldots} = \sqrt{\Nkl + 1}
    }
$$
При такой постановке начального и конечного условий вклад матричного элемента даёт единственное слагаемое из всей суммы.

Пока мы не знаем энергию. Введём обозначение $ \bf p_{n'n}$. Это среднее значение оператора $\hat{\bf p}$.
$$
\boxed{
    \bf p_{n'n} = \int d^3 \bf r \Psi_{n'}^{(0) \ast} (\bf r) e^{-i \bf k \bf r} \hat{\bf p} \Psi_n^{(0)} (\bf r)
}
$$
Находим квадрат модуля матричного элемента.
$$
    |V_{n'n}^{0+}|^2 = \left(\dfrac{e}{mc} \right)^2 \dfrac{1}{L^3} \left(
        \dfrac{2 \pi c^2 \hbar}{\omega}
    \right) |\bf e_{\bf k \lambda}^{\ast} \bf p_{n'n}|^2 \cdot (\Nkl + 1)
$$
Пока мы не готовы провести все преобразования до конца. При квантовании электромагнитного поля мы рассматривали его в кубе большого размера $L$  циклическими граничными условиями. При этом как энергия, так и компоненты волнового вектора дискретны. В обычной жизни фотоны имеют непрерывный спектр.

Дельта-функция должна сниматься в результате интегрирования по энергиям конечных состояний. Должен получиться множитель, представляющий энергетическую плотность конечных состояний.

Расстояние между ближайшими значениями компонент волнового вектора
$$
    \Delta k_x = \Delta k_y = \Delta k_z = \dfrac{2\pi}{L}
$$
Таким образом, прорабатывая переход от суммам к интегралу, получаем
$$
    \dfrac{1}{L^3} \sum_{\bf k} F(\bf k) = \sum_{\bf k} F(\bf k) \dfrac{\Delta k_x}{2 \pi} \dfrac{\Delta k_y}{2 \pi} \dfrac{\Delta k_z}{2 \pi} \to \int (\cdots)
$$
$$
\boxed{
    \dfrac{1}{L^3} \sum_{\bf k} F(\bf k) \underset{L\to \infty}\to \int \dfrac{d^3 \bf k}{(2\pi)^3} F(\bf k)
}
$$
В сферической системе координат
$$
    d^3 \bf k = k^2 \; dk \; d \Omega
$$
По углам вылета фотона $\Omega$ интегрировать пока не будем. Получим дифференциальное сечение вероятности.

Совершая эти преобразования, учитывая $E_k = \hbar \omega = \hbar k c$, окончательно получаем
\begin{eqnarray*}
    d W_{nn'} &=& \dfrac{2 \pi}{\hbar} \left(
        \dfrac{e}{mc}
    \right)^2 \dfrac{1}{L^3} \left(
        \dfrac{2 \pi c^2 \hbar}{\omega}
    \right)
    |\bf e_{\bf k \lambda}^\ast \bf p_{n'n}|^2 (\Nkl + 1) \cdot \dfrac{L^3}{(2\pi)^3} \dfrac{k^2 dk d \Omega}{dE_k}\\
    && \boxed{\hbar \omega_{nn'} = E_n - E_{n'}}\\
    &=& \left(
        \dfrac{e}{m}
    \right)^2 \left(
        \dfrac{1}{\omega}
    \right)
    |\bf e_{\bf k \lambda}^\ast \bf p_{n'n}|^2 (\Nkl + 1) \cdot \dfrac{1}{(2\pi)} \dfrac{k^2 dk d \Omega}{dE_k}
\end{eqnarray*}
Кроме того,
$$
    \dfrac{k^2 dk d \Omega}{d E_k} = \dfrac{k^2 d  k d \Omega}{\hbar c dk} = \dfrac{\omega^2 d \Omega}{\hbar c^3}
$$
Подставляя, получаем
$$
    \boxed{
        d W_{nn'}^{\lambda} = \dfrac{e^2}{\hbar c} \cdot \dfrac{\omega_{nn'}}{2 \pi}
        \left|
            \dfrac{(\bf e_{\bf k \lambda}^\ast \bf p_{n'n})}{mc}
        \right|
        (\Nkl + 1) d \Omega
    }
$$
Эта формула предполагает излучение фотонов с определённой поляризацией. Хочется уметь измерять излучение всех фотонов независимо от поляризации. Просуммируем вероятность по поляризациям конечных фотонов.

\begin{eqnarray*}
    \sum_{\lambda = 1, 2} (\bf e_{\bf k \lambda})_i (\bf e_{\bf k \lambda})_j \Nkl &=&
    A(\bf k) \dta_{ij} + B(\bf k) k_i k_j
\end{eqnarray*}
Ответ не может иметь другой вид, потому что это единственно возможные симметричные тензоры. Здесь $A, B$~--- некоторые коэффициенты. Найдём их.

\begin{itemize}
  \item Поперечность поля. $(\bf e \bf k) = 0$
  $$
    0 =\Big( A(\bf k) + B(\bf k) k^2 \Big) k_j
  $$
  $$
    B(\bf k) = -\dfrac{A(\bf k)}{k^2}
  $$
  \item $(\bf e^{\ast} \bf e) = 1$.
  $$
    \sum_{\lambda} \Nkl = N_1 + N_2 = 3A(\bf k) + B(\bf k) k^2 = 2A(\bf k)
  $$
\end{itemize}

Таким образом,
$$
    A(\bf k) = \dfrac{N_1 + N_2}{2} \eqdef N(\bf k)
$$
Это усреднённое по поляризациям число начальных фотонов.

Сформулируем само правило.
$$
    \boxed{
        \sum_{\lambda = 1, 2} (\bf e_{\bf k \lambda})_i(\bf e_{\bf k \lambda}^\ast)_j (\Nkl + 1)
        = (N(\bf k) + 1) \left(
            \dta_{ij} - \dfrac{k_i k_j}{k^2}
        \right)
    }
$$
Применим эту формулу к чему-нибудь. Обозначим для краткости  $\bf k^0 \eqdef \bf k / |\bf k|$
$$
    \sum_{\lambda = 1,2 } (\bf e_{\bf k \lambda})_i(\bf e_{\bf k \lambda}^\ast)_j
    (\bf p _{n'n}^\ast)_i(\bf p _{n'n})_j (\Nkl + 1) =
    (N(\bf k) + 1)
    \Big(
        (\bf p_{n'n}^\ast \bf p_{n'n}) - (\bf p_{n'n}^\ast \bf k^0)(\bf p_{n'n} \bf k^0)
    \Big)
$$
Если ввести систему координат, в которой ось $z$ направлена вдоль волнового вектора, то выражение приобретает вид
$$
\boxed{
    \sum_{\lambda = 1,2 } (\bf e_{\bf k \lambda})_i(\bf e_{\bf k \lambda}^\ast)_j
    (\bf p _{n'n}^\ast)_i(\bf p _{n'n})_j (\Nkl + 1) =
    | [\bf k^0 \times \bf p_{n'n}] |^2 (N(\bf k) + 1)
}
$$
$$
    d W_{nn'} = \dfrac{e^2}{\hbar c} \cdot \dfrac{\omega_{nn'}}{2 \pi} S_{\text{изл.}} d\Omega,
$$
где
$$
    S_{\text{изл.}} = \left|
        \dfrac{[\bf k \times \bf p_{n'n}]}{mc}
    \right|^2 (N(\bf k) + 1)
$$
\textbf{Замечание.} Эта формула показывает, что дифференциальная вероятность состоит из двух слагаемых. Первое слагаемое пропорционально числу квантов электромагнитного поля, уже бывших в системе. Таким образом, имеется некоторое поле, излучение которого, воздействуя на нашу систему, вызывает излучение фотона (индуцированное излучение).

Второе слагаемое пропорционально единице. Это слагаемое обеспечивает возможность излучения в случае, если в системе фотонов не было (спонтанное излучение).

\subsection{Излучение в дипольном приближении}
Именно эта часть имеет практический смысл.

Приблизим экспоненту рядом Тейлора, затем отбросим всё кроме единицы.
$$
    \bf p_{n'n} \approx \int d^3 x \Psi_{n'}^{(0) \ast} (\bf r) (1 - i (\bf k \bf r) + \ldots) \hat{\bf p} \Psi_n^{(0)} (\bf r)
$$
$$
    |\bf k \bf r| \ll |\bf k| \cdot |\bf r| \sim \dfrac{\omega}{c} a \sim \dfrac{\hbar \omega}{\hbar c} a \sim \dfrac{e^2}{\hbar c} = \alpha \approx \dfrac{1}{137} \ll 1
$$
Если учитывать это слагаемое, получим \emph{квадрупольное излучение}.

Есть такая формула
$$
    \dfrac{d \hat{\bf r}}{dt} = \dfrac{i}{\hbar} [\hat H, \hat{\bf r}] = \dfrac{\hat{\bf p}}{m}
$$
Отсюда
\begin{eqnarray*}
    \dfrac{1}{m} \bf p_{n'n} &=& \dfrac{i}{\hbar}
    \qtri{n'}{\hat H \hat{\bf r} - \hat{\bf r} \hat H}{n}\\
    &=& \dfrac{i}{\hbar} (E_{n'} - E_n) \qtri{n'}{\bf r}{n}\\
    &=& -i \omega_{n'n} \bf r_{n'n}
\end{eqnarray*}
$$
    \boxed{
        \dfrac{1}{m} \bf p_{n'n} = -i \omega_{nn'} \bf r_{n'n}
    }
$$
Для дипольного момента $\hat {\bf d} = e \hat{\bf r}$.

В дипольном приближении дифференциальная вероятность принимает вид.
$$
\boxed{
    d W_{nn'}^{\text{дип}} = \dfrac{e^2}{\hbar c} \dfrac{\omega_{nn'}^3}{2 \pi c^2} 
    |[\bf k^0 \times \bf r_{n'n}]|^2 (N(\bf k) + 1) d\Omega
}
$$
Полная вероятность получается интегрированием по углам.
$$
    W^{\text{дип.}}_{nn'}
$$