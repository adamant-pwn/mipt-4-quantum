***

Нужно получить простую физическую одночастичную интерпретацию. Такая теория должна использовать состояния с одним знаком энергии, и использовать такие операторы, которые не смешивают состояния с разными знаками энергии:
$$
    E > 0
$$ 
Оказывается, что если в начальный момент состояния с отрицательным знаком энергии исключить, то далее они и не появятся. Сейчас мы поймём, какие приближения могут привести нас к такой теории.

\begin{itemize}
  \item \textbf{Свободная частица.} Оператор знака $\hat \Lambda$
  $$
    \Hat \Lambda = \dfrac{\hat H_D}{(\hat H_D^2)^{1/2}} = \dfrac{c(\hat{\bf \alpha} \bf p) + \hat \beta m c^2}{E_p}, \quad |E| = E_p
  $$
  Он обладает такими свойствами:
  \begin{itemize}
    \item $\hat \Lambda^{+} = \hat \Lambda = \hat \Lambda^{-1}$, так как $\hat \Lambda^2 = \hat 1$
    \item $[\hat \Lambda \hat H_D] = [\hat \Lambda, \hat{\bf p}] = 0$
    \item $\hat \Lambda \Psi(\xi) = \xi \Psi(\xi), \quad \xi = \pm 1$
  \end{itemize}
  \item \textbf{Чётные операторы.}

    \Def
  Любой оператор можно представить в виде \emph{чётного} и \emph{нечётного} оператора
    $$
        \hat F = \big[\hat F\big] + \big\{ \hat F \big\}
    $$ 
  При этом
  $$
     \big[ \hat F \big] \Psi(\pm) = \tilde \Psi(\pm),
  $$
  $$
     \big\{ \hat F \big\} \Psi(\pm) = \tilde \Psi(\mp)
  $$
  То есть к чётности знака энергии прибавляется чётность оператора. Теория строится при помощи применения чётных операторов.
  
  Найдём правило для представления произвольного оператора $\Hat F$ в виде такой суммы.
  $$
    \hat F \psi(\pm) = [\hat F] \Psi(\pm) + \{\hat F\} \Psi(\pm)
  $$
  \def \La {\Lambda}
  \begin{eqnarray*}
    \hat \La \hat F \hat \La \Psi(\pm) &=& \pm \hat \La \hat F \Psi(\pm)\\
    &=& [\hat F] \Psi(\pm) - \{\hat F\} \Psi(\pm)
  \end{eqnarray*}
  
  Отсюда получаем искомое выражение
  $$
    \left\{
      \begin{array}{lcl}
        [\hat F] & = & \frac12 \left( \hat F + \hat \La \hat F \hat \La \right) ,\\
        \{\hat F\} & = & \frac12 \left( \hat F - \hat \La \hat F \hat \La \right).
      \end{array}
    \right.
  $$
  \textbf{Выводы.}
  \begin{itemize}
    \item Чётная часть оператора является интегралом движения\footnote{Лишь в том случае, если оператор, от которого берётся чётная часть, не является функцией координат.}, коммутирует с гамильтонианом, и имеет простую физическую интерпретацию.
    \item Нечётная часть оператора антикоммутирует с гамильтонианом\footnote{При тех же предположениях}, осциллирует с частотой $\dfrac{2 E_p}{\hbar}$, и является причиной <<дрожащего движения>>.
  \end{itemize}
  Покажем, что нечётная часть антикоммутирует с гамильтонианом.
  \begin{eqnarray*}
    \dfrac{\hat H_D}{E_p} \{ \hat F \} &=& \dfrac{\hat H_D}{2 E_p} (\hat F - \hat \La \hat F \hat \La)\\
    &=& \dfrac{1}{2} (\hat \La \hat F - \hat F \hat \La) 
  \end{eqnarray*}
  \begin{eqnarray*}
    \{ \hat F \} \dfrac{\hat H_D}{E_p}  &=&  (\hat F - \hat \La \hat F \hat \La)\dfrac{\hat H_D}{2 E_p}\\
    &=& \dfrac{1}{2} (\hat F \hat \La - \hat \La \hat F)
  \end{eqnarray*}
  Если оператор не является функцией координат, то используя представление Гайзенберга, можно записать:
  $$
    \dfrac{d \{\hat F\}}{dt} = \dfrac{i}{\hbar} \big[
        \hat H_D, \{\hat F\}
    \big] = -\dfrac{2i}{\hbar}\{\hat F\} \hat H_D
  $$
  Получили дифференциальное уравнение. Если гамильтониан явно не зависит от времени $\left( \ud{\hat H_D}{t} = 0\right)$, то решая это уравнение, получаем
  $$
    \big\{
        \hat F(t)
    \big\} = \big\{
        \hat F(0)
    \big\} \exp \left(
        - \dfrac{2i \hat H_D t}{\hbar}
    \right)
  $$
  Выделим чётную часть у известных нам операторов, чтобы построить законченную версию теории.
  
  \textbf{Оператор скорости.}
  
  $$
    \hat v_x = c \hat \alpha_x
  $$
  Чётная часть:
  \begin{eqnarray*}
    [\hat v_x] &=& \dfrac{1}{2} \big(
        \hat v_x + \hat\La \hat v_x \hat \La
    \big)\\
    &=& \dfrac{1}{2} (\hat v_x \hat \La + \hat\La \hat v_x) \hat \La\\
    &=& \dfrac{c}{2 E_p} (\hat \alpha_x \hat H_D + \hat H_D \hat \alpha_x) \hat \La\\
    &=& \dfrac{c^2 \hat p_x}{E_p} \hat \La
  \end{eqnarray*}
  Если <<отбросить>> множитель $\hat \La$, получим формулу для релятивистской скорости, где вместо чисел стоят операторы.
  
  Нечётная часть оператора скорости:
  $$
    \{v_x\} = c \{\hat \alpha_x (0)\} \exp \left(
        -\dfrac{2i \hat H_D t}{\hbar}
    \right)
  $$
  
  \textbf{Операторы момента.}
  
  Орбитальный и спиновый операторы по отдельности не являются интегралами движения. Был получен ответ: это связано с интерференцией состояний.
  $$
    \dfrac{d \hat{\bf L}}{d t} = \dfrac{i}{\hbar} [\hat H_D, \hat{\bf L}] = c [\hat{\bf \alpha} \times \hat{\bf p}]
  $$
  При этом $ \hat \alpha$ является (с точностью до множителя) оператором скорости.
  $$
    \hat {\bf \alpha} = [\hat{\bf \alpha}] + \{ \hat{\bf \alpha}\}
  $$
  Отсюда
  $$
    \dfrac{d \hat{\bf L}}{d t} = c \Big[
        \big\{
            \hat{\bf \alpha(0)}
        \big\} \times \hat{\bf p}
    \Big] \exp \left(
        -\dfrac{2 i \hat H_D t}{\hbar}
    \right)
  $$
  $$
    \dfrac{d \hat{\bf S}}{d t} = \dfrac{i}{\hbar} [\hat H_D, \hat{\bf S}] = -с
    \Big[
        \big\{
            \hat{\bf \alpha(0)}
        \big\} \times \hat{\bf p}
    \Big] \exp \left(
        -\dfrac{2 i \hat H_D t}{\hbar}
    \right)
  $$
  Эти операторы не коммутируют, потому что они осциллируют во времени <<в противофазе>>. Суммарная фаза компенсируется, что интерпретируется, как сохранение полного момента.
  
  \textbf{Оператор координаты.}
  
  \begin{eqnarray*}
    \hat x \, : \, \hat v_x &=& \dfrac{d \hat x}{d t} = [\hat v_x] + \{\hat v_x\}\\
    &=& \dfrac{c^2 \hat p_x}{E_p} \La + c \{\bf \alpha(0)\} \exp \left(
        -\dfrac{2 i \hat H_D t}{\hbar}    
\right)
  \end{eqnarray*}
  $$
  \boxed{
    \hat x(t) = \hat x(0) + \dfrac{c^2 \hat p_x}{E_p} \La t + c \big\{
        \hat \alpha_x (0)
    \big\} \exp \left(
        -\dfrac{2i \hat H_D t}{\hbar}
    \right) (i\hbar) (2 \hat H_D)^{-1}
  }
  $$
  В классической физике первое слагаемое описывает <<макродвижение>>, то есть изменение положения со временем, движение по траектории.
  
  Второе слагаемое~--- это дрожание, или квантовые флуктуации траектории. Можно найти их (максимальную) амплитуду.
  $$
    |\{\hat x\}| \simeq \dfrac{c \hbar}{2 E_p} \leqslant \dfrac{c \hbar}{2 m c^2} \approx \dfrac{\hbar }{mc} = r_{qu}
  $$
  Квантовый радиус электроны равняется комптоновской длине волны.
  
  Координата квантовой релятивистской частица <<размазана>> на радиус порядка квантовой длины волны электрона.
  
  Условия, при которых можно отбросить осцилляцию, совпадают с условиями одночастичной интерпретации (появления электрон-позитронных пар). Если находиться на расстоянии квантового радиуса, то движение будет проходить по классической траектории. 
  
  Итог: релятивистская квантовая механика обычно подразумевает многочастичную модель.
\end{itemize}

\section{Квазирелятивистское приближение в теории Дирака}

\subsection{Уравнение Паули как нерелятивистский предел уравнения Дирака}

\def \bf {\boldsymbol}

Вспомним уравнение Дирака (во внешнем поле)
$$
    i \hbar \ud{}{t} \Psi(\bf r, t) = \left(
        c (\hat{\bf \alpha} \hat{\bf{\mathcal P}}) + \hat\beta mc^2 - e \Phi
    \right) \Psi(\bf r, t)
$$
Используем обозначения
$$
    \hat{\bf{\mathcal P}} = \hat{\bf p} + \dfrac{e}{c} \bf A, \quad \hat{\bf p} = - i \hbar \nabla, \quad -e < 0
$$
Потенциал имеет вид
$$
    A^{\mu} = \{\Phi, \bf A\}
$$
В стационарном случае
$$
    \Psi(\bf r, t) = \exp \left(
        -\dfrac{iEt}{\hbar}
    \right) \psi(\bf r), \quad E > 0
$$
$$
    \psi(\bf r) = \begin{pmatrix}
                    \psi(\bf r) \\
                    \chi(\bf r) \\
                  \end{pmatrix}
$$
Уравнение переписывается в виде системы
$$
    \left\{
      \begin{array}{lcl}
        c (\hat{\bf \sigma} \hat{\bf{\mathcal P}}) \chi(\bf r) & = & (E - mc^2 + e \Phi) \phi(\bf r) ,\\
        c (\hat {\bf \sigma}) \phi(\bf r) & = & (E + mc^2 + e \Phi) \chi(\bf r).
      \end{array}
    \right.
$$
Пользуемся следующими приближениями:
$$
    E = mc^2 + \mathcal E, \quad |\mathcal E + \exp| \ll mc^2
$$
Уравнения принимают вид
$$
    \left\{
      \begin{array}{lcl}
        c (\hat{\bf \sigma} \hat{\bf{\mathcal P}}) \chi(\bf r) & = & (\mathcal E + e \Phi) \phi(\bf r) ,\\
        c (\hat {\bf \sigma}) \phi(\bf r) & = & (2mc^2 + \mathcal E + e \Phi) \chi(\bf r).
      \end{array}
    \right.
$$
$$
    \boxed{
        \chi(\bf r) \simeq \dfrac{(\hat{\bf \sigma} \hat{\bf{\mathcal P}})}{2mc} \phi(\bf r)
    }, \qquad \dfrac{v}{c} \to 0
$$

Принято говорить, что нижние компоненты малы по сравнению с верхними двумя компонентами (здесь существенно, что энергия имеет положительный знак) 

После подстановки получаем уравнение на $\phi(\cdot)$:
$$
    (\mathcal E + e \Phi) \phi(\bf r) = \dfrac{1}{2m} (\hat{\bf \sigma} \hat{\bf{\mathcal P}})(\hat{\bf \sigma} \hat{\bf{\mathcal P}}) \phi(\bf r)
$$
Здесь нужно воспользоваться вспомогательной формулой (она получена на семинарах):
$$
    (\hat{\bf \sigma} \hat{\bf{\mathcal P}})(\hat{\bf \sigma} \hat{\bf{\mathcal P}}) = \hat{\bf{\mathcal P}}^2 + \dfrac{e \hbar}{c} (\hat{\bf \sigma} \bf {\mathcal H})
$$
Уравнение Паули:
$$
    \boxed{
        \mathcal E \phi(\bf r) = \left(
            \dfrac{\hat{\bf{\mathcal P}}^2}{2m} + \dfrac{e \hbar}{2mc} (\hat{\bf \sigma} \bf{\mathcal H}) - e \Phi
        \right) \phi
    }
$$
Для $\phi(\bf r)$ нужны 4 компоненты (спин, функции энергии могут принимать оба знака).
$$
    \xi = \pm 1, \quad s = \pm 1
$$
В этом уравнении Паули коэффициент $\dfrac{e \hbar}{2mc}$~--- это \emph{магнетон Бора}.

Достоинство теории Дирака состоит в том, что в ней уже содержатся спиновые свойства электрона, $g$-фактора Ланде, и т.д.

Свойства, которые раньше были получены с помощью принципа соответствия, теперь могут быть получены непостредственнно из теории Дирака.

Возможен переход в подпространство <<больших>> (верхних) компонент. При этом 3, 4-я координаты отбрасываются. Что при этом происходит с операторами? 

\subsection{Квазирелятивистское приближение в атоме водорода}

Физическая интерпретация спиновых и релятивистских эффектов в атоме водорода наиболее удобна в нерелятивистском приближении. Уравнение Дирака допускает \emph{точное решение} в случае атома водорода.

Ограничимся нерелятивистским приближением, и в его рамках получим все необходимые физические свойства, тонкую структуру, и так далее.

Что является малым параметром? Отношение скорости движения электрона к скорости света.
$$
    \dfrac{v_{\text{ат}}}{c} \approx \dfrac{p}{mc} \approx \dfrac{\hbar}{a} \dfrac{1}{mc} \approx \dfrac{\hbar}{mc} \dfrac{me^2}{h^2} = \dfrac{e^2}{\hbar c} = \alpha \approx \dfrac{1}{137} \ll 1
$$ 
Эта константа $\alpha$ называется \emph{постоянной тонкой структуры}.

\textbf{Терминология.}

Спиновые эффекты в магнитном поле проявляются в первом порядке по $v/c$, и этот случай называют \emph{нерелятивистским приближением}.

Электрическое поле: $(v/c)^2$, квазирелятивистское приближение. Это название необязательное, допустимо оба приближения называть нерелятивистскими (при этом необходимо указывать, какой порядок приближения имеется в виду).

Релятивистскую поправку можно учесть в классической задаче (Кеплер).
\begin{eqnarray*}
    E &=& (m^2c^4 + c^2 \bf p^2)^{1/2} \\
      &=& mc^2\left(1 + \left(\frac{p}{mc}\right)^2\right)^{1/2}\\
      &\approx& mc^2 + \dfrac{p^2}{2m} - \frac{p^4}{8m^3 c^2}
\end{eqnarray*}

Эта поправка приводит к прецессии Кеплерова эллипса вокруг одного из фокусов. (За период эллипс поворачивается вокруг одного из фокусов. Получается замкнутая кривая, которая называется \emph{розеточная траектория Зоммерфельда}. Это эффект специальной теории относительности.) 