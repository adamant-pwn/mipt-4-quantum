\chapter{Введение в релятивистскую квантовую механику}

\section{Релятивистские уравнения}

\subsection{Предварительные замечания}

Традиционная квантовая механика базируется на уравнении Шрёдингера и теории спиновых частиц Паули. Эта теория не является Лоренц-ковариантной. Принцип относительности требует Лоренц-ковариантности. Уравнение Шрёдингера является нерелятивистским.

В теории Паули спин электрона вводится <<руками>>. В хорошей теории он должен следовать из основ теории. Всё дело в том, что он имеет релятивистскую природу.

Попытаемся обобщить квантовую механику в релятивистском случае и сформулировать её в Лоренц-ковариантном виде.

На пути релятивистского продвижения мы также получим ответ на происхождение спина электрона, и продвинемся в обосновании гипотезы Уленбека и Гаудсмита.

Рассмотрим нашу модель (идеализацию). 
\begin{itemize}
  \item Частицы будем считать точечными. Под точечностью имеется в виду бесструктурность. В случае электронов точечность доказана с точностью до $10^{-18}$ см. В случае нуклонов, напротив, известно, что частицы имеют структуру в виде кварков. Тем не менее, будем предполагать точечность.
  \item Частицы будут изолированными. Если рассматривать частицы больших энергий, то могут происходить различные превращения частиц, и непонятно, можно ли выделить и рассмотреть одну частицу. Нужно указывать границы применимости такого подхода.
  \item Сохранение числа частиц. При больших энергиях ($E \gtrsim mc^2$ = 0.5 МэВ) это совершенно необязательно (могут рождаться электрон-позитронные пары).
\end{itemize}

Процессы, которые описывают рождение и аннигиляцию частиц, мы рассматривать не будем. О квантовой теории поля мы поговорим совсем немного.

Этот подход называется иногда <<одночастичное приближение>>. При этом основные уравнения будут релятивистскими, и энергии будут большими (как же так?).

\subsection{Общая идея построения релятивистских волновых уравнений}

\textbf{Уравнение Шрёдингера.}

Само по себе уравнение изначально является нерелятивистским. Будем $\psi(\cdots)$ обозначать нерелятивистскую волновую функцию, а большой буквой $\Psi(\cdots)$~--- релятивистскую.
\def \bf{\mathbf}
$$
    i \hbar \dfrac{\pd}{\pd t} \psi(\bf r, t) = \left(
        -\dfrac{\hbar^2}{2m} \nla^2 + U(\bf r)
    \right) \psi(\bf r, t)
$$ 
Заменим $E \to i \hbar \ud{}{t}$, $\bf p \to - i \hbar \nla$.

Какому же соотношению между энергией и импульсом отвечает уравнение Шрёдингера?
$$
    E = \dfrac{\bf p^2}{2m} + U(\bf r)
$$
Ясно, что по отношению к преобразованию Лоренца оно не может быть ковариантным. Более того, в релятивистской механике пространство и время равноправны. В квантовой механике это не так!

Заметим, что $E, \bf p$ являются компонентами 4-вектора импульса.
$$
    p^{\mu} = \left\{ \dfrac{E}{c}; \, \bf p \right\}
$$ 
При этом $\mu, \nu, \alpha, \ldots$ обозначают индексы в пространстве Минковского. Латинские индексы будут соответствовать Евклидову пространству.

\begin{itemize}
  \item 4-вектор $
    X^{\mu} = \left\{
        ct, \bf r
    \right\}
$
  \item Четырёхмерный градиент:
$$
    \pd_{\mu} = \dfrac{\pd}{\pd x^{\mu}} = \left\{
        \dfrac{1}{c} \dfrac{\pd}{\pd t}, \vec \nla
    \right\}
$$
\end{itemize}

После замены из принципа соответствия, получаем
$$
    \boxed{
        p^{\mu} = \left\{
            \dfrac{E}{c}, \bf p
        \right\} \to
        \left\{
            \dfrac{i \hbar}{c} \dfrac{\pd}{\pd t}, -i \hbar \vec \nla
        \right\} = 
        i \hbar \dta_{\mu}
    }
$$
Положим в основу нашего будущего уравнения релятивистскую связь.
$$
\boxed{
    E = \sqrt{c^2 \bf p^2 + m^2 c^4}
    }
$$
\begin{itemize}
  \item $E^2 = c^2 \bf p^2 + m^2 c^4$, откуда вытекает уравнение Клейна-Фока-Гордона
  \item А что если пытаться <<извлечь>> этот корень?
  
  Эта процедура называется линеаризация радикала.
  $$
    E = c \big(
        \hat{\vec \alpha} \bf p
    \big) + \hat \beta m c^2
  $$
  Из полученного уравнения вытекает уравнение Дирака.
\end{itemize}

\section{Уравнение Клейна-Фока-Гордона}

$$
    \left(
        -\dfrac{1}{c^2} \dfrac{\pd^2}{\pd t^2} + \Dta - \Big(
            \dfrac{mc}{\hbar}
        \Big)^2
    \right) \Psi(\bf r, t) = 0
$$
Здесь принято $\dfrac{mc}{\hbar} = k_0 = \dfrac{1}{\lam_{\text{компт}}}$.

\subsection{Различные способы записи уравнения}

Заметим, что в уравнении фигурирует оператор Д'Аламбера.
$$
    \square = \Dta - \dfrac{1}{c^2} \dfrac{\pd^2}{\pd t^2} = -\dfrac{\pd}{\pd x^{\mu}} \dfrac{\pd}{\pd x_\mu} = - \pd_\mu \pd^\mu
$$
Уравнение принимает вид:
$$
    (\square - k_0^2)\Psi = 0
$$
Или
$$
    (\pd_\mu \pd^\mu + k_0^2) \Psi = 0
$$
\subsection{Анализ уравнения}
Принцип соответствия требует выполнения релятивистской ковариантности.
$$
    x^\mu \to {x'}^{\mu} = \Lambda_\nu^{\mu} x^{\nu}
$$
$$
    \square \to \square' = \square
$$
При этом волновая функция преобразовывается как
$$
    \Psi(x) \to \Psi'(x) = \Psi(x)
$$
Волновая функция должна быть релятивистским скаляром. При преобразованиях Лоренца волновая функция не должна преобразовываться. 

$\bullet$ При $k_0 \to 0$ $(m \to 0)$ получаем уравнение д'Аламбера.
При $m \ne 0$ уравнение описывает квантовое поле, состоящее из мезонов. 

%\subsection{Сложности}
%To be announced. Всё дело в том, что это уравнение 2 порядка по времени.