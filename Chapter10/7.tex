\section{Эффект Зеемана}
\subsection{Исходные уравнения}
\def \bf{\boldsymbol}
\def \l {\ell}
Гамильтониан для водородоподобного атома в магнитном поле:
$$
\boxed{
    \Hat H = \dfrac{1}{2m} (\hat{\bf p} + \dfrac{e}{c} \hat A)^2 - \dfrac{Z e^2}{r} + \dfrac{e}{mc} (\hat{\bf S} \bf{\mathcal H}) + \hat V^{qu} = \hat H^{pauli} + \hat V^{qu}
}
$$
Проанализируем этот гамыильтониан.
$$
    \hat H = \hat H^{\text{без поля}} + \dfrac{e}{2mc} (\hat {\bf p} \bf A + \bf A
    \hat{\bf p}) + \dfrac{e^2}{2mc^2} \bf A^2 + \dfrac{e}{mc} \bf{\mathcal H} \hat{\bf S}
$$
\def \div{\mathrm{div}\,}
$$
    (\hat{\bf p} \bf A + \bf A \hat{\bf p}) = 2(\bf A \hat{\bf p}) + [\hat{\bf p}, \bf A] = 2(\bf A \hat{\bf p}) - i \hbar \div \bf A
$$
Калибровка Лоренца: $\div \bf A = 0$. При этом
$$
    \bf A = \dfrac{1}{2}[\bf{\mathcal H} \times \bf r] + \underbrace{\nabla f}_{=0}
$$
\def \rot{\mathrm{rot}\,}
$$
    \bf{\mathcal H} = \rot A
$$
Первое слагаемое:
$$
    \dfrac{e}{2mc} \left(
        2 \left(\frac12 [\bf{\mathcal H} \times \bf r]\right) \hat{\bf p}
    \right) = \dfrac{e}{2mc} (\bf{\mathcal H} \hat{\bf L})
$$
Второе слагаемое:
$$
    \dfrac{e^2}{8mc^2}[\bf{\mathcal H} \times \bf r]^2
$$
В результате гамильтониан тождественно преобразуется к виду
$$
\boxed{
    \hat H = \hat H^{\text{без поля}} + \dfrac{e}{2mc} \Big(
        \bf{\mathcal H} (\hat{\bf L} + 2 \hat{\bf S})
    \Big) + \dfrac{e^2}{8mc^2} [\bf{\mathcal H} \times \bf r]^2
}
$$
\subsection{Выбор приближения}
Подразумеваем, что внешнее поле слабое, и квадратичная поправка по полю гораздо меньше, чем линейная. Проверим, при каких условиях это может быть выполнено.

$$
    \dfrac{e^2}{mc^2} \mathcal H^2 \langle r^2 \rangle \ll \dfrac{e \hbar}{mc} \mathcal H
$$
Приходим к условию
$$
    \mathcal H \ll \left( \dfrac{e}{a^2} \right) \dfrac{\hbar c}{e^2} \approx 10^9 \text{\, Гс}
$$
Это условие уже встречалось в эффекте Штарка. Такие поля в земных условиях недостижимы. Это условие нельзя отбросить в случае сложных атомов. В них слагаемое, соответствующее линейному члену, может обратиться в ноль.

\def \H {\mathcal H}
Считаем, что $\bf{\mathcal H} \| Oz$, то есть $\bf{\mathcal H} = \{0, 0, \mathcal H\}$.

Обозначим ларморовскую частоту $\Omega = \dfrac{e\H}{2mc}$.
$$
\boxed{
    \hat H = \hat H^{\text{Б.П.}} + \Omega(\hat L_z + 2 \hat S_z)
}
$$

Поле называется <<слабым>>, если поправки, которые оно вносит, меньше, чем соответствующие квазирелятивистские.
$$
    \dfrac{e \H}{mc} \hbar \ll E_{\text{тонк}}^{(1)} = \left(
        \dfrac{e^2}{a}
    \right) \alpha^2
$$
Отсюда
$$
    \H \ll \dfrac{e}{a^2} \alpha \approx 1.2 \cdot 10^5 \text{\, Гс}
$$
Если поле слабое, открывается возможность повторного применения теории возмущения. Первый раз теория возмущений применяется для нахождения уровней энергий тонкой структуры. После этого найденное приближение используется для нахождения возмущения, вносимого магнитным полем.

Отсюда получается \emph{аномальный эффект} Зеемана. При этом снимается вырождение уровней по магнитному квантому числу, и очень богатый спектр.

В случае \emph{сильного поля} наоборот, квазирелятивистские поправки меньше, чем поправки поля.
$$
    \left(\dfrac{e}{a^2}\right) \alpha \ll \H \ll \left( \dfrac{e}{a^2} \right) \alpha^{-1}
$$
\subsection{Расчёт в случае слабого поля}
$$
    \hat H^{(0)} = \hat H^{\text{Б.П.}}, \quad \hat V = \Omega (\hat L_z + 2 \hat S_z)
$$
\Quest{Какие функции являются функциями нулевого приближения?}

\Ans Сферические спиноры.

Они являются общими собственными функциями операторов $\hat{\bf L}^2, \hat{\bf S}^2, \hat{\bf J}^2, \hat J_z$.
\def \l{\ell}
$$
    E_{\H}^{(1)} = \Omega \qtri{\l s j m_j}{\hat L_z + 2 \hat S_z}{\l s j m_j}
$$
\subsubsection{Теорема Вигнера-Эккарта}
Это теорема из теории Групп.

\Def Неприводимое представление группы трёхмерных вращений (см.выше)

Некоторое преобразование векторов состояний можно представить по-другому, как преобразование операторов. Примером могут являться представление Шрёдингера и Гайзенберга.

Рассмотрим векторный оператор (три наблюдаемых).
$$
    \hat{\bf V} = \{\hat V_x, \hat V_y, \hat V_z\}
$$
Векторный оператор~--- это оператор, который при вращении преобразуется как вектор.

Пусть оператор $\hat U(\phi)$ реализует неприводимое представление группы трёхмерных вращений.
$$
    \hat U^{-1}(\bf\phi) \hat{\bf V} \hat U (\bf \phi) = \hat V_{\phi} \hat{\bf V}
$$
Рассмотрим инфинитезимальное преобразование
$$
    (1 - \dfrac{i}{\hbar} \dta \phi (\bf n \hat{\bf J}) \hat{\bf V} (1 + \dfrac{i}{\hbar} \dta \phi(\bf n \hat{\bf J}))
$$
Как гласит формула Эйлера из механики:
$$
    = \hat{\bf V} - \dta \phi [\bf n \times \hat{\bf V}]
$$
Приравнивая обе части, получаем
$$
    -\dfrac{i}{\hbar} \dta \phi \big[
        (\bf n \hat {\bf J}), \hat V
    \big] = - \dta \phi [\bf n \times \hat{\bf V}]
$$
В тензорных обозначениях\footnote{Это верно для любых проекций $n_i$, поэтому можно сократить}:
$$
    n_i [\hat J_i, \hat V_j] = - i \hbar e_{jik} n_i \hat V_k
$$
$$
    [\hat J_i, \hat V_j] = i \hbar e_{ijk} \hat V_k
$$
Введём обозначения
$$
    \hat T_0^1 = \hat V_z, \qquad \hat V_{\pm 1}^{1} = \hat V_x \pm i \hat V_y
$$
В новых обозначениях
$$
\left\{
  \begin{array}{lcl}
    [\hat J_z, \hat T_m^1] & = & \hbar m \hat T_m^1 ,\\{}
    [\hat J_{\pm}, \hat T_m^1] & = & \hbar \sqrt{j(j+1) - m(m\pm 1)} T_{m \pm 1}^{1}.
  \end{array}
\right.
$$
Эти операторы ведут себя как базисные векторы неприводимого представления группы трёхмерных вращений.

Система $\hat T_m^j$, $j = -j, \ldots, j$, которая удовлетворяет коммутационным соотношениям, называется неприводимой системой тензорных операторов ранга $j$.

\Th Пусть дана система $\hat T_m^j$. Общий собственный вектор $\qu{\gamma_1 j_1 m_1}$ системы $\hat {\bf J}^2, \hat J_z$ с собственными значениями $\hbar j_1 (j_1 + 1)$, $\hbar m_1$.

Второй такой вектор: $\qu{\gamma_2 j_2 m_2}$, с собственными значениями $\hbar j_2 (j_2 + 1)$, $\hbar m_2$.

Тогда
$$
    \qtri{\gamma_2 j_2 m_2}{T_m^j}{\gamma_1 j_1 m_1} = \qs
    {j_2 m_2}{jm j_1 m_1} \qtri{\gamma_2 j_2}{\hat T_j}{\gamma_1 j_1}
$$
Здесь имеется в виду коэффициент сложения моментов Клебша-Гордана
$$
    \qs{j_2m_2}{jm j_1 m_1} = C_{jm j_1m_1}^{j_2m_2}, \quad \bf j_2 = \bf j + \bf j_1
$$
Смысл теоремы в том, что мы представили матричный элемент в виде произведения некоторых величин. Первый множитель описывает поведение при поворотах. Он зависит только от свойств векторов при вращении. Второй сомножитель~--- редуцированный матричный элемент, и содержит в себе всю информацию относительно характеристик квантовой системы и квантовой величины, от которой берётся среднее. Он не зависит ни от одного из магнитных квантовых чисел.

Проверим критерий того, что оператор является векторным. (?)

$$
    \qtri{\l s j' m_j'}{\hat L_z + 2 \hat S_z}{\l s j m_j} =
\qtri{\l s j' m_j'}{\hat J_z + \hat S_z}{\l s j m_j} = g \qtri{\l s j' m_j'}{\hat J_z}{\l s j m_j}
$$
$g$~--- это фактор Ланде. Можно его явно выразить.

$$
    E_{\H}^{(1)} = \Omega g \hbar m_j
$$
\textbf{Вычисление фактора Ланде.}

$$
    \langle
    \hat{\bf J} (\hat{\bf J} + \hat{\bf S})
\rangle
$$
\begin{eqnarray*}
    \hat{\bf J}^2 + (\hat{\bf J} \hat{\bf S}) &=& \langle
    \hat{\bf J}^2 + \frac12 (\hat{\bf J}^2 + \hat{\bf S}^2 - \hat{\bf L}^2)
\rangle\\
&=& \dfrac{\hbar^2}{2} (3 j(j+1) + s(s+1) - \l(\l+1))
\end{eqnarray*}
$$
    \sum_{i=1}^{3} \sum_{j} \qtri{\gamma m_j}{\hat{J_i}}{\gamma m_j'}
\qtri{\gamma m_j'}{\hat J_i + \hat S_i}{\gamma m_j}
= g \hbar^2 j(j+1)
$$
$$
    \omega = \dfrac{E_1^{(0)} - E_2^{(0)}}{\hbar} + \hbar \Omega (g_1 m_{j_1} - g_2 m_{j_2})
$$