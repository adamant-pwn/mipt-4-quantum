\subsection{Сложности в теории Клейна-Фока-Гордона}
Мы получим выражение для \emph{плотности вероятности} и для \emph{плотности тока вероятности}.

Мы получили выражение вида
$$
    (\pd_\mu \pd^\mu + k_0^2)\Psi = 0,
$$
где
%%должнен быть лям перечеркнутый
\def \xlam{\lam}
$$
    k_0 = \dfrac{mc}{\hbar} = \xlam_{\text{компт.}}^{-1}
$$
Рассмотрим комплексно сопряжённое данного уравнения:
$$
    \Psi^\ast (\pd_\mu \pd^\mu) \Psi - \Psi(\pd_\mu \pd^\mu) \Psi^\ast = 0
$$
Отсюда получаем закон непрерывности
\def \bf{\mathbf}
$$
    \dfrac{\pd}{\pd x^\mu} j^\mu = 0, \qquad j^\mu = \{c \rho, \bf j\}
$$
\def \div{\mathrm{div} \,}
\begin{equation}
    \boxed{
        \dfrac{\pd \rho}{\pd t} + \div  \bf j = 0
     }
     \label{eq::continuous_eq}
\end{equation}
Можно получить явные выражения для $j, \rho$:
$$
    \bf j = \dfrac{i \hbar}{2 m} \left(
        (\nla \Psi^\ast)\Psi - \Psi^\ast (\nla \Psi)
    \right)
$$
$$
    \rho = \dfrac{\hbar}{2mci} \left(
        (\ud{}{t} \Psi^\ast) \Pi - \Psi^\ast(\ud{}{t} \Psi)
    \right)
$$
Проблема заключается в том, что в такой постановке плотность вероятности не является знакоопределённой.

При больших энергиях ($E \gtrsim mc^2$) нельзя отдельно проследить за каждой частицей. Вместо того, чтобы интерпретировать выражение~\eqref{eq::continuous_eq} как закон сохранения тока, будем интерпретировать его как закон сохранения электрического тока:
$$
    \rho \to \rho_e = e \rho, \quad \bf j \to \bf j_e = e \bf j
$$
Когда меняется заряд, число частиц может меняться. По сути это есть теория одного заряда, но не теория одной частицы.
$$
    \boxed{
        \dfrac{\pd \rho_e}{\pd t} + \div \bf j_e = 0
    }, \quad \boxed{
        \int \rho_e d^3 \bf x = Q
    }
$$

\textbf{Решение с отрицательным знаком энергии.}
Посмотрим на волновую функцию решения этого уравнения для одной частицы. Для нерелятивистского случая это плоская волна.
$$
    \Psi(\bf r,  t) = A \exp \left(
        -\dfrac{i}{\hbar} (Et - \bf p \bf r)
    \right)
$$
\def\sqr{\square}
$$
    (\sqr - k_0^2)\Psi = 0,
$$
откуда
$$
    \boxed{
        E = \pm \sqrt{c^2 \bf p^2 + m^2 c}
    }
$$
Знак здесь нельзя отбросить, потому что мы ищем \emph{полное} решение системы.\footnote{Вспомните, откуда взялось это выражение. Мы возвели в квадрат выражение для энергии и воспользовались принципом соответствия. Наличие знака --- по сути особенности релятивистских квантовых уравнений.}

Перед тем, как переходить к уравнению Дирака, проанализируем нерелятивистский предел уравнения Клейна-Фока-Гордона.

\subsection{Уравнение Клейна-Фока-Гордона для заряженной частицы во внешнем электромагнитном поле}
Поле задаётся при помощи четырёхмерного потенциала
$$
    A^\mu = \{\psi, \bf A\}
$$
Классическая функция Гамильтона имеет вид
$$
    H_{cl} = \sqrt{c^2 \left(\bf p - \dfrac{e}{c} A \right) + m^2c^4} + e\phi
$$
Иногда также пишут
$$
    H_{cl} = \sqrt{c^2 \vec \rho^2 + m^2 c^4} + e \phi
$$
Введём \emph{оператор удлинённого импульса}:
\def \P{\mathcal P}
$$
    \P^\mu \to \hat \P^\mu = \hat p^\mu - \dfrac{e}{c} A^\mu = i \hbar D^\mu,
$$
и оператор \emph{удлинённой производной}:
$$
    D^\mu = \pd^\mu + \dfrac{ie}{\hbar c} A^\mu
$$
Описание частицы во внешнем электромагнитном поле делается заменой импульса на удлинённый, либо производной на удлиннённую производную.
$$
    (D^\mu D_\mu + k_0^2) \Psi = 0
$$
Перейдём к нерелятивистскому пределу. Посмотрим, получим ли мы при этом уравнение Шрёдингера или уравнение Паули.

Для начала, рассмотрим стационарный случай:
$$
    \Psi(\bf r, t) = \exp \left( -\dfrac{iEt}{\hbar} \right) \psi(\bf r), \quad \phi = 0
$$
Подставляя в уравнение, получаем:
$$
    \left(
        E^2 - c^2 \hat{\vec \P}^2 - m^2 c^4
    \right) \psi(\bf r) = 0
$$
$$
    E = mc^2 + \eps, \quad \eps \ll mc^2
$$
$$
    (m^2 c^4 + 2 \eps mc^2 - c^2 \hat{\vec \P}^2 - m^2 c^4) \psi(\bf r) = 0
$$
Отсюда получаем уравнение Шредингера:
$$
\boxed{
    \left(
        \eps - \dfrac{\hat{\vec\P}^2}{2m}
    \right) \psi = 0
}
$$
Итак, уравнение Клейна-Фока-Гордона, описывает скалярные, или бесспиновые частицы.

\section{Уравнение Дирака}
Уравнение Дирака --- это самое главное уравнение физики XX века. Идей, которые мы будем обосновывать, довольно мало, но выводы из этого уравнения сами по себе довольно мощные.

Уравнение описывает любые частицы со спином $\dfrac{1}{2}$.

Нужно вложить возможность описания частицы с помощью многокомпонентной волновой функции, отдельные компоненты которой при переходе между инерциальными системами отсчёта будут преобразовываться в соответствии с группой Лоренца.
$$
    \psi(\bf r, t, s) = \psi_s(\bf r, t) = \left(\begin{array}{c}
                                            \psi_1 \\
                                            \psi_2 \\
                                            \vdots \\
                                            \psi_N
                                          \end{array}\right), \quad S = 1, \ldots, N
$$
$$
    \psi(\bf r, t, s) = \qs{ \bf r, s}{\Psi(t)}
$$
Это находится в пространстве, являющимся тензорным произведением орбитального и внутреннего (спинового) пространства
\def \H{\mathcal H }
$$
    \H^{orb} \otimes \H^{spin}
$$

\subsection{Идеи, которые могут привести к уравнению Дирака}
$$
     E = \sqrt{c^2 \bf p^2 + m^2 c^4}
$$
Рассматривая $\bf p$ как оператор, попробуем <<извлечь>> корень, то есть линеаризовать выражение.

Введём специальные коэффициенты $\hat\alpha_i, \hat\beta$:
$$
    E = c \sum_{i=1}^{3} \hat \alpha_i p_i + \hat \beta m c^2
$$
По-другому это можно записать в виде
$$
    E = c \big(
        \hat{\vec \alpha} \bf p
    \big) + \hat \beta m c^2
$$
Поймём, каким условиям удовлетворяют эти четыре коэффициента $\{\hat \alpha_1,\hat \alpha_2, \hat \alpha_3,\hat \beta\}$.
\begin{eqnarray*}
    E^2 &=& c^2 \bf p^2 + m^2 c^4 = \left[
        c \sum_{i=1}^{3} \hat \alpha_i p_i + \hat \beta m c^2
    \right] \left[
        c \sum_{i=1}^{3} \hat \alpha_i p_i + \hat \beta m c^2
    \right]\\
    &=& c^2 \sum_{i=1}^{3} \sum_{j=1}^{3} \dfrac{\hat \alpha_i \hat \alpha_j + \hat \alpha_j \hat \alpha_i}{2} p_i p_j + mc^3 \sum_{i=1}^{3} (\hat \alpha_i \hat \beta + \hat \beta \hat \alpha_i) p_i + \hat \beta^2 m^2 c^4_{}
\end{eqnarray*}
Отсюда видно, какие (антикоммутационные) соотношения нужно наложить на операторы:
$$
    \{\hat \alpha_i, \hat \alpha_j\} = \hat \alpha_i \hat \alpha_j + \hat \alpha_j \hat \alpha_i = 2 \dta_{ij} \hat 1
$$
$$
    \{\hat \alpha_i, \hat \beta\} = \hat \alpha_i \hat \beta + \hat \beta \hat \alpha_i = 0
$$
$$
    \hat \beta^2 = \hat 1
$$
Эти соотношения, по сути, и определяют эти объекты. Можно считать их матрицами, тогда получаем матричные уравнения. Есть также подход Зоммерфельда\footnote{Зоммерфельд. Теория атомов и спектры.}.

%break

Итак, получили уравнение Дирака в Гамильтоновой форме:
$$
    i \hbar \ud{}{t} \Psi(\bf r, t) = \hat H_D \Psi(\bf r, t), \quad \hat H_D = c \big(\hat{\vec \alpha} \bf p
    \big) + \hat\beta mc^2
$$
где $\hat H_D$~--- Дираковский гамильтониан.
$$
    \hat \alpha_i^+ = \hat \alpha_i, \quad \hat \beta^+ = \hat \beta
$$

\subsection{Пространство спиновых переменных}

\Quest{Какие операторы действуют в этом пространстве?}

\Ans Любые комбинации векторов $\hat \alpha_i, \hat \beta$, и только они. Формально говоря, это пространство должно быть неприводимо по отношению к набору операторов $\hat \alpha_i, \hat \beta$.  Это должно быть пространство минимальной размерности, не содержащее внутри себя нетривиальных инвариантных подпространств.

\textbf{В теории Паули} (нерелятивистской) пространство $\H^{spin}$ было неприводимо по отношению к набору матриц Паули.
$$
    \hat \sigma_i \hat \sigma_j = \dta_{ij} \hat 1 + i e_{ijk} \hat \sigma_k
$$
В самом деле, спиновое пространство в теории Паули натягивается на базис из собственных векторов $\chi_{\pm 1/2}$, или, в матричном виде, $\left(\begin{array}{c}
                          1 \\
                          0
                        \end{array}\right), \left(\begin{array}{c}
                          0 \\
                          1
                        \end{array}\right)
$.

По самому построению в нём нет нетривиальных инвариантных подпространств.

\textbf{В теории Дирака}. Используя алгебраические выражения для этих операторов, можно построить ровно два набора матриц, которые будут подчиняться тем же самым алгебраическим соотношениям, что и матрицы Паули.
\begin{eqnarray*}
    \hat \Sigma_1 &=& - i \hat \alpha_2 \hat \alpha_3\\
    \hat \Sigma_2 &=& - i \hat \alpha_3 \hat \alpha_1\\
    \hat \Sigma_3 &=& - i \hat \alpha_1 \hat \alpha_2\\
    \hat p_1 &=& \Hat \Sigma_3 \hat \alpha_3\\
    \hat p_2 &=& - \hat \beta \hat \alpha_1 \hat \alpha_2 \hat \alpha_3s\\
    \hat p_3 &=& \hat \beta
\end{eqnarray*}
При этом выполняются следующие коммутационные соотношения
\begin{eqnarray*}
    \hat \Sigma_i \hat \Sigma_j &=& \dta_{ij} \hat 1 + i e_{ijk} \hat \Sigma_k\\
    \hat p_i \hat p_j &=& \dta_{ij} \hat 1 + i e_{ijk} \hat p_k\\
    {}[\Hat \Sigma_i, \hat p_j] &=& 0
\end{eqnarray*}

Искомое пространство $\H^{spin}$ можно искать в виде $\H^{(\Sigma)} \otimes \H^{(p)}$.

\begin{enumerate}
  \item Пространство, неприводимое по отношению к набору $\Sigma$. Его можно построить таким же образом, как спиновое пространство Паули. Минимальная размерность такого пространства равна двум.

  \item Пространство $\H^{(p)}$ строится аналогично.
\end{enumerate}
Размерность такого тензорного произведения равна $2 \times 2 = 4$.

Если считать операторы матрицами, то это будут квадратные матрицы $4$ порядка. Называются такие спиноры \emph{дираковскими спинорами}, или \emph{ди-спинорами}.

Получим явный вид этих матриц в стандартном (дираковском) представлении.
$$
    \Hat \Sigma_i = \hat 1 \otimes \hat \sigma_i = \begin{pmatrix}
                                                     \hat \sigma_i & 0 \\
                                                     0 & \hat \sigma_i \\
                                                   \end{pmatrix}
$$
$$
    \hat p_i = \hat \sigma_i \otimes \hat 1, \quad
    \hat p_1 = \begin{pmatrix}
                 0 & \hat 1 \\
                 \hat 1 & 0 \\
               \end{pmatrix}, \,
    \hat p_2 = \begin{pmatrix}
                 0 & -i\hat 1 \\
                 i\hat 1 & 0 \\
               \end{pmatrix}, \,
    \hat p_3 = \hat \beta = \begin{pmatrix}
                 \hat 1 & 0 \\
                 0 & -\hat 1 \\
               \end{pmatrix}
$$
Не у всех из этих матриц есть физический смысл.

\subsection{Уравнение Дирака для заряженной частицы во внешнем электромагнитном поле}
Вводится удлинённый импульс и удлинённая производная (см. выше)
$$
    A^{\mu} = \{\phi, \bf A\}
$$
$$
    \left\{
      \begin{array}{lcl}
        \hat p^\mu & \to & \hat \P^\mu - \dfrac{e}{c} A^\mu ,\\
        \pd^\mu & \to & D^\mu = \pd^mu + \dfrac{i e}{\hbar c} A^\mu
      \end{array}
    \right.
$$
Четырёхмерность пространства-времени никак не связана с четырёхмерностью Диракова пространства.
$$
    \Psi = \begin{pmatrix}
              \psi_1 \\
              \psi_2 \\
              \psi_3 \\
              \psi_4 \\
            \end{pmatrix}, \quad \Psi^+ = (\psi_1^\ast \, \psi_2^\ast \, \psi_3^\ast \, \psi_4^\ast)
$$
Уравнение Дирака:
$$
    \left\{
        \left(i \hbar \ud{}{t} - e \phi\right)
        - c (\hat{\vec \alpha} \hat{\vec \P})
        -\hat \beta m c^2
    \right\} \Psi = 0
$$
После подстановки получаем:
$$
    i \hbar \ud{}{t} \Psi - e \phi \Psi+ i \hbar c (\hat{\vec \alpha} \nla) \Psi + e (\hat{\vec \alpha} \bf A) \Psi - \hat \beta m c^2 \Psi = 0
$$
Эрмитово сопрягаем данное уравнение (предполагая, что все операторы имеют матричный вид)
$$
    -i \hbar \ud{}{t} \Psi^{+} - e \phi \Psi^{+} + i \hbar c \left( (\nla \Psi^+) \hat{\vec \alpha} \right) + e \Psi^+ (\hat{\vec\alpha} \bf A) - \Psi^{+} mc^2 \hat \beta = 0
$$
После сокращения получаем:
$$
    i \hbar \ud{}{t} (\Psi^+ \Psi) + i \hbar c \left(
        \Psi^+ (\hat{\vec \alpha} \nla) \Psi + (\nla \Psi^+) \hat{\vec \alpha} \Psi
    \right) = 0
$$
Отсюда вытекает уравнение непрерывности
$$
\boxed{
    \ud{\rho}{t} + \div \bf j = 0, \quad \rho = \Psi^+ \Psi \succeq 0, \quad \bf j = c \Psi^+ \hat{\vec \alpha} \Psi
}
$$
\subsection{Ковариантная форма записи уравнения Дирака}
$$
    \left(
        i \hbar \ud{}{t} - c(\hat{\vec \alpha} \hat{\vec \beta}) - \hat \beta mc^2
    \right) \Psi(\bf r, t) = 0
$$
Нужно показать, что время и координаты входят в уравнения на <<равных правах>>.

Введём ещё один оператор $\gamma$.
$$
\left\{
  \begin{array}{lcl}
    \gma^0 = \gma_0 & = & \hat \beta ,\\
    \gma^i = -\gma_i & = & \hat \beta \hat \alpha_i.
  \end{array}
\right.
$$
Принимая эти обозначения, мы увидим, что уравнение Дирака переписывается в очень лаконичном виде.
$$
    \gma^\mu = g^{\mu \nu} \gma_{\nu}
$$
\def \diag{\mathrm{diag}\,}
$$
    g^{\mu \nu} = g_{\mu \nu} = \diag (+ \, - \, - \, -)
$$
$$
    \boxed{
        \left(
            i \hbar \gma^\mu \ud{}{x^\mu} - mc
        \right) \Psi(\bf r, t) = 0
    }
$$
Можно придумать ещё много обозначений
$$
    \gma^\mu \pd_\mu = \hat \pd = \not \pd,
$$
и уравнение Дирака переписывается в виде
$$
\boxed{
    (i \hat \pd - m)\Psi = 0
    }
$$
$$
    \{\gma^\mu, \gma^\nu\} = \gma^\mu \gma^\nu + \gma^\nu \gma^\mu = 2 g^{\mu \nu} \hat 1
$$