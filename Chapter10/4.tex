\section{Операторы физических величин в теории Дирака}

%Уравнение Дирака получено из принципа соответствия. Нужно дать явный последовательный вывод вида операторов, которые присутствуют в этой теории. Нам пригодятся элементы теории групп.

\subsection{Операторы преобразования волновых функций}

Нам понадобятся выражения для следующих операторов.
\begin{itemize}
  \item Преобразования Лоренца
  \item Трёхмерные вращения
  \item Пространственно-временные трансляции
  \item Отражения координатных осей
\end{itemize}

Нужно рассмотреть операторы, которые действуют во всём пространстве
\def \H{\mathcal H}
$$
    \H = \H^{(orb)} \otimes \H^{(spin)}
$$
Если оператор $\Hat T$ определён в таком пространстве
$$
    \Psi(x) \to \Psi'(x) = \Hat T \Psi(x),
$$
то он однозначно представляется в виде
$$
    \Hat T = \Hat T^{(orb)} \otimes \Hat T^{(spin)}
$$
\textbf{Замечание.} Выше мы рассматривали преобразования вида
$$
    \Psi(x) \overset{\Lambda}{\to} \Psi'(x') = S(\Lambda) \Psi(x)
$$
Эти преобразования производились при помощи матрицы $4 \times 4$ (явно её компоненты мы не выписывали)

С другой стороны,
$$
    S(\Lambda) \Psi(x) = S(\Lambda) \Psi(\Lambda^{-1} x')
$$
Поэтому для преобразований Лоренца действие оператора задаётся по формуле
$$
    \Psi(x) \to \Psi(x) = S(\Lambda) \Psi(\Lambda^{-1} x)
$$

Уравнения Дирака должны сохранять свой вид.
$$
    \Hat D \Psi(x) = 0, \qquad \Hat D = i \hbar \left(
        \gamma^{\mu} \dfrac{\pd}{\pd x^{\mu}}
    \right) - mc
$$
$$
    \underbrace{\Hat T \hat D \hat T^{-1}}_{\hat D}\underbrace{\hat T \Psi(x)}_{\Psi'(x)} = 0
$$
Отсюда возникает условие перестановочности с оператором преобразования волновой функции:
$$
    \Hat T \Hat D \hat T^{-1} = \hat D
$$
$$
\boxed{
    [\hat D, \hat T] = 0
}
$$

\begin{itemize}
  \item \textbf{Трансляции.}  С точки зрения пространства Минковского:
$$
    \hat T^{(spin)} = x^{\mu} \to x'^{\mu} = x^{\mu} - a^{\mu}
$$
$$
    \Psi'(x) = \Hat T_a \Psi(x) = \Psi(\Hat T_a^{-1} x) = \Psi(x^{\mu} + a^{\mu})
$$
Это рассматривалось в прошлом семестре. Нужно провести разложение в ряд Тейлора при малом $a$. Отсюда получится выражение для инфинитезимальных операторов и генераторов группы Ли.
$$
    \boxed{
        \hat p^{\mu} = i \hbar \pd^{\mu}
    } \quad \boxed
    {
        E \to i \hbar \ud{}{t}
    } \quad \boxed{
        \bf p \to - i \hbar \nabla
    }
$$
  \item \textbf{Трёхмерные вращения.} Соответствующий оператор будет действовать в обоих пространствах:
  $$
    \Psi(x) \to \Psi'(x) = \hat R(\phi) \Psi(x) = S(\vec \phi) \Psi(\hat T^{-1} (\vec \phi) \bf r)
  $$
  Необходимо рассмотреть, что представляют из себя операторы бесконечно малых поворотов.
  
  Ещё раз обратимся к соответствующей главе лекций, в которых получались аналогичные результаты.
  $$
    \hat{\bf J} = \hat{\bf L} \otimes \hat 1 + \hat 1 \otimes \hat{\bf S}
  $$
  где
  $$
    \hat {\bf L} = [\hat {\bf r} \times \hat{\bf p}], \qquad \hat {\bf S} = \dfrac{\hbar}{2} \hat{\boldsymbol \Sigma}
  $$
  Полный момент есть сумма орбитального и спинового. Орбитальный момент затрагивает только аргумент, спиновый оператор действует в пространстве спиновых переменных.
\end{itemize}

\subsection{Интегралы движения в теории Дирака}

Среди всех преобразований четырёхмерных координат есть такие, которые затрагивают время.

Если рассматривать только операторы, которые не затрагивают время (пространственные трансляции, трёхмерные вращения и отражения), то операторы этих преобразований коммутируют с $\gamma^0 = \hat \beta$.

Например, [$\hat \sigma, \hat \beta] = 0$, так как операторы действуют в разных пространствах.

Можно переформулировать требования инвариантности нашей теории на языке законов сохранения.
Первое требование: 
$$
[\hat D, \hat T] = 0.
$$

С другой стороны, оператор Дирака можно переписать в виде
$$
    \Hat D = \dfrac{\gamma^0}{c} \left(
        i \hbar \dfrac{\pd}{\pd t} - \hat H_{D}
    \right) = i \hbar \left(
        \gamma^{\mu} \ud{}{x^{\mu}}
    \right) - mc
$$
$$
    [\hat D, \hat T] = 0 \, \to \, [\hat H_D, \hat T] = 0
$$
Операторы бесконечно малых преобразований были интегралами движения.

Отсюда можно указать, какие операторы будут являться интегралами движения для свободной частицы (и коммутировать с гамильтонианом Дирака).

$$
    \hat{\bf p}, \quad \hat {\bf J}, \quad \hat{\mathcal P}
$$ 
Последний оператор означает <<чётность>>.

\subsection{Сложность интерпретации операторов в теории Дирака}
Уравнение Дирака в дальшейшем будем записывать в Гамильтоновой (а не в ковариантной) форме:
$$
    i \hbar \dfrac{\pd}{\pd t} \Psi(\bf r, t) = \hat H_D \Psi(\bf r, t) = 
    \Big(
        c (\hat{\boldsymbol \alpha}, \hat{\bf p}) + \hat \beta mc^2
    \Big)\Psi
$$
Вероятностная интерпретация, постулат средних значений и другие свойства, полученные при изучении нерелятивистской квантовой механики, сохраняются.

Интегралы движения, которые имеют очевидную интерпретацию:
$$
    \hat {\bf p}, \quad \hat{\bf J}, \quad \hat{\mathcal P}, \quad \Hat{H}_D
$$ 
\Quest{Как интерпретировать оператор $\hat {\bf L}$? Он не является интегралом движения дираковских частиц.}

\Quest{Аналогично для оператора $\hat {\bf S}$.}

Спин не является полностью кинематическим свойством, он связан с движением частицы. То, что эти операторы не коммутируют \emph{порознь} есть именно свойство релятивистского движения.

Общая причина связана с особым движением дираковского электрона. Есть интерференция состояний с разными знаками уровня энергии.

При решении свободного уравнения Дирака получаем
$$
    E^2 = c^2 \bf p^2 + m^2 c^4,
$$ 
откуда
$$
    E = \pm \sqrt{c^2 \bf p^2 + m^2 c^4}
$$
\subsection{Оператор скорости Дираковской частицы}
Необходимо перейти в представление Гайзенберга, и найти этот оператор по определению.

\begin{eqnarray*}
    \hat v_x &=& \dfrac{d \hat x}{dt} = \dfrac{i}{\hbar} [\hat H_D, \hat x]\\&=&
    \dfrac{ic}{\hbar} [(\hat {\boldsymbol \alpha} \hat{\bf p}), \hat x]\\
    &=& \dfrac{ic}{\hbar} \hat{\alpha_x} \underbrace{[\hat p_x, \hat x]}_{-i \hbar} = c \hat \alpha_x
\end{eqnarray*}
Это довольно странный результат. Принцип соответствия говорит, что матрица $\alpha_x$ соответствует оператору скорости.

Собственные значения матрицы $\alpha_x$ равны $\pm 1$, поэтому значение любой скорости равно $\pm c$.

Если есть три компоненты, то модуль скорости равнялся бы $c \sqrt{3}$. Кроме того, матрицы не коммутируют, значит даже две компоненты скорости измерить нельзя.

Происходит <<расщепление>> понятий скорости и импульса. В нерелятивистской теории Шрёдингера такая связь присутствовала: $\hat v_x = \hat p_x / m$.

\textbf{Вывод.}  Принцип соответствия не работает в полной мере. Это связано с интерференцией состояний с разными знаками энергии.

Чтобы объяснить, что это за явление, покажем, что эта интерференция из себя представляет.

Представим решение в виде волнового пакета из двух плоских волн:
$$
    \Psi(\bf r, t) = A_1 e^{-i \Omega t} \psi(+) + A_2 e^{i \Omega t} \psi(-)
$$
$$
    \Omega = \dfrac{|E|}{\hbar}
$$
При помощи такой функции вычислим несколько средних. Система функций гамильтониана полна и ортонормирована, и в ней присутствуют состояния с разными знаками энергии.

Подчеркнём, что волновые функции состояний с разными знаками энергии ортогональны.

\begin{itemize}
  \item \textbf{Вероятность}.
  $$
    \int d^3 x \Psi ^+ \Psi = \int d^3 x \Big(
        |A_1|^2 \psi^+(+) \psi(+) + |A_2|^2 \psi^+(-) \psi(-)
    \Big) = 1
  $$
  \item \textbf{Плотность вероятности.}
  \begin{eqnarray*}
    \bf j &=& c \int d^3 x \Psi^+ \hat{\boldsymbol \alpha} \Psi\\
    &=& \int d^3 x
    \Big(
        |A_1|^2 \psi^+ (+) \hat{\boldsymbol \alpha} \psi(+)\\ &&+
        |A_2|^2 \psi^+ (-) \hat{\boldsymbol \alpha} \psi(-)\\ &&+
        A_1^{\ast} A_2 \psi^+(+) \hat {\boldsymbol \alpha} \psi(-) e^{2i \Omega t}\\
        &&+ A_1 A_2^{\ast} \psi^+(-) \hat {\boldsymbol \alpha} \psi(+) e^{-2i \Omega t}
    \Big)
  \end{eqnarray*}
  Последние два слагаемых имеют осциллирующий характер. (Zitterbewegung\footnote{<<Дрожащее>> движение.})
  
  \item $\Omega \sim 1.6 \cdot 10^{21} c^{-1}$
\end{itemize}

\subsection{Операторы с дефинитной чётностью}
Если мы допускаем интерференцию состояний, то мы отходим от одночастичного описания модели.

Для простоты объяснения сути явления перейдём к модели, которая имеет ряд изъянов.

Представим, что в вакууме движется релятивистский электрон. Его путь можно рассматривать как последовательность отрезков пути, причём в стыках отрезков происходит аннигиляция виртуальной электрон-позитронной пары. Затем из вакуума снова происходит рождение, и так далее. В результате направление движения постоянно хаотично изменяется, причём на некоторых отрезках электрона вообще нет.

Виртуальность обозначает лишь то, что эти пары живут в течение времени, не большее, чем
$$
    \Delta t \gtrsim \dfrac{\hbar}{\Delta E} \simeq \dfrac{\hbar}{mc^2}
$$ 
Даже если электрон двигается со скоростью света, он отодвигается лишь на небольшое конечное расстояние
$$
    c \Delta t \sim \dfrac{\hbar}{mc} = \bar \lambda_{\text{компт.}} \simeq 3.9 \cdot 10^{-11} cm
$$
Это комптоновская длина волны электрона. Такое явление не имеет интерпретации в смысле одночастичной теории.

Если выйти за пределы <<трубки>> с характерным радиусом $\dfrac{\hbar}{mc}$ на довольно большое расстояние, то эффекты рождения электрон-позитронных пар будут пренебрежимы. Электрон имеет флуктуацию координаты.

% Второй подход.
% На следующих лекциях. Сложный ломаный характер движения дираковского электрона может быть объяснён осцилляцией. Одночастичное приближение: отсутствие осцилляции.
% Последовательная одночастичная теория должна использовать состояния только одного знака энергии. Должна использовать операторы, которые не смешивают состояния с разными знаками энергии, такие как $\alpha$
% 