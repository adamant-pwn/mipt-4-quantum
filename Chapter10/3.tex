\section{Релятивистская ковариантность в уравнении Дирака}

Чтобы анализировать релятивистски-ковариантный смысл уравнения Дирака, запишем его в виде
$$
   \left[ i \hbar \left( \gamma^\mu \ud{}{x^{\mu}}\right) - mc \right] \Psi(\bf r, t) = 0,
$$
где
$$
    \gamma^0 = \gamma_0 = \hat \beta =  \begin{pmatrix}
                 \hat 1 & 0 \\
                 0 & -\hat 1 \\
               \end{pmatrix}, \quad 
    \gamma^i = - \gamma_i = \hat \beta \hat \alpha_i = 
    \begin{pmatrix}
      0 & \hat \sigma_i \\
      -\hat \sigma_i & 0 \\
    \end{pmatrix}
$$
Такой вид выбирается по той причине, что производные по времени и координате входят в это уравнение симметрично.

Нужно найти преобразование
$$
    \Psi(\bf r, t) = \Psi(x^\mu) = \Psi(\bf x)
$$
при переходе от одной системы отсчёта к другой

\subsection{Краткий обзор группы Лоренца}
Предположим, что имеются два наблюдателя, каждый из которых связан с некоторой инерциальной системой отсчёта. \emph{События} при этом имеют для них разные координаты.

\begin{itemize}
  \item \textbf{Матрица преобразования Лоренца}
$$
    x^\mu \to x^{i \mu} = \Lambda_\nu^\mu x^\nu
$$
  \item \textbf{Классификация преобразований Лоренца.}
  
  Как известно искушённому читателю, интервал в теории поля имеет вид
  $$
    ds^2 = dx_\mu dx^\mu = g_{\mu \nu} dx^{\mu} dx^{\nu},
  $$
  где $g_{\mu\nu}$~--- метрический тензор.
  
  Какие условия должны накладываться на матрицу преобразований Лоренца?
  $$
    dx'^{\sigma} dx'_{\sigma} = \Lambda^{\sigma\mu}\Lambda_{\sigma \nu} dx_{\mu} dx^{\nu} = \dta_{\nu}^{\mu} dx_{\mu} dx^{\nu}
  $$
  Отсюда следует, что матрица должна обладать свойством
  $$
    \Lambda^{\sigma\mu} \Lambda_{\sigma \nu} = \dta^\mu_\nu
  $$
  Свёртывая по индексам $\mu, \nu$, получаем
  $$
    (\det \Lambda)^2 = 1
  $$
  Значит, преобразования можно классифицировать на 2 типа:
  $$    
    \begin{array}{lcl}
         \nearrow \\
         \searrow
    \end{array}
    \left[
      \begin{array}{lcl}
        \det \Lambda &=& 1 ,\\
        \det \Lambda &=& -1.
      \end{array}
    \right.
  $$
  В первом случае это \emph{собственные преобразования Лоренца}.
  
  Во втором случае это дискретные преобразования типа симметрии плоскости, которые не сводятся к непрерывным движениям.
  
  Будем рассматривать только собственные преобразования Лоренца.
  \item \textbf{Конкретные примеры матриц преобразований.}
  
  Собственные преобразования Лоренца включают тождественные преобразования. Можно рассмотреть инфинитезимальные преобразования. На основе их и единичного (тождественного) можно построить любое непрерывное преобразование.
  
  Пусть верно <<приближение>>
  $$
  \boxed{
    \Lambda_{\nu}^{\mu} \simeq \dta^\mu_{\nu} + \lambda_\nu^{\mu} \dta \phi,
  }
  $$
  где $\lambda_{\nu}^{\mu}$~--- генератор однопараметрической подгруппы преобразования Лоренца.
  
  \textbf{Свойства.}
  \begin{itemize}
    \item Антисимметрия. $\lambda_{\mu\nu} = -\lambda_{\nu\mu}$. Это свойство непосредственно следует из определения, данного выше.
    \item Если вернуться к первой части курса, перед генератором был множитель $i$. Это связано с тем, что в группе Лоренца не все генераторы эрмитовы.
  \end{itemize}
\end{itemize}
Построим генераторы для \textbf{буста} и для \textbf{поворота}.

Пусть $\gamma = (1 - \beta^2)^{-1/2}$~--- это релятивистский фактор, $\beta = \dfrac{v}{c}$.

\begin{enumerate}
  \item Буст вдоль оси $Ox$.
  $$
    \left\{
      \begin{array}{lcl}
        x^{10} & = & \gamma (x^0 - \beta x^1) ,\\
        x^{11} & = & \gamma (x^1 - \beta x^0) ,\\
        x^{12} & = & x^2 ,\\
        x^{13} & = & x^3.
      \end{array}
    \right.
  $$
  Такой вид не очень приятен, так как нужна каноническая параметризация с аддитивным параметром. Для этого делают замену ($\tilde \phi$ иногда называют <<быстрота>>)
  $$
    \beta = \th \tilde \phi
  $$
  \def \ch {\mathrm{ch}\,}
  \def \ch {\mathrm{sh}\,}  
  $$
    \left\{
      \begin{array}{lcl}
        x^{10} & = & x^0 \ch \tilde \phi - x^1 \sh \tilde \phi ,\\
        x^{11} & = & -x^0 \sh \tilde \phi + x^1 \ch \tilde \phi ,\\
        x^{12} & = & x^2 ,\\
        x^{13} & = & x^3.
      \end{array}
    \right.
  $$
  $$
    \Lambda_{\nu}^{\mu} = \begin{bmatrix}
                            \ch \tilde \phi & -\sh \tilde \phi & 0 & 0 \\
                            -\sh \tilde \phi & \ch \tilde \phi & 0 & 0 \\
                            0 & 0 & 1 & 0 \\
                            0  & 0 & 0 & 1 \\
                          \end{bmatrix}
  $$
  Генератор преобразования (при дифференцировании матрицы по $\tilde \phi$):
  $$
    \lambda_{(x^0x^1)}{}_{\nu}^{\mu} = \dfrac{\pd}{\pd \tilde \phi} \Lambda^\mu_{\nu} \Big|_{\tilde \phi = 0}
    = \begin{bmatrix}
        0 & -1 & 0 & 0 \\
       -1 & 0 & 0 & 0 \\
        0 & 0 & 0 & 0 \\
        0 & 0 & 0 & 0 \\
      \end{bmatrix}
  $$
  \item Поворот вокруг оси $O_z$.
  $$
    \lambda_{x^1 x^2} {}_{\nu}^{\mu} =\dfrac{\pd}{\pd \phi} \Lambda_{\nu}^{\mu} (\phi) \Big|_{\phi = 0}
    = \dfrac{\pd}{\pd \phi} \left.\begin{bmatrix}
        1 & 0 & 0 & 0 \\
        0 & \cos \phi & \sin \phi & 0 \\
        0 & -\sin \phi & \cos \phi & 0 \\
        0  & 0 & 0 & 0 \\
      \end{bmatrix} \right|_{\phi = 0} = \begin{bmatrix}
                                           0 & 0 & 0 & 0 \\
                                           0 & 0 & 1 & 0 \\
                                           0 & -1 & 0 & 0  \\
                                           0 & 0 & 0 & 0 \\
                                         \end{bmatrix}
  $$
  \item   В этой группе можно выделить 6 параметрических подгрупп (3 поворота и 3 буста). Любое преобразование строится как комбинация бесконечно малых.
\end{enumerate}

\subsection{Доказательство релятивистской ковариантности уравнения Дирака}
Уравнение Дирака в системе $s$:
$$
    \left(
        i \hbar \left(
            \gamma^{\mu} \ud{}{x^{\mu}} - mc
        \right) \Psi(x) = 0
    \right)
$$
Хотим найти волновую функцию, соответствующую этому состоянию, в системе $s'$.
$$
    \Psi(x) \to \Psi'(x')
$$
В системе $s'$ уравнение Дирака должно сохранять свой вид. В этом и заключается ковариантность уравнений.
$$
    \left(
        i \hbar \left(
            \gamma^{\mu} \ud{}{x'^{\mu}} - mc
        \right) \Psi'(x') = 0
    \right)
$$
\begin{itemize}
  \item В первой половине указывалось, что при поворотах матрицы Паули не преобразуются. Сейчас мы примем, что матрицы $\gamma^{i}$ не преобразуются при переходе в другую систему координат.
  \item Операторы действуют в пространстве волновых функций. Говоря точнее, в пространстве внутренних переменных Дираковской частицы. 
  \item Будем считать, что и преобразование волновой функции тоже линейно.
  $$
    \Psi'(x') = S(\Lambda)\Psi(x)
  $$
  Операторы преобразования волновой функции будут реализовывать некоторое представление группы Лоренца.
  \item Должно быть обратное преобразование.
  $$
    \Psi(x) = S^{-1} (\Lambda) \Psi'(x')
  $$
  \item Умножаем обратное преобразование таким образом:
  \begin{equation}
    \left(
        i \hbar S(\Lambda) \Big(
            \gamma^\mu \ud{}{x^\mu}
        \Big) S^{-1} (\Lambda) - mc
    \right) \underbrace{S(\Lambda) \Psi(x)}_{\Psi'(x')} = 0
    \label{eq::inverse_multiply}  
  \end{equation}
  Как преобразовывается четырёхмерный градиент:
  $$
    \ud{}{x^\mu} = \ud{}{x'^\nu} \ud{x'^{\nu}}{x'^{\mu}} = \Lambda_\mu^{\nu} \ud{}{x'^{\nu}}, \quad x'^{\mu} = \Lambda_\nu^{\mu} x^{\nu}
  $$
  Таким образом,~\eqref{eq::inverse_multiply} преобразовывается как
  $$
    \left(
        i \hbar S(\Lambda) \Big(
            \gamma^\mu \Lambda^{\nu}_{\mu} \ud{}{x'^{\nu}}
        \Big) S^{-1} (\Lambda) - mc
    \right) \Psi'(x') = 0
  $$
  $$
    S(\Lambda) \Lambda_\mu^{\nu} \gamma^\mu S^{-1}(\Lambda) = \gamma^{\nu}
  $$
  Уравнение, определяющее оператор $S(\Lambda)$
  $$
    \boxed{
        S^{-1} (\Lambda) \gamma^\mu S(\Lambda) = \Lambda_\nu^{\mu} \gamma^\nu
    }
  $$
  \item Напомним, что мы рассматриваем только случай $\det \Lambda = 1$. Ограничимся только бесконечно малыми преобразованиями, потому что любое другое можно через них выразить.
  
  Для пространства Минковского
  $$
    \Lambda_\nu^{\mu} \simeq \dta_\nu^{\mu} + \lambda_\nu^{\mu} \dta \phi,
  $$
  где
  $$
    \hat \Lambda \simeq \hat 1 + \hat \lambda \dta \phi
  $$
  При этом оператор $S$ действует следующим образом:
  $$
    S(\hat 1 + \hat \lambda \dta \phi) \simeq \hat 1 + \hat T \dta \phi
  $$
  $$
    S^{-1} (\hat 1 + \hat \lambda \dta \phi) \simeq \hat 1 - \hat T \dta \phi
  $$
  Подставляя в соотношение, связывающее $S$ и обратную матрицу $S^{-1}$ (См. пособие Тернова), получаем выражение для оператора $\hat T$:
  $$
    \hat T = \dfrac{1}{8} \lambda^{\mu\nu}
    (\gamma_\mu \gamma_\nu - \gamma_\nu \gamma_\mu) = \dfrac{-i}{4} \lambda^{\mu \nu} \sigma_{\mu \nu},
  $$
  где $\sigma_{\mu \nu} = \dfrac{i}{2}[\gamma_\mu, \gamma_nu]$.
  
  Различных матриц $\sigma$ есть всего 6.
  $$
    \hat \sigma_{i0} = - \hat \sigma_{0i} = i \hat \alpha_i, \quad \hat \sigma_{ij} = e_{ijk} \hat \Sigma_k
  $$
  \item В качестве конкретных случаев рассмотрим буст и поворот в пространстве Минковского.
  
  Какие операторы будут действовать на волновые функции при поворотах и бустах?
  
  Пользуемся соотношениями
  $$
    \boxed{
        \lambda_{(x^0 x^1)}^{01} = - \lambda_{(x^0 x^1)}^{10} = +1
    } \, , \quad 
    \boxed{
        \lambda_{(x^1 x^1)}^{12} = - \lambda_{(x^1 x^2)}^{21} = -1
    }
  $$
  $$
  \boxed{
    \Hat T_{(x^0 x^1)} = \dfrac{i}{2} \hat \sigma_{10} = -\dfrac{1}{2} \hat \alpha_1
  }
  $$
  Воспользуемся теоремой Ли. Согласно ей,
  $$
    S_{(x^0 x^1)} (\tilde \phi) = e^{-\frac{1}{2} \hat \alpha_1 \tilde \phi} = \ch \dfrac{\tilde \phi}{2}- \hat \alpha_1 \sh \dfrac{\tilde \phi}{2}
  $$
  Видно, что преобразование не является унитарным.
  \item Подставляем поворот вокруг оси $Oz$.
  $$
    \lambda^{12} = - \lambda^{21} = -1
  $$
  $$
    \Hat T_{(x^1 x^2)} = \dfrac{i}{2} \hat \sigma_{12} = \dfrac{i}{2} \hat \sigma_3
  $$
  Это оператор бесконечно малого поворота.
  $$
    S_{(x^1 x^2)}(\phi) = e^{\frac{i}{2} \hat \Sigma_3 \phi} = \cos \dfrac{\phi}{2} + i \hat \sigma_3 \sin \dfrac{\phi}{2}
  $$
  Представление реализовано в пространстве внутренних переменных.
\end{itemize}

Чуть позже мы выясним, что это \emph{спинорное представление}. Волновая функция будет преобразовываться так же, 
как и частица со спином $\dfrac{1}{2}$.

\subsection{Выводы и замечания}
\subsubsection{Спин частиц Дирака}
Спин определяется трансформационными свойствами внутренних переменных по отношению к трёхмерным вращениям.

\begin{center}
\begin{tabular}{|c|c|}
  \hline
  % after \\: \hline or \cline{col1-col2} \cline{col3-col4} ...
  Теория Паули & Теория Дирака \\
  \hline
  $\frac{i}{2} \hat \sigma_z$ & $\frac{i}{2} \hat \Sigma_3$ \\
  \hline  
  $\exp(\frac{i}{2} \hat \sigma_z \phi)$ & $\exp( \frac{i}{2}\hat \sigma_z \phi)$ \\
  \hline
   $\hat {\bf S} = \frac{\hbar}{2} \hat{\vec \sigma}$ & $\hat {\bf S} = \frac{\hbar}{2} \hat{\vec \sigma}$\\
  \hline   
\end{tabular}
\end{center}

Спин, определённый таким образом, не коммутирует с гамильтонианом и не обладает ковариантными свойствами.\footnote{О построении ковариантного спина см. в пособии Тернова}

\subsubsection{Некоторые общие свойства матриц преобразований $S(\Lambda)$} 
При рассмотрении группы вращений общим свойством была унитарность. Здесь же такого не наблюдается.

\begin{itemize}
  \item Для поворотов унитарность есть:
  $$
    S_{(x^1 x^2)}^{+} = S_{(x^1 x^2)}^{-1}
  $$
  \item Для бустов нет унитарности
  \item Общее свойство (без доказательства)
  $$
  \boxed{
    \gamma^0 S^+ (\Lambda) \gamma^0 = S^{-1} (\Lambda)
  }
  $$
\end{itemize}
\subsection{Ковариантность уравнения непрерывности}
$$
    \Psi(x) \overset{\Lambda}\to \Psi'(x') = S(\Lambda) \Psi(x)
$$
$$
    \Psi^{+}(x) \overset{\Lambda}{\to} {\Psi^{+}}'(x') = \Psi^{+}(x)S^{+}(\Lambda) 
$$
\textbf{Определение.}

Дираковски сопряжённый спинор:
$$
    \bar \Psi(x) = \Psi^{+} (x) \gamma^0
$$

$$
    \bar\Psi(x) \overset{\Lambda}\to \bar\Psi'(x') = \bar \Psi(x) S^{-1}(\Lambda)
$$
Обозначим $j^\mu$ четырёхмерный вектор плотности вероятности (тока).
$$
    j^\mu = \{
        c \rho, \bf j
    \} = \{
        c \Psi^+ \Psi, c \Psi^+ \hat{\bar \alpha} \Psi
    \}, \quad j^\mu = c \Psi^+(x) \gamma^0 \gamma^\mu \Psi(x) =  \bar \Psi(x) \gamma^\mu \Psi(x)
$$
\begin{eqnarray*}
    j'^{\mu} &=& c \bar \psi'(x') \gamma^\mu \Psi'(x')\\
    &=& c \bar \Psi(x) \underline{S^{-1}(x) \gamma^\mu S(\Lambda)} \Psi(x)\\
    &=& \Lambda^{\mu}_{\nu} \underline{c \bar \Psi(x) \gamma^{\nu} \Psi(x)}\\
    &=& \Lambda^{\mu}_{\nu} j^{\nu}
\end{eqnarray*}
Действительно, данная величина при преобразованиях ведёт себя как четырёхмерный вектор.

\textbf{Заключение.} Ковариантные свойства уравнение Дирака довольно богаты.

Отдельно хотелось бы заметить, что если перемножить всевозможными способами $\gamma$-матрицы, то получится 16 всевозможных матриц. Эти матрицы образуют \emph{алгебру Клиффорда}. Если хотим эти матрицы изображать в виде квадратных,  то наименьшая размерность равна 4. Число компонент волновой функции тоже должно быть равно 4.