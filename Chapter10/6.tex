
Итак, квазирелятивистское приближение решения в атоме водорода характеризуется величиной
$$
    \left(\dfrac{v}{c}\right)^2 \sim \left(
        \dfrac{1}{mc}
    \right)^2 \sim \alpha^2
$$
$$
    \bf A = 0,\qquad -e \Phi = v(r)
$$
Можно провести разложение с точностью до $v/c$ в первой, второй степени. Могут встречаться некоммутирующие выражения, поэтому для второго порядка следует проявлять осторожность. Эта тема разобрана более подробно в семинарских занятиях, но входит в экзаменационный материал. Материал есть в задачнике.

Есть один существенный момент, на котором нужно заостроить внимание. При переходе к уравнению Паули две нижние компоненты спинора $(\phi \, \chi)$  малы, и имеют порядок $(v/c)$ по отношению к верхним. Просто отбросить их \emph{нельзя}, потому что порядок приближения $(v/c)$. Истинная двух компонентная функция Паули $\phi^{P}$  должна строиться следующим образом:
$$
     \psi^{P} (\bf r) = \hat A \phi
$$ 
Результат разложения и гамильтониан в квазирелятивистском приближении (в стационарном случае) имеют вид
$$
    \eps \psi^{P}(\bf r) = \hat H^{q} \psi^P(\bf r), \quad \hat H^{q} = \underbrace{\hat H^{S}}_{\text{Schr\"odinger}} + \hat V^{q},
$$
где
$$
    \hat H^{S} = \dfrac{\hat{\bf p}^2}{2m} + U(r), \quad \hat V^{q}
     = - \dfrac{\hat{\bf p}^4}{8 m^3 c^2} + \dfrac{1}{2 m^2 c^2} \dfrac{dv}{dr} \dfrac{(\hat{\bf S} \hat{\bf L})}{r} + \dfrac{1}{8}\left(
        \dfrac{\hbar}{mc}
     \right)^2 (\vec \nabla^2 U(r))
$$
Интерес представляет именно анализ приближённого вида гамильтониана. (На экзамене можно предоставлять его без вывода)

\subsubsection{Анализ приближённого вида гамильтониана}
Если считать магнитное поле равным нулю, то получится гамильтониан Паули.
\begin{itemize}
  \item \emph{Основной член.} Это может быть $\hat H^{S}$ или $\hat H^{P}$.
  \item \emph{Релятивистская поправка}. $\boxed{\hat V^{rel} = - \dfrac{\hat{\bf p}^4}{8m^3c^2}}$. Это слагаемое учитывает зависимость энергии от импульса частицы. У этого слагаемого есть классический аналог. Учёт этой классической поправки приводит к изменению траектории. Кеплеровский эллипс прецессирует вокруг одного из фокусов (розеточная траектория Зоммерфельда). Такая же поправка есть в уравнении Клейна-Фока-Гордона.
  \item \emph{Спин-орбитальное взаимодействие}. $\boxed{\hat V^{so} = \dfrac{Ze^2}{2m^2c^2} \dfrac{(\hat{\bf S} \hat{\bf L})}{r^3}}$. Именно в атоме водорода впервые при рассмотрении этих поправок возник этот термин. Его название идёт от соответствующего скалярного произведения.\footnote{Будьте внимательны при физической интерпретации! Не следует говорить, что это буквально является взаимодействием механических моментов. Из ниоткуда не может появиться новый вид взаимодействия, не описанный в теории. Это слагаемое обязано своим происхождением электромагнитному взаимодействию.}
      \begin{itemize}
        \item $\hat{\bf \mu} = - \mu_0 \hat{\bf \sigma} = - \dfrac{e\hbar}{2mc} \hat{\bf \sigma} = -\dfrac{e}{mc} \hat{\bf S}$, $-e < 0$
            
            Двигаясь по орбите со скоростью $v$, электрон начинает обладать дипольным моментом:
        \item $\bf d = \dfrac{1}{c}[\bf v \times \bf \mu] = \dfrac{1}{mc} [\bf p \times \bf \mu]$.
        
        Если у электрона, летящего по орбите, есть также дипольный электрический момент, он начинает взаимодействовать с ядром.
        
        Дополнительная энергия взаимодействия:
        \item $U^{add} = - (\bf d, \bf{\mathcal E})$
        
        После подстановки
        $$
            U^{add} = \dfrac{-Ze}{mc r^3} \Big(
                [\bf p \times \bf \mu]
            \Big) \bf r = \dfrac{Ze^2}{m^2c^2} \dfrac{(\bf S \bf L)}{r^3}
        $$
        В операторном виде:
        $$
            \hat U^{add} \overset{?}= \dfrac{Ze^2}{m^2c^3} \dfrac{(\hat {\bf S} \hat {\bf L})}{r^3}
        $$
        (см. прим\footnote{
        Полезно сравнить с тем, что получается в результате точного расчёта. Возникает коэффициент $\frac{1}{2}$! Это оказалось достаточно драматическим моментом. Если пользоваться принципом соответствия так, как продемонстрировано выше, то нельзя получить <<правильный>> коэффициент $\frac{1}{2}$, который сходится с экспериментом.
        
        Задача была решена в 1926 году. Оказалось, что спин испытывает дополнительную Томасовскую прецессию. Пришлось более точно анализировать формулы перехода в разных системах отсчёта.
        }
        )
      \end{itemize}
  \item \emph{Контактное взаимодействие}. 
  $$
  \boxed{
    \hat V^{c} = \dfrac{\pi}{2} Z e^2 \left(
        \dfrac{\hbar}{mc}
    \right)^2 \dta(\bf r)}, \qquad \nabla^2 (1/r) = -4 \pi \dta^{(3)} (\bf r)
  $$
  Это взаимодействие так называется из-за присутствия дельта-функции. Электрон находится в <<контакте>> с ядром, есть вероятность того, что электрон находится в одной точке с ядром.
  
  Это слагаемое можно связать с не-локальностью взаимодействия и с наличием дрожания. Весь комплекс физических явлений, который обсуждался выше, проявляется здесь. Электрон размазан по некоторому объёму с линейными размерами порядка квантового радиуса.
  $$
    r_{qu} \sim \dfrac{\hbar}{mc}
  $$
  Он <<чувствует>> не только Кулоновское поле, но и усреднённое поле.
  $$
    U^{add} = \langle \dta U(\bf r) \rangle = \langle U(\bf r + \dta \bf r) - U(\bf r) \rangle
  $$
  Если $\dta \bf r$ мало, то при усреднении получаем
  $$
    U^{add} = \langle
        (\bf{\dta r} \bf\nabla) U(r) + \dfrac{1}{2} \sum_{i} \sum_{j} \dta {x_i} \dta x_j \nabla_i \nabla_j U(r)
    \rangle = \dfrac{1}{6} (\dta \bf r)^2 (\nabla^2 U(r))
  $$
  Поступим с этим слагаемым достаточно <<грубо>>. Заменяем $\dta \bf r$ на комптоновскую длину волны:
  $$
  \boxed{
    U^{add} = \dfrac{1}{6} \left(
        \dfrac{\hbar}{mc}
    \right)^2 (\nabla^2 U(r))
  }
  $$
\end{itemize}

\section{Тонкая структура уровней энергии атома водорода}
\def \bf{\boldsymbol}
Если решать задачу об уровнях энергии атома водорода, используя уравнение Шрёдингера, то получим результат
$$
    E_n^{(0)} = -\dfrac{Z^2}{2 n^2} \left(
        \dfrac{e^2}{a}
    \right)
$$
Смысл круглых скобок~--- переход в другую систему единиц.

Это надо воспринимать как нулевое приближение. Уравнение Шрёдингера не включает ни релятивистских, не орбитальных поправок. Нужно привлекать уравнение Дирака. Этим путём мы не пойдём.

Пора возвращаться к физическому смыслу. Поправки к энергии мы получим с помощью теории возмущений.
$$
    \hat H = \hat H^{(0)} + \hat V^{qu\, rel}
$$ 
\def \l {\ell}
\subsection{Постановка задачи}
\begin{itemize}
  \item \textbf{Невозмущённая задача.} Если взять привычные нам водородные функции,

  $$
    \psi_{n\l m} = C_{n \l} E_{n \l} Y_{\l}^{(m)} (\theta, \phi) \leftarrow \hat{\bf L}^2 \hat{\bf S}^2, \hat L_z, \hat S_z
  $$
  $$
        [(\hat{\bf L}, \hat{\bf S}), \hat J_z] = 0
  $$
  Правильными функциями будут те функции, которые являются общими собственными веторами для следующего набора операторов:
  $$
    \hat{\bf L}^2, \hat{\bf S}^2, \hat{\bf J}^2, \hat{J_z}
  $$
  Эти операторы связаны с предыдущими некоторым линейным преобразованием: $\bf J = \bf L + \bf S$.
  
  В координатном представлении $\Psi_{n\l m_j}^{(j)} (\bf r)$~--- это шаровые спиноры.
\end{itemize}
\subsection{Вычисление тонкого расщепления}
\begin{itemize}
  \item \emph{Релятивистская поправка.}
  \begin{eqnarray*}
    E_{rel}^{(1)} &=& \int d^3 x \Psi_{n \l m_j}^{(j) +} (\bf r) \left(
        - \dfrac{\hat{\bf p}^4}{3 m^3 c^2}
    \right) \Psi_{n \l m_j}^{(j)} (\bf r)\\
    &=& -\dfrac{1}{2mc^2} \int d^3 \left(
        \Psi_{n \l m_j}^{(j) + }(\bf r) \dfrac{\hat{\bf p}^2}{2m} 
    \right)
    \left(
        \dfrac{\hat{\bf p}^2}{2m} \Psi_{n \l m_j}^{(j)} (\bf r)
    \right)
  \end{eqnarray*}
  Если $\Psi_{n \l m_j}^{(j)}$~--- решение уравнения Дирака, то
  \begin{eqnarray*}
    E^{(1)}_{rel} &=&-\dfrac{1}{2mc^2} \qtri
    {n \l m_j}
    { \Big( E_n^{(0)}  + \dfrac{Z e^2}{r} \Big)^2 }
    { n \l m_j }\\
    &=&
    -\dfrac{1}{2mc^2} \Big((E_n(0))^2  + 
        2 E_n^{(0)} Ze^2 \langle 1/r \rangle + (Ze^2)^2 \langle 1/r^2 \rangle
    \Big)
  \end{eqnarray*}
  Средние для $1/r^\alpha$ вычислялись в первом семетре\footnote{
  $$
    \langle 1/r \rangle = \left(\dfrac{Z}{a}\right) \dfrac{1}{n^2}
  $$
  $$
    \langle 1/r^2 \rangle = \left(\dfrac{Z}{a}\right)^2 \dfrac{1}{n^3 (\l + 1/2)}
  $$
  $$
    \langle 1/r^3 \rangle = \left(\dfrac{Z}{a}\right)^3 \dfrac{1}{n^3 (\l + 1/2) \l (\l + 1)}, \, \l \neq 0
  $$
  }.
  С помощью несложных преобразований:
  $$
    \boxed{
        E_{rel}^{(1)} = E_n^{(0)} \dfrac{Z^2 \alpha^2}{n^2} \left(
            \dfrac{n}{\l + 1/2} - \dfrac34
        \right)
    }
  $$
  Возникает зависимость от квантового числа $\l$. Вырождения нет.
  \item \emph{Спин-орбитальная поправка.} 
  \begin{eqnarray*}
    E_{so}^{(1)} &=& \dfrac{Ze^2}{2m^2c^2} 
    \qtri
    {n \l m_j}
    { \dfrac{(\hat{\bf S} \hat{\bf L})}{r^3} }
    {n \l m_j}
  \end{eqnarray*}
  Воспользуемся вспомогательным соотношением
  $$
    (\hat{\bf S} \hat{\bf L}) = \dfrac{1}{2} (\hat{\bf J}^2 - \hat{\bf L}^2 - \hat{\bf S}^2)
  $$
  Продолжая равенство, получаем:
  \begin{eqnarray*}
    E_{so}^{(1)} &=& \dfrac{Ze^2}{2m^2c^2} \dfrac{\hbar^2}{2}
    \Big(
        j(j+1) - \l(\l+1) - s(s+1)
    \Big) \langle 1/r^3 \rangle
  \end{eqnarray*}
  $$
    \boxed{
        E_{so}^{(1)} = - E_n^{(0)} \dfrac{Z^2 \alpha^2}{n} 
        \dfrac
        {j(j+1) - \l(\l+1) - s(s+1)}
        {2(\l+1/2) \l (\l+1)}
        (1 - \dta_{\l 0})
    }
  $$
  Сложение моментов производится по правилу $j = \l \pm 1/2, \, \l \ne 0$, и $j = 1/2, \, \l = 0$.
  \item \emph{Контактная поправка.}
  $$
    E_{cont}^{(1)} 
  $$
  Заметим, что
  $$
    \langle \dta^{(3)} (\bf r) \rangle = \Big|
        \Psi^{(j)}_{n \l m_j} (0)
    \Big|^2 = \left(
        \dfrac{1}{\pi} \dfrac{1}{n^3} \big(
            \dfrac{Z}{a}
        \big)^3
    \right) \dta_{\l 0}
  $$
  $$
    \boxed{
        E_{cont}^{(1)} = - E_n^{(0)} \dfrac{Z^2 \alpha^2}{n}
    }
  $$
  Если сложить результат по всем $j$, то он не будет зависеть от $\l$ (формула Зоммерфельда).
  $$
    \boxed{
        E_{nj} = -\dfrac{Z^2}{2n^2} \left(
            \dfrac{e^2}{a}
        \right) \left(
            1 + \dfrac{Z^2 \alpha^2}{n^2} \big(
                \dfrac{n}{j + 1/2} - \dfrac34
            \big)
        \right)
    }
  $$
  Этот результат был получен без использования теории Шрёдингера, с помощью правила квантования Зоммерфельда.
\end{itemize}
\subsection{Анализ формулы Зоммерфельда}
\begin{itemize}
  \item $$
    E_{nj} = mc^2 \left[
        1 = \dfrac{Z^2 \alpha^2}{(\sqrt{(j+1/2) - Z^2 \alpha^2} - (n - (j + 1/2)))^2}
    \right]^{-1/2}
  $$
  Если эту формулу разложить при малом $\alpha$, то получится то, что написано выше.
  
  Нужно, чтобы подкоренное выражение в знаменателе было неотрицательно.
  $$
    j_{\min} = 1/2, \qquad Z < Z_{critical} = 137
  $$
  В противном случае может возникнуть нестабильность вакуума, что проявляется в рождении электрон-позитронных пар (парадокс Клейна).
  \item \emph{Квантовые числа.}
  \begin{itemize}
    \item главное: $n = 1, 2, \ldots$
    \item орбитальное: $\l = 1, 2, \ldots, n-1$
    \item внутреннее: $j = \ell \pm 1/2, \, \l \ne 0$; $j = 1/2, \, \l = 0$
    \item магнитное: $m_j = -j, j+1, \ldots, j$
  \end{itemize}
  Можно видеть, что по $\l$ снова наблюдается вырождение (это особенность не только приближённого решения, но даже точного). Есть специфическое Кулоновское вырождение, которое встречается и в релятивистском атоме водорода, $\l = j \pm 1/2$. 
    \item \emph{Кратность вырождения}
    \begin{itemize}
      \item По $m_j$ кратность $(2j+1)$
      \item По $\l$ кратность $2(2j+1)$ кроме $j=j_{\max} = n - 1/2$. Тогда кратность вырождения $1$, $\l = \l^{\max} = n-1$.
    \end{itemize}
    \item Уровни энергии атома водорода обозначаются $n^{\varkappa} \ell_j$, $\varkappa = 2s + 1$.
\end{itemize}
\subsection{Экспериментальная проверка формулы тонкой структуры Зоммерфельда}
Эксперимент ставят с помощью спектральных линий. На этом эксперименте будут наблюдаться только те спектральные линии, которые разрешены правилами отбора (им будет посвящена отдельная лекция).

Мощность излучения дипольного электричества
$$
    W^{d} \sim \Big|
        \qtri{f}{\bf r}{i}
    \Big|^2
$$
$$
    \qtri{n' \l' m_j'}{\bf r}{n \l m_j}
$$
\textbf{Правила отбора.}
\begin{itemize}
  \item Величина $\dta n = n' - n$ может быть любая
  \item $\dta j = j' - j = 0, \pm 1$
  \item $\dta m_j = m_j' - m_j = 0, \pm 1$
\end{itemize}
$$
    \nu_{nn'} = \dfrac{E_n - E^{n'}}{h}
$$
Спектр объясняется формулой Зоммерфельда, кроме двух исключений. (Одно из них объяснила квантовая электродинмаика, второе исключение~--- сверхтонкая структура, связанная с магнитным моментом ядра, который возникает из-за того, что ядро~--- не точечное.)

\section{Трудности в теории Дирака. Античастицы.}
$$
    E = \pm \sqrt{c^2 \bf p^2 + m^2 c^4}
$$
Было выяснено, что уровни энергии образуют два континуума состояний. Через запрещённую зону нельзя <<перепрыгнуть>>. Решения с отрицательным знаком энергии~--- часть общей системы решений.

\subsection{Дырочная интерпретация}
Это была первая теория электрон-позитронного вакуума. Там были объяснены все эффекты, связанные с вакуумом. Все состояния с отрицательным знаком энергии уже заняты, и ничего туда <<впихнуть>> нельзя, в соответствии с принципом Паули. Выходит так, что вокруг нас есть бесконечно большая энергия с отрицательным знаком. Постулируется, что это ненаблюдаемо.

Если частица перескакивает с нижнего слоя на верхний, то рождается электрон-позитронная пара, а <<дырка>> интерпретируется как античастица.

