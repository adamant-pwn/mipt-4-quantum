\setcounter{section}{0}
\section{Теория возмущений}
В этой части будем исследовать приближённые решения уравнения Шрёдингера с возмущённым гамильтонианом.

Рассмотрим возмущённое уравнение Шрёдингера:
$$
\left\{
    \dfrac{i}{\hbar} \ud{}{t} - \hat H
\right\} \qu{\Psi} = 0
$$
Пусть, при этом,
$$
    \hat H = \hat H^\circ + \hat V,
$$
где $\hat H^\circ$~--- гамильтониан невозмущённой задачи, а $\hat V$~--- оператор возмущения.

Чтобы считать возмущение <<малым>>, считаем, что
$$
    \hat V = \lambda \hat U, \qquad \lambda \ll 1
$$
В зависимости от постановки для приближённого решения задачи используют следующие два метода:
\begin{itemize}
  \item Метод Релея-Джинса (стационарная теория возмущений, $V(t) \equiv \mathrm{const}$).
  \item Метод Дирака
\end{itemize}
Позже мы увидим, что нестационарную задачу можно свести к стационарной.

\section{Стационарная теория возмущений}

\subsection{Стационарная теория возмущений в отсутствии вырождения}
В стационарном случае
$$
    \ud{\hat H}{t} = 0, \qquad \qu{\psi(t)} = \exp \big( -\dfrac{iEt}{\hbar} \big) \qu{\psi}
$$
Стационарное уравнение Шрёдингера:
$$
    \hat H \qu{\psi} = E \qu \psi, \qquad \hat H^\circ \qu{\psi^{(\circ)}} = E^{(\circ)} \qu{\psi^{(\circ)}}
$$
Точное решение будем искать методом последовательных приближений при разложении в ряд Тейлора.

Разобьём рассуждение на несколько шагов.

\begin{itemize}
  \item \textbf{Шаг 1.} Предположим, что энергия и решение уравнения Шрёдингера разложимы в ряды (по $\lambda$):
  $$
  \left\{
    \begin{array}{lcl}
      E & = & E^{(0)} + E^{(1)} + \ldots ,\\
      \qu{\psi} & = & \qu{\psi^{(0)}} +  \qu{\psi^{(1)}} + \ldots.
    \end{array}
  \right.
  $$
  Вопросы о сходимости рядов рассматривать не будем. Для этого нужно рассматривать сходимость для каждого конкретного гамильтониана.

  Ряды \emph{асимптотические}, и совершенно не обязаны сходиться в том смысле, в котором мы предполагаем.

  Подставляя в уравнение Шрёдингера, получаем:
  $$
    \Big(\hat H^{(\circ)} + \hat V\Big)
    \Big(
        \qu{\psi^{(0)}} + \qu{\psi^{(1)}} + \ldots
    \Big) =
    \Big(
        E^{(0)} + E^{(1)} + \ldots
    \Big) \Big(
        \qu{\psi^{(0)}} + \qu{\psi^{(1)}} + \ldots
    \Big)
  $$
  Раскрывая скобки и приравнивая коэффициенты при соответствующих степенях $\lambda$, получаем:
  \begin{equation}
      \Big(
        \Hat H^{(\circ)} - E^{(0)}
    \Big) \qu{\psi^{(0)}} = 0
    \label{eq::zero_term}
  \end{equation}
  \begin{equation}
      \Big(
        \hat H^{(\circ)} - E^{(0)}
    \Big) \qu{\psi^{(1)}} = (E^{(1)} - \hat V) \qu{\psi^{(0)}}
    \label{eq::first_term}
  \end{equation}
  \begin{equation}
	    \Big(
        \hat H^{(\circ)} - E^{(0)}
    \Big) \qu{\psi^{(k)}} = (E^{(1)} - \hat V) \qu{\psi^{(k-1)}} + E^{(2)} \qu{\psi^{(k-2)}} + \ldots + E^{(k)} \qu{\psi^{(0)}}
    \label{eq::k_term}  
  \end{equation}
  \item \textbf{Шаг 2.} Считаем спектр энергий уравнения~\eqref{eq::zero_term} невырожденным.

  Тогда каждому значению $E_n^{(0)}$ соответствует единственная волновая функция $\psi_n^{(0)}$, причём семейство функций $\Big\{ \qu{\psi_n^{(0)}} \Big\}$ образует полную ортонормированную систему и является базисом.
  
  Точное решение ищем в виде разложения по этому базису:
  $$
    \qu{\psi_n} = \sum_{m} C_{nm} \qu{\psi_{m}^{(0)}}
  $$
  Таким образом мы ищем поправки к \emph{начальному состоянию с номером $n$}.
  
  Отсюда $C_{nm}^{(0)} = \dta_{nm}$, чтобы в начальном приближении было только одно слагаемое.
  
  \textbf{Теорема Фредгольма.} Неоднородное уравнение имеет решение тогда и только тогда, когда правая часть ортогональна пространству решений однородного уравнения.
  
  Пусть $E_n^{(0)} \to \qu{\psi_n^{(0)}}$~--- пространство решений является одномерным.
  
%  Пользуясь теоремой Фредгольма, находим матричное представление оператора $\hat V$
  $$
    \uq{\psi_n^{(0)}} \hat H^{(\circ)} - E_n^{(0)} \qu{\psi_n^{(1)}} = \uq{\psi_n^{(0)}} E_n^{(1)} - \hat V \qu{\psi_n^{(0)}}
  $$
  Пользуясь ортогональностью, получаем:
  $$
    0 = E_n^{(1)} - \qtri{\psi_n^{(0)}}{\hat V}{\psi_n^{(0)}},
  $$
  откуда получается формула для энергии в первом приближении:
  $$
    \boxed{
        E_n^{(1)} = \qtri{\psi_n^{(0)}}{\hat V}{\psi_n^{(0)}} = V_{nn}
    }
  $$
  Найдём поправки к вектору состояния.
  $$
    \qu{\psi_n^{(1)}} = \sum_m C_{nm}^{(1)} \qu{\psi_m^{(0)}} = \sum_m \qu{\psi_m^{(0)}} \qs{\psi_m^{(0)}}{\psi_n^{(1)}}
  $$
  Подставляя в уравнение~\eqref{eq::first_term}, получаем
  $$
    \sum_m C_{nm}^{(1)} (\hat H^{(\circ)} - E_n^{(0)}) \qu{\psi_m^{(0)}} =  (E_n^{(1)} - \hat V) \qu{\psi_n^{(0)}}
  $$
  Домножая скалярно на $\uq{\psi_k^{(0)}}$, в силу ортогональности, получаем:
  $$
    \sum_{m} C_{nm} ^{(1)} (E_k^{(0)} - E_n^{(0)}) \dta_{km} = E_n^{(1)}\dta_{kn} - \hat V_{kn}
  $$
  Отсюда получается окончательное выражение для коэффициентов $C_{nk}^{(1)}$.
  $$
    \boxed{
        C_{nk}^{(1)} = 
        \left\{
          \begin{array}{lcl}
            \dfrac{V_{kn}}{E_n^{(0)} - E_{k}^{(0)}} & , & n \neq k ,\\
            \text{не определён} & , & n = k.
          \end{array}
        \right.
    }
  $$
  Мы не фиксировали нормировку, поэтому соответствующие коэффициенты не определены.
  $$
    \qu{\psi_n}  \approx 1 + C_{nm}^{(1)} \qu{\psi_n^{(0)}} + \sum_{k \neq n} \dfrac{V_{kn}}{E_{n}^{(0)} - E_{k}^{(0)}} \qu{\psi_k^{(0)}}
  $$
  В этом приближении нормировка не выполняется:
  $$
    1 = |1 + C_{nn}^{(1)}|^2 + \sum(\cdots)
  $$
  $$
    1 = 1 + 2 \mathrm{Re}\, C_{nn}^{(1)}
  $$
  \def \Re{\mathrm{Re}\,}
  \def \Im{\mathrm{Im}\,}  
  Для удобства можно положить $\Re C_{nn}^{(1)} = 0$, $\Im C_{nn} ^{(1)} = 0$, то есть
  $$
    C_{nn}^{(1)} = 0
  $$
  $$
    \boxed{
        \qu{\psi_n} \approx \qu{\psi_n^{(0)}} + \sum_{k \neq n} \dfrac{V_{kn}}{E_n^{(0)} - E_k^{(0)}} \qu{\psi_k^{(0)}}
    }
  $$
  В непрерывном случае:
    \boxed{
        \qu{\psi_n} \approx \qu{\psi_n^{(0)}} + \int d \nu \dfrac{V_{n\nu}}{E_n^{(0)} - E_\nu^{(0)}} \qu{\psi_\nu^{(0)}}
    }
  \item \textbf{Шаг 3.} Поиск поправки к энергии во втором приближении.
  
  Домножим уравнение с номером~\eqref{eq::k_term} скалярно на $\uq{\psi_n^{(0)}}$:
  \begin{eqnarray*}
    0 &=& E_n^{(1)} \qs{\psi_n^{(0)}}{\psi_{n}^{k-1}} - \qtri{\psi_n^{(0)}}{\hat V}{\psi_n^{k-1}} \\
    &&+ E_n^{(2)} \qs{\psi_n^{(0)}}{\psi_n^{k-2}} + \ldots + E_n^{(k)}\underbrace{ \qs{\psi_n^{(0)}}{\psi_n^{(0)}}}_{1}
  \end{eqnarray*}
  $$
    \boxed{
        E_n(k) = \qtri{\psi_n^{(0)}}{\hat V}{\psi_n^{(k-1)}} - \sum_{s=1}^{k-1} E_n^{(s)} \qs{\psi_n^{(0)}}{\psi_n^{k-s}}
    }
  $$
\end{itemize}
