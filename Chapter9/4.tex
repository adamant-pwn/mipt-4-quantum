...о месте квантовой механики...

...эйконал...

%4.3
\subsection{Переход к уравнениям Гамильтона-Якоби}
Перейдём от уравнения Шрёдингера к уравнениям Гамильтона-Якоби.
\def \nla{\nabla}
$$
    i \hbar \ud{}{t} \Psi(\vec r, t) = \left\{
        -\dfrac{\hbar^2}{2m} \vec \nla^2 + U(\vec r)
    \right\} \Psi(\vec r, t)
$$
\begin{equation}
    \Psi(\vec r, t) = A \exp(i/\hbar S(\vec r, t) )
    \label{eq::psi}
\end{equation}

Перед тем, как провести вычисления, сделаем несколько замечаний.

Рассмотрим неизвестную функцию $S(\vec r, t)$. Она имеет размерность \emph{действия}. Хочется назвать её действием, но пока этого делать нельзя. 

Для свободной частицы классическая функция действия имеет вид
$$
    S(\vec r, t) = -E \tau + \vec p \vec r
$$
Эта функция \emph{вещественная}, и строго говоря, функция действия всегда должна быть вещественной. Про уравнения~\eqref{eq::psi} такого гарантировать нельзя. 

$$
    \vec \nla \Psi = \left(
        \dfrac{i}{\hbar} \vec \nla S
    \right) \Psi
$$
$$
    i\hbar \left(
        \dfrac{i}{\hbar} \dfrac{\pd S}{\pd t}
    \right)\Psi = \left\{
        -\dfrac{\hbar^2}{2m} \left(
            (\dfrac{i}{\hbar} \vec \nla S)^2 + \dfrac{i}{\hbar} (\vec \nla^2 S)   
        \right) + U(\vec r) 
    \right\} \Psi
$$
\begin{equation}
\underline{
    -\dfrac{\pd S}{\pd t} = \dfrac{(\vec \nla S)^2}{2m} + U(\vec r) - \dfrac{i\hbar}{2m} \vec \nla^2 S
    }
    \label{eq::quantum_jacobi}
\end{equation}

Если бы последнего слагаемого в~\eqref{eq::quantum_jacobi} <<не было>>, то мы бы имели уравнение Гамильтона-Якоби. Это слагаемое иногда называют <<квантовая поправка>>, если $\hbar$ устремить к нулю, то оно исчезнет.

Представим ряд миров, в которых постоянная Планка принимает всё меньшие и меньшие значения. Что бы было, если бы постоянная Планка была очень велика? Что бы случилось с нашим миром?

Нужен обходной манёвр, чтобы воспользоваться свойством $\hbar \to 0$. (Это константа! Константа никогда не стремится к нулю)

\begin{itemize}
  \item Рассмотрим снова уравнение Шрёдингера, и сгрупируем константы при старшей производной
  $$
    \left\{
        \Dta + \dfrac{2m}{\hbar^2} (E - U(\vec r))
    \right\}\psi(\vec r) = 0
  $$
  Потенциальная энергия существенно изменяется на расстояниях порядка $a$.
  $$
    \dfrac{\hbar^2}{2m E a^2} \to 0
  $$
  Это может означать только то, что $E \to \infty$ ($n \gg 1$)
  \item \textbf{Условие квазиклассичности, применимость квазиклассического приблжения.}
  
  Перейти к квазиклассическому приближению можно, если поправка мала.
  $$
    \left|
        \dfrac{i\hbar \nla^2 S}{(\nla S)^2}
    \right| \ll 1
  $$
  $$
    \dfrac{\hbar |\nla^2 S|}{(\nla S)^2} \ll 1
  $$
  В классике импульсу соответствует величина $\nla S = \vec p$. Тогда выполнялось бы
  $$
    \dfrac{\hbar}{p^2} |\mathrm{div} \, \vec p | \ll 1
  $$
  $$
    \left|
        \dfrac{\hbar}{p^2} \ud{p}{x}
    \right|\ll 1
  $$
  $$
    \left|
        \dfrac{\hbar}{p^2} \ud{p}{x}
    \right| = \left|
    ud{}{x} \left(
        \dfrac{\hbar}{p}
    \right)
    \right|
  $$
  Получаем условия квазиклассического приближения
  $$
    \left|
        \ud{\lam_{\text{д.Б.}}}{x}
    \right|
  $$
  Другими словами, изменение длины волны де-Бройля на расстояниях порядка волны де-Бройля имеет порядок волны де-Бройля.
  \item Перейдём к более, чем одномерному пространству.
  $$
    p(x) = \sqrt{2m (E - U(x))}
  $$
  $$
    \ud{p}{x} = \dfrac{1}{p} m \left(
        -\ud{U}{x}
    \right) = \dfrac{m}{p} \cdot F
  $$
  $$
    \left|
        \dfrac{\hbar m}{p^2} \ud{U}{x}
    \right| \ll 1
  $$
  Эти условия плохо работают при малых импульсах (импульс пытается попасть в знаменатель).
  
  Совсем не применима эта теория в классических точках поворота: $p(x) = p$.
  
  Из этого равенства вытекает также, что нужен плавный ход потенциальный кривой. (Аналогия с геометрической оптикой: плавный ход показателя преломления).
\end{itemize}

\section{Это должен быть 5 параграф. Квазиклассический метод ВКБ}
\begin{flushright}(Вентцель, Крамерс, Бриллюэн)\end{flushright}

Это изучалось как метод приближённого решения уравнений типа уравнения Шрёдингера с малым параметром при старшей производной.

\begin{itemize}
  \item Карлини (1817)
  \item Лиувилль (1837)
  \item Дебай (1909)
  \item Релей (1912)
  \item Джеффрис (1923)
  \item ВКБ (1926)
\end{itemize}

Идея этого метода шире, чем просто классическое приближение. Он позволяет не просто решать уравнения Гамильтона-Якоби, а получать квантовые поправки к решению этого уравнения с учётом малости постоянной Планка.

\subsection{Общий вид решения} 
Нужно представить функцию $S$ (решение~\eqref{eq::quantum_jacobi}) в виде
$$
    S(\vec r, t) = \underset{\sim \hbar^2}{S_0} + S_1 + S_2 + \ldots
$$
Метод нахождения поправок является очень общим, потому что приближения можно проводить не только в квазиклассической области ($E > U(x)$), но и в классически запрещённой.

Рассмотрим конкретный вид потенциала, $x_0$~--- классическая точка поворота. Ось $x$ разбивается на области $\mathrm{I}$ и $\mathrm{II}$.

\centerline{\includegraphics[width=0.4\textwidth]{pic/9/vkb.pdf}}
$$
    S(x, t) = -Et + S(x)
$$
$$
    (S')^2 - i\hbar (S'') = 2m(E - U(x)) = p^2(x)
$$
$$
    (S_0' + S_1' + \ldots)^2 = i\hbar (S_0'' + S_1'' + \ldots) = p^2(x)
$$
Малый параметр равен $\dfrac{\hbar}{p a} \sim \dfrac{1}{p}$
\begin{itemize}
  \item \textbf{Область \texttt{I}}.
  
  В этой области $E > U(x)$, $p^2(x) > 0$
  $$
    \left\{
      \begin{array}{lcl}
        (S_0')^2 & = & p^2(x) ,\\
        2S_0'S_1' - i\hbar S_0'' & = & 0 .
      \end{array}
    \right.
  $$
  Из первого уравнения
  $$
      S_0  =  \pm \int\limits_{x}^{x_0} p(x') dx'
  $$
  Из второго
  \begin{eqnarray*}
    S_1' &=& \dfrac{i\hbar}{2} \dfrac{S_0''}{S_0'} = \dfrac{i\hbar}{2} \dfrac{\mp p'(x)}{\pm p(x)}\\
    &=& \dfrac{i\hbar}{2} \dfrac{d (\ln p(x))}{dx}
  \end{eqnarray*}
  Отсюда
  $$
    S_1 = \dfrac{i\hbar}{2} \ln p(x) = i \hbar \ln \sqrt{p(x)}
  $$
  $$
    \psi_{x < x_0} = A e^{\frac{i}{\hbar} S} \approx \dfrac{A}{\sqrt{p(x)}} \exp\left(
        \pm \dfrac{i}{\hbar} \int\limits_{x}^{x_0} p(x') dx'
    \right)
  $$
  \item \textbf{Область \texttt{II}}.
  
  В этой области $E < U(x)$, $p^2(x) > 0$.
  $$
    p(x) = \pm i \sqrt{ 2m(U(x) - E)} = \pm i |p(x)|
  $$
  Если ввести такое обозначение, то дальше решение получается по аналогии
  $$
    \psi_{x > x_0} = A e^{\frac{i}{\hbar} S} \approx \dfrac{A}{|\sqrt{p(x)}|} \exp\left(
        \pm \dfrac{1}{\hbar} \int\limits_{x_0}^{x} |p(x')| dx'
    \right)
  $$
\end{itemize}

\textbf{Анализ решения.}

В классически разрешённой области имеем осциллирующее решение. Бросается в глаза импульс, стоящий в знаменателе.

$$
    \left\{
      \begin{array}{lcl}
        \psi_{\mathrm{I}} & = & \dfrac{1}{\sqrt{p}} \left\{
            a \sin (z + \gamma_1) + b \cos (z + \gamma_2)
        \right\} , \quad z = \dfrac{1}{\hbar}\int\limits_{x}^{x_0} p(x') dx'\\
        \psi_{\mathrm{II}} & = & .
      \end{array}
    \right.
$$
Оказывается, что нужно сшивать эти функции в тех точках, где приближение не работает!