%paragraph 3
%\setcounter{\section}{2}
\section{Нестационарная теория возмущений. Теория квантовых переходов}
\subsection{Постановка задачи. Общие формулы}
Рассмотрим естественное уравнение Шрёдингера
\def \pd{\partial}
\begin{equation}
   \left\{ i \hbar \dfrac{\pd}{\pd t} - \hat H \right\} \qu{\Psi(t)} = 0
   \label{eq::schrodinger}
\end{equation}

Считается, что гамильтониан можно представить в виде суммы
$$
    \Hat H = \Hat H^{(0)} + \Hat V, \quad\hat  V = \hat V(t)
$$
В отличие от предыдущего подхода, тут будем считать, что возмущение \emph{зависит от времени}. В этом заключается основное отличие.

Зависимость $\Hat H$ от времени характерна для незамкнутых систем.

Например, поле атома, взаимодействующее с внешним полем волнового типа.

\begin{itemize}
\item При постановке задачи будем считать, что
$$
    \Hat V(t) =
      \begin{cases}
        \Hat W(t), & 0 < t < T \\
        0, & t \le 0, \, t \ge T.
      \end{cases}
$$
Другими словами, если одно слагаемое зависит от времени (система нестационарная), то и весь гамильтониан зависит от времени.  У такой системы вообще не может быть стационарных состояний.

Считаем, что в задаче есть некоторый безразмерный параметр $\lam \ll 1$.
\item При $t \le 0, \quad \Hat H = \Hat H^{(0)}$. Возмущение ещё не включилось.

Положим, что в этой системе у гамильтониана существовала полная система собственных векторов, отвечающая дискретному спектру.
$$
    \left\{
    i\hbar \dfrac{\pd}{\pd t} - \hat H^{(0)}     \right\} \qu{\Psi_{n}^{(0)} (t)} = 0
$$
Как и положено для стационарных состояний,
$$
    \qu{\Psi_{n}^{(0)} (t)} = e^{- i E_n^{(0)} / \hbar}, \quad \Hat H^{(0)} \qu{\psi_{n}^{((0)}} = E_n^{(0)} \qu{\psi_n^{(0)}}
$$
Для удобства введём обозначения
$$
    \omega_n = \dfrac{E_n^{(0)}}{\hbar}, \quad \qu{\Psi_n^{(0)} (t)} = e^{-i \omega_n t} \qu{\psi_n^{(0)}}
$$
Эти <<уровни энергии>> $E_n^{(0)}$ такими и останутся (задача нестационарная), поэтому индекс $(0)$ сверху можно не писать.

Полное решение уравнения Шрёдингера до включения возмущения можно записать в виде разложения по этой системе
$$
    \qu{\Psi^{(0)} (t)} = \sum_n C_n e^{-i \omega_n t} \qu{\psi_n^{(0)}} =
\sum_n C_n \qu{\Psi_n^{(0)} (t)}
$$
В духе вероятностной интерпретации, коэффициенты $C_n$~--- это амплитуды соответствующих вероятностей.
$$
    |C_n|^2 = w_n
$$
\item  Посмотрим на уравнение~\eqref{eq::schrodinger}. Пусть <<включено>> возмущение $\Hat V(t)$.

Считаем, что указанные выше коэффициенты являются функциями времени:
\begin{eqnarray*}
    \qu{\Psi (t)} &=& \sum_n C_n (t) e^{-i \omega_n t} \qu{\psi_n^{(0)}} \\
&=& \sum _n C_n (t) \qu{\Psi_n^{(0)} (t)}
\end{eqnarray*}
Вероятности тоже являются функциями времени
$$
    |C_n(t)|^2 = w_n(t)
$$
$$
    i \hbar \sum_{k} \left\{
    \dot C_k \qu{\Psi_k^{(0)} (t)} + C_k \dfrac{\pd}{\pd t} \qu{\Psi_k^{(0)} (t)}
    \right\} =
\sum_{k} C_k \left(
        \hat H^{(0)} \qu{\Psi_k^{(0)} (t)} + \hat V \qu{\Psi_k^{(0)} (t)}
    \right)
$$
Для начала, учтём уравнение Шрёдингера нулевого (невозмущённого) порядка приближения. Из него:
$$
    i \hbar \sum_{k} \left\{
    \dot C_k \qu{\Psi_k^{(0)} (t)}
    \right\} =
\sum_{k} C_k \left(
         \hat V \qu{\Psi_k^{(0)} (t)}
    \right)
$$
Умножим исходное равенство слева на $\uq{\Psi_{m}^{(0)}} (t)$ и учтём ортонормированность системы.
$$
    i \hbar \dot C_m = \sum_k C_k V_{mk} e^{i \omega_{mk} t},
$$
где
$$
    V_{mk} = \qtri{\psi_m^{(0)}}{\hat V}{\psi_k^{(0)}}, \qquad \omega_{mk} = \dfrac{E_m - E_k}{h}
$$
Важно понимать: получившийся коэффициент $V_{mk}$ зависит от времени.

Для начала отметим, что данное уравнение является \emph{точным}. Хорошее оно или нет? Если уровней энергии не очень много ($k = 1, 2, 3$), то уравнения можно решать. А что делать в случае, когда их больше, или когда спектр непрерывный? Получается интегро-дифференциальное уравнение. Решить его в общем виде нельзя. Значит, нужно применять какие-то приближения. Они представляют собой \emph{метод Дирака}, также известный как метод квантовых переходов.
\item \textbf{Метод Дирака}

Предположим, что $C_n(t)$ можно представить в виде функционального ряда
$$
    C_k(t) = C_k^{(0)} (t) + C_k^{(1)} (t) + \ldots,
$$
где $C_k^{(m)} (t)$ содержит малый параметр в качестве <<множителя>> в степени $m$.

Пусть до <<включения>> возмущения система находилась в стационарном состоянии с номером $n$.

Включается возмущение $C_k(0) = C_k^{(0)} (0) = C_k^{(0)} = \dta_{nk}$.

Итак, в разложении $\Psi(t)$ в начальный момент времени есть только одно слагаемое, с номером $n$.

Удобно обозначить вместо $C_k$ коэффициентом $C_{nk}$, где $n$~--- начальное состояние, а $k$~--- состояние, к которому этот коэффициент относится, поправка.

Подставляя в уравнение, получаем
$$
    i \hbar (\underbrace{\dot C_{nm}^{(0)}}_{=0} + \dot C_{nm}^{(1)} + \ldots) \simeq
    \sum_k (C_{nk} ^{(0)} + \underbrace{C_{nk} ^{(1)}}_{\sim \lam^2} + \ldots) V_{mk} e^{i \omega_{mk} t}\\ 
$$
\textbf{Уравнение в первом приближении:}
$$
    i \hbar \dfrac{\pd}{\pd t} C_{nm}^{(1)} = V_{mn} e^{i \omega_{mn} t}
$$
Если формально провести интегрирование по времени, то соответствующее выражение для $C_{nm}$ запишется вот каким образом:
$$
    C_{nm}^{(1)} (T) = \dfrac{1}{i\hbar} \int\limits_{0}^{T} V_{mn} (t') e^{i\omega_{mn} t'} dt'
$$
Смысл этого коэффицента, ещё раз напомним, в том, что он даёт поправку \emph{первого порядка}.
$$
    \qu{\Psi(t)} = \sum_{m} C_{nm} (t) \qu{\Psi_m^{(0)} (t)}
$$
до включения возмущения $n = m$. Теперь эти коэффициенты, отличные от нуля, становятся ненулевыми. Следовательно, если рассмотреть
$$
    |C_{nm}^{(1)} (T)|^2 = P_{nm},
$$
где $P_{nm}$~--- вероятность обнаружить систему в состоянии $m$ в момент $T$, если она изначально находилась в состоянии $n$. Другими словми, это \emph{вероятность квантового перехода} из состояния $n$ в состояние $m$ за время $T$.
\end{itemize} 

Говорить о стационарных состояних можно тогда, когда возмущение снято. Именно поэтому мы рассматриваем конечный промежуток времени. Промежутки времени должны быть не очень велики, тогда можно говорить о дискретных квантовых переходах.

Вероятность перехода:
$$
    P_{nm} (T) = \dfrac{1}{h^2}\left|
\int\limits_{0}^{T} V_{mn} (t') e^{i\omega_{mn} t'} dt'
    \right|^2
$$

%\ast
\subsection{Адиабатическое и внезапное включение возмущения}
\textbf{Первый случай.}
%\centerline{\includegraphics[width=0.6\textwidth]{pic/9/v.pdf}}
\begin{center}
\begin{tikzpicture}
\begin{axis}[width=6in,axis equal image,
    axis lines=middle,
    xmin=-0.1,xmax=3,samples=201,
    xlabel=$t$,ylabel=$V(t)$,
    ymin=-0.1,ymax=1.5,
    enlargelimits={abs=1cm},
    axis line style={-latex},
    ticklabel style={font=\small,fill=white},
    xtick={\empty},ytick={\empty},
    extra x ticks={3},
	extra x tick style={xticklabel=$T$}
    ]
\addplot+[no marks, thick, black, rounded corners=1ex] coordinates {(0,0)  (0.1, 0) (0.2,0.5) (0.3, 1) (2.9, 1) (3,0.5) (3.1, 0) (4, 0)};
\addplot+[dashed, no marks, black] coordinates {(0.05,-1) (0.05,1.5)};
\addplot+[dashed, no marks, black] coordinates {(0.31,-1) (0.31,1.5)};
\addplot[latex-latex, no marks, black] coordinates {(0.05, 1.3) (0.31, 1.3)} node[above, midway] {$\tau$};
\addplot+[dashed, no marks, black] coordinates {(2.88,-1) (2.88,1.5)};
\addplot+[dashed, no marks, black] coordinates {(3.13,-1) (3.13,1.5)};
\addplot[latex-latex, no marks, black] coordinates {(2.88, 1.3) (3.13, 1.3)} node[above, midway] {$\tau$};
\end{axis}
\end{tikzpicture}
\end{center}
$$
    C_{nm} (t) = \underbrace{\left.\dfrac{V_{mn} (t')}{i\hbar} \dfrac{e^{i \omega_{mn} t'}}{i \omega_mn} \right|_{0}^{t}}_{=0} +
\dfrac{1}{\hbar \omega_{mn}} \int\limits_0^T \dfrac{d V_{mn} (t')}{dt'} e^{i \omega_{mn} t'} dt'
$$
Это будет работать, если изменение матричного элемента на временах характерного периода мало изменяется по сравнению с разностью уровней соответствующих состояний.
$$
    \dfrac{d V_{mn}}{dt} \dfrac{1}{\omega_{mn}} \ll \hbar \omega_{mn}
$$
$$
    T_{mn} \simeq \dfrac{2\pi}{\omega_{mn}}
$$
Из теоремы Лагранжа о среднем можно получить некую оценку на выписанный выше интеграл:
$$
    P_{nm} (T) \cong \left|
    \dfrac{d V_{mn}}{dt}
\right|_{t = T} \left(
    \dfrac{2}{\hbar \omega_{mn}^2}
\right)^2 \sin^2 \left(
    \dfrac{\omega_{mn} T}{2}
\right)
$$
Эта вероятность экспоненциально мала, $P_{nm} (T) \ll 1$. В первом приближении система остаётся в исходном состоянии. Адиабатические изменения не приводят к изменению состояния.\footnote{Существует целая теория адиабатических возмущений. Она включена в задание.} 

\textbf{Второй случай.}
$$
    \tau \ll T_{mn} \sim \dfrac{1}{\omega_{mn}}
$$
Выпишем те же уравнения.

Теперь основному изменению подвержен матричный элемент. Под интегралом от производной получим полый скачок этой функции
$$
    P_{nm} \cong \dfrac{|V_{mn}|^2}{(\hbar \omega_{mn})^2}
$$ 
\subsection{Вероятность переходов в единицу времени}
\subsubsection{Постоянное возмущение}
$$
    P_{nm} (T) = \dfrac{1}{\hbar^2} \left|
    V_{mn} \dfrac{e^{i \omega_{mn} t} - 1}{i \omega_{mn}}
    \right|^2 = \dfrac{|V_{mn}|^2}{\hbar^2 \omega_{mn}^2} 2(1 - \cos (\omega_{mn} T))
$$
$$
    f(\omega, T) = \dfrac{2}{\omega^2} \big( 1 - \cos(\omega T) \big)
$$

%\centerline{\includegraphics[width=0.6\textwidth]{pic/9/fwt.pdf}}
\begin{center}
\begin{tikzpicture}
\begin{axis}[width=6in,axis equal image,
    axis lines=middle,
    xmin=-8,xmax=8,samples=400,
    xlabel=$\omega$,ylabel={$f(\omega, T)|_{T=2}$},
    unbounded coords=jump,
    ymin=0,ymax=4.5,
    enlargelimits={abs=1cm},
    %enlargelimits=false,
    axis line style={-latex},
    ticklabel style={font=\small,fill=white},
    xtick={0, 3.14159,-3.14159, 6.28318, -6.28318},
    ytick={4},
    xticklabels={$0$, $\frac{2\pi}{T}$, $-\frac{2\pi}{T}$, $\frac{4\pi}{T}$, $-\frac{4\pi}{T}$},
    yticklabel={$T^2$},
    restrict y to domain=-4:4
%   extra x ticks={3},
%	extra x tick style={xticklabel=$T$}
	]
\addplot[black,no marks,domain=1e-1:10] {(1 - cos(2*deg(x))) * 2 / x^2};
\addplot[black,no marks,domain=-10:-1e-1] {(1 - cos(2*deg(x))) * 2 / x^2};
\addplot[black,no marks,domain=-1e-1:1e-1] {4 - 4*x^2 / 3};
\end{axis}
\end{tikzpicture}
\end{center}

$$
    \int\limits_{-\infty}^{+\infty} f(\omega, T) d\omega = 2 \pi T = \textrm{const}
$$
При $T \to \infty$ функция переходит в $\dta$-функцию.
$$
    \underset{T \to \infty}\lim \left(
        \dfrac{f(\omega, T)}{T}
    \right) = 2 \pi \dta(\omega)
$$
Итак, будем полагать, что время действия возмущения достаточно велико.
\begin{eqnarray*}
    P_{nm} (T) \Big|_{t \gg \frac{1}{\omega_{mn}}} &\simeq& \dfrac{2 \pi}{\hbar^2} |V_{mn}|^2 \cdot T \dta(\omega_{mn})\\
&=& \dfrac{2\pi}{\hbar} |V_{mn}|^2 T \dta (E_m - E_n)
\end{eqnarray*}
Поэтому можно ввести вероятность переходов в единицу времени:
$$
    w_{mn} \eqdef \dfrac{P_{mn} (T)}{T} = \dfrac{2\pi}{\hbar} |V_{mn}|^2 \dta (E_m - E_n)
$$
\begin{itemize}
  \item Спектр дискретный или непрерывный?
  \item Нужно найти вероятность перехода во все прочие состояния
$$
    \left\{
    \begin{array}{rl}
        & \nearrow \\
      n & \rightarrow \\
        & \searrow \\
    \end{array}
    \right\} m
$$
  \item $$
    W = \sum_m w_{nm} \to \int w_{nm} d \nu_m
$$
Раньше было дискретное квантовое число $m$. Теперь в единицу времени происходит какое-то непонятное число переходов (спектр непрерывен?). В непрерывном спектре уровни энергии вырождены. Состояние определяется не только модулем импульса, но и его направлением. В общем случае $dE_m \ne d \nu_m$.
$$
    d \nu_m = \rho(E_m) dE_m
$$
Проинтегрируем по энергии конечных состояний.
$$
    W_n = \int w_{nm} \rho(E_n) dE_m = \left.\dfrac{2\pi}{\hbar} |V_{mn}|^2 \rho (E_m) \right|_{E_m = E_n}
$$
Получили <<золотое>> правило Ферми (полная вероятность перехода в единицу времени в группу состояний).
\end{itemize}

\subsubsection{Периодическое во времени возмущение}
$$V(t) = V^{\pm} e^{\pm i \omega t}$$
Обычно требуется эрмитовость оператора возмущения,
$$
    \Hat V(t) = V^{-} e^{-i \omega t} + V^{+} e^{i\omega t} \qquad (\omega \to \pm \omega_{mn})
$$
$$
    C_{nm}^{(1)} (t) = \dfrac{1}{i \hbar} \int\limits_0^T V_{mn}^{\pm} \underline{e^{\pm i \omega t}} e^{i \omega_{mn} t} dt
$$
Аналогичным образом можно получить вероятность переходов в единицу времени.
$$
    w_{nm}^{\pm} = \dfrac{2\pi}{\hbar} |V_{mn}^{\pm}|^2 \dta(E_m - E_n \pm \hbar \omega)
$$
Если проинтегрировать по энергии конечных состоянии, то придём к аналогу правила Ферми
$$
    2 \dfrac{\pi}{\hbar} |V_{mn}^{\pm}|^2 \rho(E_n \mp \hbar \omega)
$$
\begin{itemize}
  \item Если брать нижний знак, то новая энергия~--- это старая энергия плюс $\hbar \omega$, то есть система поглощает энергию.
$$
    \hat V^{-} e^{-i \omega t} \to E_m = E_n + \hbar \omega
$$
  \item Если брать верхний знак, то описывается \emph{излучение} энергии.
$$
    \hat V^{+} e^{+i \omega t} \to E_m = E_n - \hbar \omega
$$
\end{itemize}
