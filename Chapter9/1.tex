\setcounter{chapter}{8}

\chapter{Симметрия в квантовой механике и законы сохранения}

\setcounter{section}{9}

\section{Задача о сложении угловых моментов}
%advanced

\subsection{Прямое произведение пространств}

Квантовую систему можно рассматривать как совокупность некоторых подсистем. При этом можно считать, что они
\begin{itemize}
  \item либо не взаимодействуют совсем
  \item взаимодействие настолько мало, что может быть учтено методом последовательных приближений
\end{itemize}

\Quest{Как строить векторы состояния \emph{всей системы}, если известны векторы состояния подсистем?}

\Quest{Как строить наблюдаемые величины в такой системе?}

\Def Пусть имеется две кинематически независимые подсистемы. Пусть пространство состояний первой системы $\H_1$, второй системы~--- $\H_2$. Тогда пространство состояний всей системы обозначается
$$
    \H(1,2) = \H(1) \otimes \H(2)
$$
и называется \emph{прямым произведением} пространств состояний первой и второй системы.

Пусть имеется некоторое отображение $f$, которое ставит каждой паре
$
\left\{
  \begin{array}{ll}
    \qu{\psi_1} \in \H(1) & \hbox{} \\
    \qu{\psi_2} \in \H(2) & \hbox{}
  \end{array}
\right.
$
некоторый новый вектор
$$
    \qu{\psi_1} \otimes \qu{\psi_1} \in \H(1) \otimes \H(2)
$$
\def \ot{\otimes}
Это отображение должно обладать следующими свойствами:
\begin{enumerate}
  \item Линейность по первому и второму сомножителю:
  $$
    \qu{\psi_1} \ot \qu{\psi_3} + \qu{\psi_2} \ot \qu{\psi_3} = (\qu{\psi_1} + \qu{\psi_2}) \ot \qu{\psi_3}
  $$
  \item Множеству пар линейно независимых векторов $\big\{ \qu{\psi_1}_i, \qu{\psi_2}_j \big\}$ ставятся тоже \emph{линейно независимые векторы} $\qu{\psi_1}_i \ot \qu{\psi_2}_j$ в пространстве $\H(1,2)$
\end{enumerate}

Тогда множество линейных комбинаций
$$
    \sum_{ij} a_{ij} \qu{\psi_1}_i \ot \qu{\psi_2}_j
$$
образует линейное пространство, которое обозначается $\H(1) \ot \H(2)$ и является подпространством $\H(1,2)$.

\Rem Определим \emph{скалярное произведение} в новом пространстве.

Будем пользоваться обозначением $\qu{\psi_1 \psi_2} = \qu{\psi_1} \ot \qu{\psi_2}$.

Положим по определению
$$
    \qs{\psi_1 \psi_2}{\psi'_1 \psi'_2}  = \qs{\psi_1}{\psi'_1} \qs{\psi_2}{\psi'_2}
$$
Как следует из физического смысла, вероятности независимых событий перемножаются. Квадраты модулей скалярных произведений есть ни что иное как вероятностиы.

Произведение размерностей $\textrm{dim} (\H(1)) \cdot \textrm{dim} (\H(2)) = \textrm{dim} (\H(1) \ot \H(2))$.

\subsection{Система двух угловых моментов}

В системе есть две частицы, обладающие соответственно угловыми моментами (это не квантовые операторы!)
$$
    \mathbf J(1), \quad \mathbf J(2)
$$

\Reminder Для любого углового момента очень важно помнить, как устроена общая система собственных векторов $\Hat{ \mathbf{J}}^2, \, \Hat{J}_z$:
\def \jsq {\Hat{\mathbf J}^2}
\def \jz  {\Hat J_z}
\begin{equation}
\left\{
  \begin{array}{ll}
    \jsq \qu{jm} = \hbar^2 j(j+1) \qu{jm}, & \hbox{} \\
    \jz \qu{jm} = \hbar m \qu{jm}. & \hbox{}
  \end{array}
\right.
\label{eq::common_eigs}
\end{equation}

Пространство размерности $j+1$ и есть пространство состояний углового момента.

Для частицы 1 есть операторы $\jsq(1)$, $\jz(1)$. Их собственные векторы образуют систему из $2j_1 + 1$ собственных векторов. Пространство состояний первой частицы есть $E_{(2j_1 + 1)}$ (евклидово пространство).

Для частицы 2 есть аналогичное пространство $E_{(2j_2 + 1)}$.

Так как частицы \emph{не взаимодействуют}, то компоненты угловых моментов должны быть независимо измеримы:
$$
    [\Hat J_i (1), \Hat J_j(2)] = 0
$$

Пространство состояний для этих двух частиц:
$$
    E_{(2j_1 + 1)} \ot E_{(2j_2 + 1)} = E_{(2j_1 + 1)(2j_2 + 1)}
$$
Полная система совместных наблюдаемых (операторов):
$$
    \jsq (1), \jz(1), \jsq(2), \jz(2)
$$

%Пробил час перейти к задаче сложения моментов в квантовой механике.
\textbf{Физическая интерпретация сложения моментов}

\def \vecJ {\Hat{\mathbf J}}
Определим оператор сложения моментов $\Hat {\mathbf J} = \Hat {\mathbf J}(1) + \Hat {\mathbf J}(2)$.

\textbf{Проблема!} Записывая правую часть в таком виде, сложение операторов, действующих в \emph{разных пространствах}, некорректно. Сумму надо будет записать немного по-другому.

\textbf{Теоретико-групповой подход} 

\Quest{Что говорилось про группу трёхмерных вращений?}

Для любого углового момента $\vecJ$ верна система уравнений~\eqref{eq::common_eigs}. Построенное пространство \emph{не содержит}  внутри себя инвариантных подпространств относительно действия операторов $\Hat J_i$.

В соответствии с леммой Шура, в этом пространстве реализовано неприводимое представление группы вращений $D(j)$.

***

Для частицы 1 в пространстве состояний реализовано неприводимое представление $D(j_1)$, аналогично для частицы 2.

\subsection{Прямое произведение представлений}

\Reminder Представление~--- это конструкция, которая предполагает, что любому элементу группы ставится в соответствие преобразование пространства, при этом должна сохраняться групповая операция.

Группа трёхмерных вращений~--- это группа преобразования координат. Исходная координата $\mathbf r$ при этом переходит в $\mathbf r'$.
\def \bf {\mathbf}
$$
    \bf r \to \bf r' = \Hat G \bf r = \Hat T_{\vec \phi} \bf r
$$
Генераторы представления:
$$
    \Hat U(\vec \phi) = \exp \left(
        \dfrac{i}{\hbar} \phi \big(\bf n, \vecJ (1)\big)
    \right)
$$
в $E_{(2j_1 + 1)} \equiv E_1$,
$$
    \Hat U(\vec \phi) = \exp \left(
        \dfrac{i}{\hbar} \phi \big(\bf n, \vecJ (2)\big)
    \right)
$$
в $E_{(2j_2 + 1)} \equiv E_2$.

При этом из $E_1$, $E_2$ мы составили $E_1 \ot E_2$.

***

\Quest{Какие преобразования отвечают унитарным преобразованиям в построенном пространстве?}
$$
    \left\{
      \begin{array}{ll}
        \qu{\psi_1} \to \qu{\psi'_1}, & \hbox{} \\
        \qu{\psi_2} \to \qu{\psi'_2}, & \hbox{}
      \end{array}
    \right.
    \RA
    \qu{\psi'_1} \ot \qu{\psi'_2} = (\Hat U_1 \qu{\psi_1}) \ot (\Hat U_2 \qu{\psi_2})
    =
    (\Hat U_1 \ot \Hat U_2)(\qu{\psi_1} \ot \qu{\psi_2})
$$
\Lem Для операторов $\Hat U_1$, $\Hat U_2$ групповая операция представления группы $SO(3)$ в $E_1 \ot E_2$ сохраняется.
$$
    \Hat{U'} = \Hat{U_1} \ot \Hat{U_2}
$$
$$
    \Hat{U'} (G_1 G_2) = \Hat{U'}(G_1) \Hat{U'} (G_2)
$$
\Def Представление группы $G$, которое реализовано при помощи операторов $\Hat U_1 \ot \Hat U_2$ в пространстве состояний $E_1 \ot E_2$,  называется \emph{прямым произведением представлений}.

\subsection{Генераторы прямого произведения представлений}

\Reminder Инфинитезимальные операторы нужны для того, чтобы описать окрестность единичного элемента группы преобразований. Говоря другим языком, они определяют соответствующую группу Ли.
$$
    \left\{
      \begin{array}{ll}
        \Hat U_1 (\dta \vec \phi) \simeq \Hat 1 + \dfrac{i}{\hbar} \dta \phi \big( \bf n, \vecJ(1) \big), & \hbox{} \\
        \Hat U_2 (\dta \vec \phi) \simeq \Hat 1 + \dfrac{i}{\hbar} \dta \phi \big( \bf n, \vecJ(2) \big). & \hbox{} \end{array}
    \right.
$$
$$
    \Hat U_1 (\dta \vec \phi) \ot \Hat U_2 (\dta \vec \phi) \simeq \Hat 1 + \dfrac{i}{\hbar}
    \dta \phi \left(
        \bf n ,
        \big(
            \vecJ(1) \otimes \Hat 1 + \Hat 1 \ot \vecJ(2)
        \big)
    \right)
$$
Роль инфинитезимальных операторов играют операторы
$$
    \vecJ = \vecJ(1) \ot \Hat 1 + \Hat 1 \ot \vecJ(2)
$$
Введением <<единичек>> устраняется некорректность, описанная выше (операторы действуют в разных пространствах).

***

В пространстве $E_1 \ot E_2$ есть независимая система наблюдаемых
$$
    \jsq (1), \jsq(2), \jz(1), \jz(2).
$$
Система собственных векторов:
$$
    \qu{j_1 j_2 m_1 m_2}.
$$
Оказывается, что построенное нами представление является \emph{приводимым}. Поэтом построенное пространство содержит инвариантное подпространство. Выделим его, выделив все базисные векторы.

Чтобы построить базис такого пространства, надо построить общие собственные векторы $\jsq, \jz$.
$$
    \left\{
      \begin{array}{ll}
        \jsq \qu{JM} = \hbar^2 J(J+1) \qu{JM}, & \hbox{} \\
        \jz \qu{JM} = \hbar M \qu{JM}. & \hbox{}
      \end{array}
    \right.
$$
Вспомним, что для фиксированного $J$ число $M$ пробегает значения
$$
    M = \underbrace{-J, -J +1, \ldots, J}_{2J + 1}
$$
Задача сводится к нахождению всевозможных значений $J$ (то есть $J_1$, $J_2$, $\ldots$).

Построенное пространство раскладывается в прямую сумму
$$
    E_1 \ot E_2 = \H(J_1) \oplus \H(J_2)
$$
\Th (о сложении моментов)
$$
    \qu{j_1 j_2 J M} = \sum_{m_1 m_2} C_{j_1 j_2 m_1 m_2}^{JM} \qu{j_1 j_2 m_1 m_2},
$$
где $C_{j_1 j_2 m_1 m_2}^{JM}$~--- коэффициенты Клебша-Гордана.

\textbf{Правила.}
\begin{enumerate}
  \item $\jz \qu{j_1 j_2 m_1 m_2} = (\jz(1) \ot \Hat 1 + \Hat 1 \ot \jz(2)) \qu{j_1 m_1} \ot \qu{j_2 m_2}$
  $$
    \jz \qu{j_1 j_2 m_1 m_2} = \hbar(m_1 + m_2) \qu{j_1 j_2 m_1 m_2} = \hbar M\qu{j_1 j_2 m_1 m_2}
  $$
  Таким образом, матрица $\jz$ является диагональной уже в старом базисе. Отсюда
  $$
    M = m_1 + m_2
  $$
  \item Связь между квантовыми числами $J$ устроена гораздо сложнее.
  Для начала,
  $$
    M_{\max} = j_1 + j_2
  $$
  Затем, сдвигаемся по диагонали таблицы:
  $$
    \left\{
      \begin{array}{ll}
        J = j_1 + j_2, & \hbox{} \\
        M = j_1 + j_2 - 1. & \hbox{}
      \end{array}
    \right.
  $$
  $$
    \left\{
      \begin{array}{ll}
        J = j_1 + j_2 - 1, & \hbox{} \\
        M = j_1 + j_2. & \hbox{}
      \end{array}
    \right.
  $$
  Сдвинулись до следующей диагонали, и т.д.
  
  \centerline{\includegraphics[width=0.6\textwidth]{./pic/9/qu.pdf}}
  
  Надо обратить внимание, что далее порядок матричного уравнения для диагонализации $\jsq$ увеличиваться не будет (в прямоугольнике количество точек не будет увеличиваться).
  $$
    \begin{array}{ccc}
      J = j_1 + j_2 & : & 2J + 1 = 2 (j_1 + j_2) + 1 \\
      J = j_1 + j_2 - 1 & : & 2 J + 1 = 2(j_1 + j_2) - 1 \\
      \hdotsfor{3} \\
      J = |j_1 - j_2| & : & 2|j_1 - j_2| + 1
    \end{array}
  $$
  Количество разных значений $J$ равно $2 j_{\min} + 1$.
\end{enumerate}

Находим количество линейно независимых собственных векторов
\begin{eqnarray*}
    \sum_{|j_1 - j_2|}^{j_1 + j_2} (2J + 1) &=& \dfrac{1}{2} \left(
        2 |j_1 - j_2| + 1 + 2(j_1 + j_2) + 1
    \right) \cdot (2 j_{\min} + 1)\\
    &=& (2j_{\max} + 1)(2j_{\min} + 1)\\
    &=& (2j_1 + 1) (2j_2 + 1)
\end{eqnarray*}
Итак, подводя итог, правила выглядят следующим образом:
$$
    \left\{
      \begin{array}{ll}
        M = m_1 + m_2, & \hbox{} \\
        J = j_1 + j_2, \ldots, |j_1 - j_2|. & \hbox{}
      \end{array}
    \right.
$$
