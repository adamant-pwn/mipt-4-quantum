\chapter{Одномерное движение}
Будем отталкиваться от одномерного уравнения Шрёдингера
$$
    i \hbar \ud{}{t} \qu{\psi(t)} = \hat H \qu{\psi(t)} \eqno{(\ast)}
$$ 
Гамильтониан определяется из принципа соответствия
\section{Линейный гармонический осциллятор}
Будем решать $(\ast)$ в абстрактном гильбертовом пространстве, без конкретного представления.

\textbf{Задача:} найти собственные функции оператора $\hat H$,
\begin{eqnarray*}
    \Hat H &=& \dfrac{\Hat p^2}{2m} + \dfrac{m \omega^2 \Hat x^2}{2}\\\index{Оператор! гамильтониан}
    \Hat H \qu{\psi_E} &=& E \qu{\psi_E}
\end{eqnarray*}
\subsection{Обезразмеривание}
Уравнение приводится к виду
$$
    \dfrac{\Hat H}{\hbar \omega} = \dfrac{1}{2} \left( \Hat P^2 + \Hat X^2 \right),
$$
где
$$
    \Hat P = \dfrac{\Hat p}{p_0}, \qquad \Hat X = \dfrac{\Hat x}{x_0},
$$
$$
    x_0 = \sqrt{\dfrac{\hbar}{m\omega}}, \qquad p_0 = \sqrt{\hbar m \omega}
$$
$x_0$, $p_0$~--- осцилляторные единицы координаты и импульса.\index{Осцилляторная единица! координаты}\index{Осцилляторная единица! импульса}

Новое значение коммутатора имеет вид:
$$
    [\Hat X, \Hat P] = i \hbar \dfrac{1}{\hbar} \Hat 1 = i
$$
\subsection{<<Разложение на множители>>}
Обозначим
$$
    \begin{cases}
        \Hat a = \dfrac{1}{\sqrt{2}} \left(\Hat X + i \Hat P \right)\\
        \Hat a^+ = \dfrac{1}{\sqrt{2}} \left(\Hat X - i \Hat P \right)        
    \end{cases}
$$\index{Оператор! рождения частиц}\index{Оператор! уничтожения частиц}
Эти операторы всего лишь сопряжены друг другу\wikiq{Почему?}, они не являются эрмитовыми или унитарными.
\begin{eqnarray*}
    \Hat a^+ \Hat a &=& \dfrac12 (\Hat X - i \Hat P)(\Hat X + i \Hat P)\\
                    &=& \dfrac12 (\Hat X^2 + \Hat P^2 - i[\Hat P, \Hat X])\\
                    &=& \dfrac{\Hat H}{\hbar \omega} - \dfrac12 \hat 1
\end{eqnarray*}
Окончательное выражение для гамильтониана\footnote{единичный оператор обычно опускают}:
$$
    \underline{\hat H = \hbar \omega (\Hat a^+ \Hat a + \dfrac12)}
$$\index{Оператор! гамильтониан}
\subsection{Алгебра операторов $\Hat a^+$, $\Hat a$, и функций от них}
$$
    \Hat N \eqdef \Hat a^+ \Hat a
$$\index{Оператор! числа частиц}
\Lem $[\Hat a, \Hat a^+] = \Hat 1$

\Proof
\begin{eqnarray*}
    [\hat a, \hat a^+] &=& \dfrac12 [\Hat X + i \Hat P, \Hat X - i \hat P]\\
                        &=& \dfrac12 (-i [\Hat X, \Hat P] + i [\Hat P, \Hat X]) = \Hat 1
\end{eqnarray*}
Оператор
$$
    \hat N = \Hat a^+ \Hat a
$$
является эрмитовым, и его собственные векторы~--- собственные векторы гамильтониана (он \emph{линейно связан} с гамильтонианом)
$$
\begin{cases}
    [\Hat N, \Hat a] = [\Hat a^+ \Hat a, \Hat a] = [\Hat a^+, \Hat a]\Hat a = -\Hat a,\\
    [\Hat N, \Hat a^+] = \Hat a^+
\end{cases}
$$
\subsection[Собственные векторы и значения $\Hat H, \Hat N$]{Собственные векторы и значения гамильтониана (оператора $\Hat N$)} 
\begin{enumerate}
  \item Предполагаем, что найдётся хотя бы одно решение уравнения
  $$
    \Hat N \qu{\lam} = \lam \qu{\lam}, \qquad \qu{\lam} \in \H,
  $$
  причём $\qs{\lam}{\lam} = 1$, то есть $\qu{\lam}$ лежит в дискретном спектре.\index{Спектр! дискретный}
  \item
  \begin{eqnarray*}
    \Hat N \Hat a^+ \qu{\lam} &=& (\Hat a^+ \Hat N + [\Hat N, \, \hat a^+]) \qu{\lam}\\
    &=& (\Hat a^+ \Hat N + \Hat a^+ ) \qu \lam \\
    &=& (\lam+1) \Hat a^+ \qu \lam
  \end{eqnarray*}
  Если предположить, что у оператора $\Hat N$ существует хотя бы один собственный вектор, то действие на собственный вектор оператора $\Hat a^+$ даёт собственный вектор для собственного значения $(\lam + 1)$.
  \item Аналогично,
  $$
    \Hat N \Hat a \qu \lam = (\lam - 1) \Hat a \qu \lam
  $$
\end{enumerate}
\textbf{Вывод:}
Действуя операторами $\Hat a$, $\Hat a^+$ на какой-нибудь собственный вектор $\qu \lam$, можно построить цепочку собственных векторов с собственными значениями $\lam \pm 1, \lam \pm 2, \ldots$
$$
    \begin{cases}
        \Hat a^+ \qu \lam = C \qu{\lam + 1}\\
        \Hat a \qu \lam = D \qu{\lam - 1}
    \end{cases}
$$
$\Hat a^+, \hat a$~--- <<лестничные>> операторы.\index{Оператор! лестничный}
\subsection{Лестничные операторы}
%Для множества собственных значений $\lam$ могут выполняться две возможности:
%\begin{enumerate}
%  \item Пусть $\lam$~--- любое вещественное. Тогда $\lam \geqslant 0$ или $\lam < 0$
%  \item 
%\end{enumerate}
Покажем, что спектр собственных значений $\Hat N$ ограничен снизу.
$$
    \lam = \qtri{\lam}{\Hat N}{\lam} = \qtri{\lam}{\hat a^+ \hat a}{\lam} = \big\| \hat a \qu \lam  \big\|^2 \geqslant 0
$$
\textbf{Утверждение:} должен быть собственный вектор, отвечающий собственному значению 0.\wikiq{Почему?}
$$
    \hat a \qu 0 = 0
$$
Если бы было дробное собственное значение $\lam'$, то они бы порождало цепочку $\lam' \pm 1, \ldots$, получили бы спектр, не ограниченный снизу. Этого быть не может.

Таким образом, собственные значения имеют вид $n = 0, 1, 2, \ldots$,
$$
    E_n = \hbar \omega \left( n + \dfrac12 \right)
$$
Других собственных значений нет.

\textbf{Интерпретация:}
\begin{enumerate}
  \item Традиционный подход.
  
  $\Hat H$ описывает систему с одной частицей,
  $$
    \Hat H = \dfrac{\Hat p^2}{2m} + \dfrac{m \omega^2 \Hat x^2}{2}
  $$\index{Оператор! гамильтониан}
  При этом $\qu 0$~--- основное состояние\index{Состояние! основное}, $E_0 = \dfrac{\hbar \omega}{2}$
  \item Другой подход.
  $$
    \Hat H = \hbar \omega (\Hat a^+ \Hat a + \dfrac12)
  $$
  Система из $n$ тождественных неотличимых частиц. $n$~--- собственное значение.
  
  \begin{itemize}
    \item $\Hat N$~--- оператор числа частиц.\index{Оператор! числа частиц} $n$~--- его собственное значение~--- число частиц.
    \item $\qu 0$~--- основное состояние без частиц (вакуум)\index{Состояние! вакуум}
    \item $\qu n$~--- состояние с $n$ частицами (возбуждённое)\index{Состояние! вобужденное}
  \end{itemize}
  Эти частицы называют \emph{квантами}, или \emph{квазичастицами}\index{Квант}\index{Квазичастица} (пример\footnote{род частиц зависит от значения $\omega$}: фотоны, фононы).
  \begin{itemize}
    \item $\Hat a^+$~--- оператор рождения частиц\index{Оператор! рождения частиц}
    \item $\Hat a$~--- оператор уничтожения частиц\index{Оператор! уничтожения частиц}
  \end{itemize}
\end{enumerate}
$$
    \begin{cases}
        \Hat a^+ \qu n = C \qu{n+1},\\
        \Hat a \qu n = D \qu{n-1}
    \end{cases}
$$
Наша задача~--- найти $C$, $D$.
$$
    \begin{cases}
        \Hat a^+ \qu n = C \qu{n+1}\\
        \uq{n} \Hat a = C^\ast \uq{n+1}
    \end{cases}
$$
$$
    \qtri{n}{\Hat a \Hat a^+}{n} = |C|^2 \qs{n+1}{n+1} = |C|^2
$$
$$
    \qtri{n}{\hat a^+ \hat a + \hat 1}{n} = \qtri{n}{\Hat + \hat 1}{n} = n+1
$$
С точностью до общей фазы (в любом случае, мы её не можем наблюдать),
$$
\underline{    C = \sqrt{n+1}}
$$
Отсюда получаем:
$$
    \begin{cases}
        \hat a^+ \qu n = \sqrt{n+1} \qu{n+1}\\
        \hat a \qu{n} = \sqrt{n} \qu{n-1}
    \end{cases}
$$
\Quest{Как получить $\qu n$?}

\Ans Найдём $\qu 0$, затем
\begin{eqnarray*}
    \qu 1 &=& \dfrac{\Hat a^+}{\sqrt{1!}} \qu 0\\
    \qu 2 &=& \dfrac{\big(\Hat a^+\big)^2}{\sqrt{2!}} \qu 0\\
    && \cdots\\
    \qu n &=& \dfrac{\big(\Hat a^+\big)^n}{\sqrt{n!}} \qu 0\\
\end{eqnarray*}
\Exercise Проверить, что $\qs{n}{n'} = \dta_{nn'}$
\subsection{Замечания математического характера}
\begin{itemize}
  \item Можно доказать, что $\Hat N$ образует сам по себе полный набор\index{Полный набор} совместных наблюдаемых, то есть его собственные значения однозначно характеризует собственные векторы.
      
      Это означает, что не требуется дополнительных операторов для  различения собственных векторов, с оператором $\Hat N$ коммутируют только $f(\Hat N)$:
      $$
        [\Hat A, \Hat N] = 0 \LRA \Hat A = f(\Hat N)      $$
  \item Система собственных векторов $\Hat N$ $\big\{ \qu n \big\}$ полна.
\end{itemize}
\subsection{Координатное представление}
\index{Представление! координатное}
$$
    \qs{\xi}{n} = \qs{\frac{x}{x_0}}{n} = \psi_n (\xi)
$$
Сопоставим:
$$
    \begin{cases}
        \Hat a \to \dfrac{1}{\sqrt{2}} \left( \xi + \ud{}{\xi} \right)\\
        \Hat a^+ \to \dfrac{1}{\sqrt{2}} \left( \xi - \ud{}{\xi} \right)
    \end{cases}
$$
Подставляя $\Hat a \qu 0 = 0$, получаем:
$$
    \psi_0(\xi) = \left( \xi + \ud{}{\xi} \right) \qs{\xi}{0} = 0
$$
$$
    \dfrac{d \qs{\xi}{0}}{\qs{\xi}{0}} = -\xi d \xi
$$
Основное состояние:
$$
    \qs{\xi}{0} = c_0 e^{-\nicefrac{\xi^2}{2}}
$$
В координатном представлении:
\begin{eqnarray*}
    \qs{\xi}{n} = \psi_n(\xi) &=& \dfrac{1}{2^n n!} \left( \xi - \ud{}{\xi} \right)^n \dfrac{1}{\sqrt[4]{\pi}} e^{-\nicefrac{\xi^2}{2}}\\
                              &=& \dfrac{1}{2^n n! \sqrt{\pi}} H_n(\xi)
\end{eqnarray*}
$H_n(\xi)$~--- полином Эрмита.\index{Полином Эрмита}
$$
    H_n(\xi) = (-1)^n e^{\xi^2} \dfrac{d^n}{d \xi^n} e^{-\xi^2}
$$
Система эрмитовых полиномов полна в $\mathcal{L}_2$
