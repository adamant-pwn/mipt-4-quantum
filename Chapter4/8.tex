\chapter{Одномерное движение}
Будем отталкиваться от одномерного уравнения Шрёдингера
\begin{equation}
    i \hbar \ud{}{t} \qu{\psi(t)} = \hat H \qu{\psi(t)}
    \label{eq::1d_schrodinger}
\end{equation}

Гамильтониан определяется из принципа соответствия

\section{Линейный гармонический осциллятор}
Будем решать~\eqref{eq::1d_schrodinger} в абстрактном гильбертовом пространстве, без конкретного представления.

\textbf{Задача:} найти собственные векторы и спектр оператора $\hat H$,
\begin{eqnarray*}
    \Hat H &=& \dfrac{\Hat p^2}{2m} + \dfrac{m \omega^2 \Hat x^2}{2}\\
    \Hat H \qu{\psi_E} &=& E \qu{\psi_E}
\end{eqnarray*}
\subsection{Обезразмеривание}
Гамильтониан приводится к виду
$$
    \dfrac{\Hat H}{\hbar \omega} = \dfrac{1}{2} \left( \Hat P^2 + \Hat X^2 \right),
$$
где
$$
    \Hat P = \dfrac{\Hat p}{p_0}, \qquad \Hat X = \dfrac{\Hat x}{x_0},
$$
$$
    x_0 = \sqrt{\dfrac{\hbar}{m\omega}}, \qquad p_0 = \sqrt{\hbar m \omega}
$$
$x_0$, $p_0$~--- осцилляторные единицы координаты и импульса.

Новое значение коммутатора имеет вид:
$$
    [\Hat X, \Hat P] = i \hbar \dfrac{1}{\hbar} \Hat 1 = i
$$
\subsection{<<Разложение на множители>>}
Обозначим
$$
    \begin{cases}
        \Hat a = \dfrac{1}{\sqrt{2}} \left(\Hat X + i \Hat P \right)\\
        \Hat a^+ = \dfrac{1}{\sqrt{2}} \left(\Hat X - i \Hat P \right)
    \end{cases}
$$
Эти операторы всего лишь сопряжены друг другу\wikiq{Почему?}, они не являются эрмитовыми или унитарными. \textsl{ЛЕКТОР : $\Hat{X}^{+}=\Hat{X}$, а также
$\Hat{P}^{+}=\Hat{P}$, остальное очевидно, исходя из явного вида операторов $\Hat a$ и $\Hat a^+$}.
\begin{eqnarray*}
    \Hat a^+ \Hat a &=& \dfrac12 (\Hat X - i \Hat P)(\Hat X + i \Hat P)\\
                    &=& \dfrac12 (\Hat X^2 + \Hat P^2 - i[\Hat P, \Hat X])\\
                    &=& \dfrac{\Hat H}{\hbar \omega} - \dfrac12 \hat 1
\end{eqnarray*}
Окончательное выражение для гамильтониана\footnote{Единичный оператор обычно опускают}:
$$
    \underline{\hat H = \hbar \omega (\Hat a^+ \Hat a + \dfrac12)}
$$
\subsection{Алгебра операторов $\Hat a^+$, $\Hat a$ и функций от них}
$$
    \Hat N \eqdef \Hat a^+ \Hat a
$$
\Lem $[\Hat a, \Hat a^+] = \Hat 1$

\Proof
\begin{eqnarray*}
    [\hat a, \hat a^+] &=& \dfrac12 [\Hat X + i \Hat P, \Hat X - i \hat P]\\
                        &=& \dfrac12 (-i [\Hat X, \Hat P] + i [\Hat P, \Hat X]) = \Hat 1
\end{eqnarray*}
Оператор
$$
    \hat N = \Hat a^+ \Hat a
$$
является эрмитовым, и его собственные векторы~--- собственные векторы гамильтониана (он \emph{линейно связан} с гамильтонианом)
$$
\begin{cases}
    [\Hat N, \Hat a] = [\Hat a^+ \Hat a, \Hat a] = [\Hat a^+, \Hat a]\Hat a = -\Hat a,\\
    [\Hat N, \Hat a^+] = \Hat a^+
\end{cases}
$$
\subsection[Собственные векторы и значения $\Hat H, \Hat N$]{Собственные векторы и значения гамильтониана (оператора $\Hat N$)}
\begin{enumerate}
  \item Предполагаем, что найдётся хотя бы одно решение уравнения
  $$
    \Hat N \qu{\lam} = \lam \qu{\lam}, \qquad \qu{\lam} \in \H,
  $$
  причём $\qs{\lam}{\lam} = 1$, то есть $\lam$ лежит в дискретном спектре.
  \item
  \begin{eqnarray*}
    \Hat N \Hat a^+ \qu{\lam} &=& (\Hat a^+ \Hat N + [\Hat N, \, \hat a^+]) \qu{\lam}\\
    &=& (\Hat a^+ \Hat N + \Hat a^+ ) \qu \lam \\
    &=& (\lam+1) \Hat a^+ \qu \lam
  \end{eqnarray*}
  Если предположить, что у оператора $\Hat N$ существует хотя бы один собственный вектор, то действие на собственный вектор оператора $\Hat a^+$ даёт собственный вектор, отвечающий собственному значению $(\lam + 1)$.
  \item Аналогично,
  $$
    \Hat N \Hat a \qu \lam = (\lam - 1) \Hat a \qu \lam
  $$
\end{enumerate}
\textbf{Вывод:}
действуя операторами $\Hat a$, $\Hat a^+$ на какой-нибудь собственный вектор $\qu \lam$, можно построить цепочку собственных векторов с собственными значениями $\lam \pm 1, \lam \pm 2, \ldots$
$$
    \begin{cases}
        \Hat a^+ \qu \lam = C \qu{\lam + 1}\\
        \Hat a \qu \lam = D \qu{\lam - 1}
    \end{cases}
$$
$\Hat a^+, \hat a$~--- <<лестничные>> операторы.
\subsection{Лестничные операторы}
%Для множества собственных значений $\lam$ могут выполняться две возможности:
%\begin{enumerate}
%  \item Пусть $\lam$~--- любое вещественное. Тогда $\lam \geqslant 0$ или $\lam < 0$
%  \item 
%\end{enumerate}
Покажем, что спектр собственных значений $\Hat N$ ограничен снизу.
$$
    \lam = \qtri{\lam}{\Hat N}{\lam} = \qtri{\lam}{\hat a^+ \hat a}{\lam} = \big\| \hat a \qu \lam  \big\|^2 \geqslant 0
$$
\textbf{Утверждение:} должен быть собственный вектор, отвечающий собственному значению 0.\wikiq{Почему?}

\textsl{Квадрат нормы вектора всегда неотрицателен: $\||\psi\rangle\|^2=\langle\psi|\psi\rangle\geqslant0$, причем равенство в данной формуле достигается тогда и только тогда, когда $|\psi\rangle=0$, поэтому и должен
существовать собственный вектор оператора ${\Hat N}$, отвечающий собственному значению $\lam =0$, т.е. должно выполняться условие $\hat{ a}|0\rangle=0$. Это -- очень важное условие, потому что оно препятствует неограниченному "спуску по лестнице вниз"{}, т.е. именно оно ограничивает спектр оператора ${\Hat N}$ снизу значением $\lam =0$.}
$$
    \hat a \qu 0 = 0
$$
Таким образом, собственные значения имеют вид $n = 0, 1, 2, \ldots$,
$$
    E_n = \hbar \omega \left( n + \dfrac12 \right)
$$
Других собственных значений нет.

\Quest{А все ли собственные значения $\lam$ мы нашли?}
\Ans Все! Любое собственное значение оператора ${\Hat N}$, например,  $\lam'$ порождает цепочку собственных значений $\lam' \pm 1, \ldots$. А если это -- не целые неотрицательные числа, -- то мы получаем спектр, не ограниченный снизу. Этого быть не может.

\textbf{Интерпретация.}
\begin{enumerate}
  \item Традиционный подход.

  $\Hat H$ описывает систему с одной частицей (одночастичную систему),
  $$
    \Hat H = \dfrac{\Hat p^2}{2m} + \dfrac{m \omega^2 \Hat x^2}{2}
  $$
  При этом $\qu 0$~--- основное состояние, $E_0 = \dfrac{\hbar \omega}{2}$, $|n\rangle$ -- возбужденные состояния.
  \item Другой подход.
  $$
    \Hat H = \hbar \omega (\Hat a^+ \Hat a + \dfrac12)
  $$
Гамильтониан  ${\Hat H}$ описывает \emph{не одночастичную систему}, а систему из $n$ тождественных (неразличимых в квантовом смысле\footnote{Об этом подробнее -- в следующем семестре!}) частиц. Каждая частица имеет энергию $\hbar\omega$. Число частиц при этом \emph{не фиксировано}, оно может изменяться, но собственные векторы оператора ${\Hat H}$ (или ${\Hat N}$) отвечают определенным значениям "числа частиц"{}, т.е. $n$.

  \begin{itemize}
    \item $\Hat N$~--- оператор числа частиц. $n$~--- его собственное значение~--- число частиц.
    \item $\qu 0$~--- основное состояние или состояние без частиц (вакуум).
    \item $\qu n$~--- состояние, в котором возбуждено $n$ частиц (возбуждённое).
  \end{itemize}
  Эти частицы называют \emph{квантами}, или \emph{квазичастицами} (пример\footnote{Род частиц зависит от значения $\omega$}: фотоны, фононы).
  \begin{itemize}
    \item $\Hat a^+$~--- оператор рождения частиц.
    \item $\Hat a$~--- оператор уничтожения частиц.
  \end{itemize}
\end{enumerate}
\textbf{Действие операторов рождения и уничтожения частиц:}
$$
    \begin{cases}
        \Hat a^+ \qu n = C \qu{n+1},\\
        \Hat a \qu n = D \qu{n-1}
    \end{cases}
$$
Наша задача~--- найти $C$, $D$.
$$
    \begin{cases}
        \Hat a^+ \qu n = C \qu{n+1}\\
        \uq{n} \Hat a = C^\ast \uq{n+1}
    \end{cases}
$$
$$
    \qtri{n}{\Hat a \Hat a^+}{n} = |C|^2 \qs{n+1}{n+1} = |C|^2
$$
$$
    \qtri{n}{\hat a^+ \hat a + \hat 1}{n} = \qtri{n}{\Hat N + \hat 1}{n} = n+1
$$
С точностью до общей фазы (в любом случае, мы её не можем наблюдать),
$$
\underline{    C = \sqrt{n+1}}
$$
Отсюда получаем:
$$
    \begin{cases}
        \hat a^+ \qu n = \sqrt{n+1} \qu{n+1}\\
        \hat a \qu{n} = \sqrt{n} \qu{n-1}
    \end{cases}
$$
\Quest{Как получить $\qu n$?}

\Ans Найдём $\qu 0$, для этого нужно решить операторное уравнение $\hat{ a}|0\rangle=0$, затем
\begin{eqnarray*}
    \qu 1 &=& \dfrac{\Hat a^+}{\sqrt{1!}} \qu 0\\
    \qu 2 &=& \dfrac{\big(\Hat a^+\big)^2}{\sqrt{2!}} \qu 0\\
    && \cdots\\
    \qu n &=& \dfrac{\big(\Hat a^+\big)^n}{\sqrt{n!}} \qu 0\\
\end{eqnarray*}
\Exercise Проверить, что $\qs{n}{n'} = \dta_{nn'}$
\subsection{Замечания математического характера}
\begin{itemize}
  \item Можно доказать, что $\Hat N$ образует сам по себе полный набор совместных наблюдаемых, то есть его собственные значения однозначно характеризует собственные векторы (ни одно собственное значение оператора $\Hat N$ не является вырожденным).

      Это означает, что не требуется вводить дополнительных операторов, коммутирующих с $\Hat N$ для различения собственных векторов (по набору собственных значений), с оператором $\Hat N$ коммутируют только $f(\Hat N)$:
      $$
        [\Hat A, \Hat N] = 0 \LRA \Hat A = f(\Hat N)      $$
  \item Система собственных векторов $\big\{ \qu n \big\}$ оператора $\Hat N$ полна. С одной стороны, мы нашли все его собственные векторы (см. выше), а с другой стороны, мы знаем, что система собственных векторов эрмитова (самосопряженного) оператора полна. Отсюда следует полнота системы собственных векторов оператора $\Hat N$. Это значит, что если "натянуть" на базис $\big\{ \qu n \big\}$ пространство $\mathcal{H}$, то это будет именно гильбертово пространство\footnote{Детали можно найти в книге А. Мессиа <<Квантовая механика>> 2 тома, М: Наука, 1978}.
\end{itemize}

\subsection{Координатное представление}

Оно строится при помощи функций
$$
    \qs{\xi}{n} = \qs{\frac{x}{x_0}}{n} = \psi_n (\xi)
$$
Перейдем в координатное представление. Тогда будет:
$$
    \begin{cases}
        \Hat a \to \dfrac{1}{\sqrt{2}} \left( \xi + \ud{}{\xi} \right)\\
        \Hat a^+ \to \dfrac{1}{\sqrt{2}} \left( \xi - \ud{}{\xi} \right)
    \end{cases}
$$
Подставляя явный вид оператора $\Hat a$ в уравнение $\Hat a \qu 0 = 0$, получаем дифференциальное уравнение для волновой функции основного состояния $\psi_0(\xi)=\qs{\xi}{0}$:
$$
     \left( \xi + \ud{}{\xi} \right) \qs{\xi}{0} = 0
$$
$$
    \dfrac{d \qs{\xi}{0}}{\qs{\xi}{0}} = -\xi d \xi
$$
Основное состояние:
$$
    \qs{\xi}{0} = c_0 e^{-\nicefrac{\xi^2}{2}}\quad \text{где} \quad c_0=1/\sqrt[4]{\pi}.
$$
В координатном представлении:
\begin{eqnarray*}
    \qs{\xi}{n} = \psi_n(\xi) &=& \dfrac{1}{\sqrt{2^n n!}} \left( \xi - \ud{}{\xi} \right)^n \dfrac{1}{\sqrt[4]{\pi}} e^{-\nicefrac{\xi^2}{2}}\\
                              &=& \dfrac{1}{\sqrt{2^n n! \sqrt{\pi}}} e^{-\nicefrac{\xi^2}{2}} H_n(\xi)
\end{eqnarray*}
$H_n(\xi)$~--- полином Эрмита.
$$
    H_n(\xi) = (-1)^n e^{\xi^2} \dfrac{d^n}{d \xi^n} e^{-\xi^2}
$$
Известный из математики факт: система функций $\psi_n(\xi)$ (см. выше) полна в пространстве $L_2(\R)$. Значит, найдены действительно все стационарные состояния осциллятора. 