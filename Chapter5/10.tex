%\section{Общие собственные функции оператора $\Hat I^2$}
Как было показано выше, в сферических координатах справедлива запись:
$$
    \begin{cases}
        \Hat L^2 &\to -\hbar \Dta_{\theta, \phi}\\
        \Hat L_z &\to -i \hbar \ud{}{\phi}
    \end{cases}
$$
Уравнение
$$
    \begin{cases}
        -\Dta_{\theta, \phi} Y(\theta, \phi) = \ell (\ell + 1) Y(\theta, \phi)\\
        -i \ud{}{\phi} Y(\theta, \phi) = m Y(\theta, \phi)
    \end{cases}
$$
 решается методом разделения переменных:
$$
    Y(\theta, \phi) = Y_\ell^{(m)} (\theta, \phi), \quad 0 \leqslant \theta \leqslant \pi, \quad 0 \leqslant \phi \leqslant 2\pi
$$

Если поставить условие ограниченности, получаем решение в виде сферических функций.
$$
    Y_\ell^{(m)} (\theta, \ell) = C_\ell^{(m)} e^{i m\phi} P_\ell^{(m)} (\cos \theta), \quad \ell = 0, 1, 2, \ldots, \quad m = 0, \pm 1, \ldots, \pm \ell
$$

$P_\ell^{(m)}$~--- присоединённый полином Лежандра.
$$
    P_\ell^{(m)} = (1-x^2)^{m/2} \dfrac{d^{\ell + m}}{dx^{\ell + m}} (x^2 - 1)^{\ell} \dfrac{1}{2^\ell \ell!}
$$

Известный факт: $\L_2(\R^3)$ на сфере $(r = \mathrm{const})$ образует ортогональную систему функций:
\begingroup

$$
    \int_0^{2\pi} d\phi \int_0^\pi \sin \theta d \theta
    Y_\ell ^{(m)\ast} Y_{\ell'}^{(m')}= \dta_{\ell \ell'}\dta_{mm'}
$$

Значит, любую функцию на единичной сфере можно разложить по единичному базису:
$$
    \psi(\theta, \phi) = \sum_{\ell = 0}^\infty \sum_{m = -\ell}^{\ell} A_\ell ^{m} Y_{\ell }^{(m)} (\theta, \phi)
$$
\Reminder Состояния со значениями квантового числа $\ell = 0, 1, 2, \ldots$ носят соответственно названия $s, \, p, \, d, \, \ldots$


\subsection{Чётность состояния}
В сферических координатах при замене $\vec r \to -\vec r $:
$$
    \begin{cases}
        r & \to -t\\
        \theta &\to \pi - \theta\\
        \phi &\to \phi + \pi
    \end{cases}
$$
При этом $\cos \theta \to - \cos \theta$

Преобразование волновых функций:
$$
    \begin{cases}
        \Phi_m (\phi + \pi) = (-1)^m \Phi_m (\phi)\\
        \Theta_\ell^m (\pi - \theta) = (-1)^{\ell + m} \Theta_\ell ^{(m)} (\theta)
    \end{cases}
$$
$$
    \Hat I Y_\ell ^{(m)} (\theta, \phi) = (-1)^{\ell + 2m} Y(\theta, \phi),
$$
откуда
$$
    \lam = (-1)^{\ell }
$$

\section{Спин, или собственный механический момент частицы}
\subsection{Гипотеза Уленбека и Гаудсмита}
Некоторые эксперименты в атомах потребовали модификации теории (гипотеза спина, не имеет классического аналога, постулат тождественности в система со многими частицами).
\begin{itemize}
  \item Опыт Эйнштейна-де Гааза (1921).
  Гиромагнитное отношение:
  $$
    \gamma = \dfrac{\mu_z}{L_z} = \dfrac{\ell}{2mc} g
  $$
  где $g$~--- фактор Ланде. Для орбитального движения $g = 1$. В опыте получилось $g = 2$.
  \item Опыт Штерна-Герлаха. Требование полуцелого квантования:
  $$
    2s + 1 = 2n
  $$
  \textbf{Гипотеза}
  \begin{itemize}
    \item   Электрону, наряду с орбитальным, нужно приписать также и собственный (спиновый) механический момент, не связанный с перемещением в пространстве. Его проекция на любое направление может принимать два значения
  $$
    \pm \dfrac{\hbar}{2} \qquad (m_s = \pm\dfrac12)
  $$
    \item Электрон должен обладать и собственным магнитным моментом:
    $$
        \pm \mu_0 = \pm \dfrac{e \hbar}{2mc},
    $$
    где $\mu_0$~--- магнетон Бора.
  \end{itemize}
  \Rem
  \begin{itemize}
    \item $\dfrac{\mu_z}{L_z} = \dfrac{e\hbar}{2mc} \cdot \dfrac{2}{\hbar} = \dfrac{2}{2mc} \cdot g, \qquad g = 2$
    \item
    $$
        L = \hbar^2 \ell (\ell + 1), \qquad s^2 = \hbar^2 s(s+1)
    $$
    При $\hbar \to 0$, $s \to 0$:
    $$
        L^2 = \mathrm{const}
    $$
  \end{itemize}
  \item Первая теория спина Паули
\end{itemize}
\endgroup

\subsection{Теория спина Паули}
Координатное представление в этой части невозможно, так как нарушается эрмитовость операторов.

Оператор спина имеет вид:
$$
    \vec S = \{ \Hat S_x, \Hat S_y, \Hat S_z \}
$$
\textbf{Требования к оператору спина:}
\begin{enumerate}
  \item Эрмитовость
  \item $[\Hat S_i, \Hat S_j] = i \hbar e_{ijk} \Hat S_k$
  \item Операторы компонент спина коммутируют со всеми операторами, которые зависят от координат и импульса. Это происходит потому, что спин~--- внутреннее свойство. Оно должно быть измеримо \emph{независимо} по отношению к возможным перемещениями в пространстве.
\end{enumerate}
Отсюда следует, что оператор спина действует \emph{не в пространстве координат}, а в другом пространстве, \emph{пространстве внутренних переменных}.

\Quest{Как это пространство можно построить?}
Вспомним свойство оператора спина:
\begingroup
\def \Ssq {\vec {\Hat S}^2}
\def \Sz {\Hat S_z}
$$
    \begin{cases}
        \Ssq \qu{s, m_s} = \hbar^2 s (s+1) \qu{s, m_s}\\
        \Sz \qu{s, m_s} = \hbar m_s \qu{s, m_s}\\
    \end{cases}
$$
Мы знаем, что общая система собственных векторов оператора спина и оператора проекции образует базис некоторого пространства размерности $2s + 1$. В этом базисе матрицы, которые соответствуют этим операторам $\Ssq, \Sz$, имеют диагональный вид.

Нас интересует \emph{спиновая функция} $\qs{\sigma}{s, m_s}$, где $s, m_s$~--- квантовые числа, характеристики состояний, $\sigma$~--- индекс представления.

\def \X {\mathcal{X}}

Если $\sigma = m_s'$, то $\X_{s, m_s} \eqdef \qs{m_s'}{s, m_s}$.

Эту функцию принято представлять в виде столбца:
$$
    \X_{s, m_s = s} = \begin{bmatrix}
                        1 & 0 & \ldots & 0 \\
                      \end{bmatrix}^\top
$$
Такие столбцы называются \emph{спиноры}. Всё пространство натянуто на эти два спинора, поэтому спиновое пространство реализуется как $\mathbb{C}^2$.

\textbf{Свойства спиновых функций}
\begin{enumerate}
  \item
  $
    \qs{m'_2}{s, m_s} = \dta_{m_s' m_s},
  $
  $$
    \sum_\sigma = \X_{m_s}^\ast (\sigma) \X_{m_s'}(\sigma) = \dta_{m_s m_s'}
  $$
  \item Полнота
  $$
    \sum_{m_s} \qu{m_s}\uq{m_s} = \Hat 1
  $$
\end{enumerate}
Норма. Как было сказано на прошлой лекции,
  $$
    \|\Hat S\|_{m_s' m_s} =
    \qtri{\frac12, m_s'}{\hat S_x}{\frac12, m_s}
    =
$$
$$
=
    \hbar/2 \left(
        \sqrt{s(s+1) - m_s (m_s + 1)} \dta_{m+_s' m_s + 1} + \sqrt{s(s+1) - m_s ( m_s + 1)} \dta_{m_s' m_s - 1}
    \right) = \dfrac{\hbar}{2} \begin{bmatrix}
                                 0 & 1  \\
                                 1 & 0 \\
                               \end{bmatrix}
  $$
Таким образом,
\begin{itemize}
  \item $$
    \Hat{\vec{S}} = \dfrac{\hbar}{2} \Hat{\vec \sigma}
  $$
  \item
  $$
    \Hat \sigma_x = \begin{bmatrix}
                      0 & 1 \\
                      1 & 0 \\
                    \end{bmatrix},
                    \qquad
    \Hat \sigma_y = \begin{bmatrix}
                      0 & -i \\
                      i & 0 \\
                    \end{bmatrix}, \qquad
    \Hat \sigma_x = \begin{bmatrix}
                      1 & 0 \\
                      0 & -1 \\
                    \end{bmatrix}
  $$
\end{itemize}
\Quest{Почему матрица спина двухрядная? Почему $\sigma_z$ диагональная?}

\Ans{
  $\Hat\sigma_i$ -- оператор над $\mathbb{C}^2$, а базис выбран так,
  что $\Hat\sigma_z \qu{sm_s} = m_s \qu{s,m}$
}
% 1) Гипотеза
% 2) мы работаем в таком представлении

Ещё есть орбитальное пространство. При этом справедливо представление
$$
    \H = \H^{(orbital)} \otimes \H^{(spinal)} \cong L_2 (\R^3) \otimes \C^{2}
$$

\subsection{Волновая функция}
$$
    \Psi (\vec r, \sigma) \to \Psi (\vec r, m_s)
$$
Рассмотрим функцию $\Psi(\vec r, t)$, в которой нет определённого значения проекции спина на ось~$z$.

Так как всего этих проекций может быть всего две, то в каждой пространственно-временной точке можно функцию разложить по функциям, имеющим определённое значение проекции спина.

\begin{eqnarray*}
    \Psi(\vec r, t) &=& \sum_{m_s} \Psi(\vec r, t, m_s) \X_{m_s}\\
    &=& \underbrace{\Psi(\vec r, t, m_s = +\frac12)}_{\psi_1 (\vec r, t)} \begin{bmatrix}
                                          1 \\
                                          0 \\
                                        \end{bmatrix}
    + \underbrace{\Psi(\vec r, t, m_s = -\frac12)}_{\psi_2 (\vec r, t)} \begin{bmatrix}
                                          0 \\
                                          1 \\
                                        \end{bmatrix}\\
    &=& \begin{bmatrix}
    \psi_1 (\vec r, t) \\
          \psi_2 (\vec r, t) \\
        \end{bmatrix}
\end{eqnarray*}
$\psi_{12}$ тоже называются \emph{спинорами}.

$$
    1 = \sum_{m_s} \int d^3 x \big|
        \Psi(\vec r, t, m_s)
    \big|^2 = \int d^3 x \big(
        |\psi_1|^2 + |\psi_2|^2
    \big) = 1
$$

Из вида нормировочного условия следует смысл верхней и нижней компоненты.

$$
    \begin{cases}
        \psi_1^\ast(\vec r, t) \psi_1(\vec r, t) =  W(\vec r, t, \uparrow) d^3 x\\
        \psi_2^\ast(\vec r, t) \psi_2(\vec r, t) =  W(\vec r, t, \downarrow) d^3 x
    \end{cases}
$$

\begin{itemize}
  \item $\psi_1^\ast\psi_1$~--- вероятность обнаружить частицу в момент времени $t$ в объёме $\vec r, \vec r + d \vec r$ на ось $z$, равной $\hbar/2$.

  \item $\psi_2^\ast\psi_2$~--- вероятность обнаружить частицу в момент времени $t$ в объёме $\vec r, \vec r + d \vec r$ на ось $z$, равной $-\hbar/2$.

\end{itemize}

\endgroup

\textbf{Альтернативный способ записи нормировочного условия.}
Сопряжённый вектор:
$$
    \Psi^+ (\vec r, t) = (\psi_1^\ast, \psi_2^\ast)
$$
$$
    \int d^3 x \Psi^+ \Psi = \int d^3 x \left(
        w(\vec r, t, \uparrow) + w(\vec r, t, \downarrow)
    \right) = 1
$$
С помощью этого аппарата можно пользоваться матричным представлением.

Берём единичные операторы из полного пространств состояний (орбитального и спинового)
\begin{eqnarray*}
    \qfive{\Psi_A}{\Hat 1}{\Hat F}{\Hat 1}{\Psi_B} &=&
    \sum_{m_s m_s'} \int d^2 x d^3 x' \qs{\Psi_A}{\vec r, m_s}
    \underbrace{\qtri{\vec r, m_s}{\Hat F}{\vec r', m_s'}}_{\text{матрица $2s+1 \times 2s+1$}}
    \qs{\vec r' m_s'}{\Psi_B}
\end{eqnarray*}
В координатном базисе $\Hat F $ сводится к некоторому оператору $\F \dta(\vec r - \vec r')$ (дельта-функция трёхмерная). Эта функция помогает снять один из интегралов:
\begin{eqnarray*}
    &=& \sum_{m_s m_s'} \int d^3 x
    \underbrace{\qs{\Psi_A}{\vec r, m_s}}_{\text{\parbox{2cm}{\centering сопряженный спинор}}}
    \underbrace{\qtri{m_s}{\Hat\F}{m_s'}}_{\text{матрица $2 \times 2$}}
    \underbrace{\qs{\vec r, m_s}{\Psi_B}}_{\text{спинор}}\\
    &=&\int d^3 x \Psi_A^+ \Hat \F \Psi_B\\
    &=& \qtri{\Psi_A}{\Hat F}{\Psi_B}
\end{eqnarray*}

Оператор магнитного момента:
$$
    \Hat{\vec \mu} = \mu_0 \Hat \sigma = \dfrac{e \hbar}{2mc} \Hat{\vec \sigma} = \gamma \Hat{\vec S}
$$ 