\chapter{Теория углового момента}
\section{Каноническое квантование}
\index{Квантование!каноническое}
Каноническое квантование~--- это квантование при помощи коммутационных соотношений.

\subsection{Коммутационные соотношения}
\Reminder В классической механике есть \emph{орбитальный момент}, или \emph{момент количества движения}
\index{Момент количества движения!квантовый}
%\index{орбитальный момент}
$$
    \vec L = [\vec r \times \vec p] = (L_x, L_y, L_z)^\top
$$
\Quest{Как поставить этой величине в соответствие квантовый оператор?}
Заметим, что:
$$
    \Pois{L_i}{L_j} = e_{ijk} L_k
$$
Имеется две возможности для применения принципа соответствия:
\begin{enumerate}
  \item Заменить $p_i, q_j$ на соответствующие им квантовые операторы $\Hat p_i, \Hat q_j$
  \item Заменить классические скобки Пуассона на квантовые.
\end{enumerate}

При втором подходе оказывается возможным описать вектор $\Hat{\vec S}$~--- \emph{спиновый, или собственный момент}, для которого нет классического аналога.

Под угловым моментом понимается более общее понятие (в частном случае он может являться как спиновым, так и орбитальным)

\Def
\newcommand{\hvec}[1]{\hat{\vec #1}}
$$
    \hvec J = \{\Hat J_x, \Hat J_y, \Hat J_z\}
$$
является оператором углового момента, если выполняется 2 условия:
\begin{enumerate}
  \item Все компоненты являются \emph{наблюдаемыми}
  \item Для них выполняется классическое коммутационное соотношение
  $$
\underline{    [\Hat J_i, \, \Hat J_j] = i \hbar e_{ijk} \Hat J_k}
  $$
\end{enumerate}
\Rem В квантовой механике мы не имеем возможности измерить две координаты углового момента, ибо они попарно не коммутируют.

%make table of definitions

\textbf{Обозначения:}
\begin{enumerate}
  \item
  $$
    \hvec J = \hbar \hvec j
  $$
  $$
    [\hat j_i, \, \hat j_j] = i e_{ijk} \Hat j_k
  $$
  \item Оператор квадратного углового момента
  $$
    {\hvec j}^2 = \hat j_x^2 + \hat j_y^2 + \hat j_z^2
  $$
  $$
    [\hvec j^2, \, \Hat j_\ell] = 0, \quad \forall \ell = 1, 2, 3
  $$
  \item
  $$
    \Hat j_\pm = \Hat j_x \pm i \Hat j_y
  $$
  $$
    [\Hat j_z, \, \Hat j_\pm] = \pm \Hat j_\pm
  $$
  Они не являются эрмитовыми. Роль этих операторов напоминает роль операторов $\Hat a, \Hat a^+$ в одномерном движении.
\end{enumerate}
\Cor $[\hvec j^2, \hat j_\pm] = 0$
\subsection{TODO!!!}
%\subsection{Спектр и общие собственные векторы оператора $\vec j^2$ и $j_z$}

%here points begin
\begin{enumerate}
  \item Для дальнейшего исследования надо предположить, что имеется хотя бы один собственный вектор.
$$
    \begin{cases}
        \hvec j^2 \qu{\lam, \mu} = \lam \qu{\lam, \mu},\\
        \Hat j_z \qu{\lam, \mu} = \mu \qu{\lam, \mu}
    \end{cases}
$$
Предполагаем, что векторы нормированы на 1:
$$
    \qs{\lam, \mu}{\lam, \mu} = 1
$$

  \item \begingroup %orbital operators
\def \jpm {{\hat j}_\pm}
\def \jx {\Hat j_x}
\def \jy {\Hat j_y}
\def \jz {\Hat j_z}
\def \jsq {{\hvec j}^2}
\def \qlm {\qu{\lam,\,\mu}}
Поскольку оператор $\jpm$ коммутирует с оператором $\jsq$,
$$
    \jz \jpm \qlm = \big( \jpm \jz + \underbrace{[\jz,\, \jpm]}_{= \pm \jpm} \big) \qlm = (\mu \pm 1) \jpm \qlm
$$
Можно построить собственные векторы, отвечающие собственным значениям $\mu \pm 1$, и так далее.

\textbf{Вывод:}
\begin{equation}
    \jpm \qlm = N_\pm \qu{\lam, \, \mu \pm 1}
    \label{eq::eigenvec_j_pm}
\end{equation}
Хотим показать, что спектр оператора $\jz$ является ограниченным (см. главу 4)

  \item %\subsection[Спектр проекции $\Hat{j_z}$]{Спектр оператора $\jz$}
\def \lmu {\lam, \, \mu}
\begin{eqnarray*}
    \lam &=& \qtri{\lam,\mu}{\jsq}{\lam,\mu} \\
    &=& \qtri{\lmu}{\Hat j_x^2 + \Hat j_y^2 + \Hat j_z^2}{\lmu}\\
    &=& \qtri{\lmu}{\jx^+ \jx}{\lmu} + \qtri{\lmu}{\jy^+ \jy}{\lmu} + \qtri{\lmu}{\jz^+ \jz}{\lmu} \geqslant 0
\end{eqnarray*}
Каждое слагаемое может быть записано в виде квадрата нормы некоторого вектора.
$$
    \lam = \underbrace{\qtri{\lmu}{\jx^2}{\lmu} + \qtri{\lmu}{\jy^2}{\lmu}}_{\geqslant 0} + \mu^2 \geqslant 0
$$
Отсюда следует, что если при фиксированном $\lam$ число $\mu^2$ ограничено сверху.
А это означает, что
$$
    \mu_{\min} \leqslant \mu \leqslant \mu_{\max}
$$
\begin{center}
Обозначим $\mu_{\max} = j$
\end{center}
Должны выполняться два условия:
\def \jp {\Hat j_+}
\def \jm {\Hat j_-}
$$
    \begin{cases}
        \jp \qu{\lam, j} = 0\\
        \jm \qu{\lam, \mu_{\min}} = 0
    \end{cases}
$$
Для того, чтобы реализовать эти условия, надо найти $N_\pm$. Воспользуемся~\eqref{eq::eigenvec_j_pm}.

  \item %\subsection{Оператор $N_\pm$}
\def \jmp {\Hat j_\mp}
\begin{eqnarray*}
    N_\pm^2 &=& \qtri{\lmu}{\jpm^+ \jpm}{\lmu}\\
    &=& \qtri{\lmu}{\jmp \jpm}{\lmu}\\
    &=& \qtri{\lmu}{ (\jx \mp i \jy)(\jx \pm i \jy) }{\lmu}\\
    &=& \qtri{\lmu}{ \jx^2 + \jy^2 \pm i [\jx, \jy] }{\lmu}\\
    &=& \qtri{\lmu}{\jsq - \jz^2 \mp \jz}{\lmu}\\
    &=& \lam - \mu^2 \mp \mu
\end{eqnarray*}
$$
    \jpm \qlm = \sqrt{\lam - \mu^2 \mp \mu} \qu{\lam, \mu \pm 1}
$$

  \item %\subsection{}
$$
    \begin{cases}
        \jp \qu{\lam, j} = 0\\
        \jm \qu{\lam, \mu_{\min}} = 0
    \end{cases}
    \to
    \begin{cases}
        \lam - j^2 - j = 0\\
        \lam - \mu_{\min}^2 + \mu_{\min} = 0
    \end{cases}
$$
Приравняв $\lam$, получим уравнения, которые связывают $j$, $\mu_{\min}$:
$$
    j^2 + j = \mu_{\min}^2 - \mu_{\min}
$$
$$
    (j + \mu_{\min})\underbrace{(j - \mu_{\min} + 1)}_{\neq 0} = 0
$$
Окончательно:
$$
\underline{    \mu_{\min} = -j}
$$

  \item %\subsection{}
$$
    \jsq \qlm = \lam \qlm
$$
Воспользуемся тем, что
$$
    \jm \jp \qu{\lam, j} = 0
$$
$$
    (\jsq - \jz^2 - \jz) \qu{\lam, j} = 0
$$
Откуда
$$
\underline{    \jsq \qu{\lam, j} = j(j+1) \qu{\lam, j}}
$$
Так как $\lam = j(j+1)$, то в векторе $\qlm$ составляющую $\lam$ не пишут.

Но $\mu$ пробегает все значения от $-j$ до $+j$. Вся система общих собственных векторов может быть найдена из одного любого вектора с помощью наших операторов. Двигаясь с целым шагом, можно получать значения от $-j$ до $j$, откуда
$$
    2j = n \in \mathbb{Z} \RA j = 0, \dfrac12, 1, \dfrac32
$$

  \item %\subsection{}
$$
    \mu = m = -j, -j+1, \ldots, j
$$
Если $\mu$ полуцелое, то $j$ полуцелое. Если $\mu$ целое, то $j$ целое.
$$
    \begin{cases}
        \jsq \qu{j, m} = j(j+1) \qu{j, m}\\
        \jz \qu{j, m} = m \qu{j, m}
    \end{cases}
$$
Для фиксированного числа $m$ (\emph{магнитного квантового числа}) число $j$ пробегает значения от (?)

\Rem В этом пространстве размерности $2j+1$ все векторы линейно независимы. Если действовать на векторы этого пространства операторами $\jsq, \jx, \jy, \jz, \jpm$, мы не будем выходить за пределы этого пространства (то есть оно инвариантно). Несколько позже мы выясним, что пространство также инвариантно относительно вращений.

$\jsq$, $\jz$ как правило не образуют полный набор.

\textbf{Для всех этих операторов построим матричное представление:}
$$
\underline{    \jpm \qu{jm} = \sqrt{j(j+1) - m(m\pm 1)} \qu{j, m \pm 1}}, \qquad \qs{j_{m'}}{j_m} = \dta_{m'm}
$$
$$
    \qtri{j_{m'}}{\jpm}{j_m} = \sqrt{j(j+1) - m(m\pm 1)} \dta_{m', \, m \pm 1}
$$
Берём полусумму и полуразность из формулы $\jpm = \jx \pm i \jy$
\begin{itemize}
  \item $
    \qtri{j,m'}{\jx}{j,m} = \dfrac{1}{2} \sqrt{j(j+1) - m(m+1)} \dta_{m', m+1} +
    \dfrac12 \sqrt{j(j+1) - m(m-1)} \dta_{m', m-1}
$
  \item $
    \qtri{j,m'}{\jy}{j,m} = -\dfrac{i}{2} \sqrt{j(j+1) - m(m+1)} \dta_{m', m+1} +
    \dfrac{i}{2} \sqrt{j(j+1) - m(m-1)} \dta_{m', m-1}
$
  \item $
    \qtri{j,m'}{\jsq}{j,m} = j(j+1) \dta_{m'm}
$
  \item $
    \qtri{j,m'}{\jz}{j,m} = m \dta_{m'm}
$
\end{itemize}
\Quest{Как в этом представлении выглядит вектор состояния?}
$$
    \qs{j_{m'}}{j_m} = (\underbrace{1, 0, \ldots, 0}_{2j+1})^\top
$$
\endgroup

\end{enumerate}

\section{Орбитальный момент количества движения}
\subsection{}
Были написаны две возможности для того, чтобы реализовать принцип соответствия. Вернёмся к первому способу.
Заменим $\vec p$, $\vec r$ на операторы.
\begingroup
\def \nla {\nabla}
\def \hvr {\hat{\vec r}}
\def \hvp {\hat{\vec p}}
\def \hvL {\hat{\vec L}}
    $$
        \vec L = [\vec r \times \vec p] \to
        \hvL = [\hvr \times \hvp ] \to
        -i\hbar [\vec r \times \vec \nla], \quad \hvp = -i\hbar \vec \nla
    $$
Наиболее удобная для этих описаний система координат~--- сферическая.
$$
    \begin{cases}
        x &= r \cos \phi \sin \theta \\
        y &= r \sin \phi \sin \theta \\
        z &= r \cos \phi
    \end{cases}, \, \text{где} \,
    \begin{cases}
        0 \leqslant \phi \leqslant 2\pi&\\
        0 \leqslant \theta \leqslant \pi&
    \end{cases}
$$
$$
    \Hat L_z \to -i\hbar [\vec r \times \vec \nla]_z = -i \hbar \left(
        x \ud{}{y} - y \ud{}{x}
    \right) = -i \hbar \ud{}{\phi}
$$
$$
    \ud{}{\phi} = \ud{}{x} \ud{x}{\phi} + \ud{}{y} \ud{y}{\phi} + \ud{}{z} \ud{z}{\phi}
$$
Поставим вопрос о собственных функциях оператора $L_z$.

\subsection{}

В дираковских обозначениях:
$$
    \qs{\phi}{m} = \Phi_m(\phi), \qquad -i \ud{}{\phi} \qs{\phi}{m} = m \Phi_m(\phi)
$$
$$
    \Phi_m(\phi) = C e^{i m\phi}
$$
\Quest{Чему равно $m$?}

\Ans Нужно найти граничные условия, которые бы определяли значения этого числа. Ставится граничное циклическое условие:
$$
    \Phi_m(\phi + 2\pi) = \Phi_m(\phi)
$$
Циклическое условие означает \emph{однозначность} волновой функции при повороте на $2 \pi$.

Система будет нормирована:
$$
    \int\limits_0^{2\pi} \Phi_m^\ast (\phi) \Phi_{m'} (\phi) d \phi = \dta_{mm'}
$$
Отсюда
$$
    C = \dfrac{1}{\sqrt{2\pi}}
$$

На самом деле, целочисленное квантование появляется и без требования однозначности волновой функции (почему это требование есть? Ведь наблюдаем только квадрат волновой функции).

В случае полуцелого квантования оператор $\hvL$ теряет свойство эрмитовости. В случае целочисленного квантования всё хорошо.

\subsection{}
$$
    \hvL^2 = -\hbar^2 \left(
        \dfrac{1}{\sin \theta} \ud{}{\theta} \left(
            \sin \theta \ud{}{\theta}
        \right) +
        \dfrac{1}{\sin \theta} \left(\ud{}{\phi}\right)^2
    \right) = -\hbar^2 \Delta_{\theta, \phi}
$$
$$
    \Dta = \Dta_r + \dfrac{1}{r^2} \Dta_{\theta, \phi}
$$
\def \hvl {\Hat {\vec \ell}}
\def \qlm {\qu{\ell, m}}
\def \lz {\Hat{\ell_z}}
$$
    \begin{cases}
        \hvl^2 \qlm  = \ell (\ell + 1) \qlm\\
        \lz \qlm = m \qlm
    \end{cases}
$$
%Решением этой системы являются сферические функции.
%$$
%    \begin{cases}
%        -\Dta_{\theta, \phi} Y(\theta, \phi) = \ell (\ell + 1) Y(\theta, \phi)\\
%        -i \ud{}{\phi} Y(\theta, \phi) = m Y(\theta, \phi)
%    \end{cases}
%$$
%$$
%    Y(\theta, \phi) = Y_\ell^{(m)} (\theta, \phi)
%$$
%$$
%    Y_\ell^{(m)} (\theta, \ell) = C_\ell^{(m)} e^{i m\phi} P_\ell^{(m)} (\cos \theta), \quad \ell = 0, 1, 2, \ldots, \quad m = 0, \pm 1, \ldots, \pm \ell
%$$
\endgroup

